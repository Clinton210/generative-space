\chapter{Alignment and Tail Sewing Inside Collapse Fibers}
\label{chap:alignment-sewing}

\section{Introduction}

Collapse fibers contain a vast collection of generative identities that share
the same collapse coordinate.  
Inside a fiber, the selector, digit, and meta-information layers may vary
freely so long as they preserve the ordered list of selected digits.  
The diagonalizer constructed in the next chapter exploits this freedom to
produce identities that agree with a reference identity on all observed
prefixes while diverging arbitrarily on their unobserved tails.

To accomplish this, we need two technical tools:

\begin{itemize}
    \item \emph{alignment}, which identifies corresponding selection indices
    across identities in the same fiber, and

    \item \emph{tail sewing}, which replaces the tail of one identity with the
    tail of another without leaving the fiber.
\end{itemize}

These tools rely on the fact that identities in a collapse fiber have identical
collapse coordinates, even if their selector positions differ.  
Similar alignment ideas appear in the study of synchronized and coded shift
spaces in symbolic dynamics \cite{LindMarcus}, though here the setting is
simpler and more rigid.  
Alignment and tail sewing provide the structural freedom necessary for
controlled constructions inside the fiber.

\section{Alignment of Selected Digits}

Let $H, A \in \mathcal{F}(x)$ be two identities in the collapse fiber of a
real number $x$.  
Let
\[
X(H) = (x_0, x_1, x_2,\ldots)
\qquad \text{and} \qquad
X(A) = (x_0, x_1, x_2,\ldots)
\]
be their collapse coordinates.  
Since both collapse to $x$, these sequences coincide.

Let
\[
h_0 < h_1 < h_2 < \cdots
\qquad\text{and}\qquad
a_0 < a_1 < a_2 < \cdots
\]
be the indices at which $H$ and $A$ expose digits.  
By definition of collapse, the identity $H$ produces digit $x_k$ at position
$h_k$, and $A$ produces the same digit at $a_k$.

\begin{proposition}[Index Alignment]
\label{prop:index-alignment}
For every $k\in\mathbb{N}$, both $H$ and $A$ expose the $k$th digit of the
collapse coordinate at the positions $h_k$ and $a_k$.  
Thus any identity obtained by taking the prefix of $H$ through $h_k$ and the
tail of $A$ starting at $a_k$ preserves the collapse coordinate.
\end{proposition}

\begin{proof}
Since $H$ and $A$ lie in $\mathcal{F}(x)$, both expose the digit $x_k$ at
their respective $k$th selection indices.  
Thus $h_k$ and $a_k$ exist and correspond to the same position in the collapse
coordinate.
\end{proof}

This alignment property ensures that splicing the two identities at matched
selection indices preserves the entire collapse coordinate.

\section{Tail Sewing Inside a Fiber}

Given alignment indices $h_k$ and $a_k$, we may splice the beginning of $H$
to the tail of $A$ to obtain a new identity.

\begin{proposition}[Tail Sewing]
\label{prop:tail-sewing}
Fix $k\in\mathbb{N}$.  
Define an identity $G$ by
\[
G(n) =
\begin{cases}
H(n), & n \le h_k,\\
A(n - h_k + a_k), & n > h_k.
\end{cases}
\]
Then $G \in \mathcal{F}(x)$.
\end{proposition}

\begin{proof}
The identity $G$ agrees with $H$ through the $k$th selected digit, which
occurs at $h_k$.  
Beyond that point it reproduces the $(k+1)$st, $(k+2)$nd, and all later
selected digits from $A$ in the correct order.  
Since $A$ and $H$ have identical collapse coordinates, $G$ reproduces this same
sequence.  
Therefore $\pi(G)=x$.
\end{proof}

Tail sewing shows that once an initial segment of the collapse coordinate has
been fixed, the remainder of the generative identity may be replaced freely by
another tail from the same fiber.

\section{Controlled Tail Replacement via Dependency Bounds}

In diagonalization we must ensure that the sewn identity preserves not only
the collapse coordinate but also the values of a finite family of observers
up to a chosen precision.  
Dependency bounds make this possible.

Let $\mathcal{P}=\{\Phi_1,\ldots,\Phi_m\}$ be a finite family of structural
projections.  
Let $B_{\mathcal{P}}$ be the uniform dependency bound:
\[
B_{\mathcal{P}}(\varepsilon)
  = \max_{1\le i\le m} B_{\Phi_i}(\varepsilon).
\]

Fix $\varepsilon>0$ and set
\[
N = B_{\mathcal{P}}(\varepsilon).
\]
If two identities agree on the first $N$ coordinates, then every projection in
$\mathcal{P}$ evaluates them within $\varepsilon$.

\begin{proposition}[Controlled Tail Sewing]
\label{prop:controlled-tail-sewing}
Let $H,A\in\mathcal{F}(x)$ and let $\mathcal{P}$ be a finite family of
projections.  
Fix $\varepsilon>0$ and let $N=B_{\mathcal{P}}(\varepsilon)$.  
Choose $k$ such that $h_k \ge N$.  
Let $G$ be obtained by sewing the prefix of $H$ through $h_k$ to the tail of
$A$ from $a_k$ onward.  
Then for every $\Phi\in\mathcal{P}$,
\[
|\Phi(G)-\Phi(H)| < \varepsilon.
\]
\end{proposition}

\begin{proof}
Since $G$ and $H$ agree on $[0..h_k]$ and $h_k\ge N$, they agree on the first
$N$ coordinates.  
By the definition of $B_{\mathcal{P}}$, this agreement implies
$|\Phi(G)-\Phi(H)|<\varepsilon$ for every $\Phi\in\mathcal{P}$.
\end{proof}

Thus, once the observers are satisfied on a sufficiently long prefix, the tail
may be freely replaced without affecting any projection in the family at the
chosen precision.

\section{Alignment and Sewing as Fiber Geometry}

Alignment and sewing formalize two structural freedoms inside collapse fibers:

\begin{itemize}
    \item \emph{prefix determination}: the first $k$ selected digits may be
    fixed using any identity in the fiber;

    \item \emph{tail freedom}: after a matched alignment index, the remainder
    of the identity may be replaced arbitrarily by another fiber member.
\end{itemize}

Together with prefix stabilization, these tools allow observers to be frozen
at finite stages while the tail remains open for divergence.  
This is the operational core of the meta-diagonalizer built in the next
chapter.

\section{Summary}

Identities inside a collapse fiber share the same collapse coordinate, even
though their selector and meta-information layers may differ dramatically.  
Index alignment identifies the matching positions at which the same collapse
digit is exposed in different identities.  
Tail sewing uses this alignment to splice identities together while remaining
inside the fiber.

When combined with dependency bounds and prefix stabilization, alignment and
sewing enable the construction of identities that satisfy any finite family of
observers on long prefixes while differing arbitrarily in their tails.  
These constructions form the technical foundation for the indistinguishability
and diagonalization arguments of the next chapter.
