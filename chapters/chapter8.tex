\chapter{Projective Incompatibility}

\section{Introduction}

Chapters~6 and~7 introduced structural projections and their computable
subclass.  
These projections represent the observable features that an effective
measurement procedure can detect in a generative identity.  
Although every such projection is prefix-determined and continuous in the
product topology, different projections may impose mutually incompatible
constraints on the prefix of a mechanism.

This phenomenon—\emph{projective incompatibility}—is central to the
generative viewpoint.  
It occurs whenever two observers require conflicting finite-prefix conditions
that no single generative identity can satisfy simultaneously.  
Incompatibility reveals that finite observational systems cannot jointly
classify the internal structure of collapse fibers, even among computable
generators.

This chapter formalizes projective incompatibility, develops examples, and
establishes fundamental properties that prepare the ground for the
meta-diagonalizer of Chapter~9.

\section{Prefix Constraints Induced by Projections}

A computable structural projection $\Phi$ with dependency bound $B_\Phi$
stabilizes on finite prefixes.  
To compute $\Phi(G)$ to within a tolerance $\varepsilon$, the observer needs to
examine only the prefix of length $B_\Phi(\varepsilon)$.  
Thus each such projection induces a family of finite-prefix constraints.

\begin{definition}[Prefix Constraint]
Let $\Phi$ be a computable structural projection with dependency bound
$B_\Phi$.  
For $\varepsilon>0$ and $G \in \mathcal{X}$, the \emph{$(\Phi,\varepsilon)$-prefix
constraint of $G$} is the finite word
\[
(M{\upharpoonright}B_\Phi(\varepsilon), \,
 D{\upharpoonright}B_\Phi(\varepsilon), \,
 K{\upharpoonright}B_\Phi(\varepsilon)).
\]
Any identity $H$ agreeing with $G$ on this prefix must satisfy
$|\Phi(H)-\Phi(G)|<\varepsilon$.
\end{definition}

Thus, for fixed $\varepsilon$, the projection $\Phi$ partitions $\mathcal{X}$
into finitely many prefix cylinders, each determining $\Phi$ to within
$\varepsilon$.

\section{Compatibility of Projections}

Two projections may be jointly satisfied only if their prefix constraints are
consistent.

\begin{definition}[Compatibility]
Two computable structural projections $\Phi$ and $\Psi$ are
\emph{compatible} if for every $\varepsilon>0$ there exists a mechanism
$G \in \mathcal{X}$ and a prefix length $n$ such that any identity $H$ agreeing
with $G$ on the first $n$ coordinates satisfies
\[
|\Phi(H)-\Phi(G)|<\varepsilon
\quad\text{and}\quad
|\Psi(H)-\Psi(G)|<\varepsilon.
\]
\end{definition}

Informally, $\Phi$ and $\Psi$ are compatible if they can simultaneously
stabilize on arbitrarily small tolerance levels within the same sufficiently
large prefix.

\begin{remark}
This definition is aligned with the meet operation in the projection lattice:
$\Phi$ and $\Psi$ are compatible if their meet $\Phi \wedge \Psi$ is computable.
\end{remark}

\section{Incompatibility via Conflicting Prefix Requirements}

Structural projections may force incompatible prefix conditions on the selector
or meta layers.  
A classical example arises when one observer requires a prefix to exhibit
a high digit-selection frequency, while another demands long stretches of
meta selections.

\begin{proposition}[Basic Incompatibility Criterion]
\label{prop:incompatibility}
Let $\Phi$ and $\Psi$ be computable structural projections with dependency bounds
$B_\Phi$ and $B_\Psi$.  
If for some $\varepsilon>0$ the $\varepsilon$-prefix cylinders required by
$\Phi$ and $\Psi$ disagree on at least one coordinate, then $\Phi$ and $\Psi$
are incompatible.
\end{proposition}

\begin{proof}
Suppose the $\varepsilon$-prefix required by $\Phi$ forces a symbol $a$ at some
position $k < B_\Phi(\varepsilon)$ while the $\varepsilon$-prefix required by
$\Psi$ forces a different symbol $b \neq a$ at the same position.  
No mechanism can satisfy both constraints simultaneously.  
Thus, there is no prefix on which both projections stabilize within
$\varepsilon$, proving incompatibility.
\end{proof}

The criterion is easy to verify in practice and captures many natural cases.

\section{Examples of Incompatible Projections}

\subsection*{Digit-Frequency vs.~Sparse-Selector Projections}

Let $\Phi$ estimate digit density at precision $\varepsilon$ (via short-run
frequency estimates) and let $\Psi$ detect large gaps between successive digit
selections.  
To satisfy $\Phi$ to within $\varepsilon$, the selector must contain many digit
selections in a short prefix.  
To satisfy $\Psi$ to the same tolerance, the selector must display a very long
meta-only interval in the same prefix.

These cannot coexist if both dependency bounds fall below the location of the
forced gap or density peak.  
Thus $\Phi$ and $\Psi$ are incompatible.

\subsection*{Pattern-Detection vs.~Pattern-Avoidance}

Let $\Phi$ detect frequent occurrences of a specific meta-block $w$, while $\Psi$
detects long intervals where $w$ never appears.  
For sufficiently small $\varepsilon$, both projections place contradictory
requirements on $K{\upharpoonright}n$ for the same prefix length.
Thus they are incompatible.

\section{Incompatibility in Collapse Fibers}

Compatibility and incompatibility are defined at the level of the full
generative space, but the phenomenon persists when restricted to collapse fibers.

\begin{proposition}
Let $\Phi$ and $\Psi$ be incompatible computable structural projections.  
Then for any $x \in [0,1]$, no mechanism in the effective fiber
$\mathcal{F}_{\mathrm{eff}}(x)$ can simultaneously satisfy arbitrarily small
tolerance constraints for both $\Phi$ and $\Psi$.
\end{proposition}

\begin{proof}
If $G \in \mathcal{F}_{\mathrm{eff}}(x)$ could satisfy both projections at all
tolerance levels, then the $(\Phi,\varepsilon)$- and $(\Psi,\varepsilon)$-prefix
constraints would be simultaneously satisfiable for arbitrarily small
$\varepsilon$, contradicting incompatibility as in
Proposition~\ref{prop:incompatibility}.
\end{proof}

Thus incompatibility is intrinsic: it does not vanish under the collapse map.

\section{Prefix Conflict and Tail Freedom}

Projective incompatibility reflects a deeper tension: projections impose
prefix-level constraints, while the generative space permits arbitrary
modifications in the tail.  
This tension is exploited by the meta-diagonalizer in Chapter~9, which forces
incompatible constraints to arise on different tails, ensuring that no finite
family of projections can correctly classify all effective identities.

\section{Outlook}

Projective incompatibility reveals that even simple computable observers may
fundamentally disagree about the structure of a generative identity.  
Finite-prefix requirements can contradict each other, and no single mechanism
can satisfy incompatible projections within arbitrarily small tolerances.

This phenomenon sets the stage for Chapter~9, where the meta-diagonalizer uses
tail freedom to escape all finite families of computable projections.  
Incompatibility is therefore a key precursor to the
\emph{Structural Incompleteness Theorem} of Part~IV.
