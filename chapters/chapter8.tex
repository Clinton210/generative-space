\chapter{Finite-N Invariants}
\label{chap:finite-n-invariants}

\section{Introduction}

Observer towers summarize finite information from generative identities. 
A structural projection depends on only a finite prefix of the identity at each precision. 
Finite-N invariants formalize this principle by assigning a numerical value to each identity based on the inspection of the first N entries in the chosen representation. 
These quantities are continuous in the ambient topology, computable when N is fixed, and form the basic building blocks of all asymptotic invariants studied in the next chapter.

Finite-N invariants do not capture structural information about the full generative identity. 
They record only prefix behavior and therefore belong to the observer layer in the hierarchy from Chapter~\ref{chap:incompleteness}. 
They are continuous structural projections, and their dependency bounds coincide with their finite observation radius. 
Appendix~B contains the general theory of dependency bounds used throughout this chapter.

\section{Definition of Finite-N Invariants}

We fix the collapse representation introduced in Chapter~\ref{chap:collapse-map}. 
For each generative identity $G \in \mathcal{G}^{*}$, let $X(G)$ denote the collapse coordinate. 
Finite-N invariants evaluate simple empirical quantities based on the first N entries of $X(G)$ or other prefix-limited features of the identity.

\begin{definition}[Finite-N Invariant]
A finite-N invariant is a continuous function
\[
I_{N} : \mathcal{G}^{*} \to \mathbb{R}
\]
of the form
\[
I_{N}(G) = F\bigl(X(G)\!\upharpoonright\! N,\, G\!\upharpoonright\! N\bigr)
\]
where $F$ is a computable function depending only on the prefix of length N.
\end{definition}

Finite-N invariants include quantities such as:

\begin{itemize}
    \item empirical digit frequencies in $X(G)$,
    \item block-frequency counts,
    \item partial sums of deviations,
    \item local fluctuation measures,
    \item prefix complexity indices defined on raw symbolic coordinates.
\end{itemize}

All such objects are well defined for any fixed N and are continuous in the product topology.

\section{Continuity of Finite-N Invariants}

Finite-N invariants are continuous because they depend on a finite number of coordinates. 
The argument is straightforward.

\begin{proposition}
Every finite-N invariant is a structural projection.
\end{proposition}

\begin{proof}
Fix $N$. 
If two identities $G$ and $H$ agree on the first $N$ coordinates, then both
\[
X(G)\!\upharpoonright\! N = X(H)\!\upharpoonright\! N
\quad\text{and}\quad
G\!\upharpoonright\! N = H\!\upharpoonright\! N.
\]
Since $I_{N}$ depends only on these prefixes, we have $I_{N}(G) = I_{N}(H)$. 
This implies continuity in the product topology, where finite agreement implies arbitrarily small distance.
\end{proof}

Thus each finite-N invariant lies inside the structural projection layer defined in Chapter~\ref{chap:structural-projections}. 
Its dependency bound is exactly $B_{I_{N}}(\varepsilon) = N$ for all sufficiently small $\varepsilon$.

\section{Examples of Finite-N Invariants}

Finite-N invariants are neutral with respect to coordinate meaning. 
They record only symbolic behavior in the fixed representation.

\subsection*{Digit frequency approximants}

Let $a$ be a value in the observed-value alphabet. 
Define
\[
I_{N}^{(a)}(G)
=
\frac{1}{N}\sum_{j < N} \mathbf{1}\bigl[X(G)_{j} = a\bigr].
\]
This counts the empirical frequency of $a$ in the first N observed digits of $X(G)$.

\subsection*{Block frequency approximants}

For a finite block $w$ of observed symbols, define
\[
I_{N}^{(w)}(G)
=
\frac{1}{N}\sum_{j < N} 
\mathbf{1}\bigl[X(G)\!\upharpoonright\! |w| = w\bigr]_{j},
\]
where $[\,\cdot\,]_{j}$ denotes an indicator of $w$ occurring at position $j$. 
These are classical block frequency statistics as used in symbolic dynamics \cite{LindMarcus}.

\subsection*{Deviation or fluctuation approximants}

For example,
\[
I_{N}(G)
=
\frac{1}{N} \sum_{j< N} 
\bigl|X(G)_{j} - \tfrac{1}{b}\sum_{a} a\bigr|.
\]
These quantify deviations from the average value of the observed alphabet.

\subsection*{Prefix-complexity approximants}

Let $K_{N}(G)$ denote the prefix complexity of $X(G)\!\upharpoonright\! N$, using any fixed universal machine. 
Define
\[
I_{N}(G) = \frac{K_{N}(G)}{N}.
\]
This is computable relative to $X(G)\!\upharpoonright\! N$ and is continuous as a finite-N observer.

All these follow the same pattern: an N-limited, continuous functional.

\section{Dependency Bounds and Prefix Behavior}

Because finite-N invariants explicitly depend only on the first N positions, their dependency bounds are trivial:

\[
B_{I_{N}}(\varepsilon) = N \quad \text{for all } 0 < \varepsilon < 1.
\]

Prefix stabilization from Chapter~\ref{chap:prefix-stabilization} therefore states:
if $G$ and $H$ agree on the first N coordinates, then
\[
|I_{N}(G) - I_{N}(H)| = 0.
\]

Finite-N invariants therefore occupy the simplest part of the structural projection layer. 
They are the primary inputs to observer towers.

\section{Finite-N Invariants as Observer Towers}

An observer tower is a sequence
\[
(\Phi_{0}, \Phi_{1}, \Phi_{2}, \ldots),
\]
where each $\Phi_{n}$ is a structural projection with a computable dependency bound. 
Finite-N invariants naturally form such towers by defining
\[
\Phi_{n}(G) = I_{n}(G)
\]
for a chosen invariant template.

\begin{proposition}
Let $(I_{N})_{N\in\mathbb{N}}$ be any sequence of finite-N invariants. 
Then $(I_{N})$ is a computable observer tower.
\end{proposition}

\begin{proof}
Each $I_{N}$ is continuous, and for fixed N the evaluation of $I_{N}(G)$ depends only on the prefix $G\!\upharpoonright\! N$. 
Thus $I_{N}$ has dependency bound $B_{I_{N}}(\varepsilon) = N$ and is computable from this prefix. 
Therefore the sequence $(I_{N})$ is a computable family of structural projections with uniformly computable dependency bounds.
\end{proof}

Finite-N invariants therefore ground the entire observer-based viewpoint of Part~\ref{part:invariants}: asymptotic invariants will be defined as limits of these prefix-limited observers.

\section{Interpretation in the Observer Hierarchy}

Finite-N invariants occupy the lowest level of the invariant layer. 
They provide coarse summaries of the prefix through simple statistics, finite block counts, local deviations, and other computable quantities. 
They do not reveal tail structure or infinite-dimensional features of the generative identity. 
Their behavior is fully determined by the fixed representation, and they are incapable of distinguishing identities that diverge only in the unobserved tail.

The Structural Incompleteness Theorem of Chapter~\ref{chap:incompleteness} implies that even the full tower $(I_{N})_{N\in\mathbb{N}}$ cannot classify any collapse fiber. 
Finite-N invariants are therefore best understood as tools for summarizing prefix behavior, not as structural invariants of generative identities.

\section{Summary}

Finite-N invariants are continuous, computable, finite-prefix observers that summarize the initial segment of a generative identity under the fixed collapse representation. 
They form computable observer towers and serve as the foundation for asymptotic invariants, which appear as limits of these prefix-limited quantities in the next chapter.

The transition from finite-N invariants to asymptotic invariants mirrors the transition from finite observation to idealized infinite observation. 
This shift introduces discontinuities and oscillatory behavior characteristic of Baire class 1 functions, which is the central focus of Chapter~\ref{chap:asymptotic-invariants}.
