\chapter{Alignment and Tail Sewing Inside Fibers}

\section{Introduction}

The collapse fiber $\mathcal{F}(x)$ contains a vast collection of generative
identities that all yield the same classical real number.  
The diagonalizer developed in the next chapter constructs a new identity
inside the effective fiber that matches a reference identity on all observed
prefixes while diverging arbitrarily in its unobserved tail.  
To carry out this construction, we need two technical tools.

The first tool is an alignment procedure.  
Since the collapse depends only on the sequence of selected digits in the
order they appear, we must ensure that when we splice the tail of one
identity onto the prefix of another, the resulting identity produces the same
canonical output.  
The second tool is a sewing procedure, which replaces the tail of one
identity with the tail of another while retaining membership in the same
collapse fiber.

These constructions rely on the fact that identities in a fiber agree on
their selected digits when listed in order, even though the positions of
these digits in the raw sequence may differ.  
This kind of alignment appears in various areas of symbolic dynamics, in
particular in the study of synchronized shift spaces, but here it plays a
more basic role.  
The alignment and sewing tools allow us to replace long tails without
changing the collapsed value.

\section{Alignment of Selected Digits}

Let $H$ and $A$ be two identities in the fiber $\mathcal{F}(x)$, and let
\[
d_H(0), d_H(1), d_H(2), \ldots
\quad\text{and}\quad
d_A(0), d_A(1), d_A(2), \ldots
\]
be their canonical output sequences.  
Since $H$ and $A$ lie in the same fiber, these sequences are identical and
represent the expansion of $x$.

Let
\[
h_0 < h_1 < h_2 < \cdots
\quad\text{and}\quad
a_0 < a_1 < a_2 < \cdots
\]
be the indices at which $H$ and $A$ select digits.  
For any $k$, both identities expose the $k$th digit of $x$ at their
respective indices $h_k$ and $a_k$.

\begin{proposition}[Index Alignment]
For any $k$, there exist positions in $H$ and $A$ at which the $k$th canonical
digit is selected, namely $h_k$ and $a_k$.  
Thus an identity obtained by taking the prefix of $H$ up to $h_k$ and the tail
of $A$ beginning at $a_k$ produces the same canonical output as $H$.
\end{proposition}

\begin{proof}
Since both identities lie in $\mathcal{F}(x)$, the value of the $k$th selected
digit in each must be $x_k$.  
Therefore the alignment indices $h_k$ and $a_k$ exist by definition of the
canonical output.
\end{proof}

This proposition ensures that splicing the two identities at matching digit
indices preserves the canonical output sequence.

\section{Sewing of Tails}

Given two identities $H$ and $A$ in the same fiber, consider the identity
$\tilde{G}$ that agrees with $H$ up to $h_k$ and with $A$ beyond $a_k$.  
Alignment ensures that the canonical output of $\tilde{G}$ equals that of
$H$, so $\tilde{G}$ lies in $\mathcal{F}(x)$.

\begin{proposition}[Tail Sewing]
Fix $k \in \mathbb{N}$.  
Let $G$ be the identity defined by
\[
G(n) =
\begin{cases}
H(n) & n \le h_k,\\
A(n - h_k + a_k) & n > h_k.
\end{cases}
\]
Then $G \in \mathcal{F}(x)$.
\end{proposition}

\begin{proof}
The identity $G$ agrees with $H$ on the prefix containing the first $k$ selected
digits.  
Beyond that prefix it reproduces the $(k+1)$st, $(k+2)$nd, and all later
selected digits of $A$ in order.  
Since $A$ and $H$ have the same canonical output, $G$ reproduces this same
sequence.  
Therefore $\pi(G) = x$.
\end{proof}

This construction replaces the tail of one identity with that of another
without altering the canonical output.  
The ability to modify the tail freely inside the fiber is one of the key
structural freedoms used in the diagonalizer.

\section{Controlled Tail Replacement}

In diagonalization, we do not splice tails arbitrarily.  
Instead, we choose $A$ to satisfy a specific structural property that we want
the final identity to inherit, and we sew its tail onto a reference identity
$H$ after a sufficiently long prefix.

Let $\mathcal{P}$ be a finite family of projections that we wish to match up
to precision $\varepsilon$.  
Let $N = B_{\mathcal{P}}(\varepsilon)$ be the uniform dependency bound.  
If $H$ and $A$ agree on their first $N$ symbols, then sewing the tail of $A$
onto the prefix of $H$ at any alignment point beyond $N$ preserves the
projections to within $\varepsilon$.

\begin{proposition}[Controlled Tail Sewing]
Let $\mathcal{P}$ be a finite family of projections with uniform dependency
bound $B_{\mathcal{P}}$.  
Fix $\varepsilon > 0$ and set $N = B_{\mathcal{P}}(\varepsilon)$.  
Let $h_k$ be the $k$th selection index for $H$, and choose $k$ such that
$h_k \ge N$.  
Similarly, let $a_k$ be the $k$th selection index for $A$.  
Define $G$ by sewing the prefix of $H$ up to $h_k$ to the tail of $A$ from
$a_k$ onward.  
Then for every $\Phi \in \mathcal{P}$,
\[
|\Phi(G) - \Phi(H)| < \varepsilon.
\]
\end{proposition}

\begin{proof}
Since $G$ and $H$ agree on their first $h_k$ symbols and $h_k \ge N$, we have
agreement on the first $N$ symbols.  
By definition of $B_{\mathcal{P}}$, agreement on the first $N$ symbols ensures
agreement of all projections in the family to within $\varepsilon$.
\end{proof}

This shows that once observers are satisfied on the prefix of length $N$, the
tail may be replaced freely without altering their outputs at the chosen
precision.  
This powerful freedom is the main technical ingredient of the diagonalizer.

\section{Summary}

Alignment of selected digits ensures that identities in the same fiber expose
their canonical digits in a coherent order.  
Tail sewing uses this alignment to replace the entire tail of one identity
with the tail of another while remaining inside the collapse fiber.

When combined with dependency bounds and prefix stabilization, these tools
allow us to construct identities that satisfy any finite family of observers
on arbitrarily long prefixes while diverging freely in the unobserved tail.
The next chapter uses these tools to build the meta-diagonalizer, which
demonstrates the impossibility of recovering generative structure from any
finite collection of continuous observers.
