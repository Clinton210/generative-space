\chapter{The Meta Diagonalizer Construction}

\section{Introduction}

Chapters~6 and~7 introduced secondary projections and established that each computable projection depends on only a finite prefix of any effective input.  
This finite dependence is the central mechanism exploited in the diagonalizer construction.  
The goal of this chapter is to build a generative identity that evades a finite family of projections by altering its structure outside their joint dependency bound.

The diagonalizer is constructed by stitching together a reference identity and an adjustment identity.  
The stitching is performed beyond a region of the prefix where all projections stabilize.  
This technique follows the classical method of diagonalization in computability theory while respecting the layered structure of the generative space.

Most technical lemmas used in this chapter appear with full detail in Appendix~B.

\section{Setting and Objectives}

Let
\[
\mathcal{F} 
=
\{ \Phi_1, \ldots, \Phi_m \}
\]
be a finite family of computable secondary projections defined on the effective core $\mathcal{G}_{\mathrm{eff}}$.  
Our objective is to construct an effective generative identity $G^{*}$ such that the values $\Phi_i(G^{*})$ differ from the corresponding values for a given reference identity $H$.

The construction will satisfy three constraints.

\begin{enumerate}
    \item $G^{*}$ is effective.
    \item $G^{*}$ agrees with $H$ on a long initial segment of $(M,D,K)$.
    \item Beyond that segment, $G^{*}$ diverges in a controlled way that forces disagreement with each $\Phi_i$.
\end{enumerate}

This balance of agreement and divergence is achieved through tail modifications that respect dependency bounds.

\section{Uniform Dependency Bound}

Since the family $\mathcal{F}$ is finite, it possesses a uniform dependency bound.

\begin{proposition}[Uniform Dependency Bound]
\label{prop:uniform-bound}
There exists a function $B : \mathbb{Q}^+ \to \mathbb{N}$ such that for every $\varepsilon > 0$ and every $i \le m$, whenever two effective generative identities agree on the first $B(\varepsilon)$ coordinates, the values of $\Phi_i$ differ by at most $\varepsilon$.
\end{proposition}

\begin{proof}
Given the individual dependency bounds $B_{\Phi_i}$, define
\[
B(\varepsilon) = \max_{1 \le i \le m} B_{\Phi_i}(\varepsilon).
\]
This is a dependency bound for each projection in the family.
\end{proof}

This uniform bound isolates the finite region on which all projections base their approximations.

\section{Prefix Agreement and Tail Freedom}

Let $L = B(\varepsilon)$ for some fixed $\varepsilon > 0$.  
Suppose we require $G^{*}$ to agree with a reference identity $H$ on the first $L$ positions of each layer.  
Then each $\Phi_i(G^{*})$ and $\Phi_i(H)$ differ by less than $\varepsilon$.  
To force a strict difference we will choose $\varepsilon$ to be small and modify the tail structure of $G^{*}$ in a direction that produces a shift in the projection values greater than $2\varepsilon$.

Tail freedom refers to the ability to alter the coordinates $(M,D,K)$ for indices $n > L$ without affecting any projection to within error $\varepsilon$ until the modification crosses the dependency threshold.  
This freedom allows the diagonalizer to evade the collective influence of $\Phi_1, \ldots, \Phi_m$.

\section{Adjustment Zones}

We introduce an auxiliary identity $A = (M_A, D_A, K_A)$ that will serve as the adjustment source.  
The principle is that $A$ differs from $H$ in a direction that changes each projection, and that the difference is concentrated in a tail region.

\begin{definition}[Adjustment Zone]
Given $L \in \mathbb{N}$, an adjustment zone is a finite interval
\[
I = [N_1, N_2]
\quad \text{with } N_1 > L,
\]
on which $G^{*}$ is assigned the corresponding coordinates of $A$.
\end{definition}

Outside the adjustment zone, $G^{*}$ continues to follow $H$ until the next adjustment zone.  
The adjustment process may use multiple zones to force disagreements with all projections simultaneously.

\section{Construction of the Diagonalizer}

Choose a decreasing sequence $(\varepsilon_k)$ converging to zero, for example $\varepsilon_k = 2^{-k}$.  
For each $k$, compute $L_k = B(\varepsilon_k)$ using Proposition~\ref{prop:uniform-bound}.  
These bounds determine a sequence of prefix regions in which agreement with $H$ guarantees that the values of all projections are stable to within $\varepsilon_k$.

Let $(I_k)$ be a sequence of disjoint adjustment zones satisfying
\[
I_k = [N_{k,1}, N_{k,2}] \quad \text{with} \quad N_{k,1} > L_k \quad \text{and} \quad N_{k,1} \to \infty.
\]

Define $G^{*}$ by the following rules.

\begin{enumerate}
    \item For $n \le L_1$, set $G^{*}(n) = H(n)$.
    \item For each $k \ge 1$:
    \begin{enumerate}
        \item On $I_k$, set $G^{*}(n) = A(n)$.
        \item For $L_k < n < N_{k,1}$ and for $n > N_{k,2}$, set $G^{*}(n) = H(n)$ unless this conflicts with previous assignments.
    \end{enumerate}
\end{enumerate}

This defines $G^{*}$ as a pointwise limit of effective partial assignments.  
Since $H$ and $A$ are effective and the adjustment zones are computable, $G^{*}$ is also effective.

\section{Forcing Disagreement}

To show that $\Phi_i(G^{*}) \ne \Phi_i(H)$ for each $i$, consider any projection $\Phi_i$.  
Choose $k$ large enough that the modifications in $I_k$ produce a shift in $\Phi_i$ of at least $3\varepsilon_k$.  
Since $G^{*}$ and $H$ agree on the first $L_k$ positions, the values of $\Phi_i$ differ by at most $\varepsilon_k$ on the shared prefix.  
The forced shift on $I_k$ ensures that the total difference exceeds $2\varepsilon_k$, and therefore $\Phi_i(G^{*}) \ne \Phi_i(H)$.

Since this argument applies to each projection in the family and since the adjustment zones do not overlap, the diagonalizer evades all projections simultaneously.

\section{Summary}

The generative identity $G^{*}$ constructed in this chapter serves as the diagonalizer for the Structural Incompleteness Theorem of Chapter~9.  
It preserves agreement with a reference mechanism on all positions that influence a chosen finite family of projections and diverges outside these regions in a way that forces disagreement.  
This technique generalizes classical diagonalization to the layered structure of the generative space.

The technical lemmas underpinning this construction are collected in Appendix~B.  
Chapter~9 uses the diagonalizer to prove that no finite family of secondary projections can classify the effective core $\mathcal{G}_{\mathrm{eff}}$.
