\chapter{Entropy Balance as a Secondary Invariant}

\section{Introduction}

Extended invariants quantify internal structure that collapse ignores.  
The most fundamental of these invariants is the \emph{entropy balance}
$\eta(G)$, which measures the long-term density with which the selector chooses
the digit layer.  
This invariant distinguishes hybrid, intermediate, and null-density behaviors
within a single collapse fiber and provides the simplest example of a
computable, non-collapsing generative coordinate.

Entropy balance is the archetype of a secondary invariant:  
it is prefix-determined, continuous, computable, and sensitive to the internal
structure of the mixer.  
This chapter formalizes entropy balance within the projection-theoretic
framework of Chapters~6–12 and establishes its role as a canonical generative
coordinate.

\section{Digit Density and Entropy Balance}

The selector $M(n)$ chooses at each position whether the digit layer or the meta
layer contributes the symbol to the canonical output.  
The long-term behavior of this choice determines how frequently the digit layer
is used.

\begin{definition}[Digit Density / Entropy Balance]
For a generative identity $G=(M,D,K)$, the \emph{entropy balance} is
\[
\eta(G)
=
\liminf_{n\to\infty}
\frac{1}{n}\, \bigl|\{\,0\le k<n : M(k)=D\,\}\bigr|.
\]
\end{definition}

The term “entropy balance’’ reflects that a higher digit-selection frequency
injects more base-$b$ entropy into the canonical output, while a lower frequency
shifts structural load toward the meta layer.

\begin{remark}
The use of $\liminf$ ensures that $\eta$ is defined for all selector sequences,
including oscillatory and irregular patterns.
\end{remark}

\section{Continuity and Prefix Dependence}

Entropy balance is determined by the long-run behavior of the selector, but it
still satisfies the continuity and prefix-determination conditions required of a
structural projection.

\begin{proposition}[Prefix-Determined Continuity]
\label{prop:eta-continuity}
The map $G \mapsto \eta(G)$ is continuous on $\mathcal{X}$.
\end{proposition}

\begin{proof}
For any $\varepsilon>0$, choose $N$ such that if two selectors agree on the
first $N$ positions, then their finite-sample digit frequencies in the first
$N$ steps differ by at most $\varepsilon$.  
Since $\eta(G)$ is approximated from below by such finite-sample frequencies,
agreement on the prefix determines the $\liminf$ to within $\varepsilon$.
\end{proof}

Entropy balance is therefore a legitimate generative invariant.

\section{Computability of Entropy Balance}

Entropy balance is computable because:

\begin{enumerate}
    \item the selector $M$ is computable on the effective core,
    \item digit-frequency estimates converge uniformly from below,
    \item the $\liminf$ can be approximated by a computable sequence of rational
    lower bounds.
\end{enumerate}

\begin{proposition}
The entropy balance map
\[
\eta : \mathcal{G}_{\mathrm{eff}} \to [0,1]
\]
is computable in the sense of Type--2 computability.
\end{proposition}

\begin{proof}
The prefix frequency
\[
F_n(G) = \frac{1}{n} \bigl|\{0\le k<n : M(k)=D\}\bigr|
\]
is computable from the prefix $M{\upharpoonright}n$.  
Since $\eta(G)=\liminf_n F_n(G)$, the value of $\eta(G)$ can be approximated to
any precision by computing a sufficiently long prefix.
\end{proof}

Thus $\eta$ is an effective secondary invariant.

\section{Entropy Balance as a Structural Projection}

Entropy balance fits neatly into the projection lattice introduced in
Chapter~6.

\begin{proposition}[Projection-Theoretic Interpretation]
The map $G \mapsto \eta(G)$ is the infimum (in the projection lattice) of the
family of finite-frequency projections
\[
\Phi_n(G) = \frac{1}{n} \bigl|\{0\le k<n : M(k)=D\}\bigr|.
\]
\end{proposition}

\begin{proof}
Each $\Phi_n$ is a structural projection depending only on the first $n$
coordinates.  
Their pointwise $\liminf$ is $\eta(G)$, which is the greatest lower bound of
the family in the refinement order.
\end{proof}

This viewpoint clarifies how entropy balance captures the long-run behavior of
the selector while respecting the finite-observation constraints of structural
projections.

\section{Behavior Inside Collapse Fibers}

Entropy balance varies widely within each collapse fiber.

\begin{proposition}
For every $x \in [0,1]$, the set
\[
\{\eta(G) : G \in \mathcal{F}(x)\}
\]
is the entire interval $[0,1]$.
\end{proposition}

\begin{proof}
For any $\alpha \in [0,1]$, construct a selector that chooses the digit layer
with lower density $\alpha$ and align its digit-selected positions with the
expansion of $x$.  
This yields a generator in $\mathcal{F}(x)$ with entropy balance $\alpha$.
\end{proof}

Thus entropy balance is highly non-collapsing: it completely varies over each
fiber.

In the effective setting:

\begin{proposition}
For any computable real $x$ and any computable $\alpha \in [0,1]$, the effective
fiber $\mathcal{F}_{\mathrm{eff}}(x)$ contains an effective generator $G$ with
$\eta(G)=\alpha$.
\end{proposition}

\begin{proof}
Choose a computable selector whose digit-selection density is $\alpha$ (for
example, periodic or block-structured).  
Assign digits and meta symbols algorithmically as in earlier constructions.
\end{proof}

Thus $\eta$ is a refining invariant in the sense of Chapter~12: it separates
infinitely many effective identities representing the same real number.

\section{Hybrid and Null-Density Structure Revisited}

Entropy balance provides a unified language for the behaviors introduced in
Part~II:

\begin{itemize}
    \item \textbf{Hybrid identities:} $\eta(G)>0$.  
    Digit usage has positive asymptotic density.
    \item \textbf{Null-density identities:} $\eta(G)=0$.  
    Digit usage is sparse and dominated by the meta layer.
\end{itemize}

Both classes appear in every collapse fiber.  
Entropy balance quantifies the spectrum between the two extremes.

\section{Compatibility with the Diagonalizer}

Entropy balance is sensitive to tail modifications but only within its
prefix-determined dependency structure.  
Thus the meta-diagonalizer can produce generators with prescribed entropy
balances that still evade any finite family of other projections.

\begin{proposition}
Entropy balance is compatible with the diagonalizer: for any
computable $\alpha\in[0,1]$, there exists $G^\#\in\mathcal{F}_{\mathrm{eff}}(x)$
with $\eta(G^\#)=\alpha$ such that $G^\#$ diagonalizes against any given finite
family of computable projections.
\end{proposition}

\begin{proof}
Construct a tail identity $A$ with entropy balance $\alpha$ and use it as the
tail source in the diagonalizer construction of Chapter~9.
\end{proof}

Thus entropy balance survives the incompleteness landscape as a robust extended
coordinate.

\section{Outlook}

Entropy balance is the simplest non-collapsing invariant and the foundation for
constructing higher-order generative coordinates.  
The next chapter introduces the \emph{fluctuation index}, which measures
irregularity and long-range variation in selector behavior.  
This tertiary invariant refines entropy balance, providing a richer perspective
on selector complexity within collapse fibers.
