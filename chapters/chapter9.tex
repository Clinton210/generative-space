\chapter{Structural Indistinguishability}
\label{chap:indistinguishability}

\section{Introduction}

Earlier chapters established the finite information principle. 
Every continuous observer on the exposure domain depends on a finite prefix of the generative identity at each precision level. 
Each observer has a dependency bound, so the observer sees only an initial window of the identity.

The Structural Incompleteness Theorem in Chapter~\ref{chap:incompleteness} shows that no computable tower of observers can classify a collapse fiber. 
This chapter develops the explicit mimicry construction that realizes this incompleteness. 
Given any computable reference identity in a collapse fiber, we construct another computable identity in the same fiber that agrees with the reference on every observer-visible prefix while diverging on infinitely many coordinates.

The construction uses the following tools:

\begin{itemize}
    \item prefix synchronization from Chapter~\ref{chap:prefix-stabilization},
    \item alignment and tail replacement from Chapter~\ref{chap:alignment-sewing},
    \item effective closedness of fibers from Appendix~A,
    \item computable prefix extension and sewing machinery developed in Appendices~C and~D.
\end{itemize}

This construction produces a new identity that is indistinguishable from the reference identity to any computable observer or computable invariant derived from observers. 
It is the operational form of the incompleteness phenomenon.

\section{Setup and Effective Enumeration}

Fix a computable real number $x$ and choose a computable identity
\[
H \in \mathcal{F}_{\mathrm{eff}}(x).
\]
The identity $H$ serves as the reference.

Let
\[
\{\Phi_{k}\}_{k \in \mathbb{N}}
\]
be an effective enumeration of all computable structural projections on $\mathcal{G}^{*}$. 
Each $\Phi_{k}$ has a computable dependency bound $B_{k}(\varepsilon)$ by the results of Chapter~\ref{chap:prefix-stabilization}. 
Finite-N invariants and derived invariants from Chapter~\ref{chap:finite-n-invariants} all appear in the enumeration.

Our objective is to construct a computable identity
\[
G^{\ast} \in \mathcal{F}_{\mathrm{eff}}(x)
\qquad\text{with}\qquad 
G^{\ast} \neq H
\]
such that
\[
\Phi_{k}(G^{\ast}) = \Phi_{k}(H)
\quad\text{for all } k.
\]
This shows that $G^{\ast}$ and $H$ are structurally indistinguishable.

\section{Effective Non-Isolation Inside Fibers}

The mimicry construction requires the freedom to replace the tail of a computable identity without changing the collapse value. 
This relies on the compactness and perfectness of collapse fibers as established in Chapter~\ref{chap:fibers}.

\begin{lemma}[Effective Non-Isolation]
\label{lem:non-isolation}
Let $H \in \mathcal{F}_{\mathrm{eff}}(x)$ and let $N \in \mathbb{N}$. 
There exists a computable identity
\[
A \in \mathcal{F}_{\mathrm{eff}}(x)
\]
such that
\[
A \upharpoonright N = H \upharpoonright N
\quad\text{and}\quad
A \neq H.
\]
\end{lemma}

\begin{proof}
Collapse fibers are compact and perfect subsets of the ambient space by Chapter~\ref{chap:fibers}. 
Thus they contain no isolated points. 
Appendix~A shows that fibers are effectively closed sets, and Appendices~C and~D develop the prefix extension and sewing machinery. 
Given $N$, choose any computable identity $A'$ in the fiber different from $H$. 
Use alignment and sewing (Proposition~\ref{prop:tail-sewing}) to splice the prefix $H \upharpoonright N$ with the tail of $A'$ beginning at a matched alignment point. 
The resulting identity $A$ is computable, lies in the fiber, agrees with $H$ on the prefix of length $N$, and differs from $H$ on infinitely many coordinates.
\end{proof}

This gives the ability to enforce divergence at any scale while preserving collapse.

\section{Construction of the Mimicry Sequence}

We build a sequence
\[
G_{0}, G_{1}, G_{2}, \ldots
\]
of computable identities that stabilizes coordinatewise.

\subsection*{Initialization}

Let $G_{0} = H$. 
Define $\varepsilon_{k} = 2^{-(k+1)}$. 
Set $N_{0} = 0$.

\subsection*{Inductive Step}

Suppose $G_{k}$ and $N_{k}$ are known.

\paragraph{Stage 1. Determine the required prefix length.}

To ensure agreement with $\Phi_{k}$ at precision $\varepsilon_{k}$, compute
\[
L_{k} = B_{k}(\varepsilon_{k}).
\]
Define
\[
N_{k+1} = \max(N_{k}, L_{k}) + 1.
\]

\paragraph{Stage 2. Freeze prefix agreement.}

We enforce
\[
G_{k+1} \upharpoonright N_{k+1}
=
H \upharpoonright N_{k+1}.
\]
Any identity with this prefix agrees with $H$ on all observers $\Phi_{0},\ldots,\Phi_{k}$ at the required precision.

\paragraph{Stage 3. Force divergence.}

Apply Lemma~\ref{lem:non-isolation} to obtain a computable
\[
A_{k} \in \mathcal{F}_{\mathrm{eff}}(x)
\]
with
\[
A_{k} \upharpoonright N_{k+1} = H \upharpoonright N_{k+1}
\quad\text{and}\quad 
A_{k} \neq H.
\]
Set $G_{k+1} = A_{k}$.

Each stage preserves prefix agreement with $H$ up to $N_{k+1}$ while ensuring divergence beyond that prefix.

\subsection*{Coordinatewise Stabilization}

Since
\[
G_{k+1} \upharpoonright N_{k}
=
G_{k} \upharpoonright N_{k},
\]
the sequence $(G_{k})$ is coordinatewise nondecreasing in domain of definition. 
Thus $(G_{k})$ converges to a limit identity $G^{\ast}$ in the product topology.

Because each $G_{k}$ lies in the fiber and the fiber is closed, the limit $G^{\ast}$ remains in the fiber.

\section{Verification of Indistinguishability}

\begin{theorem}[Structural Indistinguishability]
\label{thm:indistinguishability-final}
Let $x$ be a computable real and let $H \in \mathcal{F}_{\mathrm{eff}}(x)$ be computable. 
Then there exists a computable identity
\[
G^{\ast} \in \mathcal{F}_{\mathrm{eff}}(x),
\qquad 
G^{\ast} \neq H,
\]
such that for every computable structural projection $\Phi$,
\[
\Phi(G^{\ast}) = \Phi(H).
\]
\end{theorem}

\begin{proof}
Fix any computable structural projection $\Phi_{m}$. 
For all $k \ge m$, we have forced
\[
G_{k} \upharpoonright N_{k}
=
H \upharpoonright N_{k},
\qquad
N_{k} \ge B_{m}(\varepsilon_{k}).
\]
By prefix stabilization from Chapter~\ref{chap:prefix-stabilization},
\[
|\Phi_{m}(G_{k}) - \Phi_{m}(H)| < \varepsilon_{k}.
\]
Since $\varepsilon_{k} \to 0$, continuity gives
\[
\Phi_{m}(G^{\ast}) = \Phi_{m}(H).
\]
Distinctness holds because $G^{\ast}$ differs from $H$ on infinitely many coordinates by the construction in Step 3. 
Since $m$ was arbitrary, equality holds for every computable structural projection.
\end{proof}

\section{Interpretation}

This result formalizes the observational limitations of any finite-information system. 
Observers cannot see beyond their dependency bounds. 
For each observer in the enumeration, the mimicry construction synchronizes all visible prefixes while the tail remains a free parameter. 
Because collapse fibers are compact and perfect, tail freedom persists at every scale.

The conclusion is that representation systems that rely on finite prefix visibility cannot classify the infinite-dimensional structure of generative identities. 
Even when all continuous computable observers are consulted at once, there always exist pairs of distinct identities that appear identical to all of them.

The next chapter develops asymptotic invariants and shows that these invariants, as limits of finite-N observers, inherit the same prefix-dependence and therefore cannot recover generative structure.
