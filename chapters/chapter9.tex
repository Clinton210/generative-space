\chapter{The Meta-Diagonalizer}

\section{Introduction}

The collapse fiber $\mathcal{F}(x)$ contains many generative identities that
produce the same classical real number.  
In earlier chapters we saw that continuous projections cannot observe the full
internal structure of a generative identity and depend only on finite prefixes
at any fixed precision.  
This chapter constructs a new identity inside the effective fiber that agrees
with a given reference identity on every observed prefix, yet diverges from it
in structural properties that no fixed finite collection of projections can
detect.

The construction is a form of diagonalization.  
Given a countable sequence of projections, we build an identity that avoids
agreement with a prescribed family of reference structures by introducing
divergence at stages beyond their dependency bounds.  
The internal freedom available inside collapse fibers ensures that these
divergent tails do not change the classical value of the identity.

The diagonalizer demonstrates the fundamental incompleteness of finite
observers and prepares the ground for the Incompleteness Theorem of the next
chapter.

\section{Setup and Notation}

Fix a computable real number $x$ and let
\[
(x_j)_{j \ge 0}
\]
denote its canonical base $b$ expansion.  
Let $H$ be a computable identity in $\mathcal{F}_{\mathrm{eff}}(x)$ that we
will use as a reference.  
Let
\[
\Phi_0, \Phi_1, \Phi_2, \ldots
\]
be a computable enumeration of all computable projections on
$\mathcal{G}_{\mathrm{eff}}$.  
For each $k$, let $B_k$ denote a computable dependency bound for $\Phi_k$.

Our goal is to construct an identity
\[
G^{\sharp} \in \mathcal{F}_{\mathrm{eff}}(x)
\]
that diverges from $H$ on every projection $\Phi_k$ by more than a prescribed
amount at stage $k$, despite agreeing with $H$ on prefixes long enough to
satisfy all earlier projections.

\section{Divergent Identities Inside the Fiber}

At each stage $k$ we will require an identity $A_k$ inside the effective fiber
$\mathcal{F}_{\mathrm{eff}}(x)$ that differs from $H$ at scale $\varepsilon_k$
with respect to $\Phi_k$, where
\[
\varepsilon_0 > \varepsilon_1 > \varepsilon_2 > \cdots \to 0
\]
is a fixed computable sequence of positive tolerances.

The following lemma ensures that such identities always exist.

\begin{lemma}[Divergence Inside the Effective Fiber]
\label{lem:divergence-fiber}
For any computable projection $\Phi$ and any computable $\varepsilon > 0$,
there exists a computable identity $A \in \mathcal{F}_{\mathrm{eff}}(x)$ such
that
\[
|\Phi(A) - \Phi(H)| > 3\varepsilon.
\]
\end{lemma}

\begin{proof}
The effective fiber $\mathcal{F}_{\mathrm{eff}}(x)$ is a nonempty
$\Pi^0_1$ class.  
If $\Phi$ were constant on this class, then $\Phi$ would depend only on the
canonical output and would assign the same value to all identities in the
fiber.  
However, by the definition of a structural projection, $\Phi$ may depend on
all components of the generative identity, including the selector and
meta-information streams.

Because the fiber allows arbitrary choices on unselected digits and on the
meta stream, we may modify these components computably without altering the
selected digits.  
By continuity of $\Phi$, local changes to the selector or meta stream beyond
any prefix can move the value of $\Phi$ through an interval.  
Thus the image of $\mathcal{F}_{\mathrm{eff}}(x)$ under $\Phi$ is an interval,
and the value $\Phi(H)$ cannot lie at the boundary of this interval for all
such modifications.

We may therefore choose a computable identity $A$ in the fiber and a
computable prefix for which the subsequent tail guarantees a divergence of at
least $3\varepsilon$.  
This identity satisfies the desired inequality.
\end{proof}

This lemma encapsulates the essential freedom inside the effective fiber: one
may force meaningful structural differences without changing the collapsed
value.

\section{The Diagonal Construction}

We now build the diagonalizer
\[
G^{\sharp} = \lim_{k \to \infty} G_k
\]
as a limit of identities that stabilize on longer and longer prefixes while
introducing controlled divergence at each stage.

Let $G_0 = H$.  
Assume inductively that $G_k$ has been defined and that $G_k$ agrees with $H$
on its first $N_k$ symbols for some computable $N_k$.

\subsection*{Stage $k$: divergence}

Choose
\[
\varepsilon_k = 2^{-(k+2)}.
\]
By Lemma \ref{lem:divergence-fiber}, choose
\[
A_k \in \mathcal{F}_{\mathrm{eff}}(x)
\]
such that
\[
|\Phi_k(A_k) - \Phi_k(H)| > 3\varepsilon_k.
\]

This identity will serve as the source of structural divergence at stage $k$.

\subsection*{Stage $k$: prefix agreement}

To preserve earlier projective agreements, we require that the first $N_k$
symbols of $G_{k+1}$ agree with $G_k$ and with $H$.  
Let
\[
N_{k+1}
  = \max\bigl( N_k,\ B_k(\varepsilon_k) \bigr).
\]
This ensures that agreement on the first $N_{k+1}$ symbols forces agreement of
$\Phi_k$ up to error $\varepsilon_k$.

\subsection*{Stage $k$: alignment}

Let $h_{k}$ be the position in $H$ corresponding to the $k$th selected digit.
Let $a_k$ be the corresponding position in $A_k$.  
Choose $j$ sufficiently large that $h_j \ge N_{k+1}$.  
The alignment lemma guarantees that the indices $h_j$ and $a_j$ correspond to
the same selected digit.

\subsection*{Stage $k$: tail sewing}

Define $G_{k+1}$ by sewing:
\[
G_{k+1}(n)
  =
\begin{cases}
G_k(n) & n \le h_j,\\
A_k(n - h_j + a_j) & n > h_j.
\end{cases}
\]

By the tail sewing proposition from the previous chapter,
\[
G_{k+1} \in \mathcal{F}_{\mathrm{eff}}(x).
\]

Moreover, $G_{k+1}$ agrees with $G_k$ on their first $N_{k+1}$ symbols, so all
projections $\Phi_0,\ldots,\Phi_k$ agree with $H$ to within $\varepsilon_k$ on
$G_{k+1}$.

\section{Existence of the Limit Identity}

The sequence $(G_k)$ stabilizes on longer and longer prefixes.  
For each index $n$, there exists a stage $k$ such that $n \le N_k$, and from
that point onward, the $n$th symbol of $G_m$ remains constant for all
$m \ge k$.

Define $G^{\sharp}$ to be the identity whose $n$th symbol is this eventual
value.  
Then the limit exists and is computable.

\begin{proposition}
The identity $G^{\sharp}$ lies in the effective fiber
$\mathcal{F}_{\mathrm{eff}}(x)$.
\end{proposition}

\begin{proof}
Each $G_k$ lies in the fiber, and the canonical output is preserved at every
stage by the alignment and sewing procedure.  
Thus $G^{\sharp}$ also collapses to $x$.  
Computability follows because each coordinate stabilizes at a computable
stage.
\end{proof}

\section{Diagonalization}

Finally, we verify that $G^{\sharp}$ diverges from $H$ along every projection
in the sequence.

\begin{proposition}
For each $k$,  
\[
|\Phi_k(G^{\sharp}) - \Phi_k(H)| \ge \varepsilon_k.
\]
\end{proposition}

\begin{proof}
At stage $k$ we chose $A_k$ so that
\[
|\Phi_k(A_k) - \Phi_k(H)| > 3\varepsilon_k.
\]
The sewing step ensures that the tail of $G_{k+1}$ beyond index $h_j$ agrees
with the tail of $A_k$ from index $a_j$ onward.  
Since the tail lies entirely beyond $B_k(\varepsilon_k)$, any effect on
$\Phi_k$ caused by the tail persists in $G_{k+1}$ and therefore in
$G^{\sharp}$.

Agreement of prefixes up to $N_{k+1}$ introduces at most $\varepsilon_k$ of
error.  
Because the divergence was chosen to exceed $3\varepsilon_k$, the final
difference remains at least $\varepsilon_k$.
\end{proof}

Thus $G^{\sharp}$ avoids finite approximation by the projections
$\Phi_k$ in the same way that classical diagonalization avoids uniform
compression of information.  
This completes the construction.

\section{Summary}

This chapter constructed a computable identity in the collapse fiber of $x$
that diverges from a reference identity on every computable projection while
agreeing with the reference on arbitrarily long prefixes.  
The construction relies on alignment and sewing tools, the existence of
divergent identities inside the fiber, and finite dependency bounds for
projections.  
In the next chapter we apply the diagonalizer to prove the Structural
Incompleteness Theorem, which states that no finite family of continuous
observers can capture the generative structure of a real number.
