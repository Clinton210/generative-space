\chapter{The Continuum as a Collapse Quotient}
\label{chap:quotient}

\section{Introduction}

The collapse representation introduced in Chapter~\ref{chap:collapse-map} assigns a classical real number to each generative identity by reading the exposed values from a designated observed-value coordinate. 
Under this representation, the real line appears as a quotient of the ambient generative space, where two identities are equivalent if they produce the same collapse value. 
This reflects the general perspective of represented space theory, in which real numbers are defined by equivalence classes of names \cite{WeihrauchComputableAnalysis}.

The aim of this chapter is to describe the quotient structure induced by collapse, relate it to the standard representation of real numbers, and clarify how the classical continuum arises as a shadow of the richer symbolic space. 
As emphasized throughout this monograph, collapse is a chosen representation rather than an intrinsic feature of the generative identity. 
This distinction is central to understanding the role of observers and invariants developed in earlier chapters.

\section{Collapse Equivalence Classes}

Fix the collapse representation from Chapter~\ref{chap:collapse-map}. 
Let
\[
\pi : \mathcal{G}^{*} \to [0,1]
\]
denote the collapse map, where $\mathcal{G}^{*}$ is the exposure domain. 
Define an equivalence relation on $\mathcal{G}^{*}$ by
\[
G \sim H
\quad\Longleftrightarrow\quad
\pi(G) = \pi(H).
\]

\begin{definition}[Collapse Fiber]
The equivalence class of $G$ under $\sim$ is
\[
[\![G]\!]
=
\pi^{-1}\bigl(\{\pi(G)\}\bigr),
\]
called the collapse fiber of $\pi(G)$.
\end{definition}

Two identities lie in the same fiber exactly when their exposed-value coordinates determine identical observed digit sequences. 
Although these sequences represent the same real number, the identities may differ arbitrarily in their latent coordinates, as described in Chapter~\ref{chap:fibers}.

\section{Quotient Topology}

Equip $\mathcal{G}$ with the product topology and $[0,1]$ with the Euclidean topology. 
By continuity of $\pi$ (Proposition~\ref{prop:collapse-continuous}), the equivalence relation $\sim$ is closed in $\mathcal{G}\times\mathcal{G}$, and therefore induces a well-behaved quotient topology.

\begin{definition}[Collapse Quotient]
The collapse quotient is the space $\mathcal{G}/\!\sim$ equipped with the quotient topology induced by the map
\[
q : \mathcal{G} \longrightarrow \mathcal{G}/\!\sim.
\]
\end{definition}

The ambient generative space $\mathcal{G}$ is compact, perfect, and totally disconnected, while the exposure domain $\mathcal{G}^{*}$ is a dense $G_{\delta}$ subset but not compact. 
The quotient is defined using all of $\mathcal{G}$, which simplifies the topological structure.

\begin{lemma}
The equivalence relation $\sim$ is closed in $\mathcal{G} \times \mathcal{G}$.
\end{lemma}

\begin{proof}
If $(G,H) \notin \sim$, then $\pi(G) \neq \pi(H)$. 
Since $[0,1]$ is Hausdorff and $\pi$ is continuous, there exist disjoint open sets separating $\pi(G)$ and $\pi(H)$, whose preimages under $\pi$ separate $G$ and $H$. 
Thus the complement of $\sim$ is open, so $\sim$ is closed.
\end{proof}

Closedness of $\sim$ ensures good behavior of the quotient. 
In particular, it guarantees that the quotient inherits compactness from $\mathcal{G}$.

\section{Identification with the Real Interval}

We now show that the quotient is homeomorphic to the classical unit interval. 
This parallels results in represented space theory, where the real line is obtained as a quotient of Baire space by an appropriate naming relation \cite{PaulyRepresentedSpaces}.

\begin{proposition}
\label{prop:quotient-homeomorphism-rewritten}
The collapse quotient $\mathcal{G}/\!\sim$ is homeomorphic to $[0,1]$.
\end{proposition}

\begin{proof}
Define a map
\[
\tilde{\pi} : \mathcal{G}/\!\sim \to [0,1]
\]
by $\tilde{\pi}\bigl(q(G)\bigr) = \pi(G)$. 
This is well defined because $G \sim H$ implies $\pi(G) = \pi(H)$.

The map $\tilde{\pi}$ is continuous since $\pi = \tilde{\pi} \circ q$ and $\pi$ is continuous. 
It is bijective because each real number has at least one collapse representative and equivalent representatives are identified in the quotient.

The domain $\mathcal{G}/\!\sim$ is compact because $\mathcal{G}$ is compact and $\sim$ is closed. 
The codomain $[0,1]$ is Hausdorff. 
Thus a continuous bijection from a compact space to a Hausdorff space is automatically a homeomorphism.
\end{proof}

This identifies the classical continuum as the space of equivalence classes created by the collapse representation. 
The interval is therefore a shadow of the richer generative space.

\section{Structure of Fibers Inside the Quotient}

Although the quotient collapses the generative space onto the simple interval $[0,1]$, the equivalence classes themselves contain highly nontrivial symbolic structure. 
From Chapter~\ref{chap:fibers}, each fiber $\mathcal{F}(x)$ is:

\begin{itemize}
    \item compact (Corollary~\ref{cor:fiber-compact}),
    \item perfect and totally disconnected (Proposition~\ref{prop:fiber-perfect}),
    \item rich in tail-coded degrees of freedom (Proposition~\ref{prop:tail-freedom}).
\end{itemize}

The first few exposed entries of the generative identity determine $x$, but infinitely many symbolic coordinates remain unconstrained. 
This infinite-dimensional tail is the source of generative freedom. 
As shown in Chapter~\ref{chap:incompleteness} and Chapter~\ref{chap:indistinguishability}, observers operating under finite prefix dependence cannot resolve these latent structures.

Thus the quotient map
\[
q : \mathcal{G} \to \mathcal{G}/\!\sim
\]
is extremely lossy. 
Nearly all generative structure is collapsed to a single real value.

\section{Computability Considerations}

The quotient interpretation has a direct analog in Type-2 computability.

If $x$ is a computable real number, then its effective fiber
\[
\mathcal{F}_{\mathrm{eff}}(x)
=
\mathcal{F}(x) \cap \mathcal{G}_{\mathrm{eff}}
\]
contains a computable identity. 
Such an identity is a computable name for $x$ in the sense of represented spaces.

\begin{proposition}
If $x$ is computable, then $\mathcal{F}_{\mathrm{eff}}(x)$ is nonempty.
If $x$ is noncomputable, then $\mathcal{F}_{\mathrm{eff}}(x)$ contains no computable identity.
\end{proposition}

\begin{proof}
If $x$ is computable, the construction in Chapter~\ref{chap:collapse-map} builds a computable identity whose collapse is $x$. 
If some computable identity collapsed to a noncomputable real, the collapse map would compute that real from a computable input, contradicting noncomputability of $x$.
\end{proof}

Thus computable points in the continuum correspond to computable generative identities, but the structure of each fiber is far richer than the subset of computable representatives. 
Appendix~A contains the effective closedness results that underpin these statements.

\section{Interpretation in the Three-Tier Hierarchy}

The quotient perspective fits naturally into the ontological hierarchy that guides the monograph:

\begin{center}
generative identity 
\quad$\longrightarrow$\quad
collapse value 
\quad$\longrightarrow$\quad
observers and invariants
\end{center}

Each level loses information relative to the one above:

\begin{itemize}
    \item Collapse discards the entire tail-coded structure of the identity.
    \item Observers, depending on finite prefixes, cannot recover what collapse has lost.
    \item Asymptotic invariants, being limits of observer values, are even more coarse.
\end{itemize}

The quotient $\mathcal{G}/\!\sim$ formalizes the collapse step of this chain. 
It explains why the continuum cannot encode generative structure and why finite observers cannot reconstruct the identity from its real value.

\section{Summary}

The real interval $[0,1]$ arises as the collapse quotient of the ambient generative space. 
Collapse identifies all identities with the same observed-value sequence, producing a compact quotient homeomorphic to the unit interval. 
Each fiber is a compact, perfect, totally disconnected space containing uncountably many latent configurations. 
This structure explains both the expressive richness of generative identities and the observational limitations formalized by incompleteness and indistinguishability.

The next chapter examines asymptotic invariants as limits of finite-N observers and shows how these coarse quantities fit into the quotient perspective.
