\chapter{Selector Patterns and Density Regimes}

\section{Introduction}

Generative identities differ not only in the symbols they carry but also in
the \emph{rate} at which their selectors expose digits from the underlying
digit stream.  
This rate---the asymptotic density of positions where $M(n) = D$---governs both
the structure of the canonical output and the degree of freedom present inside
the collapse fiber.

This chapter analyzes two fundamental regimes of selector behavior:
\begin{itemize}
    \item \emph{Hybrid selectors}, which expose digits with positive
          asymptotic density, and
    \item \emph{Null-density selectors}, which expose digits at vanishing
          density.
\end{itemize}

Although these two extremes lie on opposite ends of a broad spectrum, both
occur densely in the generative space.  
Understanding these regimes clarifies how generative identities with sharply
different internal behaviors can collapse to the same real number.

\section{Selector Density}

For a selector stream $M \in \{D,K\}^{\mathbb{N}}$, define the indicator
function
\[
\chi_M(n) =
\begin{cases}
1 & M(n) = D,\\[4pt]
0 & M(n) = K.
\end{cases}
\]

The \emph{selector density} of $M$ is the lower asymptotic density
\[
\eta(M)
  = \liminf_{N \to \infty}
      \frac{1}{N}
      \sum_{n=0}^{N-1} \chi_M(n).
\]

If $\eta(M) > 0$, the selector exposes digits at a positive rate; if
$\eta(M)=0$, exposure becomes increasingly sparse.

This density measures only the frequency of digit selections, not their
spacing; a selector may have dense clusters of selections followed by long
voids while still having positive or zero density.

\section{Hybrid Selectors}

\subsection{Definition}

A generative identity $G = (M,D,K)$ is \emph{hybrid} if $\eta(M) > 0$.
Equivalently, the indices $n$ with $M(n)=D$ have positive asymptotic density.

Hybrid identities expose digits regularly enough that, in the long run, a
non-negligible portion of the total stream contributes to the classical
output.

\subsection{Topological density}

Hybrid selectors occur densely in the generative space.

\begin{proposition}
For every nonempty basic open set in $\mathcal{X}$, there exists a hybrid
identity contained in it.
\end{proposition}

\begin{proof}
Let the open set be determined by finite prefixes of $(M, D, K)$.  
Extend these prefixes by placing $M(n) = D$ for all $n$ beyond the given
prefix.  
Then the extended identity is hybrid and remains inside the open set.  
Thus hybrid selectors form a dense subset of $\mathcal{X}$.
\end{proof}

\subsection{Interpretation}

Hybrid identities distribute their observed digits steadily throughout the
total stream.  
They represent the ``typical'' behavior of selectors when little is known
about their structure.

\section{Null-Density Selectors}

\subsection{Definition}

A generative identity is \emph{null-density} if its selector satisfies
$\eta(M) = 0$.

These selectors still expose infinitely many digits (since $G \in
\mathcal{X}^*$), but they do so with asymptotically negligible frequency.

\subsection{Examples}

A standard example uses the perfect squares:
\[
M(n) =
\begin{cases}
D & \text{if } n = k^2 \text{ for some } k,\\
K & \text{otherwise}.
\end{cases}
\]
Since the number of squares below $N$ is $\lfloor \sqrt{N} \rfloor$, the
density of $D$-positions is $N^{-1/2} \to 0$.

More intricate examples use rapidly growing computable sequences such as
$n_k = k!$, $n_k = 2^{2^k}$, or sparse polynomial-time patterns.

\subsection{Existence in every fiber}

Null-density selectors appear in every collapse fiber.

\begin{proposition}
For every $x \in [0,1]$, there exists a null-density generative identity
$G \in \mathcal{F}_{\mathrm{eff}}(x)$.
\end{proposition}

\begin{proof}
Fix the canonical expansion $(x_j)$ of $x$ and define a selector that exposes
digits only at perfect-square positions.  
At each such position $n_j$, set $D(n_j) = x_j$; elsewhere set $D$ arbitrarily.
Let $K$ be any computable meta-stream.  
This identity lies in $\mathcal{F}_{\mathrm{eff}}(x)$ and has density zero by
construction.
\end{proof}

\subsection{Interpretation}

Null-density selectors exhibit extreme sparsity.  
They expose infinitely many digits but at a rate too small to influence the
asymptotic distribution of symbols in the overall generative space.  
Such identities show that collapse fibers contain elements of dramatically
different structural complexity.

\section{Selector Diversity Inside a Fiber}

Hybrid and null-density identities coexist inside the same collapse fiber,
demonstrating that the classical output $x$ places almost no restrictions on
the internal rate of digit revelation.

Given any $x$, the effective fiber $\mathcal{F}_{\mathrm{eff}}(x)$ contains:
\begin{itemize}
    \item identities selecting digits frequently,
    \item identities selecting digits sparsely,
    \item identities with periodic or chaotic selection patterns,
    \item identities with arbitrary meta-information streams.
\end{itemize}

This freedom underscores the essential distinction between internal
generative structure and classical magnitude.

\section{Summary}

Selector density provides the first structural coordinate for generative
identities.  
Hybrid selectors expose digits with positive asymptotic density, whereas
null-density selectors do so sparsely.  
Both behaviors occur densely in the generative space and both appear in every
effective collapse fiber.  
The coexistence of such radically different regimes within a single fiber
illustrates the vast internal variability hidden beneath the collapse.

The next chapter introduces structural projections, continuous observers that
measure generative properties without disrupting the underlying identity.
