\chapter{Selector Patterns and Density Regimes}
\label{chap:density}

\section{Introduction}

The selector stream governs how a generative identity exposes digits to the
collapse coordinate.  
Although collapse depends only on the ordered sequence of selected digits, the
long term behavior of the selector layer shapes the internal structure of
collapse fibers and influences what continuous observers can detect.

This chapter introduces two coarse exposure regimes for selector streams:
positive density and null density.  
They represent opposite ends of the symbolic spectrum, yet both appear densely
in the ambient generative space and inside every collapse fiber.  
Selector density provides the first large scale invariant used to understand
selector geometry before finer structural projections are introduced in the
next chapter.

\section{Selector Density}

For a selector stream $M \in \{D,K\}^{\mathbb{N}}$, define
\[
\chi_M(n)
=
\begin{cases}
1 & \text{if } M(n)=D,\\
0 & \text{if } M(n)=K.
\end{cases}
\]

\begin{definition}[Selector Density]
The lower asymptotic density of digit exposures is
\[
\eta(M)
=
\liminf_{N\to\infty}
\frac{1}{N}\sum_{n<N}\chi_M(n).
\]
\end{definition}

The value $\eta(M)$ measures how frequently digits are exposed relative to the
full stream.  
It does not constrain local regularity or gap growth.  
Selectors with identical density may differ dramatically in the placement of
exposures.

\section{Hybrid Selectors}

\subsection{Definition}

A generative identity $G=(M,D,K)$ has a \emph{hybrid selector} if
\[
\eta(M) > 0.
\]
Hybrid selectors expose digits at a sustained rate.  
A positive fraction of positions contribute to the collapse coordinate.

\subsection{Topological Abundance}

Hybrid selectors are dense in the ambient generative space
\[
\mathcal{G}
  = \{D,K\}^{\mathbb{N}}
    \times \{0,\ldots,b-1\}^{\mathbb{N}}
    \times \Sigma^{\mathbb{N}}.
\]

\begin{proposition}
Every nonempty basic open set in $\mathcal{G}$ contains a hybrid selector.
\end{proposition}

\begin{proof}
A basic open set specifies only finitely many coordinates of $M$, $D$, and
$K$.  
Extend the specified prefix by setting $M(n)=D$ for all larger $n$.  
The resulting selector has density $1$ and the identity remains inside the
open set.
\end{proof}

This parallels standard arguments in symbolic dynamics where dense families of
sequences are constructed by extending finite prefixes in shift spaces
\cite{LindMarcus, Kechris}.

\subsection{Interpretation}

Hybrid selectors capture the regime of sustained, regular exposure.  
They represent identities in which the collapse coordinate is drawn from a
digit stream that remains frequently accessible.

\section{Null Density Selectors}

\subsection{Definition}

A generative identity has a \emph{null density} selector when
\[
\eta(M) = 0.
\]
Such identities still lie in the digit-producing subspace $\mathcal{G}^*$,
because they expose infinitely many digits, but the relative frequency of
exposures tends to zero.

\subsection{Examples}

A classical example is exposure at the perfect squares:
\[
M(n)
=
\begin{cases}
D & \text{if } n=k^2 \text{ for some } k,\\
K & \text{otherwise}.
\end{cases}
\]
Since the number of squares less than $N$ grows like $N^{1/2}$, the density of
exposures is $N^{-1/2}$, which tends to zero.

More extreme sparse selectors arise from sequences such as $n_j=j!$ or
$n_j=2^{2^j}$, which create gap growth far beyond any polynomial rate.  
Such constructions are common in the study of sparse symbolic sequences and
recurrence behavior \cite{LindMarcus}.

\subsection{Existence in Every Collapse Fiber}

\begin{proposition}
For each $x \in [0,1]$, the effective fiber
$\mathcal{F}_{\mathrm{eff}}(x)$ contains identities with null density
selectors.
\end{proposition}

\begin{proof}
Let $(x_j)$ be the collapse coordinate of $x$.  
Expose $x_j$ at the $j$th square $n_j=j^2$.  
Fill unselected digit positions arbitrarily and choose any computable
meta-information stream.  
The resulting identity collapses to $x$ and has density zero.
\end{proof}

\subsection{Interpretation}

Null density selectors emphasize that collapse imposes no restraint on the
rate of exposure.  
A collapse fiber can contain identities whose internal timing of exposures is
extremely sparse or irregular without affecting the classical value.

\section{Selector Diversity Inside a Collapse Fiber}

Collapse fibers are closed under arbitrary changes to unobserved structure.
As a consequence, selector streams within a single fiber may display an
extremely wide range of long term behaviors.  
For fixed $x \in [0,1]$, the effective fiber $\mathcal{F}_{\mathrm{eff}}(x)$
contains selectors that are:

\begin{itemize}
    \item hybrid,
    \item null density,
    \item periodic or quasiperiodic,
    \item irregular with highly variable gap growth,
    \item influenced by arbitrary choices in the meta-information layer.
\end{itemize}

This diversity reflects the fact that collapse depends only on the sequence of
exposed digits, not on the rate or structure by which exposure occurs.

\section{Summary}

Selector density provides a fundamental coarse descriptor of selector
behavior.  
Hybrid selectors expose digits at a positive rate, while null density
selectors expose digits sparsely.  
Both appear densely in the ambient generative space and inside every collapse
fiber.  
Their coexistence illustrates that classical magnitude places almost no
constraint on large scale selector behavior.

The next chapter develops structural projections, which describe how
continuous observers extract finite information from generative identities.
