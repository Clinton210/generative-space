\chapter{The Fluctuation Index as a Tertiary Invariant}

\section{Introduction}

Entropy balance $\eta(G)$ measures \emph{how often} a selector chooses the digit
layer.  
But many selectors with the same digit density behave very differently:
some distribute digit selections uniformly, while others place them in bursts
separated by long gaps.

To capture this higher-order structure, we introduce the \emph{fluctuation
index}, a tertiary invariant that measures the irregularity or dispersion of
digit selections.  
This invariant refines entropy balance in the same way that variance refines
mean: it distinguishes selectors with identical limiting densities but
different internal patterns.

The fluctuation index satisfies all criteria for extended invariants introduced
in Chapter~12.  
It is continuous, prefix-determined, computable on the effective core, and
sensitive to structure invisible to entropy balance and collapse.  
It will serve as one of the key coordinates in the extended generative space
developed in Chapters 15 and 16.

\section{Gap Sequences and Dispersion}

Let $G = (M,D,K)$ be a generative identity.  
Write
\[
S_M = \{ n_0 < n_1 < n_2 < \cdots \}
\]
for the positions where $M$ selects the digit layer.  
Define the \emph{gap sequence}
\[
g_j = n_{j+1} - n_j.
\]

Digit density depends only on the asymptotic cardinality of $S_M$; the gap
sequence captures its \emph{shape}.  
Small gaps correspond to uniform usage; large gaps indicate bursts of meta-layer
dominance.

\section{Finite-Prefix Fluctuation}

We begin with a finite version that is well-defined on prefixes.

For any $n$ such that $S_M\cap[0,n)$ contains at least two digit selections,
define the partial gap sequence
\[
g_j^{(n)} = n_{j+1} - n_j
\quad\text{for } n_{j+1} < n.
\]

Let
\[
\Phi_n(G)
=
\begin{cases}
\max_j g_j^{(n)}, & \text{if at least two gaps appear in }[0,n),\\
n, & \text{otherwise}.
\end{cases}
\]

The value $\Phi_n(G)$ measures the largest digit-free region within the first
$n$ positions.  
Taking $\Phi_n(G)=n$ in the degenerate case ensures monotonicity and prefix
dependence.

\section{Definition of the Fluctuation Index}

\begin{definition}[Fluctuation Index]
The \emph{fluctuation index} of $G$ is
\[
\phi(G)
=
\limsup_{n\to\infty}
\frac{\Phi_n(G)}{n}.
\]
\end{definition}

Thus $\phi(G)$ measures the normalized size of the largest digit-free portion of
the initial segment.  
Values near zero indicate uniformity; values near one indicate extreme
irregularity or sparsity.

\section{Continuity and Prefix Dependence}

\begin{proposition}
The fluctuation index $\phi(G)$ is a prefix-determined, continuous structural
projection.
\end{proposition}

\begin{proof}
Fix $\varepsilon>0$.  
To determine whether $\phi(G)$ exceeds a threshold $\alpha$, it suffices to
inspect all gap lengths in the prefix of length $N = \lceil 1/\varepsilon\rceil
$.  
Agreement on this prefix ensures that the normalized $\Phi_n(G)/n$ values differ
by at most $\varepsilon$ for all $n \ge N$, proving continuity and prefix
determination.
\end{proof}

\section{Computability}

\begin{proposition}
The fluctuation index $\phi$ is computable on $\mathcal{G}_{\mathrm{eff}}$.
\end{proposition}

\begin{proof}
Given a computable selector $M$, we can compute all gap lengths $g_j^{(n)}$
within the prefix $M{\upharpoonright}n$.  
Thus $\Phi_n(G)$ is computable.  
Since $\phi(G)$ is obtained as the $\limsup$ of computable rational numbers
$\Phi_n(G)/n$, it is Type--2 computable.
\end{proof}

\section{Relationship to Entropy Balance}

Entropy balance and fluctuation index measure complementary aspects of the
selector.

\begin{itemize}
    \item $\eta(G)$ captures the \emph{overall frequency} of digit usage.
    \item $\phi(G)$ captures the \emph{distributional irregularity} of digit usage.
\end{itemize}

\begin{proposition}
$\eta(G)$ does not determine $\phi(G)$, and $\phi(G)$ does not determine
$\eta(G)$.
\end{proposition}

\begin{proof}
Selectors with identical densities may differ arbitrarily in the size of gaps;
similarly, sequences with identical gap structure may differ in density by
increasing or decreasing digit selections uniformly.
\end{proof}

Thus $\eta$ and $\phi$ are independent coordinates in the extended space.

\section{Behavior Inside Collapse Fibers}

As with entropy balance, fluctuation index varies fully within each collapse
fiber.

\begin{proposition}
For every $x\in[0,1]$ and $\alpha\in[0,1]$, there exists
$G\in\mathcal{F}(x)$ such that $\phi(G)=\alpha$.
\end{proposition}

\begin{proof}
Construct selectors with gap sequences that achieve maximal gap proportions
corresponding to $\alpha$, and align digit selections to the expansion of $x$.
\end{proof}

In the effective setting:

\begin{proposition}
If $x\in\mathbb{R}_c$ and $\alpha\in\mathbb{Q}\cap[0,1]$, then
$\mathcal{F}_{\mathrm{eff}}(x)$ contains an effective generator $G$ with
$\phi(G)=\alpha$.
\end{proposition}

\begin{proof}
Use periodic or computably sparse selectors whose gap structure realizes
$\alpha$, then assign digit and meta coordinates computably as in earlier
constructions.
\end{proof}

Thus $\phi$ is a refining invariant and a genuine tertiary coordinate.

\section{Projection-Lattice Structure}

\begin{proposition}
The fluctuation index is the supremum (in the refinement order) of the family
of projections
\[
G \longmapsto \frac{\Phi_n(G)}{n}.
\]
\end{proposition}

\begin{proof}
The $\limsup$ operation yields the least upper bound in the refinement order:
any projection that dominates each $\Phi_n(G)/n$ must dominate their
$\limsup$ as well.
\end{proof}

This positions $\phi$ naturally within the projection lattice:  
entropy balance is an infimum of finite-frequency projections, while fluctuation
index is a supremum of gap-size projections.

\section{Compatibility with Diagonalization}

Fluctuation index is sensitive to highly local changes in gap structure but is
still prefix-determined.  
Thus the diagonalizer of Chapter~9 can be adapted to preserve $\phi(G)$ while
evading any finite family of other projections.

\begin{proposition}
For any computable $\alpha\in[0,1]$ and any computable real $x$, there exists a
diagonalizing mechanism $G^\#\in\mathcal{F}_{\mathrm{eff}}(x)$ with
$\phi(G^\#)=\alpha$.
\end{proposition}

\begin{proof}
Choose a tail identity $A$ with $\phi(A)=\alpha$, and sew it into the
diagonalizer construction.  
Digit-index alignment preserves collapse, and prefix stability preserves gap
structure at all prescribed scales.
\end{proof}

\section{Outlook}

The fluctuation index enriches the generative coordinate system beyond entropy
balance.  
Selectors with identical frequency patterns can have vastly different
irregularity profiles, and $\phi(G)$ captures this tertiary layer of structure.

Chapter~15 combines $\eta$ and $\phi$ into a two-dimensional extended coordinate
system, drawing an analogy with the classical complex plane:  
$\pi(G)$ corresponds to the “real axis,” while $\eta(G)$ or $\phi(G)$ act as
imaginary directions that restore structure lost under collapse.
