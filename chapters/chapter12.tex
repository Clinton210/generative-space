\chapter{Slice Geometry of Asymptotic Invariants}
\label{chap:slice-geometry}

\section{Introduction}

Extended invariants describe large scale features of the selector stream.  
Chapter~\ref{chap:invariants-eta-phi} introduced two tail-dependent invariants:
the asymptotic density $\eta(G)$ and the fluctuation index $\phi(G)$.  
Both reflect selector behavior far beyond the reach of finite-prefix
observation, and both are discontinuous everywhere in the product topology of
the generative space.

This chapter places these invariants in a geometric setting by examining three
natural slice families through the generative space:
\begin{itemize}
    \item vertical slices, which fix finite prefixes;
    \item horizontal slices, which fix invariant values;
    \item fiber slices, which fix collapsed magnitude.
\end{itemize}
Together these slices reveal how finite symbolic structure, asymptotic selector
behavior, and classical value interact.  
The geometry emphasizes the independence of finite-prefix information from
asymptotic invariants and the independence of collapse magnitude from selector
structure.

\section{Vertical Slices: Finite Prefix Constraints}

For a finite word $u$ of length $N$ in the generative alphabet, define the
vertical slice
\[
\mathcal{C}(u)
=
\{\, G \in \mathcal{G} : G[0..N-1] = u \,\}.
\]

Vertical slices are clopen cylinder sets in the product topology.  
Every structural projection with dependency bound at precision $\varepsilon$
examines exactly one such slice of depth $B_{\Phi}(\varepsilon)$.  
Thus vertical slices encode the finite-prefix geometry that constrains
continuous observation, dependency bounds, and the indistinguishability
results of Part~\ref{part:incompleteness}.

Vertical slices impose no restrictions on $\eta$ or $\phi$:  
by the nowhere continuity results of
Chapter~\ref{chap:invariants-eta-phi}, every value of $\eta$ in $[0,1]$ and
every value of $\phi$ in $[0,\infty]$ is realized densely in each
$\mathcal{C}(u)$.

\section{Horizontal Slices: Level Sets of Extended Invariants}

Fix $\alpha\in[0,1]$ and $\beta\in[0,\infty]$.  
Define the horizontal slices
\[
\mathcal{H}_{\alpha}
=
\{\, G : \eta(G)=\alpha \,\},
\qquad
\mathcal{H}^{\beta}
=
\{\, G : \phi(G)=\beta \,\}.
\]

These sets collect identities with identical asymptotic selector behavior,
regardless of their finite prefixes.  
Because $\eta$ and $\phi$ are tail-dependent, horizontal slices intersect every
vertical slice.  
Horizontal slices cut across finite-prefix geometry and collapse fibers alike.

It is helpful to visualize the pair of invariants through the (discontinuous)
mapping
\[
G \mapsto (\eta(G),\phi(G)).
\]
Vertical slices correspond to entire regions of the invariant plane.  
Horizontal slices correspond to lines of constant invariant value.  
Their dense intersection reflects the asymptotic sensitivity of the invariants
to tail behavior.

\section{Fiber Slices: Fixing Collapsed Magnitude}

Fix a real number $x\in[0,1]$.  
The fiber slice
\[
\mathcal{F}(x)
=
\{\, G : \pi(G)=x \,\}
\]
collects identities with the same collapse coordinate.  
Since $\eta$ and $\phi$ depend only on the selector stream and not on the
collapse coordinate, the image of $\mathcal{F}(x)$ under the invariant map is
typically the entire admissible region of the invariant plane.

Using tail freedom inside fibers
(Chapter~\ref{chap:alignment-sewing}) together with the constructions of
Theorems~\ref{thm:eta-nowhere} and~\ref{thm:phi-nowhere}, one obtains the
following fact.

\begin{proposition}
\label{prop:fiber-full-invariants}
For any real $x$ and any pair $(\alpha,\beta)$ with
$\alpha\in[0,1]$ and $\beta\in[0,\infty]$,  
there exists an identity $G \in \mathcal{F}(x)$ with
\[
\eta(G)=\alpha,
\qquad
\phi(G)=\beta.
\]
\end{proposition}

\begin{proof}
Choose a selector stream $M$ with the prescribed invariant values
$(\alpha,\beta)$ using the constructions from
Chapter~\ref{chap:invariants-eta-phi}.  
Combine the selector tail with the canonical digits of $x$ using
alignment and tail sewing
(Chapter~\ref{chap:alignment-sewing}) to obtain a generative identity
in $\mathcal{F}(x)$ with the desired properties.
\end{proof}

Thus a fixed collapsed magnitude imposes no restriction on the extended
invariants.

\section{Geometric Interpretation}

The three slice families demonstrate the independence of finite-prefix
structure, collapse magnitude, and asymptotic selector behavior.

\begin{itemize}
    \item \emph{Vertical slices} constrain finite-prefix information but do not
    restrict asymptotic invariants.  
    Every invariant value occurs densely inside every vertical slice.

    \item \emph{Horizontal slices} constrain asymptotic selector behavior but
    intersect every finite-prefix class and every collapse fiber.  
    They express global tail geometry that is invisible to finite observers.

    \item \emph{Fiber slices} constrain classical magnitude but allow all
    possible invariant values by tail freedom.  
    Collapse does not control selector asymptotics.
\end{itemize}

Together these slices reveal a layered geometry in the generative framework.
Finite-prefix structure governs observation, collapse governs classical value,
and asymptotic behavior describes global selector geometry.  
Extended invariants lie entirely outside the reach of finite observation and
are unaffected by collapse.

\section{Summary}

Vertical slices fix finite prefixes.  
Horizontal slices fix asymptotic invariant values.  
Fiber slices fix classical magnitude.  
The invariants $\eta$ and $\phi$ vary freely across all of these slice
families, illustrating the independence of asymptotic selector behavior from
both finite-prefix information and collapse.

This slice geometry provides the conceptual foundation for analyzing joint
invariant behavior in the next chapter.  
Appendix~E supplies explicit examples illustrating the full spectrum of
selector behaviors and invariant combinations.
