\chapter{Slice Geometry of Asymptotic Invariants}
\label{chap:slice-geometry}

\section{Introduction}

Extended invariants provide numerical summaries of the long term behavior of
the selector stream. Chapter \texttt{\ref{chap:invariants-eta-phi}} introduced
the asymptotic density $\eta$ and the fluctuation index $\phi$, both of which
depend only on the tail of the selector and therefore measure structural
information invisible to the collapse map. These invariants are asymptotically
sensitive and discontinuous at every point of the generative space.

In this chapter we place the invariants in a geometric context by examining
three natural families of slices through the generative space: vertical
slices that fix finite prefixes, horizontal slices that fix invariant values,
and fiber slices that fix collapsed magnitude. Together these slices describe
how local symbolic structure, global asymptotic behavior, and classical value
interact.

\section{Vertical Slices: Finite Prefix Constraints}

A vertical slice fixes a finite prefix:
\[
\mathcal{C}(u)
=
\{
G \in \mathcal{X} : G[0..N-1]=u
\}.
\]

These sets represent the regions accessible to structural projections, since
dependency bounds constrain each observer to examine only one vertical slice
at a time. Vertical slices are clopen in the product topology and encode the
finite dimensional geometry that governs observational limits and
indistinguishability.

Vertical slices place no constraints on the invariants $\eta$ or $\phi$. By
the nowhere continuity results of Chapter \texttt{\ref{chap:invariants-eta-phi}},
every value of $\eta$ and every value of $\phi$ occurs densely in each vertical
slice.

\section{Horizontal Slices: Invariant Level Sets}

Fix $\alpha$ in $[0,1]$ or $\beta$ in $[0,\infty]$. Define the horizontal
slices
\[
\mathcal{H}_{\alpha}
=
\{ G : \eta(G)=\alpha \},
\qquad
\mathcal{H}^{\beta}
=
\{ G : \phi(G)=\beta \}.
\]

These sets identify identities with identical long term selector behavior even
when their finite prefixes differ. Because $\eta$ and $\phi$ are tail
dependent, horizontal slices cross all vertical slices densely. They cut across
finite prefix classes and collapse fibers alike.

Geometrically, the map
\[
G \mapsto (\eta(G),\phi(G))
\]
projects the generative space into the invariant plane. Horizontal slices
correspond to lines in this plane, and their dense intersection with each
vertical slice reflects the asymptotic sensitivity of the invariants.

\section{Fiber Slices: Fixing Collapsed Value}

Fix a real number $x$. The fiber slice
\[
\mathcal{F}(x)
=
\{ G : \pi(G)=x \}
\]
collects all identities that collapse to $x$. Since $\eta$ and $\phi$ depend
only on the selector and not on the canonical output, the image of
$\mathcal{F}(x)$ under the invariant map is typically a large region of the
invariant plane.

By combining tail freedom inside collapse fibers with the constructions of
Chapter \texttt{\ref{chap:invariants-eta-phi}}, one finds that for any pair
$(\alpha,\beta)$ with $\alpha$ in $[0,1]$ and $\beta$ in $[0,\infty]$, there
exists an identity in $\mathcal{F}(x)$ with $\eta=\alpha$ and $\phi=\beta$.
Thus fiber slices contain identities exhibiting the full range of invariant
values.

\section{Geometric Interpretation}

The three families of slices illustrate the independence of finite prefix
information, asymptotic selector behavior, and collapsed magnitude.

\begin{itemize}
    \item Vertical slices constrain finite symbolic structure but do not
    restrict $\eta$ or $\phi$.
    \item Horizontal slices constrain asymptotic behavior but intersect every
    finite prefix class.
    \item Fiber slices constrain collapsed magnitude but allow all admissible
    invariant values.
\end{itemize}

Together these slices show that the invariants $\eta$ and $\phi$ capture
features orthogonal to finite observation and independent of collapse. They
provide a geometric language for understanding which aspects of selector
structure persist under prefix agreement and which aspects remain completely
unobservable to continuous projections.

\section{Summary}

The slice geometry of vertical, horizontal, and fiber slices provides a
geometric interpretation of the asymptotic invariants introduced in Chapter
\texttt{\ref{chap:invariants-eta-phi}}. Vertical slices represent finite
prefix constraints. Horizontal slices represent invariant constraints. Fiber
slices represent collapsed value constraints. The invariants vary freely in
all of these slices, reflecting their asymptotic nature and their insensitivity
to prefix information.

These geometric insights prepare the way for the study of joint invariant
behavior in the next chapter. Appendix~E contains explicit examples that
illustrate the full range of selector behaviors and invariant combinations.
