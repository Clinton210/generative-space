\chapter{Slice Geometry of Asymptotic Invariants}
\label{chap:slice-geometry}

\section{Introduction}

Extended invariants arise as limits of observer towers in the chosen collapse representation. 
They describe coarse symbolic behavior of the exposure mechanism and depend on tail geometry rather than finite prefixes. 
Chapter~\ref{chap:invariants} introduced two such quantities: the asymptotic exposure density and the fluctuation index. 
Both are Baire class 1 maps that can jump, oscillate, or take arbitrary values inside any nonempty open cylinder. 
Their instability follows from the finite information principle underlying all observers.

This chapter organizes these invariants using three coarse slicing operations on the ambient generative space. 
The slices are not geometric in a classical metric sense. 
Instead, they are symbolic partitions that illustrate how finite-prefix structure, asymptotic behavior, and collapse values interact under the fixed representation. 
The goal is visualization rather than classification. 
Everything here depends on the chosen representation and should be interpreted as a derived perspective rather than as intrinsic geometry of generative identities.

\section{Vertical Slices: Finite Prefix Constraints}

Fix a finite word $u$ of length $N$ over the ambient symbolic alphabet. 
Following standard practice in represented spaces and symbolic dynamics \cite{LindMarcus, PaulyRepresentedSpaces}, define the vertical slice
\[
\mathcal{C}(u)
=
\{\, G \in \mathcal{G} : G[0..N-1] = u \,\}.
\]

Vertical slices are clopen cylinder sets in the product topology. 
By the dependency bound of any continuous observer (Appendix~B), each projection at precision $\varepsilon$ depends only on a slice of depth $B_{\Phi}(\varepsilon)$. 
Vertical slices therefore represent the entire visible region of an identity for any finite observer. 
All prefix-based arguments in Part~II and Part~III, including the indistinguishability construction, take place inside such slices.

Because the topology controls only finite prefixes, vertical slices impose no restrictions on asymptotic invariants. 
By the nowhere continuity results in Chapter~\ref{chap:invariants}, every value of the density invariant and every value of the fluctuation index is realized densely inside each $\mathcal{C}(u)$. 
This illustrates the fundamental mismatch between finite-prefix geometry and asymptotic behavior.

\section{Horizontal Slices: Level Sets of Asymptotic Invariants}

Fix $\alpha \in [0,1]$ and $\beta \in [0,\infty]$. 
Define the horizontal slices
\[
\mathcal{H}_{\alpha}
=
\{\, G : \eta(G) = \alpha \,\},
\qquad
\mathcal{H}^{\beta}
=
\{\, G : \phi(G) = \beta \,\}.
\]

Horizontal slices group identities by their long term exposure behavior. 
They intersect every vertical slice because the invariants are tail dependent, and tail freedom is unconstrained in the product topology. 
Horizontal slices therefore cut across the entire prefix structure and depend on symbolic behavior invisible to observers with finite prefix dependence.

It is often convenient to view the mapping
\[
G \longmapsto \bigl(\eta(G), \phi(G)\bigr)
\]
as a coarse projection into the invariant plane. 
Vertical slices appear as large regions in this plane, while horizontal slices correspond to level sets. 
Because invariants are Baire class 1 maps with everywhere-discontinuous behavior, these slices have no regularity properties beyond their symbolic definition. 
They should be viewed as coarse partitions tied to the chosen representation.

\section{Fiber Slices: Fixing Collapse Values}

Fix a real number $x \in [0,1]$. 
The fiber slice
\[
\mathcal{F}(x)
=
\{\, G \in \mathcal{G}^{*} : \pi(G) = x \,\}
\]
collects identities that collapse to the same classical value under the representation. 
As established in Chapters~\ref{chap:collapse-map} and~\ref{chap:fibers}, collapse fibers are compact, perfect, and totally disconnected. 
They admit full tail freedom by alignment and sewing (Appendices~C and~D), while the exposure domain itself is not compact.

Because asymptotic invariants depend on tails rather than collapse values, their behavior in any fiber reflects the same freedom present in the ambient space.

\begin{proposition}
\label{prop:fiber-invariant-full}
For any real number $x$, any $\alpha \in [0,1]$, and any $\beta \in [0,\infty]$, there exists $G \in \mathcal{F}(x)$ with
\[
\eta(G)=\alpha,
\qquad
\phi(G)=\beta.
\]
\end{proposition}

\begin{proof}
Start with any $H \in \mathcal{F}(x)$. 
By the construction of Chapter~\ref{chap:invariants}, choose a symbolic tail realizing $(\alpha,\beta)$. 
Using alignment from Appendix~C, identify a position in $H$ beyond which tail replacement preserves the exposed-value sequence. 
Sew the constructed tail to the aligned prefix using the method of Appendix~D. 
The resulting identity lies in $\mathcal{F}(x)$ and has the required invariant values.
\end{proof}

This demonstrates that collapse values constrain only the observed-value coordinate and do not restrict exposure asymptotics.

\section{Interpretation of the Three Slice Families}

The slice families illustrate the independence of the three layers in the framework’s hierarchy.

\begin{itemize}
    \item \textbf{Vertical slices} capture finite-prefix information, the only region visible to finite observers. 
    They cannot constrain asymptotic invariants because observers have finite prefix dependence.

    \item \textbf{Horizontal slices} capture long term symbolic behavior derived from observer towers. 
    They intersect every finite-prefix class and every fiber, showing that asymptotic invariants are independent of collapse and finite prefixes.

    \item \textbf{Fiber slices} fix the collapse value but allow full tail variation. 
    They illustrate that collapse discards nearly all symbolic structure and leaves asymptotic behavior unconstrained.
\end{itemize}

Together these slices give a coarse visualization of how the chosen representation separates finite-prefix structure, collapse values, and asymptotic behavior. 
They do not define an intrinsic geometry. 
Instead, they highlight the observer limitations established in Part~III and the representation dependence discussed in Part~I.

\section{Summary}

Vertical slices fix finite prefixes. 
Horizontal slices fix asymptotic invariant values. 
Fiber slices fix collapse values. 
All asymptotic invariant values appear densely inside all three slice families because these invariants depend on tail behavior that cannot be constrained by finite prefixes or by collapse.

These slices provide a visual framework for understanding derived invariants and prepare the ground for the joint invariant analysis developed in the next chapter. 
Explicit examples appear in Appendix~G.
