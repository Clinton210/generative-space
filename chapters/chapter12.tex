\chapter{Extended Invariants: Entropy Balance and Fluctuation}

\section{Introduction}

Extended invariants measure large scale features of the selector stream that
survive tail modification and reveal structure invisible to the collapse map.
Two such invariants are the entropy balance $\eta$, which measures the lower
asymptotic density of digit exposures, and the fluctuation index $\phi$, which
measures the relative growth of gaps between selected positions.  

In this chapter we introduce these invariants, establish basic properties,
prove semicontinuity, and place them in the slice geometry of the generative
space.  
Selector behavior may be analyzed through vertical slices (fixed prefixes),
horizontal slices (fixed invariant values), and fiber slices (fixed collapsed
value).  
Appendix~E contains many worked examples illustrating these slices and the
full range of possible behaviors.

\section{Entropy Balance}

Let $G = (M,D,K)$ be a generative identity and write
\[
\chi_M(n) =
\begin{cases}
1 & \text{if } M(n) = D, \\
0 & \text{otherwise}.
\end{cases}
\]

The entropy balance is the lower asymptotic density of digit exposures:
\[
\eta(G)
  = \liminf_{N\to\infty}
      \frac{1}{N} \sum_{n < N} \chi_M(n).
\]

A selector with $\eta(G) > 0$ exposes digits frequently, while $\eta(G) = 0$
indicates sparse exposure.  
Balance is invariant under tail modification beyond a finite prefix and
depends only on the selector.

\subsection{Lower semicontinuity}

The balance is not continuous in the product topology, but it satisfies a one
sided bound.

\begin{proposition}[Lower Semicontinuity]
If $G_k \to G$ in the product topology, then
\[
\eta(G) \le \liminf_{k\to\infty} \eta(G_k).
\]
\end{proposition}

\begin{proof}
Fix $\varepsilon > 0$ and choose $N$ such that
\[
\frac{1}{N}\sum_{n<N} \chi_M(n)
    < \eta(G) + \varepsilon.
\]
For $k$ large enough, $G_k$ agrees with $G$ on the first $N$ coordinates, so
\[
\eta(G_k)
  \ge \frac{1}{N}\sum_{n<N} \chi_M(n)
  > \eta(G) - \varepsilon.
\]
Taking the limit inferior yields the claim.
\end{proof}

\section{Fluctuation Index}

Let the selection indices be
\[
n_0 < n_1 < n_2 < \cdots,
\qquad
g_j = n_{j+1} - n_j.
\]

The fluctuation index measures the growth of relative gaps:
\[
\phi(G)
  = \limsup_{j\to\infty} \frac{g_j}{n_j}.
\]

Large $\phi(G)$ indicates the presence of long, infrequent bursts of digit
exposure relative to scale.

\subsection{Upper semicontinuity}

\begin{proposition}[Upper Semicontinuity]
If $G_k \to G$, then
\[
\phi(G) \ge \limsup_{k\to\infty} \phi(G_k).
\]
\end{proposition}

\begin{proof}
Suppose $\phi(G) < c$.  
Then for sufficiently large $j$,
\[
g_j < c n_j.
\]
Agreement of $G_k$ with $G$ on a large enough prefix ensures that the first
$J$ selected digits occur at the same positions.  
Thus for all sufficiently large $k$,
\[
\phi(G_k) < c.
\]
Taking the limit superior proves the claim.
\end{proof}

\section{Slice Geometry of Selector Behavior}

Extended invariants permit a geometric interpretation of the generative space
through slices that constrain different aspects of selector behavior.  
These slices provide intuition for how invariants, collapse fibers, and finite
prefixes interact.

\subsection{Vertical slices: cylinder sets}

A vertical slice fixes a finite prefix:
\[
\mathcal{C}(u)
  = \{ G \in \mathcal{X}^{*} : G[0..N-1] = u \}.
\]

These sets represent the regions observable by structural projections.  
Dependency bounds show that each observer samples only one vertical slice at a
time, so vertical slices encode the finite informational geometry underlying
incompleteness.

Vertical slices impose no restriction on $\eta$ or $\phi$, and their images
under the map $G \mapsto (\eta(G),\phi(G))$ can cover large regions of the
invariant plane.

\subsection{Horizontal slices: invariant level sets}

Fix $\alpha$ or $\beta$.  
Define
\[
\mathcal{H}_{\alpha}
  = \{ G : \eta(G) = \alpha \},
\qquad
\mathcal{H}^{\beta}
  = \{ G : \phi(G) = \beta \}.
\]

These sets identify identities with the same long term selector behavior even
if their finite prefixes differ.  
Horizontal slices cut across collapse fibers and vertical slices, illustrating
the independence of asymptotic structure from local structure.

In the invariant plane, these slices appear as vertical or horizontal lines.

\subsection{Fiber slices: fixing collapsed value}

Fix a real number $x$.  
The fiber slice is
\[
\mathcal{F}(x)
  = \{ G : \pi(G) = x \}.
\]

Since $\eta$ and $\phi$ depend only on the selector, not on collapse, the
image of $\mathcal{F}(x)$ in the invariant plane typically occupies a broad
region.  
This illustrates how collapse conceals most of the selector structure.

Appendix~E contains examples showing that for any pair $(\alpha,\beta)$, there
exists an identity in $\mathcal{F}(x)$ with $\eta = \alpha$ and $\phi =
\beta$.

\section{Selector Behavior Through Examples}

Appendix~E provides detailed examples demonstrating selectors with:

\begin{itemize}
    \item positive balance and low fluctuation,
    \item zero balance and bounded fluctuation,
    \item zero balance and unbounded fluctuation,
    \item oscillating densities,
    \item constructed pairs $(\eta,\phi)$ with prescribed values.
\end{itemize}

These examples show that extended invariants are flexible tools for describing
large scale selector structure.  
They also demonstrate that collapse fibers contain identities with all
admissible invariant values.

\section{Summary}

The entropy balance and fluctuation index provide numerical lenses through
which to view long term selector behavior.  
Their semicontinuity properties match their intuitive roles: balance is hard
to increase by small perturbations, while fluctuation is hard to decrease.  
Slice geometry offers a conceptual framework for understanding how vertical
prefix constraints, horizontal invariant constraints, and fiber constraints
interact.

Together with the examples of Appendix~E, these tools give a geometric
understanding of selector behavior that complements the structural and
computational perspectives developed in earlier parts of the monograph.
