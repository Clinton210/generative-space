\chapter{Extended Invariants and the Expansion of Generative Coordinates}

\section{Introduction}

Parts~I–V developed the generative representation of classical real numbers and
established the Structural Incompleteness Theorem: no finite family of
computable structural projections can classify an effective collapse fiber.
Classical magnitude $\pi(G)$ is therefore only one coordinate of a much richer
object; almost all internal structure of a generative identity disappears under
collapse.

This motivates a natural question:

\begin{quote}
\emph{
If collapse discards nearly all generative structure, can we introduce
additional invariants that capture meaningful aspects of the internal
mechanism?
}
\end{quote}

Part~VI addresses this question by developing \emph{extended generative
coordinates}—quantities such as entropy balance, fluctuation indices, and
meta-pattern invariants that enrich the generative description of a real
number.

This chapter lays the theoretical foundation for these extended invariants.  
We introduce the axioms, regularity conditions, and structural constraints that
any generative coordinate must satisfy.  
These conditions ensure that new invariants are compatible with the topology,
computability, and tail-modification principles established earlier in the
monograph.

\section{What is a Generative Invariant?}

Classical magnitude is derived from the digit subsequence selected by $M$.  
Extended invariants generalize this idea by allowing observers to extract
quantities from \emph{any} layer of the mechanism.

\begin{definition}[Generative Invariant]
A \emph{generative invariant} is a map
\[
I : \mathcal{X} \to \mathbb{R}
\]
satisfying:
\begin{enumerate}
    \item \textbf{Continuity:} $I$ is continuous in the product topology.
    \item \textbf{Prefix dependence:} $I$ is prefix-determined:  
    for any $\varepsilon>0$ there exists $n$ such that agreement on the first
    $n$ coordinates ensures $|I(G)-I(H)|<\varepsilon$.
    \item \textbf{Computable dependency (optional):}  
    If $I$ is effective, then the prefix bound $n$ is computable as a function
    of $\varepsilon$.
\end{enumerate}
\end{definition}

Thus a generative invariant is precisely a structural projection of the form
introduced in Chapter~6.  
Extended invariants are computable structural projections specifically designed
to quantify internal mechanisms.

\section{Collapse as the Primary Invariant}

Magnitude $\pi$ itself is a generative invariant and occupies a special position
in the projection lattice:

\begin{itemize}
    \item $\pi$ is the coarsest invariant that preserves classical magnitude.
    \item Any invariant depending only on magnitude factors through $\pi$.
    \item $\pi$ forgets nearly all internal structure of $(M,D,K)$.
\end{itemize}

Extended invariants arise from asking: \emph{What additional coordinates can be
defined that do not factor through collapse?}

\section{Basic Requirements for Extended Invariants}

To be meaningful in the generative framework, an extended invariant must satisfy
four structural criteria.

\subsection*{1. Stability under prefix extension}

Since observers are limited to finite lookahead, an invariant must stabilize
once sufficiently many coordinates of the mechanism have been observed.

\begin{definition}[Stability]
A generative invariant $I$ is \emph{stable} if for every $\varepsilon>0$ there
exists $N$ such that for all $n \ge N$, any two identities agreeing on the
first $n$ coordinates produce invariant values within $\varepsilon$.
\end{definition}

Stability guarantees compatibility with tail sewing and the diagonalizer
construction.

\subsection*{2. Layer sensitivity}

An invariant must detect some structural aspect that collapse ignores:
digit usage patterns, meta frequencies, selector complexity, or statistical
regularity.

\begin{definition}[Non-collapse]
An invariant $I$ is \emph{non-collapsing} if it does not factor through the
collapse map $\pi$.
\end{definition}

\subsection*{3. Fiber refinement}

A meaningful invariant must distinguish at least some elements of each effective
fiber.

\begin{definition}[Refining Invariant]
$I$ is a \emph{refining invariant} if for every computable real $x$,
the effective fiber $\mathcal{F}_{\mathrm{eff}}(x)$ contains two identities
$G,H$ with $I(G)\neq I(H)$.
\end{definition}

\subsection*{4. Consistency with computable structure}

Extended invariants must be realizable by algorithmic observers.

\begin{definition}[Computable Extended Invariant]
An extended invariant $I$ is \emph{computable} if it is a computable structural
projection with a computable dependency bound.
\end{definition}

\section{Examples and Non-Examples}

\subsection*{The Constant Invariant}

$I(G)=0$ is stable, continuous, and computable, but it is collapsing and
gets ignored by the projection lattice.  
Thus it is not an extended invariant.

\subsection*{Collapse Magnitude}

$\pi(G)$ is a stable invariant but does not refine collapse fibers.  
Therefore it is the baseline invariant but not an extended one.

\subsection*{Digit-Frequency Limits}

Let
\[
I(G) = \liminf_{n\to\infty} 
\frac{1}{n} |\{0\le k<n : M(k)=D\}|.
\]
This invariant is computable, stable, and distinguishes between hybrid and
null-density generators within the same collapse fiber.  
It is a valid extended invariant.

\subsection*{Meta-Pattern Frequencies}

If $w$ is a finite meta-block, the invariant
\[
I_w(G) = \liminf_{n\to\infty} \frac{\#\text{occurrences of $w$ in }
K{\upharpoonright}n}{n}
\]
is also a legitimate extended invariant.

\section{Extended Coordinates as Higher-Dimensional Embeddings}

Classical real numbers form a one-dimensional continuum.  
Extended invariants allow us to embed generative identities into higher-dimensional
spaces, yielding richer coordinate systems.

\begin{definition}[Extended Coordinate Map]
Given invariants $I_1,\ldots,I_r$, the extended coordinate map is
\[
\mathbf{I}(G) := \bigl(\pi(G),\, I_1(G),\ldots,I_r(G)\bigr).
\]
\end{definition}

Such embeddings enlarge the representation space:
\[
\mathcal{X} \xrightarrow{\mathbf{I}} \mathbb{R}^{1+r}.
\]

A key example appears in the next chapter, where the entropy balance
$\eta(G)$ is introduced as a structured secondary invariant.

\section{Limits Imposed by Incompleteness}

Even with extended invariants, the Structural Incompleteness Theorem applies.

\begin{proposition}
Let $I_1,\ldots,I_m$ be finitely many computable extended invariants.
Then there exist distinct $G,H \in \mathcal{F}_{\mathrm{eff}}(x)$ with
\[
I_j(G) = I_j(H) \quad\text{for all } j.
\]
\end{proposition}

\begin{proof}
Each $I_j$ is a computable structural projection.  
The claim follows from the Structural Incompleteness Theorem applied to the
family $\{\pi, I_1,\ldots,I_m\}$.
\end{proof}

Thus extended invariants enrich the generative coordinate system but cannot
solve the classification problem for fibers.

\section{Outlook}

This chapter provided the formal apparatus for adding new structural invariants
to the generative representation.  
The next chapters introduce specific, mathematically significant invariants:

\begin{itemize}
    \item \textbf{Chapter~13: Entropy Balance ($\eta$)} — the basic secondary
    invariant measuring selector usage density.
    \item \textbf{Chapter~14: Fluctuation Index ($\phi$)} — a tertiary invariant
    measuring irregularity and long-range variation.
    \item \textbf{Chapter~15: Orthogonal Extensions and the Complex Analogy} —
    a conceptual embedding of the generative representation into a plane of
    invariants, analogous to the complexification of $\mathbb{R}$.
    \item \textbf{Chapter~16: Diminishing Returns and Final Outlook} — the
    limiting geometry of extended coordinates.
\end{itemize}

Extended coordinates reveal that the generative representation is not merely an
encoding of real numbers, but a multifaceted symbolic geometry whose structure
extends far beyond collapse.
