\chapter{Secondary Projections and Dependency Bounds}

\section{Introduction}

The collapse map $\pi$ from Chapter~2 is the only invariant designated as primary in the generative framework.  
Every other numerical quantity derived from a generative identity is understood as a secondary coordinate or projection.  
Secondary projections summarize aspects of the internal structure of $G = (M, D, K)$ but do not recover the classical magnitude and do not uniquely determine the generative identity inside a fiber.  

This chapter provides a general definition of secondary projections, establishes their computability constraints, and proves the finite-prefix dependency bounds that are essential to the diagonalizer construction of Chapter~8.  
These constraints are a direct consequence of the topology on the generative space and the theory of computable functionals on sequence spaces, reviewed in Appendix~A.

\section{Secondary Projections}

Secondary projections are functions that extract numerical summaries from generative identities.  
They may depend on any aspect of $M$, $D$, or $K$, and may take scalar or vector values.

\begin{definition}[Secondary Projection]
A \emph{secondary projection} is a function
\[
\Phi : \mathcal{X} \to \mathbb{R}^k
\]
that is continuous with respect to the product topology on $\mathcal{X}$.  
When restricted to the effective core, a secondary projection is required to be computable:
\[
\Phi : \mathcal{G}_{\mathrm{eff}} \to \mathbb{R}^k.
\]
\end{definition}

Continuity ensures that $\Phi(G)$ depends only on finitely many initial coordinates of $(M,D,K)$ up to an arbitrarily small precision.  
Computability ensures that the dependence on these coordinates is effective when $G$ is effective.

\begin{remark}
Examples of secondary projections appear in Chapter~7, including digit frequencies, meta frequencies, entropy-type quantities, mixer complexity, and various local variation statistics.
\end{remark}

\section{Secondary Projections as Coordinates on Fibers}

Secondary projections provide coordinate systems on collapse fibers.  
Given $x \in [0,1]$, the fiber $\mathcal{F}(x)$ contains many generative identities that share the same classical magnitude but differ in their secondary coordinates.

\begin{proposition}
Secondary projections are constant on fibers only when they depend solely on collapse information.  
In general, distinct elements of a fiber produce distinct values of $\Phi$.
\end{proposition}

\begin{proof}
If $\Phi$ depends only on the digit subsequence selected by the mixer, then $\Phi$ factors through $\pi$ and is constant on fibers.  
Otherwise, there exist two identities in $\mathcal{F}(x)$ that differ on meta or unselected digit positions, and continuity of $\Phi$ implies that they produce different values.
\end{proof}

Secondary projections therefore describe aspects of internal structure that the collapse does not capture.  
This is the foundation for the Rashomon effect developed in Chapter~7.

\section{Computability and Finite-Prefix Dependence}

When restricted to $\mathcal{G}_{\mathrm{eff}}$, secondary projections exhibit a key structural constraint: they depend on only a finite prefix of the input when computing their output to any fixed precision.  
This is a classical property of computable functionals on Baire space, reviewed in Appendix~A.

\begin{definition}[Prefix Dependency]
Let $\Phi : \mathcal{G}_{\mathrm{eff}} \to \mathbb{R}^k$ be a secondary projection.  
A function $B_\Phi : \mathbb{Q}^+ \to \mathbb{N}$ is a \emph{dependency bound} for $\Phi$ if for every $\varepsilon > 0$, whenever two effective generative identities $G$ and $H$ agree on the first $B_\Phi(\varepsilon)$ coordinates of $(M,D,K)$, then
\[
\|\Phi(G) - \Phi(H)\| < \varepsilon.
\]
\end{definition}

This definition captures the fact that any computable functional can only inspect a finite amount of input to produce a rational approximation of its value.

\begin{proposition}[Existence of Dependency Bounds]
Every computable secondary projection on $\mathcal{G}_{\mathrm{eff}}$ has a dependency bound.
\end{proposition}

\begin{proof}
Computable functionals on Baire space compute their outputs via algorithms that read only finitely many input values before producing an approximation within any prescribed accuracy.  
Since each $G \in \mathcal{G}_{\mathrm{eff}}$ is given by computable sequences for $M$, $D$, and $K$, the standard arguments in computable analysis imply the existence of a finite stage at which the algorithm halts with an $\varepsilon$ approximation.  
The bound $B_\Phi(\varepsilon)$ is obtained by taking the maximum input position inspected by the algorithm on any effective input.  
Background is reviewed in Appendix~A.
\end{proof}

Dependency bounds play a central role in Part~IV.  
They allow the diagonalizer of Chapter~8 to modify a generative identity outside the region inspected by a finite family of projections and thereby evade classification.

\section{Uniformity and Effective Bounds}

For some families of projections it is useful to work with uniform bounds across all components.

\begin{definition}[Uniform Dependency Bound]
Let $\{ \Phi_1, \ldots, \Phi_m \}$ be a finite family of computable secondary projections.  
A function $B : \mathbb{Q}^+ \to \mathbb{N}$ is a \emph{uniform dependency bound} if it is a dependency bound for each $\Phi_i$.
\end{definition}

Uniform bounds exist simply because the family is finite.

\begin{proposition}
Every finite family of computable secondary projections admits a uniform dependency bound.
\end{proposition}

\begin{proof}
Let $B_{\Phi_i}$ be a dependency bound for $\Phi_i$.  
Define $B(\varepsilon) = \max_i B_{\Phi_i}(\varepsilon)$.  
Then $B$ is a dependency bound for each projection in the family.
\end{proof}

Uniform bounds form the core technical ingredient in the meta-diagonalizer construction that appears in Chapter~8.  
They allow multiple projections to be simultaneously neutralized by modifying a generative identity outside their common region of inspection.

\section{Secondary Projections as Arbitrary Coordinates}

Secondary projections provide a wide range of possible coordinate systems on $\mathcal{G}_{\mathrm{eff}}$ and $\mathcal{X}$.  
They reflect structural aspects of generative identities but are not intrinsic to the definition of the space.  
This leads to the following conceptual statement.

\begin{proposition}
Beyond collapse, every coordinate system on the generative space is arbitrary.  
No finite family of secondary projections captures the full internal structure of a generative identity.
\end{proposition}

\begin{proof}
The proof appears in Part~IV.  
A summary is as follows.  
Every computable secondary projection has a finite dependency bound.  
The meta-diagonalizer of Chapter~8 constructs an identity that agrees with a reference identity on all positions within the bound but disagrees outside it, in a manner that forces disagreement with the predictions of the projections.  
Therefore no finite family of projections can classify $\mathcal{G}_{\mathrm{eff}}$.
\end{proof}

This result foreshadows the Structural Incompleteness Theorem of Chapter~9 and motivates the study of specific projections in Chapter~7.

\section{Outlook}

Chapter~7 develops concrete examples of secondary projections and illustrates how different projections give incompatible but equally legitimate perspectives on hybrid and ghost identities.  
This phenomenon underlies the Rashomon effect in generative analysis and prepares for the diagonalizer construction in Chapter~8.  
Dependency bounds introduced here are used throughout Part~IV and are collected with proofs in Appendix~B.
