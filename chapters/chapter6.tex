\chapter{Dependency Bounds and Prefix Stabilization}
\label{chap:prefix-stabilization}

\section{Introduction}

Structural projections observe generative identities through finite windows.
For any fixed precision level, a projection requires information from only a
finite prefix of the selector, digit, and meta-information layers.  
This finite-information principle is classical in Type-2 computability, where
continuous real-valued functionals on Baire or Cantor space admit computable
moduli of continuity  
\cite{WeihrauchComputableAnalysis, PaulyRepresentedSpaces}.

In the generative setting, these moduli appear as \emph{dependency bounds}.  
They express how many coordinates an observer must inspect to approximate its
value at a specified precision.  
Dependency bounds are the structural backbone of alignment, tail sewing, and
the diagonalization arguments of Part~\ref{part:incompleteness}.  
This chapter formalizes these bounds and develops the principle of \emph{prefix
stabilization}, which asserts that once a projection has read enough symbols,
the remainder of the identity becomes irrelevant at that precision.

\section{Finite Information and Dependency Bounds}

Let $\Phi:\mathcal{G}^*\to\mathbb{R}$ be a structural projection.  
Continuity in the product topology implies that for each $\varepsilon>0$ there
exists $N$ such that agreement on the first $N$ coordinates forces agreement
of the outputs within $\varepsilon$:
\[
G[0..N] = H[0..N]
  \quad\Longrightarrow\quad
|\Phi(G) - \Phi(H)| < \varepsilon.
\]

\begin{definition}[Dependency Bound]
A function $B_\Phi:(0,1]\to\mathbb{N}$ satisfying
\[
G[0..B_\Phi(\varepsilon)]
  = 
H[0..B_\Phi(\varepsilon)]
\quad\Longrightarrow\quad
|\Phi(G) - \Phi(H)| < \varepsilon
\]
is called a \emph{dependency bound} for $\Phi$.
\end{definition}

When $\Phi$ is computable, classical results in represented-space theory imply
that $B_\Phi$ may be chosen to be computable  
\cite{WeihrauchComputableAnalysis}.  
This computability is essential for effective constructions, where observers
must be controlled explicitly at finite stages.

\section{Uniform Bounds for Finite Families}

Many arguments require the simultaneous control of several observers.

\begin{definition}[Uniform Dependency Bound]
Let
\[
\mathcal{P} = \{\Phi_1,\ldots,\Phi_k\}
\]
be a finite family of structural projections.  
A function $B_{\mathcal{P}}:(0,1]\to\mathbb{N}$ is a \emph{uniform dependency
bound} if
\[
G[0..B_{\mathcal{P}}(\varepsilon)]
   =
H[0..B_{\mathcal{P}}(\varepsilon)]
\quad\Longrightarrow\quad
|\Phi_i(G)-\Phi_i(H)|<\varepsilon
\]
for all $i$.
\end{definition}

Since the family is finite, we may take
\[
B_{\mathcal{P}}(\varepsilon)
  = \max_{1\le i\le k} B_{\Phi_i}(\varepsilon).
\]

Uniform bounds allow an entire collection of observers to be frozen at the
same prefix depth.  
This is used repeatedly in alignment and sewing, where families of projections
must be held stable while the identity tail is manipulated.

\section{Prefix Stabilization}

Prefix stabilization is the fundamental principle that observers do not
respond to information beyond their dependency bound.

\begin{proposition}[Prefix Stabilization]
Let $\Phi$ be a structural projection, fix $\varepsilon>0$, and set
$N=B_\Phi(\varepsilon)$.  
If $G[0..N]=H[0..N]$, then
\[
|\Phi(G)-\Phi(H)| < \varepsilon.
\]
\end{proposition}

\begin{proof}
This follows directly from continuity in the product topology, where basic
neighborhoods are determined by finite prefixes.
\end{proof}

Prefix stabilization isolates the portion of the generative identity that
matters to an observer at a chosen precision level.  
The remainder of the identity becomes “invisible” to $\Phi$ at that scale.

\section{Stability Under Tail Modification}

Tail modification replaces the part of an identity beyond some index with an
arbitrary tail.  
Such modifications are central to both alignment and divergence.

\begin{proposition}[Tail Stability]
Let $\Phi$ be a structural projection, let $\varepsilon>0$, and set
$N=B_\Phi(\varepsilon)$.  
If $G$ and $H$ agree on $[0..N]$, then the identity $\tilde G$ obtained from
$G$ by replacing its tail beyond $N$ with the tail of $H$ satisfies
\[
|\Phi(\tilde G) - \Phi(G)| < \varepsilon.
\]
\end{proposition}

\begin{proof}
The identities $\tilde G$ and $G$ agree on $[0..N]$, so the result is an
immediate consequence of prefix stabilization.
\end{proof}

Tail stability shows that observers may be frozen at a finite stage while the
tail remains completely free for later use.  
This property is the cornerstone of the sewing arguments developed in
Part~\ref{part:incompleteness}.

\section{Selector-Dependent Bounds}

Many projections depend only on the collapse coordinate rather than the raw
indexing of the layers.  
For such observers, the relevant prefix is determined by the \emph{selector
positions} where $M(n)=D$, not by the absolute position $n$.

Thus if the first $N$ selected digits agree, the observer may already be
stable even when the prefix lengths in the raw indexing differ.  
Selector-dependent bounds appear naturally when controlling:
\begin{itemize}
    \item density observers,
    \item fluctuation-type invariants,
    \item block-frequency observers on the collapse coordinate.
\end{itemize}

Although the indexing differs, the principle remains identical: each observer
has a finite window of dependence at each precision.

\section{Summary}

Dependency bounds encode the finite information requirements of continuous
observers on the generative space.  
Prefix stabilization ensures that observers become insensitive to changes
beyond their dependency bounds, and stability under tail modification permits
arbitrary symbolic variation in the unobserved region.  
These tools make it possible to coordinate families of observers, freeze their
behavior at finite stages, and then use the remaining freedom in the tail for
alignment and diagonalization.  
The next chapter develops these ideas by introducing explicit alignment
constructions.
