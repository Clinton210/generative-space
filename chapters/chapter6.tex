\chapter{The Lattice of Projections}

\section{Introduction}

The collapse map $\pi$ introduced in Part~I extracts a classical magnitude from a
generative identity by reading a subsequence of its digit layer.  
Although collapse is the primary invariant of the framework, it is only one of
many possible ways to interpret or summarize the structure of a generative
identity.

This chapter develops a general theory of \emph{structural projections}:
continuous maps out of the generative space that depend on only finitely many
prefix constraints at a time.  
Such projections formalize the notion of observation or measurement.  
They extract partial information about a generative identity while ignoring
other aspects of its structure.

The key insight is that these projections naturally form a \emph{lattice}
ordered by information content.  
Collapse is an extremal element of this lattice: it is the projection that
forgets the most internal structure while still retaining classical magnitude.

The lattice-theoretic viewpoint provides the conceptual and technical
foundation for the computable secondary projections introduced in
Chapter~7 and the projective incompatibility phenomenon analyzed in Chapter~8.

\section{Prefix-Determined Observations}

A projection represents an observer that examines the generative identity only
through finitely many coordinates of $M$, $D$, and $K$ at any given stage.

\begin{definition}[Prefix-Determined Map]
A function $\Phi : \mathcal{X} \to Y$ into a Hausdorff space $Y$ is
\emph{prefix-determined} if for every $G \in \mathcal{X}$ and every open
neighborhood $U$ of $\Phi(G)$, there exists a finite prefix
$(M{\upharpoonright}n, D{\upharpoonright}n, K{\upharpoonright}n)$ such that
any $H \in \mathcal{X}$ agreeing with $G$ on those prefixes satisfies
$\Phi(H) \in U$.
\end{definition}

This condition precisely characterizes continuity in the product topology of
$\mathcal{X}$.  
Prefix-determined maps therefore coincide with continuous maps out of the
generative space.

\begin{definition}[Structural Projection]
A \emph{structural projection} is any continuous map
\[
\Phi : \mathcal{X} \to Y
\]
into a metric (or Polish) space $Y$.
\end{definition}

Examples include:

\begin{itemize}
    \item the collapse map $\pi : \mathcal{X}^* \to [0,1]$,
    \item digit-density maps $G \mapsto \eta(G)$ (when extended appropriately),
    \item digit-frequency maps on selected subsequences,
    \item meta-frequency and meta-pattern projections,
    \item bounded dependency maps, introduced in Chapter~7.
\end{itemize}

\section{Ordering Projections by Informational Refinement}

Structural projections differ in what aspects of a generative identity they
preserve or discard.  
This suggests a natural notion of refinement.

\begin{definition}[Refinement Order]
Let $\Phi,\Psi : \mathcal{X} \to Y$ be projections into metric spaces.
We say that
\[
\Phi \preceq \Psi
\quad\Longleftrightarrow\quad
\text{for all } G,H \in \mathcal{X},\;
\Psi(G)=\Psi(H) \implies \Phi(G)=\Phi(H).
\]
\end{definition}

Thus $\Phi \preceq \Psi$ means that $\Psi$ distinguishes at least as many
identities as $\Phi$ does.  
Equivalently, $\Psi$ is at least as informative as $\Phi$.

\begin{remark}
The refinement order is determined purely by the equivalence relations induced
by the projections:
\[
G \sim_\Phi H \quad \Longleftrightarrow \quad \Phi(G)=\Phi(H).
\]
Then $\Phi \preceq \Psi$ iff $\sim_\Psi \subseteq \sim_\Phi$.
\end{remark}

\section{The Projection Lattice}

The refinement order makes the collection of structural projections into a
lattice.  
This lattice expresses how different projections interact and combine.

\begin{proposition}[Existence of Infima]
Let $\Phi$ and $\Psi$ be structural projections.  
There exists a projection $\Phi \wedge \Psi$ satisfying:
\[
G \sim_{\Phi \wedge \Psi} H
\quad\Longleftrightarrow\quad
(G \sim_\Phi H) \text{ and } (G \sim_\Psi H).
\]
Moreover, $\Phi \wedge \Psi$ is the greatest lower bound of $\Phi$ and $\Psi$ in
the refinement order.
\end{proposition}

\begin{proof}
Define $\Phi \wedge \Psi$ by mapping $G$ to the pair $(\Phi(G), \Psi(G))$
in the product space $Y_\Phi \times Y_\Psi$.  
Continuity follows from continuity of $\Phi$ and $\Psi$.  
The equivalence relation induced by this projection is the intersection of the
equivalence relations induced by $\Phi$ and $\Psi$.  
Thus it is the infimum in the refinement order.
\end{proof}

\begin{proposition}[Existence of Suprema Within Observational Classes]
Let $\{\Phi_i\}_{i \in I}$ be a family of projections.  
There exists a least projection $\Psi$ such that $\Phi_i \preceq \Psi$ for all
$i$, given by mapping
\[
\Psi(G) = (\Phi_i(G))_{i \in I}.
\]
\end{proposition}

Thus the set of structural projections is closed under arbitrary meets and under
joins indexed by families of observables.  
These operations supply a rich algebra of projections representing combined or
coarsened observation systems.

\section{Collapse as an Extremal Projection}

The collapse map sits at a special position in the lattice.

\begin{proposition}[Collapse Maximizes Information Loss]
Let $\Phi$ be any structural projection whose codomain is $[0,1]$ or any space
encoding only classical magnitude.  
Then
\[
\Phi \preceq \pi.
\]
\end{proposition}

\begin{proof}
If $\pi(G)=\pi(H)$, then $G$ and $H$ share the same classical magnitude.
Any projection $\Phi$ depending only on magnitude cannot distinguish $G$ from
$H$.  
Thus $\Phi(G)=\Phi(H)$, establishing $\Phi \preceq \pi$.
\end{proof}

Collapse is therefore the \emph{coarsest} projection that still computes
classical real values.  
It discards all meta information and all unselected digit positions.

In contrast, projections that detect selector properties, digit densities, meta
patterns, or statistical features typically refine collapse.

\section{Locality, Prefix Dependence, and Tail Freedom}

Because structural projections are prefix-determined, they are insensitive to
arbitrarily large modifications in the tail that preserve a sufficiently long
prefix.  
This principle underlies the diagonalization arguments of Part~IV.

\begin{proposition}[Tail Freedom]
Let $\Phi$ be a structural projection.  
For any $G \in \mathcal{X}$ and any $\varepsilon>0$, there exists $n$ such that
if $H$ agrees with $G$ on the first $n$ coordinates of each layer, then
$\Phi(G)$ and $\Phi(H)$ lie within $\varepsilon$ in the metric on $Y$.
\end{proposition}

\begin{proof}
Prefix-determined maps are continuous in the product topology, which is defined
by agreement on sufficiently long prefixes.  
The statement is a restatement of continuity.
\end{proof}

Thus structural projections have intrinsic limitations: they cannot fully
capture differences that arise only in the distant tail.  
This aligns with their role as \emph{observables}, not full descriptions.

\section{Outlook}

The lattice of projections provides a structured vocabulary for describing
observable properties of generative identities and for comparing their
informational strength.  
Collapse occupies an extremal position: it is the coarsest projection that still
yields classical magnitude.

Chapter~7 introduces \emph{computable} structural projections, obtained by
imposing finite lookahead and algorithmic constraints on the observation
process.  
Chapter~8 develops the phenomenon of projective incompatibility, showing that
different projection families may fundamentally disagree about the structure of
a single collapse fiber.  
Together, these chapters form the backbone of Part~III's measurement theory.
