\chapter{Secondary Projections and Finite Lookahead}

\section{Introduction}

The collapse map $\pi$ introduced in Chapter~2 acts as the primary invariant of the generative framework, projecting the mechanisms in the generative space $\mathcal{X}$ onto the classical continuum.  
Every other numerical quantity derived from a generative identity is understood as a \emph{secondary projection}.

Secondary projections summarize aspects of the internal structure of a generative identity $G = (M,D,K)$.  
They may measure digit frequency, selector complexity, local fluctuation rates, or meta-layer behavior.  
Unlike the collapse map, secondary projections do not uniquely identify a mechanism within its collapse fiber.

This chapter formally defines secondary projections as computable functionals.  
The key result is the \emph{finite lookahead property}: any computable secondary projection can inspect only a finite prefix of an effective identity when producing an approximation to desired precision.  
This generates explicit \emph{dependency bounds} that are essential to the diagonalizer construction of Chapter~8.

\section{Secondary Projections}

A secondary projection is any functional that extracts numerical information from a generative identity.

\begin{definition}[Secondary Projection]
A map
\[
\Phi : \mathcal{X} \to \mathbb{R}^k
\]
is a secondary projection if it is continuous with respect to the product topology on $\mathcal{X}$ and the Euclidean topology on $\mathbb{R}^k$.

When restricted to the effective core, $\Phi$ is additionally required to be Type--2 computable:
\[
\Phi : \mathcal{G}_{\mathrm{eff}} \to \mathbb{R}^k_c.
\]
\end{definition}

Continuity ensures stability under tail modifications of $G$.  
Computability ensures that $\Phi(G)$ can be approximated uniformly from finite information.

\begin{remark}
Examples of secondary projections include digit density, block entropy of the meta layer, normalized selector complexity, and frequency-based summaries.  
A detailed catalogue appears in Chapter~7.
\end{remark}

\section{Projections as Coordinates on Fibers}

Secondary projections supply coordinate systems on collapse fibers.  
For a fixed $x \in [0,1]$, the fiber $\mathcal{F}(x)$ contains many distinct identities that share the same magnitude but differ in their secondary behavior.

\begin{proposition}
A secondary projection $\Phi$ is constant on each fiber if and only if $\Phi$ depends only on the selected digit subsequence.  
In general, distinct elements of $\mathcal{F}(x)$ produce distinct values of $\Phi$.
\end{proposition}

\begin{proof}
If $\Phi$ depends solely on the selected digit subsequence, then $\Phi$ factors through $\pi$, hence is constant on every fiber.  
Conversely, if $\Phi$ depends on any meta or unselected digit coordinate, then by modifying those coordinates within a fiber (see Chapter~3) we obtain two fiber elements that differ on an open set mapped to distinct values in $\mathbb{R}^k$.  
Continuity of $\Phi$ ensures these distinct values persist.
\end{proof}

This variability among fiber elements leads to the phenomenon of \emph{projective incompatibility}, where inequivalent projections yield conflicting descriptions of the same mechanism.

\section{Finite Lookahead and Dependency Bounds}

The fundamental limitation of any computable secondary projection is \emph{finite lookahead}.  
To compute an approximation $\Phi(G)$ to precision $\varepsilon$, a Type--2 Turing machine can inspect only finitely many symbols of $M$, $D$, and $K$.  
This limitation is formalized as follows.

\begin{definition}[Dependency Bound]
Let $\Phi : \mathcal{G}_{\mathrm{eff}} \to \mathbb{R}^k$ be a computable secondary projection.  
A function
\[
B_\Phi : \mathbb{Q}^+ \to \mathbb{N}
\]
is a dependency bound for $\Phi$ if for every rational $\varepsilon > 0$ and for all effective identities $G$ and $H$,
\[
(M_G, D_G, K_G){\upharpoonright}B_\Phi(\varepsilon)
=
(M_H, D_H, K_H){\upharpoonright}B_\Phi(\varepsilon)
\]
implies
\[
\|\Phi(G) - \Phi(H)\| < \varepsilon.
\]
\end{definition}

Thus $\Phi$ cannot distinguish between two identities whose prefixes agree beyond its dependency bound.

\begin{proposition}[Existence of Dependency Bounds]
Every computable secondary projection on $\mathcal{G}_{\mathrm{eff}}$ admits a computable dependency bound.
\end{proposition}

\begin{proof}
By a standard result in computable analysis (see Weihrauch), any computable function from a compact represented space is effectively uniformly continuous.  
The product space $\mathcal{X}$ is compact, and $\mathcal{G}_{\mathrm{eff}}$ inherits this representation.  
Therefore, for each $\varepsilon > 0$, there exists an $N$ such that inspecting any input beyond $N$ cannot affect the output by more than $\varepsilon$.  
The function that maps $\varepsilon$ to such an $N$ is computable and provides a dependency bound.
\end{proof}

The dependency bound $B_\Phi(\varepsilon)$ formalizes the finite observational horizon of $\Phi$.

\section{Uniformity for Finite Families}

The diagonalizer of Chapter~8 must evade a \emph{finite} set of projections simultaneously, which requires a uniform horizon of observation.

\begin{definition}[Uniform Dependency Bound]
Let $\mathcal{P} = \{ \Phi_1, \dots, \Phi_m \}$ be a finite family of computable secondary projections.  
A function $B_{\mathcal{P}} : \mathbb{Q}^+ \to \mathbb{N}$ is a uniform dependency bound if it is a dependency bound for each $\Phi_i$.
\end{definition}

\begin{proposition}
Every finite family of computable secondary projections admits a uniform dependency bound.
\end{proposition}

\begin{proof}
Let $B_{\Phi_i}$ be a dependency bound for $\Phi_i$.  
Define
\[
B_{\mathcal{P}}(\varepsilon)
=
\max_{1 \le i \le m} B_{\Phi_i}(\varepsilon).
\]
This maximum is computable and valid for all projections in the family.
\end{proof}

A uniform bound means that if we modify a mechanism strictly beyond $B_{\mathcal{P}}(\varepsilon)$, then no projection in $\mathcal{P}$ can detect the modification at precision $\varepsilon$.

\section{The Limit of Finite Observation}

The existence of dependency bounds implies intrinsic limitations on what secondary projections can measure.

\begin{proposition}[Arbitrariness of Coordinates]
No finite family of secondary projections can fully recover the internal structure of a generative identity.  
For every finite family $\mathcal{P}$, there exist distinct effective identities $G$ and $H$ that agree on all projections in $\mathcal{P}$ to arbitrary precision but differ at infinitely many coordinates.
\end{proposition}

\begin{proof}
This follows from the Structural Incompleteness Theorem proved in Part~IV.  
Given $\mathcal{P}$ and $\varepsilon > 0$, modify $G$ in the tail beyond $B_{\mathcal{P}}(\varepsilon)$ to obtain $H$ that differs infinitely often from $G$ but produces projection values within $\varepsilon$ under each $\Phi \in \mathcal{P}$.
\end{proof}

Thus secondary projections act as incomplete coordinate systems, revealing only finite fragments of the structure of effective generative identities.

\section{Outlook}

This chapter establishes the computational and analytical limitations of secondary projections.  
Chapter~7 develops concrete examples such as densities, entropies, and selector-growth summaries, illustrating projective incompatibility.  
These phenomena culminate in the diagonalizer of Chapter~8, which exploits finite lookahead to evade all finite families of projections and demonstrate structural incompleteness.
