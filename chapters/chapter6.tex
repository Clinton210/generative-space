\chapter{Dependency Bounds and Prefix Stabilization}

\section{Introduction}

Structural projections evaluate generative identities using only finitely many
symbols at any fixed precision.  
This finite information principle is central to Type-2 computability, where
continuous functionals on Baire space are understood through their moduli of
continuity.  
In the generative setting, these moduli appear naturally as \emph{dependency
bounds}.  

This chapter develops the machinery that allows observers to be controlled at
finite stages.  
We formalize prefix stabilization, show how dependency bounds govern
finite-stage agreement, and explain how these properties prepare the ground
for the construction of the meta-diagonalizer in Part IV.

\section{Finite Information and Dependency Bounds}

Let $\Phi : \mathcal{X}^* \to \mathbb{R}$ be a structural projection.  
Continuity implies that for every $\varepsilon > 0$ there exists an integer
$B_\Phi(\varepsilon)$ such that agreement on the first $B_\Phi(\varepsilon)$
symbols of the identity forces agreement of the projections within
$\varepsilon$:
\[
G[0..B_\Phi(\varepsilon)]
  = H[0..B_\Phi(\varepsilon)]
\quad\Longrightarrow\quad
|\Phi(G) - \Phi(H)| < \varepsilon.
\]

When $\Phi$ is computable, the classical results of Pour-El, Richards,
Weihrauch, and Pauly guarantee that the map $\varepsilon \mapsto
B_\Phi(\varepsilon)$ may be chosen computably.  
This computability requirement is essential for the effective diagonalization
argument, where observers must be controlled by explicit finite parameters.

\section{Uniform Bounds for Finite Families}

Many arguments involve finite families of projections that must be handled
simultaneously.

\begin{definition}[Uniform Dependency Bound]
Given a finite family
\[
\mathcal{P} = \{ \Phi_1, \ldots, \Phi_k \}
\]
of projections, a function $B_{\mathcal{P}} : (0,1] \to \mathbb{N}$ is a
\emph{uniform dependency bound} if
\[
G[0..B_{\mathcal{P}}(\varepsilon)]
  = H[0..B_{\mathcal{P}}(\varepsilon)]
\quad\Longrightarrow\quad
|\Phi_i(G) - \Phi_i(H)| < \varepsilon
\]
for all $i$.
\end{definition}

Since the family is finite, we may take
\[
B_{\mathcal{P}}(\varepsilon)
  = \max_i B_{\Phi_i}(\varepsilon),
\]
which is computable if each $\Phi_i$ is.

Uniform bounds allow us to freeze a finite family of observers at a single
precision parameter.  
This operation is repeated at increasing precision in the diagonalizer
construction.

\section{Prefix Stabilization}

The key structural property of projections is that agreement beyond the
dependency bound is irrelevant to their evaluation.

\begin{proposition}[Prefix Stabilization]
Let $\Phi$ be a structural projection.  
Fix $\varepsilon > 0$ and set $N = B_\Phi(\varepsilon)$.  
If $G$ and $H$ agree on their first $N$ symbols, then their projections differ
by less than $\varepsilon$:
\[
G[0..N] = H[0..N]
\quad\Longrightarrow\quad
|\Phi(G) - \Phi(H)| < \varepsilon.
\]
\end{proposition}

\begin{proof}
This is exactly the definition of continuity in the product topology.  
The basic open neighborhoods of $G$ are determined by finite prefixes.
Choosing $N$ as the length of such a prefix gives the desired result.
\end{proof}

Prefix stabilization encodes the idea that projections observe only a finite
window of the identity at any fixed resolution.  
The unobserved tail may contain arbitrary structure without being detected by
the observer.

\section{Stability Under Tail Modification}

Tail modification is the process of replacing the portion of a generative
identity beyond some index $N$ with an arbitrary tail.

\begin{proposition}
Let $\Phi$ be a structural projection, let $\varepsilon > 0$, and let
$N = B_\Phi(\varepsilon)$.  
If $G$ and $H$ agree on $[0..N]$, then replacing the tail of $G$ by the tail
of $H$ beyond $N$ produces a new identity $\tilde{G}$ that satisfies
\[
|\Phi(\tilde{G}) - \Phi(G)| < \varepsilon.
\]
\end{proposition}

\begin{proof}
Since $\tilde{G}$ and $G$ agree on their first $N$ symbols, the conclusion
follows from prefix stabilization.
\end{proof}

This invariance under tail modification is one of the central structural
properties of projections.  
It ensures that observers can be satisfied at finite stages, while the tail
remains available for divergence, which is essential for diagonalization.

\section{Interaction with Selector Density}

Many structural projections depend only on selected digits.  
For such projections, the relevant prefixes are determined by the positions
where $M(n) = D$, not by the raw index $n$.  
This leads to selector-dependent versions of dependency bounds, which appear
later when controlling density and fluctuation observers.

The general principle remains unchanged: agreement on the relevant finite
prefix of the canonical output determines agreement of the projection at the
corresponding precision.

\section{Summary}

Dependency bounds capture the finite information content of observers on the
generative space.  
Prefix stabilization and tail invariance show that structural projections
depend only on finite prefixes at any fixed precision.  
These properties enable finite-stage control of observers and are the key
technical tools for the alignment and sewing constructions that begin in the
next chapter and culminate in the meta-diagonalizer of Part IV.
