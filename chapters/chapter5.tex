\chapter{Structural Projections and the Projection Lattice}

\section{Introduction}

The collapse map extracts the classical value of a generative identity while
discarding most of its internal structure.  
To understand which aspects of this structure can be detected by continuous
observers, we introduce the general notion of a \emph{structural projection}.
These projections form a lattice under pointwise comparison and represent
effective measurements that respect the topology of the generative space.

The framework developed in this chapter draws on ideas from Type-2
Effectivity, where continuous functionals on sequence spaces are understood
through their finite information content.  
This finite information principle, central in the work of Weihrauch and
Pauly on represented spaces, appears here in an explicit combinatorial form.
It allows projections to be analyzed through their dependency on finite
prefixes and serves as the foundation for the incompleteness results proved
later.

\section{Structural Projections}

A \emph{structural projection} is any continuous function
\[
\Phi : \mathcal{X}^* \to \mathbb{R},
\]
where $\mathcal{X}^*$ carries the product topology defined in Part I.  
Continuity ensures that the value $\Phi(G)$ is determined to any fixed
precision by a finite prefix of $G$.

More precisely, for every $\varepsilon > 0$, continuity provides an integer
$B_\Phi(\varepsilon)$ such that
\[
G[0..B_\Phi(\varepsilon)]
  = H[0..B_\Phi(\varepsilon)]
\quad\Longrightarrow\quad
|\Phi(G) - \Phi(H)| < \varepsilon.
\]

The function $B_\Phi$ plays the role of a computable modulus of continuity in
the sense of Type-2 computability, which is the standard framework for
analyzing real-valued functionals on symbolic spaces.

\section{Basic Examples}

Several projections arise naturally from the structure of a generative
identity.

\subsection*{Collapse}

The collapse $\pi$ is the foundational projection.  
Its continuity was established in Chapter 2 and follows from the classical
theory of real number representations.

\subsection*{Digit statistics}

Fix a digit $a \in \{0,\ldots,b-1\}$.  
Define
\[
\Phi_a(G)
  = \liminf_{N \to \infty}
      \frac{1}{N}
      \sum_{j=0}^{N-1} \mathbf{1}[\, d_G(j) = a \,].
\]
This projection measures the lower asymptotic frequency of the digit $a$ in
the canonical output.  
Other variants include limsup frequency or empirical block frequencies.

Such projections resemble classical invariants in symbolic dynamics, where
frequency statistics determine measure-theoretic properties of subshifts.
The exposition of Lind and Marcus provides many examples of these quantities
in the context of shift spaces.

\subsection*{Selector statistics}

Define
\[
\Psi(G)
  = \liminf_{N \to \infty}
      \frac{1}{N}
      \sum_{n=0}^{N-1} \mathbf{1}[\, M(n) = D \,].
\]
This projection measures the asymptotic density with which the selector
exposes digits.  
It coincides with the selector density studied in Chapter 4 but now viewed as
an observer on $\mathcal{X}^*$.

\section{Dependency Bounds}

Dependency bounds measure the amount of information an observer requires to
determine its output to a given precision.

\begin{definition}[Dependency Bound]
Let $\Phi : \mathcal{X}^* \to \mathbb{R}$ be continuous.  
A function $B_\Phi : (0,1] \to \mathbb{N}$ is a \emph{dependency bound} for
$\Phi$ if
\[
G[0..B_\Phi(\varepsilon)]
  = H[0..B_\Phi(\varepsilon)]
\quad\Longrightarrow\quad
|\Phi(G) - \Phi(H)| < \varepsilon
\]
for all $\varepsilon > 0$.
\end{definition}

If $\Phi$ is computable, classical results from Type-2 Effectivity imply that
$B_\Phi$ can be chosen to be computable as well.  
This follows from the fact that computable functionals on Baire space admit
computable moduli of continuity.

Dependency bounds quantify the finite information content of observers and
provide the mechanism by which projections can be frozen at finite stages in
the diagonalizer construction of Part IV.

\section{The Projection Lattice}

Given two projections $\Phi$ and $\Psi$, define
\[
\Phi \le \Psi
\quad\Longleftrightarrow\quad
\Phi(G) = \Phi(H) \text{ whenever } \Psi(G) = \Psi(H).
\]
This relation expresses that $\Psi$ distinguishes at least as much structure
as $\Phi$.

\begin{proposition}
The set of structural projections on $\mathcal{X}^*$ ordered by $\le$ forms a
complete lattice.
\end{proposition}

\begin{proof}
For any family of projections $(\Phi_i)$, the pointwise supremum
\[
\Phi(G) = \sup_i \Phi_i(G)
\]
is still continuous and therefore a structural projection.  
This projection is the least upper bound with respect to $\le$.  
Similarly, pointwise infima provide greatest lower bounds.
\end{proof}

This algebraic structure parallels the lattice of continuous real-valued
functionals on represented spaces and has been extensively studied in the
context of Weihrauch degrees.  
Here it provides the organizational framework for understanding how different
projections capture different aspects of generative structure.

\section{Summary}

Structural projections are continuous observers on the generative space.  
Their finite dependency on prefixes gives rise to computable dependency
bounds, and their collective structure forms a complete lattice.  
These properties reflect classical results from Type-2 computability and
symbolic dynamics but are here adapted to the generative identity setting.

In the next chapter we formalize prefix stabilization and show how the finite
dependency of observers enables the controlled constructions that drive the
incompleteness phenomena in Part IV.
