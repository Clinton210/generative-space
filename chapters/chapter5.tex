\chapter{Structural Projections and the Projection Lattice}
\label{chap:projections}

\section{Introduction}

The collapse map extracts only the classical magnitude of a generative
identity.  
To understand what additional structure can be detected by continuous
observers, we introduce the notion of a \emph{structural projection}.  
These observers read only finitely many coordinates of the selector, digit,
and meta-information layers to achieve any prescribed precision.  
They capture the visible aspects of generative structure from the perspective
of continuous measurement.

The theory developed here is grounded in Type-2 Effectivity, where continuous
functionals on Baire and Cantor spaces are analyzed through their finite
information content  
\cite{WeihrauchComputableAnalysis, PaulyRepresentedSpaces}.  
In the generative setting, this principle appears in a concrete combinatorial
form: every continuous observer depends only on a finite prefix of a
generative identity at each precision level.

Structural projections will form the observational layer used in
Part~\ref{part:incompleteness}.  
Their dependency bounds enable the controlled constructions that lead to
alignment, divergence, and diagonalization.

\section{Structural Projections}

Let $\mathcal{G}^*$ be the digit-producing subspace of the generative space
$\mathcal{G}$ defined in Chapter~\ref{chap:generative-space}.

\begin{definition}[Structural Projection]
A \emph{structural projection} is a continuous function
\[
\Phi : \mathcal{G}^* \to \mathbb{R}
\]
with respect to the product topology on $\mathcal{G}^*$.
\end{definition}

Continuity ensures that $\Phi(G)$ can be approximated using only a finite
prefix of $G$.  
That is, for each $\varepsilon > 0$ there exists an integer $N$ such that
prefix agreement up to $N$ forces agreement of $\Phi$ up to $\varepsilon$.

Concretely, for each $\varepsilon > 0$ there exists
$B_\Phi(\varepsilon) \in \mathbb{N}$ such that
\[
G[0..B_\Phi(\varepsilon)]
   = 
H[0..B_\Phi(\varepsilon)]
\quad\Longrightarrow\quad
\bigl|\Phi(G) - \Phi(H)\bigr| < \varepsilon .
\]

This finite-information property is the analogue of a modulus of continuity
for real-valued functionals on symbolic spaces.

\section{Examples of Structural Projections}

Several observers arise naturally from the structure of generative identities.

\subsection*{The collapse map}

The collapse map $\pi(G)$ depends only on the collapse coordinate $X(G)$.  
Its continuity was established in Chapter~\ref{chap:collapse-map}.  
It is the archetypal structural projection.

\subsection*{Digit frequency observers}

Fix a digit $a\in\{0,\ldots,b-1\}$.  
Define the lower frequency of $a$ in the collapse coordinate by
\[
\Phi_a(G)
  = 
  \liminf_{N\to\infty}
     \frac{1}{N} 
     \sum_{j < N} \mathbf{1}\bigl[X(G)_j = a\bigr].
\]

These observers resemble classical frequency invariants in symbolic dynamics,
where empirical digit statistics encode measure-theoretic behavior
\cite{LindMarcus}.

\subsection*{Selector exposure observers}

Define
\[
\Psi(G)
   =
   \liminf_{N\to\infty}
      \frac{1}{N}
      \sum_{n < N} \mathbf{1}[M(n)=D].
\]

This projection coincides with the selector density introduced in
Chapter~\ref{chap:density}, now viewed as a continuous functional on
$\mathcal{G}^*$.

\section{Dependency Bounds}

Dependency bounds quantify the finite information content of a structural
projection.

\begin{definition}[Dependency Bound]
Let $\Phi:\mathcal{G}^*\to\mathbb{R}$ be continuous.  
A function
\[
B_\Phi:(0,1]\to\mathbb{N}
\]
is a \emph{dependency bound} for $\Phi$ if for all $\varepsilon>0$,
\[
G[0..B_\Phi(\varepsilon)]
  =
H[0..B_\Phi(\varepsilon)]
\quad\Longrightarrow\quad
|\Phi(G) - \Phi(H)| < \varepsilon.
\]
\end{definition}

Dependency bounds describe the number of coordinates an observer must read to
produce an $\varepsilon$-accurate estimate.  
If $\Phi$ is computable, classical results in Type-2 computability imply that
$B_\Phi$ can be chosen to be computable  
\cite{WeihrauchComputableAnalysis}.

These bounds play a central role in prefix stabilization and in the
construction of the meta-diagonalizer in Part~\ref{part:incompleteness}.

\section{The Projection Lattice}

Observers can be compared according to how much structure they distinguish.

\begin{definition}[Informational Dominance]
For structural projections $\Phi$ and $\Psi$ define
\[
\Phi \leq \Psi
\quad\Longleftrightarrow\quad
\Phi(G)=\Phi(H)
\ \text{whenever}\ 
\Psi(G)=\Psi(H).
\]
\end{definition}

Thus $\Psi$ distinguishes at least as much structure as $\Phi$: any pair of
identities indistinguishable to $\Psi$ must also be indistinguishable to
$\Phi$.

\begin{proposition}
The set of structural projections on $\mathcal{G}^*$ ordered by $\leq$
forms a complete lattice.
\end{proposition}

\begin{proof}
Given a family $(\Phi_i)$, define the pointwise supremum
\[
\Phi(G) = \sup_i \Phi_i(G).
\]
Pointwise suprema of continuous functions are continuous, so $\Phi$ is again
a structural projection.  
This function is the least upper bound of the family.  
A similar argument shows that pointwise infima give greatest lower bounds.
\end{proof}

This lattice structure parallels the order-theoretic organization of
continuous functionals on represented spaces and is closely related to the
hierarchies appearing in the study of Weihrauch degrees.

\section{Summary}

Structural projections are continuous real-valued observers on the
digit-producing generative space.  
Their finite dependency on prefixes yields computable dependency bounds, and
their informational structure forms a complete lattice.  
These features provide the analytical and combinatorial tools required for the
prefix stabilization and incompleteness phenomena that follow.

In the next chapter we develop prefix stabilization and show how dependency
bounds allow continuous observers to be controlled at finite stages during
alignment and diagonalization.
