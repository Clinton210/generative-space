\chapter{Null-Density Generators and Sparse Dynamics}

\section{Introduction}

Chapter~4 analyzed hybrid identities, whose selectors choose the digit layer on a
set of positive lower density.  
Hybrid mechanisms occupy the high-density end of the selector spectrum, where
the information encoding classical magnitude is distributed across a substantial
portion of the timeline.

This chapter develops the complementary regime: \emph{null-density generators}.
A null-density generator selects the digit layer infinitely often, ensuring that
collapse is well-defined, but does so at vanishing asymptotic density.  
In this regime the meta layer dominates the canonical output, yet the sparse
digit positions still encode a precise classical real number.

Null-density behavior demonstrates the expressive flexibility of collapse.  
Magnitude information can be encoded in a sparse subsequence of the timeline,
leaving the overwhelming majority of coordinates unconstrained.  
This allows entire regions of the mechanism to carry additional structure that is
invisible to classical magnitude but becomes relevant for secondary invariants
and structural projections.

The goal of this chapter is to formalize null-density generation, establish its
prevalence in both full and effective fibers, analyze its shift dynamics, and
contrast it with hybrid behavior.  
These results complete the classification of selector regimes that underlies
Part~III’s projection-theoretic analysis and Part~IV’s diagonalizer.

\section{Digit Sparsity and Null-Density Structure}

A null-density generator represents a real number through a selector that chooses
the digit layer rarely but still infinitely often.

\begin{definition}[Null-Density Generator]
A generative identity $G = (M,D,K)$ is a \emph{null-density generator} if
\begin{enumerate}
    \item $M$ selects the digit layer infinitely often, and  
    \item its digit density satisfies
    \[
    \eta(G)
    =
    \liminf_{n\to\infty}
    \frac{1}{n}
    \bigl|\{\,0 \le k < n : M(k)=D \,\}\bigr|
    =
    0.
    \]
\end{enumerate}
\end{definition}

Thus the distances between successive digit selections grow without bound.
The resulting canonical output consists almost entirely of meta symbols.

\begin{remark}[Symbolic-Dynamic Perspective]
From the standpoint of symbolic dynamics, a null-density selector belongs to a
subshift of extremely low combinatorial complexity.  
If the gaps between digit selections diverge, the selector subshift has
topological entropy zero.  
This contrasts sharply with many hybrid selectors, which may belong to
positive-entropy subshifts.
\end{remark}

\section{Existence in Full Fibers}

Null-density behavior is compatible with any classical magnitude.

\begin{proposition}[Null-Density Generators in Fibers]
\label{prop:nd-fiber}
For every $x \in [0,1]$, the fiber $\mathcal{F}(x)$ contains infinitely many
null-density generators.
\end{proposition}

\begin{proof}
Let $(x_j)$ be a base-$b$ expansion of $x$.  
Let $S = \{ j^2 : j \ge 1 \}$ and define the selector $M$ by $M(n)=D$ if
$n \in S$ and $M(n)=K$ otherwise.  
The number of squares below $n$ satisfies $|S \cap [0,n)| \approx \sqrt{n}$, so
$\eta(G)=0$ for any identity $G$ using this selector.

Define $D(j^2)=x_j$ and assign $D(n)$ arbitrarily for $n\notin S$.  
Choose any meta sequence $K$.  
All such mechanisms collapse to $x$, and varying $K$ yields infinitely many
distinct null-density generators in $\mathcal{F}(x)$.
\end{proof}

Sparse selectors therefore impose no restriction on which real numbers may be
represented.

\section{Null-Density Generators in the Effective Core}

The construction above remains valid in the effective setting because the sparse
selector pattern $n=j^2$ is computable.

\begin{proposition}[Effective Null-Density Generation]
\label{prop:nd-effective}
If $x \in \mathbb{R}_c$, then $\mathcal{F}_{\mathrm{eff}}(x)$ contains effective
null-density generators.  
If $x$ is not computable, $\mathcal{F}_{\mathrm{eff}}(x)$ is empty.
\end{proposition}

\begin{proof}
Let $(x_j)$ be a computable expansion of $x$.  
Define a selector $M$ that chooses $D$ exactly at positions $j^2$.  
The set of squares is recursive, so $M$ is computable and satisfies $\eta(G)=0$.

Define $D(j^2)=x_j$ and set $D(n)$ arbitrarily elsewhere; this is computable
because the map $j \mapsto j^2$ is computable and strictly increasing.  
Let $K$ be any computable sequence.  
Then $G = (M,D,K)$ is effective and collapses to $x$.
\end{proof}

Thus null-density generators exist at both the full and effective levels.

\section{Sparse Dynamics Under the Shift}

Null-density behavior interacts predictably with shift dynamics.

\begin{proposition}[Shift Invariance of Null Density]
If $G$ is a null-density generator, then the shifted identity $\sigma(G)$ is also
a null-density generator.
\end{proposition}

\begin{proof}
Let
\[
A_n = |\{\,0 \le k < n : M(k)=D\,\}|.
\]
For the shifted selector, the corresponding count is either $A_{n+1}$ or
$A_{n+1}-1$.  
Since $A_n/n \to 0$, the same holds for $A_{n+1}/n$, and thus
$\eta(\sigma(G))=0$.
\end{proof}

Null-density selectors are therefore dynamically stable under the shift, in
contrast to certain hybrid selectors whose density properties may vary under
block codings or local transformations.

\section{Contrast With Hybrid Generators}

Hybrid and null-density identities form two structural extremes within collapse
fibers:

\begin{itemize}
    \item \textbf{Hybrids ($\eta>0$):} magnitude information is distributed over
    a positive-density set of positions; meta information is interwoven with the
    digit layer.
    \item \textbf{Null-density generators ($\eta=0$):} the digit layer appears
    sparsely; almost all coordinates of the mechanism are unconstrained by
    collapse.
\end{itemize}

These regimes exemplify the tension at the core of Part~III: different
projection families respond differently to sparsity and density.  
Digit-density projections separate hybrids from null-density generators, whereas
other invariants may fail to do so.  
This mismatch is a precursor to the projective incompatibility results of
Chapter~8 and to the diagonalizer of Chapter~9, which exploits the ability to
switch between high- and low-density patterns in the tail.

\section{Outlook}

Null-density generators complete the basic classification of selector regimes.
Together with hybrid identities, they illustrate the full range of internal
freedom present in collapse fibers.  
These contrasting behaviors motivate the structural projections developed in
Part~III, where we investigate how secondary coordinates detect or miss these
internal patterns.  
The expressive power and limitations of such projections form the foundation for
the incompleteness results of Part~IV.
