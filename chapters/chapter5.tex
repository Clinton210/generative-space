\chapter{Ghost Reals and Sparse Digit Dynamics}

\section{Introduction}

Chapter~4 introduced hybrid generative identities, in which the mixer selects the digit layer with positive density.  
Hybrids represent one of the two principal structural regimes in the generative space.  
This chapter develops the complementary regime: ghost identities.  
A ghost identity collapses to a classical real number but uses the digit layer so rarely that the meta layer dominates the canonical output.  
Ghost behavior reveals the opposite extreme of internal structure and illustrates how minimal digit information can still produce a definite classical magnitude.

Ghost patterns play a dual role in the generative framework.  
They show how sparse digit mechanisms can encode classical reals, and they form contrasting examples for the hybrid patterns of Chapter~4.  
They also prepare the ground for the study of secondary invariants in Part~III, where sparse digit signals produce coordinate systems that behave very differently from the hybrid case.

\section{Digit Sparsity and Ghost Structure}

We begin by formalizing the notion of a ghost identity.

\begin{definition}[Ghost Generative Identity]
A generative identity $G = (M, D, K)$ is a \emph{ghost identity} if its digit density satisfies
\[
\eta(G)
=
\liminf_{n \to \infty}
\frac{1}{n}
\bigl| \{ 0 \le k < n : M(k) = D \} \bigr|
=
0.
\]
\end{definition}

A ghost identity may still have infinitely many digit selections, but these selections occur with vanishing frequency.  
In this regime the meta layer dominates the canonical output, yet the sparse digit subsequence still determines the collapse.

\begin{remark}
An identity may be ghost even if it collapses to a non-trivial real number.  
The collapse is determined entirely by the sparse digit subsequence, regardless of how infrequently it appears.
\end{remark}

\section{Existence of Ghost Identities in Every Fiber}

Ghost identities occur in every full fiber of the collapse map.

\begin{proposition}[Ghost Identities in Full Fibers]
For every real number $x \in [0,1]$, the full fiber $\mathcal{F}(x)$ contains infinitely many ghost identities.
\end{proposition}

\begin{proof}
Fix a base-$b$ expansion $(x_j)_{j \ge 0}$ of $x$.  
Define $M$ to select the digit layer only at positions $n$ belonging to a slowly growing sequence such as $n = j^2$.  
This guarantees that $\eta(G) = 0$.  
Assign the digit sequence so that $D(\varphi_G(j)) = x_j$, where $\varphi_G(j)$ enumerates the selected positions in increasing order.  
Complete $K$ arbitrarily.  
Varying $K$ or the precise sparse pattern of $M$ produces infinitely many ghost identities in $\mathcal{F}(x)$.
\end{proof}

This proposition contrasts with the hybrid universality theorem of Chapter~4.  
Hybrid and ghost patterns are both abundant, but they describe opposing internal behaviors.

\section{Ghost Identities in the Effective Core}

The situation in the effective core is more delicate.

\begin{proposition}
If $x \in \mathbb{R}_c$, then $\mathcal{F}_{\mathrm{eff}}(x)$ contains effective ghost identities.  
If $x \notin \mathbb{R}_c$, then $\mathcal{F}_{\mathrm{eff}}(x)$ is empty.
\end{proposition}

\begin{proof}
Let $(x_j)_{j \ge 0}$ be a computable expansion of $x$.  
Define the mixer to select the digit layer on the computable sparse set $\{j^2 : j \ge 0\}$.  
This set has density zero, so any mechanism using it for digit selection is ghost.  
Define $D$ by $D(j^2) = x_j$, and define $K$ to be any computable sequence.  
This produces an effective ghost identity in $\mathcal{F}_{\mathrm{eff}}(x)$.  
The second statement follows from collapse considerations in Chapter~2.
\end{proof}

Thus ghost identities are algorithmically available for all computable real numbers, although they represent a very different internal mode of generation from the hybrid identities of Chapter~4.

\section{Sparse Digit Dynamics and Shift Behavior}

The sparse digit patterns that define ghost identities interact in interesting ways with the shift map $\sigma$ on $\mathcal{X}$.

\begin{proposition}
If $G$ is a ghost identity and $(S_D(G))$ is infinite, then the shifted identity $\sigma(G)$ is also ghost.
\end{proposition}

\begin{proof}
The shift reduces the size of the prefix interval $[0,n)$ by exactly one position, which does not affect the asymptotic density of the digit selection set.  
Therefore the density remains zero under the shift.
\end{proof}

This stability under shift dynamics complements the hybrid regime, where shift dynamics may alter or preserve digit selection frequencies depending on the mixer.  
Sparse digit behavior is therefore structurally robust in the product topology.

\section{Contrast With Hybrids}

Hybrid and ghost identities form two conceptual extremes:

\begin{itemize}
    \item hybrids have $\eta(G) > 0$ and interleave the digit and meta layers on a substantial set of positions,
    \item ghosts have $\eta(G) = 0$ and encode classical magnitude through rare digit events.
\end{itemize}

These regimes represent two distinct ways in which the generative space supports collapse to the same classical real number.  
Chapter~7 shows that secondary invariants may emphasize or suppress different features of hybrid and ghost identities, illustrating a Rashomon effect in coordinate systems.  

This contrast will also be central to the diagonalizer construction in Chapter~8, where tail modifications are used to evade a finite family of projections.

\section{Examples}

Explicit constructions of effective and full ghost identities for classical numbers such as $\sqrt{2}$ appear in Appendix~C.  
These examples illustrate how sparse generators can encode real numbers using minimal digit information.

\section{Outlook}

Ghost identities complete the second half of the internal behavior spectrum initiated by hybrid identities in Chapter~4.  
Together these two regimes show that collapse fibers support a wide variety of meaningful internal structures.  
Part~III analyzes how coordinate systems interpret these structures and why finite families of projections cannot capture them in full.
