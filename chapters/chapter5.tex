\chapter{Null-Density Generators and Sparse Dynamics}

\section{Introduction}

Chapter~4 introduced hybrid generative identities, whose selectors choose the digit layer with positive asymptotic density.  
Hybrid mechanisms form the dense regime of the generative space, where the information encoding classical magnitude is distributed across a nontrivial fraction of the timeline.

This chapter develops the complementary regime of \emph{null-density generators}.  
A null-density generator collapses to a classical real number using a selector that chooses the digit layer infinitely often but with asymptotic digit density equal to zero.  
In this regime the meta layer dominates the canonical output, yet the sparse digit positions still encode a precise classical magnitude.

Null-density behavior illustrates the extreme flexibility of the collapse map.  
It shows that the information content of a real number can be embedded in a set of positions of vanishing density, leaving most of the mechanism unconstrained and available for additional meta-information.  
This chapter formalizes null-density generators, proves their existence in both full and effective fibers, and connects their structure to sparse subshifts in symbolic dynamics.

\section{Digit Sparsity and Null-Density Structure}

To guarantee that a null-density generator still represents a real number, we require infinitely many digit selections, even though their density tends to zero.

\begin{definition}[Null-Density Generator]
A generative identity $G = (M,D,K)$ is a null-density generator if
\begin{itemize}
    \item the selector $M$ chooses the digit layer infinitely often, and
    \item the digit density satisfies
    \[
    \eta(G)
    =
    \liminf_{n \to \infty}
    \frac{1}{n}
    \bigl|\{\, 0 \le k < n : M(k)=D \,\}\bigr|
    =
    0.
    \]
\end{itemize}
\end{definition}

Thus $M$ contains arbitrarily long runs of meta selections punctuated by rare digit selections.

\begin{remark}[Connection to Symbolic Dynamics]
From the viewpoint of symbolic dynamics, the selector sequences of null-density generators lie in subshifts of low combinatorial complexity.  
If the gaps between digit selections grow without bound, the resulting subshift has topological entropy zero.  
This behavior contrasts with hybrid identities, where the selector often belongs to a subshift with positive entropy.
\end{remark}

\section{Existence in Full Fibers}

Despite their sparse structure, null-density generators can represent any classical real number.

\begin{proposition}[Null-Density Generators in Full Fibers]
For every $x \in [0,1]$, the fiber $\mathcal{F}(x)$ contains infinitely many null-density generators.
\end{proposition}

\begin{proof}
Fix a base-$b$ expansion $(x_j)$ of $x$.  
Let $S = \{ j^2 : j \ge 1 \}$ and define the selector $M$ so that $M(n)=D$ precisely when $n \in S$.  
Since the number of squares below $n$ is approximately $\sqrt{n}$, the density of $S$ is
\[
\lim_{n \to \infty} \frac{|S \cap [0,n)|}{n} = 0.
\]
Thus any identity using this selector is a null-density generator.

Define $D(j^2) = x_j$ and assign $D(n)$ arbitrarily for $n \notin S$.  
Choose $K$ arbitrarily.  
Varying $K$ yields infinitely many distinct null-density generators in $\mathcal{F}(x)$.
\end{proof}

This demonstrates that the encoding of magnitude is independent of selector density.

\section{Null-Density Generators in the Effective Core}

The construction above is effective because the sparse selector pattern $j \mapsto j^2$ is computable.

\begin{proposition}
If $x \in \mathbb{R}_c$, then $\mathcal{F}_{\mathrm{eff}}(x)$ contains effective null-density generators.  
If $x \notin \mathbb{R}_c$, then $\mathcal{F}_{\mathrm{eff}}(x)$ is empty.
\end{proposition}

\begin{proof}
Let $(x_j)$ be a computable base-$b$ expansion of $x$.  
Define $M(n)=D$ exactly when $n=j^2$ for some $j$.  
The set of squares is computable, so $M$ is computable and has density zero.

Define $D(j^2) = x_j$ and define $D(n)$ arbitrarily elsewhere.  
Since $j \mapsto j^2$ is computable and strictly increasing, its range is recursive, so $D$ is computable.  
Choose any computable meta sequence $K$.  
The resulting identity $G=(M,D,K)$ is effective and collapses to $x$.
\end{proof}

\section{Sparse Dynamics and Shift Behavior}

Null-density patterns interact predictably with the shift map $\sigma$ introduced in Chapter~3.  
Unlike hybrid identities, whose density properties may depend on long-term stationarity, null-density behavior is shift-invariant.

\begin{proposition}
If $G$ is a null-density generator, then $\sigma(G)$ is also a null-density generator.
\end{proposition}

\begin{proof}
Let $A_n = |\{ 0 \le k < n : M(k)=D \}|$.  
For the shifted identity $\sigma(G)$, the corresponding count is either $A_{n+1}$ or $A_{n+1}-1$.  
Since $A_n/n \to 0$, the same holds for $(A_{n+1})/n$.  
Thus $\eta(\sigma(G)) = 0$.
\end{proof}

This invariance highlights the dynamical stability of sparse selector patterns.

\section{Contrast With Hybrid Generators}

Hybrid and null-density identities form two structural extremes in the generative space.

\begin{itemize}
    \item \textbf{Hybrid identities} have $\eta(G) > 0$.  
    They use the digit layer frequently and distribute magnitude information across many positions.
    \item \textbf{Null-density identities} have $\eta(G)=0$.  
    They use the digit layer rarely, allowing the meta layer to dominate the canonical output.
\end{itemize}

These regimes illustrate the phenomenon of projective incompatibility discussed in Chapter~7.  
A projection that summarizes digit-frequency behavior distinguishes hybrids from null-density generators, whereas a projection that measures gap growth cannot distinguish hybrids effectively.  
This complementarity plays a key role in the diagonalizer construction of Chapter~8, which exploits tail modifications switching between high- and low-density regimes to evade finite collections of projections.

\section{Outlook}

Null-density generators complete the classification of internal selector regimes initiated in Chapter~4.  
Together, hybrid and null-density mechanisms illustrate the range of internal structure that can occur within a single collapse fiber.  
Part~III turns to the problem of measurement: how computable coordinate systems capture fragments of this structure and why no finite family of such systems can classify generative identities.
