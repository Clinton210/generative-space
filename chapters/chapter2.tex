\chapter{Collapse as a Representation of the Real Line}
\label{chap:collapse-map}

\section{Introduction}

Generative identities inhabit an ambient symbolic space with several independent coordinates, each given by an infinite sequence over a finite alphabet. 
These coordinates are neutral and carry no intrinsic semantic meaning.
To assign a classical real value to each generative identity, we fix a representation that exposes a sequence of digits drawn from one of the coordinates. 
This exposed sequence becomes the classical expansion of a real number.
The resulting mapping is the collapse map.

The goal of this chapter is to define the collapse mechanism, establish its continuity, and describe the geometry of its fibers. 
A collapse fiber consists of all identities that produce the same real value under this representation.
These fibers play the central role in the generative framework, and they form the setting for the freedom and incompleteness phenomena explored in later chapters.

The presentation follows the treatment of infinite sequences and representations in computable analysis and symbolic dynamics \cite{LindMarcus, WeihrauchComputableAnalysis, PaulyRepresentedSpaces}. 
It also connects to the finite-information perspective emphasized throughout this monograph.

\section{Representation of Exposed Positions}

Let $\Gamma_{1}, \Gamma_{2}, \Gamma_{3}$ be finite alphabets as in Chapter~\ref{chap:generative-space}. 
To define collapse, we choose one coordinate to serve as the observed-value coordinate and another coordinate to serve as the exposure mechanism.

For each $G = (U_{1}, U_{2}, U_{3}) \in \mathcal{G}$, the exposure mechanism is specified by a sequence
\[
E(G) = (e_{n})_{n=0}^{\infty} \in \{0,1\}^{\mathbb{N}},
\]
where $e_{n} = 1$ indicates that the representation reveals the value of $U_{2}$ at position $n$. 
The positions at which $e_{n} = 1$ determine the index set of exposed digits.

\begin{definition}[Exposed Positions]
For any $G \in \mathcal{G}$, define
\[
P(G)
  = \{\, n \in \mathbb{N} : e_{n} = 1 \,\},
\]
the set of positions where the representation exposes the value of the observed-value coordinate $U_{2}$.
\end{definition}

The collapse representation requires infinitely many exposed positions in order to produce an infinite classical expansion.

\begin{definition}[Exposure Domain]
The exposure domain is the set
\[
\mathcal{G}^{*} = \{\, G \in \mathcal{G} : P(G) \text{ is infinite} \,\}.
\]
\end{definition}

The set $\mathcal{G}^{*}$ is a dense $G_{\delta}$ subset of $\mathcal{G}$. 
It is not compact. 
This distinction plays an important role when studying collapse fibers, which are compact even though $\mathcal{G}^{*}$ is not. 
This compactness property is discussed later in this chapter.

\section{Collapse Coordinate}

For $G = (U_{1}, U_{2}, U_{3}) \in \mathcal{G}^{*}$, list the exposed positions in increasing order,
\[
n_{0} < n_{1} < n_{2} < \cdots.
\]

\begin{definition}[Collapse Coordinate]
The collapse coordinate of $G$ is the sequence
\[
X(G) = \bigl( U_{2}(n_{j}) \bigr)_{j=0}^{\infty},
\]
which records the values of the observed-value coordinate at the exposed positions.
\end{definition}

Only the exposure mechanism and the observed-value coordinate influence the collapse coordinate.
The remaining coordinate carries latent structure that does not affect the classical expansion.
This asymmetry underlies the structural redundancy of collapse fibers and motivates much of the analysis in Parts~II and~III.

\section{Definition of the Collapse Map}

To translate the collapse coordinate into a real number, choose a base $b \geq 2$ and treat $X(G)$ as the base $b$ expansion of a real in the unit interval.

\begin{definition}[Collapse Map]
For each $G \in \mathcal{G}^{*}$,
\[
\pi(G)
  = \sum_{j=0}^{\infty} \frac{X(G)_{j}}{b^{j+1}}.
\]
\end{definition}

When a real number admits two expansions, we avoid the terminating representation to ensure that the collapse map is single valued.

This collapse representation is not intrinsic to the generative space. 
It is a design choice, one example of how symbolic coordinates may be interpreted as classical values. 
Different representations lead to different fibers, but the structural redundancy created by a finite-information representation is unavoidable in any naming system for real numbers.

\section{Continuity}

The collapse map is continuous on the exposure domain. 
Its continuity follows from the finite prefix nature of the representation.

\begin{proposition}
The map $\pi : \mathcal{G}^{*} \to [0,1]$ is continuous.
\end{proposition}

\begin{proof}
Fix $\varepsilon > 0$ and choose $N$ such that $b^{-(N+1)} < \varepsilon$. 
Two collapse values differ by less than $\varepsilon$ whenever the first $N$ exposed digits of the two identities coincide.
Since $P(G)$ is infinite, there exists an integer $L$ such that the first $N$ exposed positions lie below $L$.

If two identities agree on their prefixes of length $L$ in all coordinates, then their first $N$ exposed digits match, which ensures that their collapse values differ by less than $\varepsilon$. 
This proves continuity.
\end{proof}

This argument parallels classical results for maps defined on Cantor space and subshifts in symbolic dynamics \cite{LindMarcus}.

\section{Surjectivity}

Every real number arises as the collapse of many generative identities.

\begin{proposition}
The collapse map $\pi$ is surjective.
\end{proposition}

\begin{proof}
Let $x \in [0,1]$ have a non-terminating base $b$ expansion $(x_{j})_{j \geq 0}$. 
Define $e_{n} = 1$ for all $n$, so that all positions are exposed. 
Set $U_{2}(n) = x_{n}$ for all $n$, and allow $U_{1}$ and $U_{3}$ to vary freely.
Then $G = (U_{1},U_{2},U_{3})$ lies in $\mathcal{G}^{*}$ and satisfies $\pi(G) = x$.
\end{proof}

Because the auxiliary coordinates can vary arbitrarily, each real number has an uncountable collapse fiber.

\section{Effective Surjectivity}

The collapse map interacts well with computability.

\begin{proposition}
A real number is computable if and only if it is the collapse of some identity in $\mathcal{G}_{\mathrm{eff}} \cap \mathcal{G}^{*}$.
\end{proposition}

\begin{proof}
If $x$ is computable, its base $b$ expansion is computable. 
Define $G$ as in the preceding proof, using computable sequences for all coordinates. 
Then $\pi(G)=x$ and $G$ lies in $\mathcal{G}_{\mathrm{eff}}$.

Conversely, if $G \in \mathcal{G}_{\mathrm{eff}} \cap \mathcal{G}^{*}$, then $X(G)$ is computable, and the series defining $\pi(G)$ is a computable function of a computable sequence. 
Thus $\pi(G)$ is computable.
\end{proof}

Hence
\[
\pi\bigl(\mathcal{G}_{\mathrm{eff}} \cap \mathcal{G}^{*}\bigr)
  = \mathbb{R}_{c},
\]
the set of computable reals.

\section{Collapse Fibers}

For any real number $x \in [0,1]$, the collapse fiber
\[
\mathcal{F}(x)
  = \pi^{-1}(\{x\})
\]
consists of all generative identities that expose the same base $b$ digit sequence under the collapse representation.

Although $\mathcal{G}^{*}$ is not compact, each collapse fiber $\mathcal{F}(x)$ is compact. 
This is because the exposed positions determine the collapse coordinate uniquely, and any limit of identities that respect these exposed digits must respect them in the limit.
Thus the fiber inherits compactness even though the domain does not.

Collapse fibers serve as the primary geometric objects in the generative viewpoint. 
They contain arbitrarily rich latent structure, and their internal geometry is studied in Chapters~3 through~7.

\section{Summary}

The collapse map provides a fixed representation of the real line using symbolic coordinates from the generative space. 
It is continuous, surjective, and interacts cleanly with computability. 
Although the exposure domain is not compact, each collapse fiber is compact and contains many identities that share the same classical value.
These fibers form the foundation for the generative freedom and incompleteness phenomena developed in later parts of this monograph.
