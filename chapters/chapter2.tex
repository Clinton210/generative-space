\chapter{The Collapse Map}

\section{Introduction}

A generative identity contains far more symbolic structure than is visible in
its classical magnitude.  
The collapse map extracts a real number from a generative identity by reading
only the digits exposed by the selector stream.  
This operation forgets almost all of the internal generative behavior,
producing a single value in $[0,1]$ while leaving behind a large fiber of
distinct identities sharing the same classical output.

This chapter defines the collapse map, establishes its continuity, and shows
that every real number---computable or otherwise---arises as the collapse of
many different generative identities.

\section{Digit Selection}

Let $G = (M, D, K) \in \mathcal{X}^*$ be a digit-selecting generative identity.
Recall that the canonical output sequence is defined by enumerating the digits
appearing at positions where $M(n) = D$.

Let
\[
n_0 < n_1 < n_2 < \cdots
\]
be the increasing sequence of indices with $M(n_j) = D$, and define
\[
d_G(j) = D(n_j).
\]

The sequence $(d_G(j))_{j \ge 0}$ is an infinite sequence in
$\{0,1,\ldots,b-1\}^{\mathbb{N}}$ and will serve as the base-$b$ expansion of
the collapsed value.

\section{Definition of the Collapse Map}

For every $G \in \mathcal{X}^*$, define the \emph{collapse map}
\[
\pi(G)
  = \sum_{j=0}^{\infty} \frac{d_G(j)}{b^{j+1}}.
\]

When a real number has two base-$b$ expansions (a terminating expansion and a
repeating one), we adopt the standard convention of using the non-terminating
representation with trailing $(b-1)$s avoided.  
This ensures the collapse map is well defined.

The collapse map is the primary projection from the generative space to the
unit interval.  
It depends only on the canonical output and therefore only on the portions of
the digit stream selected by $M$.

\section{Continuity of Collapse}

The topology on $\mathcal{X}^*$ makes $\pi$ a continuous function onto
$[0,1]$.  
Given $\varepsilon > 0$, choosing $N$ large enough so that
$b^{-(N+1)} < \varepsilon$ shows that the first $N$ selected digits determine
$\pi(G)$ to within $\varepsilon$.

Since selected digits appear infinitely often, the first $N$ of them arise
within some initial prefix of $G$.  
Thus, for every $\varepsilon > 0$, there exists an integer $L$ such that any
two identities agreeing on their first $L$ symbols in each stream have
collapsed values within $\varepsilon$.

Therefore \( \pi : \mathcal{X}^* \to [0,1] \) is continuous.

\section{Surjectivity}

Every real number in $[0,1]$ arises as the collapse of many generative
identities.  
Fix any real number $x$ with base-$b$ expansion
\[
x = \sum_{j=0}^{\infty} \frac{x_j}{b^{j+1}}.
\]

Choose a selector $M$ that always selects digits:
\[
M(n) = D \quad \text{for all } n.
\]
Define the digit stream $D$ by $D(n) = x_n$ for all $n$, and let $K$ be any
meta-information sequence.

Then $G = (M, D, K) \in \mathcal{X}^*$ satisfies $\pi(G) = x$.  
Varying $K$ freely shows that the fiber $\pi^{-1}(\{x\})$ is uncountable.

\section{Effective Surjectivity}

The collapse map behaves correctly on the effective core.  
A real number $x \in [0,1]$ is computable if and only if it has a computable
base-$b$ expansion.  
Given such an expansion, the construction above produces a computable
generative identity $G \in \mathcal{G}_{\mathrm{eff}}$ satisfying
$\pi(G) = x$.

Conversely, if $G \in \mathcal{G}_{\mathrm{eff}} \cap \mathcal{X}^*$, then the
canonical output sequence $d_G(j)$ is computable, and so $\pi(G)$ is a
computable real.

Thus
\[
\pi(\mathcal{G}_{\mathrm{eff}} \cap \mathcal{X}^*)
  = \mathbb{R}_c,
\]
the set of computable reals.

\section{Fibers and Structural Redundancy}

The collapse map is many-to-one.  
For any $x \in [0,1]$, the fiber
\[
\mathcal{F}(x) = \pi^{-1}(\{x\})
\]
contains identities that may share no structural similarity beyond producing
the same output digits.

Two identities may:
\begin{itemize}
    \item select digits at completely different positions,
    \item carry unrelated meta-information streams,
    \item differ arbitrarily on unselected digits,
\end{itemize}
while still collapsing to the same real $x$.

This structural redundancy is essential for the development of the projection
theory and the incompleteness results of later parts.

\section{Summary}

The collapse map converts the symbolic structure of a generative identity into
a classical real number by selecting and aggregating digits according to the
selector stream.  
It is continuous, surjective, effectively surjective on computable identities,
and massively non-injective.  
The fibers of $\pi$ form the central objects of study in the Generative
Identity Framework.

The next chapter analyzes the internal geometry of these fibers and the
degrees of freedom that remain invisible after collapse.
