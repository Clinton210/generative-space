\chapter{Collapse as the Primary Invariant}

\section{Introduction}

Chapter~1 introduced the generative space $\mathcal{X}$ as a space of layered symbolic mechanisms consisting of a mixer (or selector sequence), a digit sequence, and a meta sequence.  
The structure of $\mathcal{X}$ is independent of the classical real line; it is a product space equipped with the standard cylinder topology, with an effective subspace $\mathcal{G}_{\mathrm{eff}}$ defined through Type--2 computability.

In this chapter we introduce the \emph{collapse map}, denoted by $\pi$.  
The collapse map acts as the primary invariant of the framework.  
It interprets a generative identity $G = (M, D, K)$ by extracting the digits selected by the mixer and mapping them to a classical magnitude via a base-$b$ expansion.

Collapse therefore provides the canonical projection that links the generative space with the real line.  
We show that $\pi$ maps the full generative space $\mathcal{X}$ onto the entire unit interval $[0,1]$ and maps the effective core $\mathcal{G}_{\mathrm{eff}}$ onto the computable real numbers $\mathbb{R}_c$.

\section{The Collapse Map}

The collapse map extracts the subsequence of digits chosen by the mixer $M$ and interprets those digits as a base-$b$ expansion.  
We begin by formalizing the selected subsequence.

\begin{definition}[Selected Subsequence]
Let $G = (M, D, K) \in \mathcal{X}$.  
Let
\[
S_M = \{ n_0 < n_1 < n_2 < \cdots \}
\]
be the (possibly finite) set of indices where $M(n) = D$.  
If $S_M$ is infinite, the selected digit subsequence $d_G$ is defined by
\[
d_G(j) = D(n_j) \quad \text{for } j \ge 0.
\]
If $S_M$ is finite, $d_G$ is a finite sequence.
\end{definition}

To define a real-valued collapse, we restrict attention to identities where the mixer selects digits infinitely often.

\begin{definition}[Digit-Selecting Generators]
Define
\[
\mathcal{X}^* = \{ G \in \mathcal{X} : S_M \text{ is infinite} \}.
\]
Elements of $\mathcal{X}^*$ select infinitely many digits and therefore admit a base-$b$ interpretation.
\end{definition}

\begin{definition}[Collapse Map]
For $G \in \mathcal{X}^*$, the \emph{collapse map} $\pi : \mathcal{X}^* \to [0,1]$ is defined by
\[
\pi(G)
=
\sum_{j=0}^{\infty} \frac{d_G(j)}{b^{j+1}},
\]
where $d_G$ is the selected digit subsequence of $G$.
\end{definition}

The collapse map discards:
\begin{itemize}
    \item all meta symbols,
    \item all digit symbols at positions where $M$ selects the meta layer,
    \item the higher-level structure encoded in the mixer $M$ itself.
\end{itemize}

\begin{remark}
The collapse map is a canonical projection from a high-dimensional symbolic object to a classical numerical value.  
The term ``collapse’’ emphasizes the information loss: the meta layer and most of the digit positions play no role in the magnitude extracted by $\pi$.
\end{remark}

\section{Surjectivity and Representation}

The generative viewpoint is only coherent if collapse reaches the entire continuum and represents all computable real numbers in the effective setting.

\begin{theorem}[Surjectivity onto the Continuum]
\label{thm:surjectivity}
The collapse map $\pi$ is surjective from $\mathcal{X}^*$ onto $[0,1]$.
\end{theorem}

\begin{proof}
Let $x \in [0,1]$ with base-$b$ expansion $(x_j)_{j \ge 0}$.  
Define $G = (M,D,K)$ by setting $M(n)=D$ for all $n$, $D(n)=x_n$, and $K$ arbitrary.  
Then $S_M = \mathbb{N}$ and $d_G(j)=D(j)=x_j$, so $\pi(G)=x$.
\end{proof}

We next show that the effective subspace maps precisely onto the computable reals.

\begin{theorem}[Effective Surjectivity]
\label{thm:effective-surjectivity}
The collapse map restricts to a surjection
\[
\pi(\mathcal{G}_{\mathrm{eff}} \cap \mathcal{X}^*) = \mathbb{R}_c.
\]
\end{theorem}

\begin{proof}
If $G \in \mathcal{G}_{\mathrm{eff}}$, then both $M$ and $D$ are computable.  
We can compute $n_j$, extract $D(n_j)$, and compute the resulting base-$b$ expansion.  
Thus $\pi(G)$ is computable.

Conversely, if $x \in \mathbb{R}_c$, let $(x_j)$ be a computable expansion.  
Define $G=(M,D,K)$ by $M(n)=D$ and $D(n)=x_n$.  
Then $G$ is effective and $\pi(G)=x$.
\end{proof}

\section{Collapse Equivalence and Fibers}

Collapse is far from injective: many internally different identities produce the same classical value.

\begin{definition}[Collapse Equivalence]
For $G,H \in \mathcal{X}^*$, write
\[
G \sim_\pi H
\quad\Longleftrightarrow\quad
\pi(G)=\pi(H).
\]
\end{definition}

\begin{definition}[Collapse Fiber]
For $x \in [0,1]$, the \emph{collapse fiber} at $x$ is
\[
\mathcal{F}(x)
=
\{ G \in \mathcal{X}^* : \pi(G)=x \}.
\]
\end{definition}

\begin{remark}
Each fiber $\mathcal{F}(x)$ is closed in the product topology.  
The effective fiber
\[
\mathcal{F}_{\mathrm{eff}}(x)
=
\mathcal{F}(x) \cap \mathcal{G}_{\mathrm{eff}}
\]
is a $\Pi^0_1$ class, reflecting the descriptive complexity that later enables diagonalization arguments.
\end{remark}

Fibers contain hybrid generators, null-density generators, periodic selectors, and many other internal mechanisms that nevertheless collapse to the same magnitude.

\section{Outlook}

Chapter~3 analyzes collapse fibers in detail, showing that each fiber contains a diverse range of mechanisms and substantial internal structure.  
These results form the foundation for the hybrid constructions of Part~II and for the impossibility results of structural incompleteness in Part~IV.
