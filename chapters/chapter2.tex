\chapter{The Collapse Map}
\label{chap:collapse-map}

\section{Introduction}

A generative identity contains layers of symbolic structure that extend far
beyond its classical magnitude.  
The collapse map extracts a real number by reading only those digits revealed
by the selector stream.  
This operation discards most of the internal structure and identifies large
families of generative identities whose collapse coordinates agree.  
These families are the collapse fibers, and they form the central geometric
objects of collapse theory.

This chapter defines the collapse map, establishes its continuity, and
describes the structure of its fibers.  
The presentation relies on the topological properties of the generative space
introduced in Chapter~\ref{chap:generative-space} and parallels standard ideas
in symbolic dynamics \cite{LindMarcus} and computable analysis
\cite{WeihrauchComputableAnalysis, PaulyRepresentedSpaces}.

\section{Digit Extraction}

Let $G = (M,D,K) \in \mathcal{G}^*$ be a digit-producing generative identity.
List the indices at which the selector exposes a digit:
\[
n_0 < n_1 < n_2 < \cdots,
\qquad M(n_j) = D.
\]

\begin{definition}[Collapse Coordinate]
The collapse coordinate of $G$ is the sequence
\[
X(G) = \bigl(D(n_j)\bigr)_{j=0}^\infty
    \in \{0,\ldots,b-1\}^{\mathbb{N}}.
\]
\end{definition}

Only the selector and digit layers influence $X(G)$.  
The meta-information layer does not contribute to the classical value, a fact
that later underlies the richness of collapse fibers.

\section{Definition of the Collapse Map}

The collapse map translates the collapse coordinate into a real number.

\begin{definition}[Collapse Map]
For each $G \in \mathcal{G}^*$,
\[
\pi(G)
  = \sum_{j=0}^{\infty} \frac{X(G)_j}{b^{j+1}}.
\]
\end{definition}

When a real number has two base-$b$ expansions, we adopt the usual convention
of avoiding the terminating representation with trailing $(b-1)$s.  
This ensures that $\pi$ is single-valued.

\section{Continuity}

The collapse map is continuous with respect to the product topology on
$\mathcal{G}^*$.

\begin{proposition}
The map $\pi : \mathcal{G}^* \to [0,1]$ is continuous.
\end{proposition}

\begin{proof}
Fix $\varepsilon > 0$ and choose $N$ so that $b^{-(N+1)} < \varepsilon$.  
Two identities produce collapse values within $\varepsilon$ whenever their
first $N$ selected digits agree.  
Since $G \in \mathcal{G}^*$ selects digits infinitely often, there exists
some $L$ such that the first $N$ selected digits lie within the first $L$
entries of the three streams.

If two identities agree on their prefixes of length $L$ in each layer, then
their first $N$ selected digits coincide.  
Therefore their collapse values differ by less than $\varepsilon$, proving
continuity.
\end{proof}

The proof uses the fact that basic open sets of the product topology are
determined by finite prefixes, a feature common to subshifts of finite type
and other Cantor-like symbolic systems \cite{LindMarcus}.

\section{Surjectivity}

Every real number admits many generative descriptions.

\begin{proposition}
The collapse map $\pi$ is surjective.
\end{proposition}

\begin{proof}
Let $x \in [0,1]$ have base-$b$ expansion $(x_j)_{j\geq 0}$ in the
non-terminating form.  
Define the selector $M$ by $M(n)=D$ for all $n$.  
Set $D(n)=x_n$ for all $n$ and let $K$ be any sequence in $\Sigma^{\mathbb{N}}$.
Then $G=(M,D,K)$ lies in $\mathcal{G}^*$ and satisfies $\pi(G)=x$.
\end{proof}

By varying the meta-information layer and the unused digit positions
arbitrarily, we see that each point of $[0,1]$ has an uncountable collapse
fiber.

\section{Effective Surjectivity}

Collapse behaves correctly on the effective core.

\begin{proposition}
A real number is computable if and only if it is the collapse of some
identity in $\mathcal{G}_{\mathrm{eff}} \cap \mathcal{G}^*$.
\end{proposition}

\begin{proof}
If $x$ is computable, then its base-$b$ expansion is computable.  
Using the construction above with computable streams yields
$G \in \mathcal{G}_{\mathrm{eff}}$ satisfying $\pi(G)=x$.

Conversely, if $G \in \mathcal{G}_{\mathrm{eff}} \cap \mathcal{G}^*$, then
the collapse coordinate $X(G)$ is computable, so the series defining $\pi(G)$
computes a real number.  
Hence $\pi(G)$ is computable.
\end{proof}

Thus
\[
\pi(\mathcal{G}_{\mathrm{eff}} \cap \mathcal{G}^*)
  = \mathbb{R}_c,
\]
the set of computable real numbers.

\section{Collapse Fibers}

For $x \in [0,1]$, the collapse fiber
\[
\mathcal{F}(x)
  = \pi^{-1}(\{x\})
\]
contains all identities with collapse coordinate equal to the base-$b$
expansion of $x$.

These identities may differ arbitrarily on:
\begin{itemize}
    \item the pattern of selector positions,
    \item unselected digits,
    \item the meta-information layer.
\end{itemize}

Only the selected digits influence the classical value.  
This structural redundancy forms the foundation for the projection theory of
Part~\ref{part:selector-geometry} and the incompleteness results of
Part~\ref{part:incompleteness}.

\section{Summary}

The collapse map projects generative identities to real numbers by reading the
collapse coordinate determined by the selector and digit layers.  
It is continuous, surjective, and effectively surjective onto the computable
reals.  
Its fibers are large symbolic families that share the same classical
magnitude.  
These fibers are the central geometric objects of collapse theory and will be
analyzed in detail in the next chapter.
