\chapter{Collapse as the Primary Invariant}

\section{Introduction}

Chapter~1 introduced the generative space
\[
\mathcal{X} = \{D,K\}^{\mathbb{N}} \times \{0,1,\ldots,b-1\}^{\mathbb{N}} \times \Sigma^{\mathbb{N}},
\]
and its effective core $\mathcal{G}_{\mathrm{eff}}$, consisting of computable
generative identities.  Each identity $G = (M,D,K)$ consists of a selector
$M$, a digit layer $D$, and a meta layer $K$.  
These mechanisms exist independently of the classical real line: no
numerical value is associated to an identity until a prescribed
interpretation is applied.

This chapter introduces the central interpretation map of the framework:
the \emph{collapse map}.  
Collapse extracts the digit symbols chosen by the selector and interprets
them as a base-$b$ expansion.  
The result is a classical magnitude in $[0,1]$.  
In this sense, collapse is the primary invariant of the generative space:
it is the unique invariant studied in classical analysis, and all other
secondary invariants are subordinate to it.

Two principal facts are established:

\begin{enumerate}
    \item Collapse is \emph{surjective}: every real number in $[0,1]$ is the
    collapsed value of some generative identity.
    \item On the effective core, collapse is \emph{computably surjective}:
    every computable real number arises from an effective identity.
\end{enumerate}

Collapse thus serves as the bridge between the generative
world and the classical continuum.  
Its fibers, studied in Chapter~3, encode all generative mechanisms that
produce a given classical value.

\section{Digit Selection and the Collapsible Subspace}

The collapse map interprets only those identities whose selector chooses
the digit layer infinitely often.  Otherwise, the extracted digit
subsequence would be finite and would not encode a real number.

\begin{definition}[Selected Digit Indices]
For $G = (M,D,K) \in \mathcal{X}$, let
\[
S_M = \{ n_0 < n_1 < n_2 < \cdots \}
\]
be the (possibly finite) increasing sequence of indices such that
$M(n_j) = D$.  
\end{definition}

\begin{definition}[Selected Digit Subsequence]
If $S_M$ is infinite, the \emph{selected digit subsequence} of $G$ is
\[
d_G(j) = D(n_j) \quad (j \ge 0).
\]
If $S_M$ is finite, the selected digit subsequence is finite.
\end{definition}

\begin{definition}[Digit-Selecting Identities]
The \emph{digit-selecting subspace} is
\[
\mathcal{X}^* = \{\, G \in \mathcal{X} : S_M \text{ is infinite} \,\}.
\]
\end{definition}

Identities in $\mathcal{X}^*$ select digit symbols infinitely many times
and therefore admit a classical interpretation.  
Those outside $\mathcal{X}^*$ may still possess rich structure, but they
do not define classical magnitudes.

\section{Definition of the Collapse Map}

Collapse discards all meta symbols and all unselected digit symbols,
retaining only the infinite subsequence $d_G$.  
This subsequence is interpreted as a base-$b$ expansion.

\begin{definition}[Collapse Map]
For $G \in \mathcal{X}^*$, the \emph{collapse map}
$\pi : \mathcal{X}^* \to [0,1]$ is defined by
\[
\pi(G)
=
\sum_{j=0}^{\infty} \frac{d_G(j)}{b^{j+1}}.
\]
\end{definition}

Thus, collapse maps the high-dimensional mechanism $G$ to the real number
whose digits are prescribed by the selected subsequence of $D$.

\begin{remark}
Collapse is a canonical projection: it forgets nearly all of the
mechanism.  The selector $M$ determines \emph{which}
digits are exposed to classical interpretation, but $M$ itself does not
affect the real number obtained.  
The meta layer $K$ plays no role at all in determining $\pi(G)$.
\end{remark}

\section{Surjectivity and Representation of the Continuum}

To justify the generative framework, collapse must reach the entire
continuum.  
This turns out to be immediate.

\begin{theorem}[Surjectivity]
\label{thm:collapse-surjective}
Collapse maps $\mathcal{X}^*$ onto the full unit interval:
\[
\pi(\mathcal{X}^*) = [0,1].
\]
\end{theorem}

\begin{proof}
Given any $x \in [0,1]$ with base-$b$ expansion $(x_j)$, define
\[
M(n) = D, \qquad D(n) = x_n, \qquad K \text{ arbitrary}.
\]
Then $S_M = \mathbb{N}$ and $d_G(j) = x_j$ for all $j$.  
Thus $\pi(G) = x$.
\end{proof}

The effective situation is equally straightforward.

\begin{theorem}[Effective Surjectivity]
\label{thm:effective-surjectivity}
Collapse maps the effective digit-selecting core onto the computable
real numbers:
\[
\pi(\mathcal{G}_{\mathrm{eff}} \cap \mathcal{X}^*) = \mathbb{R}_c.
\]
\end{theorem}

\begin{proof}
If $G$ is effective, then $M$ and $D$ are computable.
Thus the selected digit sequence $d_G$ is computable, and so $\pi(G)$ is
a computable real.

Conversely, if $x \in \mathbb{R}_c$, let $(x_j)$ be a computable
base-$b$ expansion.  
Define $M(n)=D$ and $D(n)=x_n$, with arbitrary $K$.  
Then $G$ is effective and $\pi(G)=x$.
\end{proof}

Collapse thus provides the exact computable representation map familiar
from Type--2 computability, via the richer structure of layered
generative mechanisms.

\section{Collapse Equivalence and Fibers}

Collapse is highly non-injective: many distinct generative identities
collapse to the same classical magnitude.  
The structure of these equivalence classes is central to the rest of the
monograph.

\begin{definition}[Collapse Equivalence]
For $G,H \in \mathcal{X}^*$,
\[
G \sim_\pi H
\quad\Longleftrightarrow\quad
\pi(G) = \pi(H).
\]
\end{definition}

\begin{definition}[Collapse Fiber]
For $x \in [0,1]$, the \emph{collapse fiber} at $x$ is
\[
\mathcal{F}(x)
=
\{\, G \in \mathcal{X}^* : \pi(G)=x \,\}.
\]
\end{definition}

\begin{remark}
Each fiber $\mathcal{F}(x)$ is closed in the product topology of
$\mathcal{X}$.  
If $x$ is computable, the effective fiber
\[
\mathcal{F}_{\mathrm{eff}}(x)
    = \mathcal{F}(x) \cap \mathcal{G}_{\mathrm{eff}}
\]
is a $\Pi^0_1$ class.  
This descriptive complexity plays a key role in the meta-diagonalizer of
Part~IV.
\end{remark}

Fibers contain identities with radically different internal structure:
high-density hybrids, sparse null-density generators, periodic selectors,
pseudo-random selectors, and intricate meta-layer patterns.  
All such behaviors are compatible with the same classical magnitude
because collapse observes only the selected digits.

\section{Outlook}

The collapse map provides the primary invariant of the generative
framework and establishes the connection between mechanisms and classical
values.  
Its surjectivity guarantees that every real number has generative
representations.  
Its non-injectivity gives rise to the rich internal geometry of collapse
fibers, which is the subject of Chapter~3.

These fibers form the foundational environment for the internal selector
regimes of Part~II, the projection-based analysis of structure
developed in Part~III, and ultimately the diagonalization and
incompleteness results of Part~IV.
