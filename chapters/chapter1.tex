\chapter{The Generative Space}

\section{Introduction}

The Generative Identity Framework begins with an ambient symbolic space in
which each real number is realized as the collapse of a richer internal
structure.  
A generative identity is not a single sequence but a triple of coordinated
layers.  
Only one of these layers contributes directly to the classical real number
obtained after collapse.  
The remaining layers encode additional structure that becomes invisible to
classical observers.

This chapter introduces the generative space, establishes its topological
properties, and identifies its effective core.  
These foundations form the geometric setting for the collapse map developed
in Chapter~\ref{chap:collapse-map}, where classical magnitude is extracted
from internal symbolic data.

Throughout, we fix a base $b \geq 2$ for numeral expansions and a finite
meta-alphabet $\Sigma$.

\section{Definition of the Generative Space}

A generative identity is a triple
\[
G = (M, D, K),
\]
where:
\begin{itemize}
    \item $M \in \{D,K\}^{\mathbb{N}}$ is the \emph{selector stream};
    \item $D \in \{0,1,\ldots,b-1\}^{\mathbb{N}}$ is the \emph{digit stream};
    \item $K \in \Sigma^{\mathbb{N}}$ is the \emph{meta-information stream}.
\end{itemize}

Each layer is equipped with the discrete topology on its alphabet, and the
ambient generative space is the product
\[
\mathcal{G}
  = \{D,K\}^{\mathbb{N}}
    \times \{0,1,\ldots,b-1\}^{\mathbb{N}}
    \times \Sigma^{\mathbb{N}}
\]
with the product (Cantor) topology.  
Basic open sets are determined by finite prefixes of the three layers.
This aligns with the usual treatment of represented spaces and infinite
symbolic products in computable analysis  
\cite{WeihrauchComputableAnalysis, PaulyRepresentedSpaces}.

The topology reflects the principle that an observer may inspect only
finitely many symbols from each coordinate.

\section{Digit-Producing Identities}

Collapse will be defined through the positions at which $M$ selects digits.
For this mechanism to produce an infinite classical expansion, the selector
layer must designate digit positions infinitely often.

\begin{definition}[Digit-Producing Identity]
A generative identity $G$ is \emph{digit-producing} if
\[
\{\, n : M(n) = D \,\}
\]
is infinite.
\end{definition}

The set of digit-producing identities is the subspace
\[
\mathcal{G}^*
  = \{\, G \in \mathcal{G} : M(n) = D \text{ for infinitely many } n \,\}.
\]

It is closed under finite modification of any layer, and it is dense in each
cylinder defined by the meta and digit streams.  
Every element of $\mathcal{G}^*$ generates an infinite sequence of classical
digits and therefore a well-defined real number once the collapse map is
introduced.

\section{The Collapse Coordinate}

Let $G = (M, D, K) \in \mathcal{G}^*$.  
List the positions at which the selector exposes a digit:
\[
n_0 < n_1 < n_2 < \cdots,
\qquad M(n_j) = D.
\]

\begin{definition}[Collapse Coordinate]
The collapse coordinate of $G$ is the infinite sequence
\[
X(G) = \bigl(D(n_j)\bigr)_{j=0}^\infty \in \{0,\ldots,b-1\}^{\mathbb{N}}.
\]
\end{definition}

Only the selector and digit layers influence the collapse coordinate.  
The meta-information layer plays no direct role in determining the classical
value.  
In later chapters, this asymmetry between the visible and invisible layers
drives the phenomena of structural incompleteness.

\section{The Effective Core}

Computable analysis views infinite symbolic objects as functions
$\mathbb{N} \to \Gamma$ for a finite alphabet $\Gamma$.  
Following standard conventions  
\cite{WeihrauchComputableAnalysis, PaulyRepresentedSpaces}, each layer of a
generative identity may be treated as a point of Baire or Cantor space.

\begin{definition}[Effective Core]
The \emph{effective core} of the generative space is
\[
\mathcal{G}_{\mathrm{eff}}
  = \{\, G = (M,D,K) \in \mathcal{G} :
        M, D, K \text{ are computable sequences} \,\}.
\]
\end{definition}

The effective core is countable and forms the computational counterpart to
the uncountable ambient space $\mathcal{G}$.  
It plays a central role in selector geometry and in the diagonalization
arguments developed in Part~\ref{part:incompleteness}.

\section{Examples}

The following examples illustrate how different internal structures give rise
to the same or different collapse coordinates.  
They also demonstrate that the meta-information layer does not influence
classical magnitude.

\subsection{Uniform Alternation}

Let
\[
M = D,K,D,K,D,K,\ldots,
\]
and let $D$ be the base-$b$ expansion of a real number $x$, repeated
infinitely often.  
The meta-layer $K$ is unrestricted.

Then $X(G)$ is obtained by reading every other digit from $D$.  
Many distinct choices of the meta-layer and the unused digit positions lead
to generative identities with the same collapse coordinate.

\subsection{Sparse Digit Selection}

Let $M(n)=D$ when $n$ is a perfect square and $M(n)=K$ otherwise.  
The set of digit positions has asymptotic density zero, yet it is infinite.
Thus $G$ lies in $\mathcal{G}^*$ and produces the classical digit sequence
formed by sampling $D$ along the perfect squares.  
The meta-layer again contributes no information to the value of $X(G)$.

\section{Summary}

The generative space $\mathcal{G}$ is a Cantor-like symbolic product capable
of encoding deep internal structure.  
Its digit-producing subspace $\mathcal{G}^*$ supports a well-defined collapse
coordinate for each identity.  
The effective core $\mathcal{G}_{\mathrm{eff}}$ identifies those structures
that are computably realizable and forms the basis for the computational
arguments developed later.

In Chapter~\ref{chap:collapse-map} we introduce the collapse map itself and
describe how classical real numbers arise as projections of generative
identities.
