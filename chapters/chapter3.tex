\chapter{Fiber Geometry of the Collapse Map}

\section{Introduction}

The collapse map introduced in Chapter~2 assigns a classical magnitude to a generative identity by extracting and decoding the digit subsequence selected by the mixer.  
Since the collapse map ignores both the meta layer and the unselected digit positions, many distinct generative identities collapse to the same real number.  
Understanding the structure of these sets of identities is central to the generative viewpoint.

This chapter studies the geometry of the collapse fibers.  
In the full generative space $\mathcal{X}$, each fiber is uncountable and forms a closed subset of the product topology.  
In the effective core $\mathcal{G}_{\mathrm{eff}}$, each fiber over a computable real is a $\Pi^0_1$ class, that is, an effectively closed set.  
These distinctions provide the structural foundation for the hybrid identities of Chapter~4 and the null-density generators of Chapter~5.

\section{Full Fibers in the Generative Space}

\begin{definition}[Full Fiber]
For $x \in [0,1]$, the \emph{full fiber} of $x$ is
\[
\mathcal{F}(x)
=
\{ G \in \mathcal{X}^* : \pi(G) = x \}.
\]
\end{definition}

A mechanism lies in $\mathcal{F}(x)$ if and only if the digit values selected by its mixer produce a base-$b$ expansion of $x$.  
The meta symbols play no role in determining $\pi(G)$, and digit symbols at unselected positions also have no influence.

\begin{proposition}[Fibers are Closed Sets]
For every $x \in [0,1]$, the fiber $\mathcal{F}(x)$ is closed in the product topology of $\mathcal{X}$.
\end{proposition}

\begin{proof}
The collapse map $\pi$ is continuous with respect to the product topology on $\mathcal{X}$ and the usual topology on $[0,1]$.  
Since $\{x\}$ is closed in a Hausdorff space, its preimage $\pi^{-1}(\{x\}) = \mathcal{F}(x)$ is closed.
\end{proof}

The closedness of fibers allows us to describe their internal structure.

\begin{proposition}[Product Structure of Fibers]
Let $x \in [0,1]$ and fix a base-$b$ expansion $(x_j)_{j \ge 0}$.  
For any $G = (M,D,K)$ in $\mathcal{X}^*$, let $\varphi_G(j)$ enumerate the positions where $M$ selects the digit layer.  
Then $G \in \mathcal{F}(x)$ if and only if
\[
D(\varphi_G(j)) = x_j \quad \text{for all } j \ge 0.
\]
All other coordinates of $D$ and all coordinates of $K$ are unconstrained.
\end{proposition}

\begin{proof}
The condition $D(\varphi_G(j)) = x_j$ ensures that the selected digit subsequence matches the expansion of $x$.  
No other coordinates affect the value of $\pi(G)$.
\end{proof}

Thus the fiber contains a full product of independently variable coordinates.

\begin{corollary}
For every $x \in [0,1]$, the fiber $\mathcal{F}(x)$ is uncountable.
\end{corollary}

\section{Effective Fibers and \texorpdfstring{$\Pi^0_1$}{Pi-0-1} Classes}

The structure of the fiber changes when restricted to the effective core.  
This connects collapse fibers to computable analysis and descriptive set theory.

\begin{definition}[Effective Fiber]
If $x \in \mathbb{R}_c$, the \emph{effective fiber} of $x$ is
\[
\mathcal{F}_{\mathrm{eff}}(x)
=
\mathcal{F}(x) \cap \mathcal{G}_{\mathrm{eff}}.
\]
If $x$ is not computable, this set is empty.
\end{definition}

\begin{remark}[Effective Fibers as $\Pi^0_1$ Classes]
The collapse map $\pi$ is computable in the Type--2 sense.  
Therefore the condition $\pi(G)=x$ defines a $\Pi^0_1$ set: membership can be disproved by finding a finite prefix that violates the required digit constraints, while confirmation requires agreement on all finite prefixes.  
Thus $\mathcal{F}_{\mathrm{eff}}(x)$ is an effectively closed subset of $\mathcal{X}$.
\end{remark}

\begin{proposition}
If $x \in \mathbb{R}_c$, then $\mathcal{F}_{\mathrm{eff}}(x)$ is infinite.
\end{proposition}

\begin{proof}
If $x$ is computable, its digit expansion is computable.  
Any computable mixer that selects digits infinitely often, together with any computable meta sequence, yields an effective identity in $\mathcal{F}(x)$.  
There are infinitely many such choices.
\end{proof}

Thus the effective fiber is a nontrivial $\Pi^0_1$ class whenever $x$ is computable.

\section{Internal Degrees of Freedom}

Every fiber $\mathcal{F}(x)$ contains three main sources of internal variation:

\begin{enumerate}
    \item positions where the mixer selects the meta layer,
    \item digit values at positions not selected by the mixer,
    \item the entire meta sequence $K$.
\end{enumerate}

These degrees of freedom produce large internal differences among identities collapsing to the same value.

\begin{proposition}
If the mixer selects digits at infinitely many positions, then for any $x \in [0,1]$ the fiber $\mathcal{F}(x)$ contains infinitely many identities that disagree on an infinite set of coordinates.
\end{proposition}

\begin{proof}
Varying infinitely many meta-layer coordinates or digit coordinates outside the selected positions yields infinitely many distinct identities with the same collapse value.
\end{proof}

This diversity anticipates the phenomena studied in hybrid identities (Chapter~4) and null-density generators (Chapter~5).

\section{Shift Dynamics and Fiber Structure}

The generative space $\mathcal{X}$ carries a natural shift map
\[
\sigma(M,D,K)(n)
=
(M(n+1), D(n+1), K(n+1)).
\]

\begin{proposition}
If $G \in \mathcal{F}(x)$, the shifted identity $\sigma(G)$ need not belong to $\mathcal{F}(x)$.  
However, the shift map is continuous on $\mathcal{X}$ and preserves the product topology.
\end{proposition}

The shift action highlights that collapse fibers collect identities that share a numerical value but may have very different dynamical signatures.  
Later chapters use this viewpoint to study hybrid and null-density behavior, which produce distinct asymptotic patterns under the shift.

\section{Outlook}

This chapter establishes the topological and combinatorial structure of collapse fibers.  
Chapter~4 develops hybrid identities, which exhibit positive digit-selection density and form a dense subset of many effective fibers.  
Chapter~5 investigates null-density generators, where the digit density is zero and the meta layer dominates.  
Together these chapters reveal the wide range of internal behaviors that collapse to a single classical magnitude.
