\chapter{Collapse Fibers and Ambient Compactness}
\label{chap:fibers}

\section{The Ambient Generative Space}

Let $\Sigma$ denote the finite alphabet used to encode the mixer, digit, and
meta streams of a generative identity. The full generative space is the
product
\[
\mathcal{X} = \Sigma^{\mathbb{N}}
\]
equipped with the product topology induced by the discrete topology on
$\Sigma$. By Tychonoff's theorem, $\mathcal{X}$ is compact and metrizable. A
convenient metric is
\[
d(G,G') = 2^{-N},
\]
where $N$ is the least index at which $G$ and $G'$ differ. This metric
generates the product topology.

We will often work with the subspace
\[
\mathcal{X}^{*}
=
\bigl\{
G \in \mathcal{X} :
M_G(n) = D \text{ for infinitely many } n
\bigr\},
\]
consisting of identities that select infinitely many digits. The set
$\mathcal{X}^{*}$ is a dense $G_{\delta}$ subset of $\mathcal{X}$ (see
\cite{Kechris} or \cite{Weihrauch}). It is not closed, and therefore not
compact.

The distinction between $\mathcal{X}$ and $\mathcal{X}^{*}$ is important.
Compactness arguments must take place in the ambient space $\mathcal{X}$.

\section{Collapse Map and Closedness of Fibers}

The collapse map
\[
\pi : \mathcal{X} \to [0,1]
\]
was defined in Chapter \texttt{\ref{chap:collapse}} by interpreting each identity
as an instruction to select digits of the output real number. Since the output
of $\pi$ depends only on the selected digits and these digits depend on finitely
many coordinates of the generative identity when viewed at any fixed precision,
the following fact is standard in symbolic dynamics.

\begin{proposition}
\label{prop:collapse-continuous}
The collapse map $\pi$ is continuous with respect to the product topology on
$\mathcal{X}$.
\end{proposition}

A proof may be found, for example, in Chapter 1 of \cite{LindMarcus} for shift
spaces and in Section 6 of \cite{Weihrauch} for continuous operators on Cantor
space.

Since singletons in $[0,1]$ are closed, the collapse fiber
\[
\mathcal{F}(x) = \pi^{-1}(\{x\})
\]
is a closed subset of the ambient generative space $\mathcal{X}$.

\begin{corollary}
\label{cor:fiber-compact}
For every $x \in [0,1]$, the collapse fiber $\mathcal{F}(x)$ is compact.
\end{corollary}

\begin{proof}
The ambient space $\mathcal{X}$ is compact and $\mathcal{F}(x)$ is closed in
$\mathcal{X}$, so $\mathcal{F}(x)$ is compact.
\end{proof}

This corrects the earlier intuition that compactness arises from
$\mathcal{X}^{*}$. The effective subspace $\mathcal{X}^{*}$ is not compact.
Compactness of the fiber is inherited from the ambient space $\mathcal{X}$.

\section{The Effective Fiber and Its Position in the Ambient Space}

Define the effective fiber by
\[
\mathcal{F}_{\mathrm{eff}}(x)
=
\mathcal{F}(x) \cap \mathcal{X}^{*}.
\]
This is the set of identities that collapse to $x$ and select infinitely many
digits. It is a dense subset of $\mathcal{F}(x)$ in the subspace topology.

The following subtlety is important.

\begin{proposition}
\label{prop:efffiber-not-closed}
The effective fiber $\mathcal{F}_{\mathrm{eff}}(x)$ is not closed in
$\mathcal{X}$ and is not closed in $\mathcal{X}^{*}$.
\end{proposition}

\begin{proof}
Consider a sequence of identities $G_k \in \mathcal{X}^{*}$ whose selectors
place the $j$th selected digit at position $n_j^{(k)}$ with
$n_j^{(k)} \to \infty$ as $k \to \infty$ for each fixed $j$. Pointwise limits
of such sequences may select only finitely many digits, so the limit lies in
$\mathcal{X} \setminus \mathcal{X}^{*}$. Since $\mathcal{F}(x)$ is closed, the
same phenomenon occurs inside fibers.
\end{proof}

Despite this lack of closedness, the effective fiber retains the same
topological richness as the full fiber.

\section{Perfectness and Cantor Geometry of the Fiber}

The collapse fiber $\mathcal{F}(x)$ is totally disconnected and has no
isolated points. This follows from standard arguments in Cantor space (see
\cite{LindMarcus} or \cite{Kechris}). We state the result here for reference.

\begin{proposition}
\label{prop:fiber-perfect}
For every $x \in [0,1]$, the collapse fiber $\mathcal{F}(x)$ is perfect,
totally disconnected, and uncountable.
\end{proposition}

\begin{proof}
Total disconnectedness follows from the product structure of $\mathcal{X}$.
Perfectness follows because arbitrary changes in the unselected coordinates or
in the meta-information stream after any finite index preserve the collapsed
value. Details are standard and may be found in the references above.
\end{proof}

Since the effective subspace $\mathcal{X}^{*}$ is dense in $\mathcal{X}$, the
effective fiber inherits these properties.

\begin{corollary}
\label{cor:efffiber-perfect}
The effective fiber $\mathcal{F}_{\mathrm{eff}}(x)$ is dense in the full fiber
and contains no isolated points.
\end{corollary}

\section{Tail Freedom Inside Collapse Fibers}

For any identity in the fiber and any finite prefix, there exist distinct
extensions in the fiber that share the prefix and differ afterward. This
property will be essential in later chapters.

\begin{proposition}
\label{prop:tail-freedom}
Let $x \in [0,1]$ and let $G \in \mathcal{F}(x)$. For every $N$ there exist
distinct identities $G'$ and $G''$ in $\mathcal{F}(x)$ such that
\[
G' \upharpoonright N
=
G'' \upharpoonright N
=
G \upharpoonright N.
\]
\end{proposition}

\begin{proof}
Modify the unselected digits or the meta stream beyond index $N$. These
changes do not alter the collapsed value. The resulting identities remain in
the fiber and are distinct.
\end{proof}

Tail freedom is one of the central sources of nondeterminacy in the generative
framework and forms the geometric basis for the indistinguishability
construction in Chapter \texttt{\ref{chap:indistinguishability}}.

\section{Summary}

This chapter clarified the topological setting of the collapse fibers. The key
points are:

\begin{itemize}
    \item The ambient generative space $\mathcal{X}$ is compact.
    \item The effective subspace $\mathcal{X}^{*}$ is dense and not compact.
    \item Collapse fibers $\mathcal{F}(x)$ are closed subsets of $\mathcal{X}$
    and are therefore compact.
    \item The effective fiber is dense inside the fiber and contains no
    isolated points.
    \item Tail freedom allows arbitrary variations after finite prefixes.
\end{itemize}

These properties form the foundation for the structural indistinguishability
results in Chapter \texttt{\ref{chap:indistinguishability}}.

