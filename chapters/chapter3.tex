\chapter{Fiber Geometry and the Effective Core}

\section{Introduction}

The collapse map $\pi : \mathcal{X}^* \to [0,1]$ associates to each
digit-selecting generative identity a classical real number.  
The purpose of this chapter is to study the structure of the fibers
\[
\mathcal{F}(x) = \pi^{-1}(\{x\})
\]
and, in particular, their topological and effective properties.

A fiber contains all identities that produce the same classical value.  
The collapse map obliterates most of the internal generative information,
leaving behind a large space of identities sharing a single output.  
Understanding this space is essential: the size and shape of fibers drive the
projection theory of Part~III and the incompleteness phenomena of Part~IV.

We show that each fiber is closed, uncountable, highly redundant, and
topologically rich.  
We then identify the \emph{effective fiber} and prove that it forms a
nonempty $\Pi^0_1$ class.  
This formalizes the computational structure of the identities surviving
collapse.

\section{Fibers of the Collapse Map}

Fix $x \in [0,1]$ and consider the set
\[
\mathcal{F}(x) = \{\, G \in \mathcal{X}^* : \pi(G) = x \,\}.
\]
Two identities $G = (M,D,K)$ and $H = (M',D',K')$ lie in the same fiber if
their canonical outputs agree:
\[
X(G) = X(H) = (x_j)_{j \ge 0},
\]
the base-$b$ expansion of $x$ we have fixed by convention.

Beyond these selected digits, the identities may differ arbitrarily.

\subsection*{Closedness}

Because $\pi$ is continuous and singletons $\{x\}$ are closed in $[0,1]$, each
fiber is a closed subset of $\mathcal{X}^*$.  
Fibers are therefore compact with respect to the product topology.

\subsection*{Degrees of freedom}

Let $(x_j)_{j \ge 0}$ denote the chosen expansion of $x$.  
A generative identity $G$ belongs to $\mathcal{F}(x)$ precisely when:
\begin{enumerate}
    \item[(i)] the selector $M$ chooses infinitely many positions;
    \item[(ii)] the digit stream $D$ satisfies
          $D(n_j) = x_j$ at the selected positions;
    \item[(iii)] the meta stream $K$ is arbitrary.
\end{enumerate}

Between successive selected positions, the identity may place any digit.  
Similarly, at each index where $M(n) = K$, the meta-symbol $K(n)$ is free.
Thus
\[
\mathcal{F}(x)
  \cong \{D,K\}^{\mathbb{N}} \times \{0,\ldots,b-1\}^{\mathbb{N}}
        \times \Sigma^{\mathbb{N}}
\]
subject to a countable collection of fixed coordinates enforcing
$D(n_j) = x_j$.  
The remaining coordinates form a product of Cantor spaces.
Hence every fiber is uncountable, totally disconnected, and perfect.

\section{The Effective Fiber}

We now restrict attention to the effective core.  
For $x \in [0,1]$ define the \emph{effective fiber}
\[
\mathcal{F}_{\mathrm{eff}}(x)
  = \mathcal{F}(x) \cap \mathcal{G}_{\mathrm{eff}}.
\]

In this section we show that $\mathcal{F}_{\mathrm{eff}}(x)$ is a nonempty
$\Pi^0_1$ class.  
This places the effective fibers squarely within the classical hierarchy of
computably presented closed sets in Cantor space.

\subsection{Nonemptiness}

If $x$ is computable, it has a computable expansion $(x_j)$, and the identity
that selects all digits and sets $D(j)=x_j$ lies in $\mathcal{F}_{\mathrm{eff}}(x)$.  
More generally, any computable selector that exposes infinitely many positions
can be combined with the given expansion to produce an element of the
effective fiber.

Thus $\mathcal{F}_{\mathrm{eff}}(x)$ is nonempty whenever $x$ is computable.

\subsection{Computable closedness}

To show that $\mathcal{F}_{\mathrm{eff}}(x)$ is a $\Pi^0_1$ class, we verify
that membership can be disproved by finite evidence.

A computable identity $G = (M,D,K)$ belongs to $\mathcal{F}_{\mathrm{eff}}(x)$
if and only if:
\begin{enumerate}
    \item $M$ selects digits infinitely often, and
    \item for each $j$, the selected digit $d_G(j)$ equals $x_j$.
\end{enumerate}

A violation of membership occurs precisely when:
\[
d_G(j) \ne x_j
\]
for some $j$.  
Because $M$ and $D$ are computable, this disagreement is detected by some
finite prefix of $(M,D)$.

Therefore:
\[
G \notin \mathcal{F}_{\mathrm{eff}}(x)
\quad\Longleftrightarrow\quad
\exists j\, [\, d_G(j) \ne x_j\,],
\]
and the right-hand condition is semidecidable (it is the disjunction of
computable finite checks).  
Hence $\mathcal{F}_{\mathrm{eff}}(x)$ is $\Pi^0_1$.

\section{Geometry of Effective Fibers}

Viewing $\mathcal{G}_{\mathrm{eff}}$ as a subset of Baire space, the effective
fiber inherits a rich internal structure:
\begin{itemize}
    \item it is infinite and compact in the subspace topology,
    \item it contains elements of both hybrid and null-density type,
    \item it admits arbitrary meta-information streams within the effective
          alphabet,
    \item it supports the tail-sewing and alignment constructions needed for
          diagonalization.
\end{itemize}

Crucially, no effective fiber collapses to a single computable identity.
Even after fixing the entire classical expansion of $x$, the selector stream
may choose digits at arbitrarily sparse or dense positions, and the
meta-information stream remains unrestricted.

This degree of freedom is the foundation for the observation and
incompatibility phenomena developed in the next part.

\section{Summary}

Each fiber of the collapse map is a large symbolic object containing every
identity that produces a given real number.  
The effective fiber $\mathcal{F}_{\mathrm{eff}}(x)$ forms a nonempty
$\Pi^0_1$ class and possesses significant internal structure.  
Understanding this geometry reveals why finite observers cannot capture all
the generative information that survives within a fiber.

In Part~II we turn to the dynamics of selector patterns, illustrating the
range of behaviors present among identities collapsing to the same value.
