\chapter{Collapse Fibers and Ambient Compactness}
\label{chap:fibers}

\section{The Ambient Generative Space}

The ambient generative space consists of three independent symbolic layers:
the selector layer, the digit layer, and the meta-information layer.
Let $\Sigma$ be a finite alphabet for the meta-information layer and fix a
base $b\geq 2$.  
The ambient space is the product
\[
\mathcal{G}
  =
  \{D,K\}^{\mathbb{N}}
  \times
  \{0,\ldots,b-1\}^{\mathbb{N}}
  \times
  \Sigma^{\mathbb{N}},
\]
equipped with the product topology induced by the discrete topology on each
alphabet.  
By Tychonoff's theorem, $\mathcal{G}$ is compact, metrizable, and totally
disconnected.  
A standard compatible ultrametric is
\[
d(G,H) = 2^{-N},
\]
where $N$ is the least index at which $G$ and $H$ differ.  
This is the classical metric used in Cantor space and shift spaces
\cite{LindMarcus, Kechris}.

\section{Digit-Producing Subspace}

A generative identity $G=(M,D,K)$ produces an infinite collapse coordinate when
the selector exposes digits infinitely often.  
The digit-producing subspace is
\[
\mathcal{G}^{*}
=
\{\, G \in \mathcal{G} : M(n)=D
     \text{ for infinitely many } n \,\}.
\]

The subspace $\mathcal{G}^{*}$ is dense in $\mathcal{G}$ and closed under
finite modifications of any layer.  
It is not closed and therefore not compact.  
Its density reflects the fact that any initial prefix of a generative identity
can later be extended to expose infinitely many digits.

\section{Collapse Map and Compact Fibers}

The collapse map
\[
\pi : \mathcal{G}^{*} \to [0,1]
\]
was defined in Chapter~\ref{chap:collapse-map} by selecting digits at the
positions where the selector layer takes the value $D$ and interpreting them
as a base-$b$ expansion.

Continuity of $\pi$ follows from the prefix topology: the value of $\pi(G)$
at precision $b^{-(N+1)}$ depends only on the first $N$ selected digits, and
these appear within some finite prefix of $G$.  
This is the standard argument for continuity of evaluation maps on symbolic
product spaces \cite{LindMarcus, WeihrauchComputableAnalysis}.

\begin{proposition}
\label{prop:collapse-continuous}
The collapse map $\pi$ is continuous.
\end{proposition}

Fibers of a continuous map into the Hausdorff space $[0,1]$ are closed.

\begin{corollary}
\label{cor:fiber-compact}
For every $x \in [0,1]$, the collapse fiber
\[
\mathcal{F}(x)
  = \pi^{-1}(\{x\})
\]
is a compact subset of $\mathcal{G}$.
\end{corollary}

\begin{proof}
The fiber is closed in the compact space $\mathcal{G}$ and is therefore
compact.
\end{proof}

\section{The Effective Fiber}

The effective generative space
\[
\mathcal{G}_{\mathrm{eff}}
=
\{\, G \in \mathcal{G} :
     M, D, K \text{ are computable sequences} \,\}
\]
defines the computational core of the framework.  
The effective fiber of $x$ is
\[
\mathcal{F}_{\mathrm{eff}}(x)
   =
\mathcal{F}(x) \cap \mathcal{G}^{*}.
\]

Since $\mathcal{G}^{*}$ is dense in $\mathcal{G}$, every basic open set
intersecting a full fiber also intersects the effective fiber.  
However, the effective fiber is never closed.

\begin{proposition}
\label{prop:efffiber-not-closed}
For any $x \in [0,1]$, the effective fiber $\mathcal{F}_{\mathrm{eff}}(x)$ is
not closed in $\mathcal{G}$ and not closed in $\mathcal{G}^{*}$.
\end{proposition}

\begin{proof}
Construct a sequence of identities whose $j$th selected digit is pushed
farther and farther out.  
For fixed $j$, let the position $n_{j}^{(k)}$ satisfying $M(n_{j}^{(k)})=D$
tend to infinity as $k\to\infty$.  
Each identity lies in $\mathcal{F}_{\mathrm{eff}}(x)$, but the limit selects
only finitely many digits and therefore lies outside $\mathcal{G}^{*}$.  
The same construction can be performed inside any fiber $\mathcal{F}(x)$.
\end{proof}

\section{Cantor Geometry of Collapse Fibers}

Collapse fibers inherit the structural properties of Cantor-type spaces.
They are uncountable, totally disconnected, and contain no isolated points.

\begin{proposition}
\label{prop:fiber-perfect}
For every $x \in [0,1]$, the collapse fiber $\mathcal{F}(x)$ is perfect,
totally disconnected, and uncountable.
\end{proposition}

\begin{proof}
Total disconnectedness follows from the product structure of $\mathcal{G}$.
To show perfectness, fix $G\in\mathcal{F}(x)$ and a finite prefix length $N$.
Modify $G$ beyond index $N$ by altering unselected digits or the meta layer.
These changes do not affect the collapse coordinate, so the resulting points
remain in $\mathcal{F}(x)$ and can be made arbitrarily close to $G$.  
Thus no point is isolated.  
Uncountability follows from the existence of infinitely many independent
choices in the unobserved coordinates, which mirrors classical arguments for
Cantor sets \cite{Kechris}.
\end{proof}

The effective fiber inherits these geometric properties through density.

\begin{corollary}
\label{cor:efffiber-perfect}
The effective fiber $\mathcal{F}_{\mathrm{eff}}(x)$ is dense in the full
fiber and has no isolated points.
\end{corollary}

\section{Tail Freedom}

Collapse depends only on the sequence of selected digits, not on the positions
at which they occur or on the values of unselected digits and meta symbols.
This leads to complete freedom to modify the symbolic structure after any
finite prefix.

\begin{proposition}
\label{prop:tail-freedom}
Let $G \in \mathcal{F}(x)$ and let $N\in\mathbb{N}$.  
There exist distinct $G'$ and $G''$ in $\mathcal{F}(x)$ such that
\[
G' \upharpoonright N
  = G'' \upharpoonright N
  = G \upharpoonright N.
\]
\end{proposition}

\begin{proof}
Make two different modifications to either the unselected digits or the
meta-information layer beyond index $N$.  
Neither modification affects the collapse coordinate, so both identities
remain in $\mathcal{F}(x)$.
\end{proof}

Tail freedom is the structural foundation for the alignment and sewing
constructions developed in Part~\ref{part:incompleteness}, where fiber
geometry interacts with observer behavior.

\section{Summary}

Collapse fibers display the following geometric features.

\begin{itemize}
    \item The ambient generative space $\mathcal{G}$ is compact and totally
    disconnected.
    \item The digit-producing subspace $\mathcal{G}^{*}$ is dense but not
    compact.
    \item Each collapse fiber $\mathcal{F}(x)$ is compact, perfect, and
    uncountable.
    \item The effective fiber is dense in the full fiber and has no isolated
    points.
    \item Symbolic structure may vary freely beyond any finite prefix without
    altering collapse.
\end{itemize}

These properties form the geometric backbone of collapse theory and prepare
the ground for the alignment and indistinguishability phenomena studied in
later chapters.
