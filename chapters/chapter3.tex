\chapter{Fiber Geometry of the Collapse Map}

\section{Introduction}

The collapse map introduced in Chapter~2 identifies the classical magnitude of a generative identity by extracting and decoding the digit subsequence selected by the mixer.  
Many distinct mechanisms collapse to the same real number, and the structure of the set of such mechanisms is central to the generative viewpoint.  
This chapter studies the geometry of the fibers of the collapse map, both in the full generative space and in the effective core.

The fibers exhibit sharp contrasts.  
In the full space $\mathcal{X}$ each fiber is uncountable and carries a rich product structure.  
In the effective core $\mathcal{G}_{\mathrm{eff}}$ each fiber over a computable real is countable, yet still infinite.  
These distinctions form the foundation for the hybrid identities developed in Chapter~4 and for the analysis of sparse digit behavior in Chapter~5.

\section{Full Fibers in the Generative Space}

\begin{definition}[Full Fiber]
For $x \in [0,1]$, the \emph{full fiber} of $x$ is the set
\[
\mathcal{F}(x) = \{ G \in \mathcal{X} : \pi(G) = x \}.
\]
\end{definition}

A mechanism belongs to $\mathcal{F}(x)$ if and only if the digit values selected by its mixer produce a base-$b$ expansion of $x$.  
The meta symbols play no role in determining $\pi(G)$, and unselected digit symbols also have no influence.  
This yields the following structural decomposition.

\begin{proposition}[Product Structure of Fibers]
Let $x \in [0,1]$ and let $(x_j)_{j \ge 0}$ be a fixed base-$b$ expansion of~$x$.  
Then $\mathcal{F}(x)$ contains all triples $G = (M, D, K)$ such that
\[
D(\varphi_G(j)) = x_j \qquad \text{for all } j \ge 0,
\]
where $\varphi_G$ enumerates $S_D(G)$.  
All other values of $D$ and all values of $K$ are unconstrained.
\end{proposition}

\begin{proof}
The condition $D(\varphi_G(j)) = x_j$ ensures that the digit subsequence selected by $M$ matches the desired expansion of $x$, which is equivalent to $\pi(G)=x$.  
No other coordinates influence $\pi(G)$.
\end{proof}

This decomposition shows that $\mathcal{F}(x)$ contains a full product of unconstrained coordinates whenever the mixer selects digits infinitely often.  
The freedom in these unconstrained positions yields the following conclusion.

\begin{corollary}
For every $x \in [0,1]$, the full fiber $\mathcal{F}(x)$ is uncountable.
\end{corollary}

The cardinality reflects the fact that infinitely many choices of meta symbols and unselected digit symbols do not affect the collapse.

\begin{remark}
When a real number admits two base-$b$ expansions, the corresponding full fibers share substantial structure.  
This ambiguity is similar to the non-uniqueness of binary expansions and does not affect later results.
\end{remark}

\section{Effective Fibers in the Computable Core}

The structure of the fiber changes dramatically when restricted to $\mathcal{G}_{\mathrm{eff}}$.

\begin{definition}[Effective Fiber]
For $x \in \mathbb{R}_c$, the \emph{effective fiber} of $x$ is the set
\[
\mathcal{F}_{\mathrm{eff}}(x) = \mathcal{F}(x) \cap \mathcal{G}_{\mathrm{eff}}.
\]
If $x$ is not computable, then $\mathcal{F}_{\mathrm{eff}}(x)$ is empty.
\end{definition}

This follows immediately from the fact that collapse on the effective core maps onto the computable reals, as established in Chapter~2.

\begin{proposition}
If $x \in \mathbb{R}_c$, then $\mathcal{F}_{\mathrm{eff}}(x)$ is infinite.  
If $x \notin \mathbb{R}_c$, then $\mathcal{F}_{\mathrm{eff}}(x)$ is empty.
\end{proposition}

\begin{proof}
If $x$ is computable, then its digit expansion is computable.  
Any effective mixer that selects digits infinitely often, combined with any computable meta sequence, produces an effective identity in $\mathcal{F}(x)$.  
Varying the effective choices of $M$ and $K$ while preserving their computability produces infinitely many such identities.  
If $x$ is not computable, then no computable mechanism can collapse to~$x$, so the effective fiber is empty.
\end{proof}

Thus the effective fibers are countable but never trivial when $x$ is computable.  
The freedom in choosing an effective mixer and an effective meta sequence allows for substantial internal variation.

\section{Internal Degrees of Freedom}

The full fiber $\mathcal{F}(x)$ contains three primary sources of variability:

\begin{enumerate}
    \item positions where the mixer selects the meta layer,
    \item digit symbols at positions not selected by the mixer,
    \item the meta sequence itself.
\end{enumerate}

None of these affect the decoded subsequence used in the collapse.  
This leads to a rich combinatorial structure inside each fiber.

\begin{proposition}
If the mixer selects digits at infinitely many positions, then for any real number $x$ the fiber $\mathcal{F}(x)$ contains infinitely many elements that disagree on an infinite set of coordinates.
\end{proposition}

This proposition is a simple consequence of the product structure and anticipates the more refined geometric phenomena that appear in hybrid identities (Chapter~4) and ghost identities (Chapter~5).

\begin{remark}
The case where the mixer selects only finitely many digits is degenerate and corresponds to rationals with exceptionally constrained fibers.  
These cases do not affect the theorems in later chapters.
\end{remark}

\section{Shift Dynamics and Fiber Structure}

The full generative space $\mathcal{X}$ carries a natural shift map $\sigma$ defined by
\[
\sigma(M,D,K)(n) = (M(n+1), D(n+1), K(n+1)).
\]
This shift interacts with fibers in a non-trivial way.

\begin{proposition}
If $G \in \mathcal{F}(x)$, then $\sigma(G)$ need not lie in the same fiber.  
However, $\sigma$ preserves the product topology and therefore acts as a continuous transformation on $\mathcal{X}$.
\end{proposition}

The shift structure highlights that mechanisms may have identical collapsed values but very different dynamical signatures under iteration.  
Later chapters use this viewpoint to analyze hybrid and ghost patterns, which exhibit distinct frequency and density behaviors under the shift.

\section{Outlook}

This chapter establishes the basic geometric framework for collapse fibers.  
Chapter~4 develops hybrid identities, which have positive density of digit selections and play a central role in both the topological and effective structures.  
Chapter~5 explores ghost identities, where the digit density is zero and the meta layer dominates.  
Together these chapters show the range of internal behaviors that collapse to a common real number.
