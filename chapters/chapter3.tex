\chapter{Fiber Geometry and Ambient Compactness}
\label{chap:fibers}

\section{Introduction}

The collapse representation introduced in Chapter~\ref{chap:collapse-map} assigns a real value to each generative identity by exposing a sequence of digits drawn from one symbolic coordinate. 
Identities that produce the same exposed digit sequence form a collapse fiber. 
These fibers are the central geometric objects of the generative viewpoint. 
They are compact, perfect, and carry substantial latent structure that is invisible to classical magnitude.

This chapter develops the topology of collapse fibers and explains how they inherit structure from the ambient space. 
A key theme is the distinction between the exposure domain, which is not compact, and individual fibers, which are compact. 
This compactness is essential for the finite-information arguments and generative freedom constructions developed in Chapters~6 and~7.

\section{Ambient Generative Space}

As introduced in Chapter~\ref{chap:generative-space}, the ambient generative space is the symbolic product
\[
\mathcal{G}
  = \Gamma_{1}^{\mathbb{N}}
    \times \Gamma_{2}^{\mathbb{N}}
    \times \Gamma_{3}^{\mathbb{N}}
\]
with the product topology. 
Each coordinate carries the discrete topology on its finite alphabet, and the product is compact by Tychonoff's theorem. 
The metric structure is given by the usual prefix metric, where the distance between two identities is determined by the length of the longest common prefix across all coordinates.

This space is compact, totally disconnected, and perfect. 
Such spaces form the standard setting for symbolic dynamics and descriptive set theory \cite{LindMarcus, Kechris}. 
All collapse fibers, and all effective constructions used later, live inside this ambient product.

\section{Exposure Domain}

The collapse representation requires infinitely many exposed positions in order to produce an infinite classical expansion. 
For each $G = (U_{1}, U_{2}, U_{3})$, the exposure mechanism described in Chapter~\ref{chap:collapse-map} determines an infinite set of positions at which the value of the observed-value coordinate $U_{2}$ is revealed.

\begin{definition}[Exposure Domain]
The exposure domain is the set
\[
\mathcal{G}^{*}
  = \{\, G \in \mathcal{G} : P(G) \text{ is infinite} \,\},
\]
where $P(G)$ is the exposure set associated to $G$.
\end{definition}

The exposure domain $\mathcal{G}^{*}$ is a dense $G_{\delta}$ subset of the compact space $\mathcal{G}$. 
It is not compact. 
This is a crucial distinction: compactness arguments apply to fibers and to the ambient space, but not to the exposure domain as a whole.

Density of $\mathcal{G}^{*}$ reflects the fact that any finite prefix of a generative identity can be extended in many ways to produce infinitely many exposed positions.

\section{Collapse Map and Closed Fibers}

The collapse map
\[
\pi : \mathcal{G}^{*} \to [0,1]
\]
assigns to each identity the real number whose base $b$ expansion is formed by the exposed values of $U_{2}$. 
The continuity of $\pi$ was established in Chapter~\ref{chap:collapse-map}, using the fact that only finitely many exposed digits are needed to determine the value of $\pi(G)$ to any prescribed precision.

A continuous map into a Hausdorff space has closed fibers.

\begin{proposition}
For each real number $x \in [0,1]$, the collapse fiber
\[
\mathcal{F}(x)
  = \pi^{-1}(\{x\})
\]
is a closed subset of $\mathcal{G}$.
\end{proposition}

Since $\mathcal{G}$ is compact, each fiber is compact as well.

\begin{corollary}
For each $x \in [0,1]$, the collapse fiber $\mathcal{F}(x)$ is compact.
\label{cor:fiber-compact}
\end{corollary}

Compactness is a fundamental property of fibers. 
It ensures that limits of identities that preserve the exposed digit sequence remain within the same fiber, even if these limits fall outside the exposure domain $\mathcal{G}^{*}$. 
This compactness is essential for the generative freedom arguments developed later.

\section{Effective Fiber}

The effective generative space
\[
\mathcal{G}_{\mathrm{eff}}
  = \{\, (U_{1}, U_{2}, U_{3}) \in \mathcal{G}
         : U_{1}, U_{2}, U_{3} \text{ are computable} \,\}
\]
forms the computational core of the framework. 
The effective fiber of $x$ is the set
\[
\mathcal{F}_{\mathrm{eff}}(x)
   = \mathcal{F}(x) \cap \mathcal{G}_{\mathrm{eff}}.
\]

The exposure domain intersects every open set meeting the full fiber, and therefore the effective fiber is dense in the full fiber. 
However, unlike the full fiber, the effective fiber is not closed.

\begin{proposition}
For any $x \in [0,1]$, the effective fiber $\mathcal{F}_{\mathrm{eff}}(x)$ is not closed in $\mathcal{G}$ or in $\mathcal{G}^{*}$.
\end{proposition}

\begin{proof}
Fix $x \in [0,1]$. 
Construct a sequence of identities $G^{(k)}$ in the effective fiber by delaying the position of the $j$th exposed digit farther and farther out for each fixed $j$. 
Each $G^{(k)}$ agrees with the target collapse coordinate, but the limit identity exposes only finitely many positions.
Hence the limit lies outside $\mathcal{G}^{*}$ and therefore outside the effective fiber.
This demonstrates that the effective fiber is not closed.
\end{proof}

This phenomenon reflects a general feature of computable presentations: computable points need not be closed under limits.

\section{Perfectness and Total Disconnectedness}

Collapse fibers inherit the structural properties of Cantor-like sets.

\begin{proposition}
For each $x \in [0,1]$, the collapse fiber $\mathcal{F}(x)$ is perfect, totally disconnected, and uncountable.
\label{prop:fiber-perfect}
\end{proposition}

\begin{proof}
Total disconnectedness follows from the product topology on $\mathcal{G}$.

To prove perfectness, fix $G \in \mathcal{F}(x)$ and a finite prefix length $N$. 
Construct a new identity $H$ by changing the value of one of the latent coordinates at a position beyond $N$. 
Such modifications do not affect the exposed digit sequence, so $H$ lies in $\mathcal{F}(x)$ and can be made arbitrarily close to $G$. 
Thus $\mathcal{F}(x)$ contains no isolated points.

Uncountability follows from the presence of infinitely many independent coordinates in the latent portion of each identity. 
Classical arguments for Cantor sets apply \cite{Kechris}.
\end{proof}

Since the effective fiber is dense within the full fiber, it also has no isolated points.

\begin{corollary}
For any $x \in [0,1]$, the effective fiber $\mathcal{F}_{\mathrm{eff}}(x)$ is dense in the full fiber and contains no isolated points.
\end{corollary}

\section{Topological Generative Freedom}

Collapse fibers permit unrestricted modification of latent coordinates beyond any fixed prefix, as long as the exposed digits remain unchanged. 
This freedom reflects the infinite-dimensional nature of the ambient space.

\begin{proposition}
Let $G \in \mathcal{F}(x)$ and let $N \in \mathbb{N}$. 
There exist distinct identities $G'$ and $G''$ in $\mathcal{F}(x)$ such that
\[
G' \upharpoonright N
  = G'' \upharpoonright N
  = G \upharpoonright N.
\]
\end{proposition}

\begin{proof}
Fix $N$. 
Modify the latent coordinates of $G$ at any position larger than $N$ to obtain two distinct identities $G'$ and $G''$. 
Since such changes do not alter the exposed digit sequence, both remain in the fiber $\mathcal{F}(x)$.
\end{proof}

This topological generative freedom forms the basis for the sewing and diagonalizer constructions developed in Appendices~C and~D and used in Chapters~6 and~7.

\section{Summary}

Collapse fibers have a rich and flexible internal structure.

\begin{itemize}
    \item The ambient generative space is compact, perfect, and totally disconnected.
    \item The exposure domain $\mathcal{G}^{*}$ is a dense $G_{\delta}$ set but not compact.
    \item Each collapse fiber $\mathcal{F}(x)$ is compact, perfect, and uncountable.
    \item The effective fiber is dense in the full fiber and has no isolated points.
    \item Latent coordinates may be modified freely beyond any finite prefix while keeping the exposed digit sequence fixed.
\end{itemize}

These properties establish the geometric foundation for generative freedom and for the incompleteness results that arise from finite-information observation in later chapters.
