\chapter{Fiber Geometry of the Collapse Map}

\section{Introduction}

The collapse map introduced in Chapter~2 assigns a classical magnitude to a
generative identity by extracting and interpreting the digit symbols selected by
its mixer.  
Because collapse ignores the meta layer and all unselected digit positions,
distinct generative identities may produce the same classical value.  
Understanding the structure of these equivalence classes is a central objective
of the generative viewpoint.

This chapter studies the \emph{collapse fibers}
\[
\mathcal{F}(x) = \{\, G \in \mathcal{X}^* : \pi(G) = x \,\},
\]
which collect all identities that collapse to a given real number $x$.
We describe their topological, combinatorial, and effective structure.  
The full fibers $\mathcal{F}(x)$ are closed and uncountable subsets of the
product space $\mathcal{X}$, while the effective fibers
$\mathcal{F}_{\mathrm{eff}}(x)$, defined for computable $x$, form
nontrivial $\Pi^0_1$ classes.  

These results lay the groundwork for the selector regimes explored in Part~II
and for the projection-based restrictions developed in Parts~III and~IV.

\section{Full Fibers in the Generative Space}

\begin{definition}[Full Fiber]
For $x \in [0,1]$, the \emph{full collapse fiber} is
\[
\mathcal{F}(x)
=
\{\, G \in \mathcal{X}^* : \pi(G)=x \,\}.
\]
\end{definition}

A mechanism belongs to $\mathcal{F}(x)$ precisely when its selected digit
subsequence equals a base-$b$ expansion of $x$.  
The selector determines \emph{where} in the timeline these digits occur, but the
meta symbols and all digit symbols at unselected positions are irrelevant to the
value of $\pi(G)$.

\begin{proposition}[Fibers Are Closed]
For every $x \in [0,1]$, the fiber $\mathcal{F}(x)$ is closed in the product
topology of $\mathcal{X}$.
\end{proposition}

\begin{proof}
Collapse is continuous and $[0,1]$ is Hausdorff.  
Therefore $\mathcal{F}(x) = \pi^{-1}(\{x\})$ is closed.
\end{proof}

The closedness of fibers reflects the fact that a finite-prefix violation of the
digit constraints suffices to prove that $G \notin \mathcal{F}(x)$.

\subsection*{Product Structure}

To describe the internal geometry of a fiber, fix $x \in [0,1]$ with a chosen
base-$b$ expansion $(x_j)_{j \ge 0}$.  
For any $G = (M,D,K) \in \mathcal{X}^*$, let
\[
\varphi_G : \mathbb{N} \to S_M
\]
enumerate the indices where $M$ selects the digit layer.

\begin{proposition}[Product Decomposition of Fibers]
\label{prop:fiber-product}
For $G \in \mathcal{X}^*$,
\[
G \in \mathcal{F}(x)
\quad\Longleftrightarrow\quad
D(\varphi_G(j)) = x_j \text{ for all } j \ge 0.
\]
All other coordinates of $D$ and all coordinates of $K$ are unconstrained.
\end{proposition}

\begin{proof}
If the selected digit subsequence is exactly $(x_j)$, the collapse value equals $x$.  
Conversely, if $\pi(G)=x$, the selected digit positions must realize the chosen
expansion of $x$.  
Other coordinates do not affect collapse.
\end{proof}

The fiber therefore has the structure of a full product:
the selector is arbitrary (subject to selecting infinitely many digits), the
digit layer is constrained only on selected positions, and the meta layer is
completely free.

\begin{corollary}
Every fiber $\mathcal{F}(x)$ is uncountable.
\end{corollary}

The uncountability reflects the enormous freedom present in the unconstrained
layers.

\section{Effective Fibers and \texorpdfstring{$\Pi^0_1$}{Pi-0-1} Structure}

Restricting to the effective core substantially changes the descriptive nature
of the fibers.

\begin{definition}[Effective Fiber]
If $x \in \mathbb{R}_c$, the \emph{effective fiber} is
\[
\mathcal{F}_{\mathrm{eff}}(x)
=
\mathcal{F}(x) \cap \mathcal{G}_{\mathrm{eff}}.
\]
If $x$ is not computable, this set is empty.
\end{definition}

\begin{remark}[Effective Fibers Are $\Pi^0_1$]
Since collapse is computable, the condition $\pi(G)=x$ can be falsified by
exhibiting a finite prefix of $G$ that violates the digit constraints.  
Thus $\mathcal{F}_{\mathrm{eff}}(x)$ is a $\Pi^0_1$ subset of $\mathcal{X}$:
membership requires agreement on all finite prefixes, but non-membership can be
witnessed by a finite prefix.
\end{remark}

\begin{proposition}[Effective Fibers Are Infinite]
If $x \in \mathbb{R}_c$, then $\mathcal{F}_{\mathrm{eff}}(x)$ is infinite.
\end{proposition}

\begin{proof}
Fix a computable expansion of $x$.  
Any computable selector that selects digits infinitely often, together with any
computable meta sequence, yields an effective generator of $x$.  
There are infinitely many such choices.
\end{proof}

Thus, even under computability constraints, the fiber contains many distinct
mechanisms with the same classical value.

\section{Internal Degrees of Freedom}

By Proposition~\ref{prop:fiber-product}, each fiber supports three independent
degrees of freedom:

\begin{enumerate}
    \item \textbf{Selector freedom:} arbitrary choice of $M$ on the positions
    where $M$ selects the meta layer.

    \item \textbf{Digit freedom:} arbitrary choice of $D(n)$ for
    $n \notin S_M$.

    \item \textbf{Meta freedom:} arbitrary choice of the entire sequence $K$.
\end{enumerate}

These give rise to significant structural diversity.

\begin{proposition}[Infinite Divergence Within a Fiber]
Let $x \in [0,1]$.  
If $G \in \mathcal{X}^*$ selects digits infinitely often, then $\mathcal{F}(x)$
contains infinitely many distinct identities that differ from $G$ on an infinite
set of coordinates.
\end{proposition}

\begin{proof}
Vary the meta layer on an infinite set, or vary the unselected digit coordinates
on an infinite set, while preserving the required digit subsequence.  
Both operations produce infinitely many distinct identities in $\mathcal{F}(x)$.
\end{proof}

This abundance of mechanisms anticipates the selector-regime dichotomy of
Part~II: hybrid identities with positive digit density and null-density identities
with asymptotically vanishing digit density.

\section{Shift Dynamics and Fiber Geometry}

The generative space carries a natural left shift:
\[
\sigma(M,D,K)(n)
=
\bigl(M(n+1),\, D(n+1),\, K(n+1)\bigr).
\]

\begin{proposition}
If $G \in \mathcal{F}(x)$, the shift $\sigma(G)$ need not belong to 
$\mathcal{F}(x)$.  
However, $\sigma$ is continuous on $\mathcal{X}$ and preserves the product
topology.
\end{proposition}

The shift map reorganizes the timeline of each fiber element.
Although collapse fibers are not invariant under $\sigma$, the shift action is a
useful tool for understanding the long-term patterns of hybrid and null-density
selectors.  
In later chapters, shift-based arguments help classify selector regimes and
analyze their structural consequences.

\section{Outlook}

This chapter describes the basic geometry of collapse fibers: closedness,
uncountability, effective $\Pi^0_1$ structure, and the degrees of freedom that
generate internal diversity.  
These fibers form the natural habitat for the selector regimes developed in
Part~II, where we study hybrid identities (positive digit density) and
null-density generators (asymptotically vanishing digit density).  
The range of behaviors found within a single fiber motivates the need for
secondary projections and structural measurement, the subject of Part~III.
