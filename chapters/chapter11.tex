\chapter{Extended Invariants: Entropy Balance and Fluctuation}

\section{Introduction}

The collapse map sends a generative identity to its classical real value, but
two identities that collapse to the same number may differ significantly in
their internal structure.  
Part IV showed that no finite collection of continuous observers can recover
this structure.  
In this chapter we introduce two extended invariants that capture broad
features of selector behavior: the entropy balance and the fluctuation index.

Both invariants measure long term properties of the selector stream.  
The entropy balance describes how frequently digits are exposed, while the
fluctuation index measures the relative size of gaps between successive
selected positions.  
Neither invariant is continuous in the product topology, and this reflects a
deeper fact: asymptotic statistical quantities on symbolic streams are rarely
continuous unless they depend only on finite windows.  
What can be proved, and what is sufficient for our purposes, is that these
quantities satisfy natural semicontinuity properties.

\section{Entropy Balance}

Let $G = (M, D, K)$ be a generative identity.  
Define the indicator function
\[
\chi_M(n)
  =
\begin{cases}
1 & M(n) = D,\\
0 & M(n) = K.
\end{cases}
\]

The entropy balance (or simply the balance) of $G$ is the limit inferior
\[
\eta(G)
  = \liminf_{N \to \infty}
      \frac{1}{N} \sum_{n=0}^{N-1} \chi_M(n).
\]

This quantity measures the lower asymptotic density with which the selector
exposes digits.  
Hybrid identities have positive balance, while null density identities have
balance zero.

\subsection{Basic properties}

The balance has two elementary properties.

\begin{itemize}
    \item It is invariant under tail modification beyond finite prefixes.
    \item It depends only on the selector stream $M$.
\end{itemize}

The balance is therefore an extended invariant that captures structural
information invisible to the collapse map.

\subsection{Lower semicontinuity}

The balance is not continuous with respect to the product topology.  
Small modifications to the selector can introduce arbitrarily large gaps or
arbitrarily many selected positions within a long prefix, which can shift the
density downward or upward.  
However, balance satisfies a one sided estimate.

\begin{proposition}[Lower Semicontinuity]
Let $(G_k)$ converge to $G$ in the product topology.  
Then
\[
\eta(G)
  \le \liminf_{k \to \infty} \eta(G_k).
\]
\end{proposition}

\begin{proof}
Fix $\varepsilon > 0$.  
Choose $N$ sufficiently large that
\[
\frac{1}{N} \sum_{n=0}^{N-1} \chi_M(n)
  < \eta(G) + \varepsilon.
\]
For all sufficiently large $k$, the identities $G_k$ agree with $G$ on the
first $N$ symbols.  
Hence
\[
\eta(G_k)
  \ge \frac{1}{N} \sum_{n=0}^{N-1} \chi_M(n)
  > \eta(G) - \varepsilon.
\]
Taking the limit inferior gives the claim.
\end{proof}

The failure of full continuity is a structural fact, but lower
semicontinuity is sufficient for all applications.

\section{Fluctuation Index}

The fluctuation index measures relative gap size between selected digit
positions.  
Let
\[
n_0 < n_1 < n_2 < \cdots
\]
be the indices at which $M(n) = D$ and define the successive gaps
\[
g_j = n_{j+1} - n_j.
\]

The fluctuation index of $G$ is the limit superior
\[
\phi(G)
  = \limsup_{j \to \infty} \frac{g_j}{n_j}.
\]

The quantity $g_j$ measures the absolute size of the gap following the
$j$th selected position, while the ratio $g_j / n_j$ measures the gap
relative to the position in the stream.  
High fluctuation indicates the presence of unusually large gaps between
selected digits, relative to scale.

\subsection{Basic properties}

The fluctuation index depends only on the selector stream and is unaffected by
any tail modifications that do not alter the sequence of selected positions.

The index captures the extent to which the selector allows sparse bursts of
digit exposure.  
Null density selectors typically have large fluctuation, while positive
density selectors have small fluctuation in many cases.

\subsection{Upper semicontinuity}

As with $\eta$, the fluctuation index is not continuous.  
A single large gap introduced in the tail can immediately increase the value
of $\phi$.  
However, the index satisfies upper semicontinuity.

\begin{proposition}[Upper Semicontinuity]
Let $(G_k)$ converge to $G$ in the product topology.  
Then
\[
\phi(G)
  \ge \limsup_{k \to \infty} \phi(G_k).
\]
\end{proposition}

\begin{proof}
Suppose $\phi(G) < c$ for some $c$.  
Then there exists $J$ such that
\[
\frac{g_j}{n_j} < c
\quad\text{for all}\quad j \ge J.
\]
The selected positions $n_j$ are determined by the selector stream.  
Agreement of $G_k$ with $G$ on sufficiently long prefixes ensures that the
positions of the first $J$ selected digits match, and that the ratios
$g_j / n_j$ for $j \le J$ match exactly.  
Therefore for all sufficiently large $k$,
\[
\phi(G_k) < c.
\]
The desired inequality follows by taking the limit superior.
\end{proof}

Upper semicontinuity reflects the fact that introducing a large gap is a
finite event that persists under limits, while eliminating a large gap is a
global tail operation that is not respected by the product topology.

\section{Selector Structure and Extended Invariants}

The invariants $\eta$ and $\phi$ reveal two complementary aspects of selector
behavior:
\begin{itemize}
    \item $\eta$ measures how frequently digits are exposed in a long term
          sense,
    \item $\phi$ measures how irregularly they are exposed relative to scale.
\end{itemize}

These quantities need not determine one another.  
For example, a selector may have positive density but also have occasional
large gaps, or a selector may have density zero but exhibit extremely regular
spacing.  
Both invariants occur densely in the generative space.

\section{Extended Invariants Inside Collapse Fibers}

Extended invariants vary widely inside a collapse fiber.  
Fix a computable real $x$.  
The fiber $\mathcal{F}_{\mathrm{eff}}(x)$ contains identities with:
\begin{itemize}
    \item positive balance,
    \item zero balance,
    \item small fluctuation indices,
    \item arbitrarily large fluctuation indices.
\end{itemize}

This diversity follows from the fact that extended invariants depend only on
the selector stream and are insensitive to positions where $M(n) = K$.  
Given any selector behavior consistent with infinite digit selection, one may
always construct an identity in the fiber by assigning the selected positions
the digits of $x$ in the correct order.

Thus the fiber contains identities with all possible behaviors permitted by
the definitions of $\eta$ and $\phi$.

\section{Summary}

The entropy balance and fluctuation index extend the collapse map by assigning
numerical values to the selector behavior of a generative identity.  
Both invariants are discontinuous in the product topology but satisfy natural
semicontinuity properties.  
These quantities illustrate the diversity of selector patterns that arise
inside collapse fibers and provide a bridge between the collapse quotient of
Part V and the generative geometry of the next chapter.

In Chapter 13 we study geometric embeddings of extended invariants and discuss
how these quantities characterize large scale features of the generative
space.
