\chapter{Extended Invariants: Asymptotic Density and Fluctuation}
\label{chap:invariants-eta-phi}

\section{Introduction}

The collapse map records only the classical real value determined by a
generative identity.  
It does not reveal the large scale structure of the selector stream.  
Part II established that continuous observers depend on finite prefixes at any
fixed precision, and Part IV showed that this restriction prevents observers
from detecting tail-dependent behavior.

In this chapter we introduce two extended invariants that describe global
aspects of selector geometry:

\begin{itemize}
    \item the asymptotic density $\eta(G)$ of selected digits, and
    \item the fluctuation index $\phi(G)$, which measures the relative scale of
    successive gaps between selected positions.
\end{itemize}

Both invariants depend only on the selector layer.  
They ignore finite prefixes and are sensitive only to tail behavior.  
Consequently they are discontinuous everywhere in the product topology of
$\mathcal{G}$ and take all admissible values inside every nonempty open set.
These invariants illustrate the asymptotic freedom that survives inside collapse
fibers and that remains invisible to any finite observer.

\section{Asymptotic Density}

Let $G=(M,D,K)$ be a generative identity.  
Define the indicator
\[
\chi_{M}(n)
=
\begin{cases}
1 & \text{if } M(n)=D,\\[3pt]
0 & \text{if } M(n)=K.
\end{cases}
\]
The asymptotic density of $G$ is the lower limit
\[
\eta(G)
=
\liminf_{N\to\infty}
\frac{1}{N}\sum_{n=0}^{N-1}\chi_{M}(n).
\]

This quantity measures how frequently digits are exposed in the long run.
Positive density indicates persistent exposure, while $\eta(G)=0$ signals that
arbitrarily long intervals of nonexposure occur.

\subsection{Basic properties}

The invariant $\eta(G)$ has two basic features:

\begin{itemize}
    \item it depends only on the selector stream $M$,
    \item it is unchanged by modifying $M$ beyond any finite index.
\end{itemize}

Thus $\eta$ is a tail invariant.  
It captures large scale structure that cannot be detected by collapse or by any
finite family of continuous observers.

\section{Fluctuation Index}

Let
\[
0 \le n_0 < n_1 < n_2 < \cdots
\]
be the indices at which $M$ exposes digits.  
Define the successive gaps
\[
g_j = n_{j+1} - n_j.
\]

The fluctuation index of $G$ is the quantity
\[
\phi(G)
=
\limsup_{j\to\infty} \frac{g_j}{n_j}.
\]

The ratio $g_j/n_j$ measures the size of the next gap relative to position.
A large value indicates that the selector allows long stretches of inactivity.
Finite values arise when gaps grow at most linearly, while $\phi(G)=\infty$
occurs when gaps grow superlinearly.

\subsection{Basic properties}

Like $\eta$, the invariant $\phi$ depends only on the selector layer.
Tail modification that preserves the selected positions eventually leaves
$\phi$ unchanged.  
Sparse selectors (for example $n_j = j!$ or $n_j = 2^{2^j}$) can produce
arbitrarily large fluctuation indices.

\section{Asymptotic Sensitivity and Nowhere Continuity}

The product topology on $\mathcal{G}$ constrains only finite prefixes.  
The tail may vary arbitrarily inside any nonempty open set.  
As a result, asymptotic invariants such as $\eta$ and $\phi$ are discontinuous
everywhere.

\begin{theorem}[Nowhere continuity of $\eta$]
\label{thm:eta-nowhere}
Let $U$ be a nonempty basic open set in $\mathcal{G}$.  
For every $\alpha\in[0,1]$ there exists an identity $G\in U$ such that
$\eta(G)=\alpha$.
\end{theorem}

\begin{proof}
Let $U$ be determined by a selector prefix $w$ of length $N$.  
Extend $w$ by appending a tail with asymptotic density $\alpha$.  
A periodic tail yields rational $\alpha$; a Sturmian or balanced sequence yields
irrational $\alpha$.  
The prefix contributes $O(1/N)$ to the average, which vanishes in the limit.
Thus the full selector has density $\alpha$.
\end{proof}

\begin{theorem}[Nowhere continuity of $\phi$]
\label{thm:phi-nowhere}
Let $U$ be a nonempty basic open set in $\mathcal{G}$.  
For every $\beta\in[0,\infty]$ there exists an identity $G\in U$ with
$\phi(G)=\beta$.
\end{theorem}

\begin{proof}
Let $U$ be determined by a prefix of length $N$.  
To obtain $\phi(G)=0$, expose every position beyond $N$.  
For finite $\beta>0$, choose $(n_j)$ satisfying
$n_{j+1}\approx(1+\beta)n_j$.  
To obtain $\beta=\infty$, set $n_j=j!$ for large $j$.  
In each case the prefix is preserved and the tail determines the fluctuation
index.
\end{proof}

Both invariants therefore realize all admissible values inside any nonempty
open set.  
Their sensitivity to tail structure reflects the fundamental disconnect between
finite-prefix topology and asymptotic geometry.

\section{Extended Invariants Inside Collapse Fibers}

Collapse fibers contain identities with arbitrarily varied selector behavior.
Fix $x\in[0,1]$.  
The fiber $\mathcal{F}(x)$ contains identities with:

\begin{itemize}
    \item every asymptotic density $\alpha\in[0,1]$,
    \item every fluctuation index $\beta\in[0,\infty]$.
\end{itemize}

To see this, combine tail freedom inside the fiber
(Chapter~\ref{chap:alignment-sewing}) with the constructions of
Theorems~\ref{thm:eta-nowhere} and \ref{thm:phi-nowhere}.  
Given any selector tail achieving the desired invariant value, alignment and
tail sewing allow it to be combined with the collapse coordinate of $x$ while
preserving membership in the fiber.

Thus extended invariants vary freely inside collapse fibers.

\section{Interpretation}

The invariants $\eta$ and $\phi$ measure large scale selector behavior and are
invisible to finite observers.  
They depend exclusively on tail geometry.  
Their nowhere continuity expresses the fact that the product topology governs
finite observation, while these invariants measure infinite scale structure.

This aligns with the indistinguishability theorem of
Chapter~\ref{chap:indistinguishability}.  
Finite observers see only finite prefixes and cannot detect the asymptotic
features encoded by $\eta$ and $\phi$.

\section{Summary}

The asymptotic density $\eta$ and fluctuation index $\phi$ are extended
invariants that detect global selector geometry beyond the reach of collapse
and continuous observation.  
Both are tail dependent, both are discontinuous everywhere, and both take all
admissible values inside any open set.  
Collapse fibers contain identities with arbitrary extended invariant values,
demonstrating the asymptotic richness of generative structure.

These invariants prepare the ground for the geometric study of invariant pairs
developed in the next chapter.
