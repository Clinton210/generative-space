\chapter{Extended Invariants: Asymptotic Density and Fluctuation}
\label{chap:invariants-eta-phi}

\section{Introduction}

The collapse map records the classical real value determined by a generative
identity, but collapse does not reveal the internal structure of the selector
stream. Part IV established that no finite collection of continuous observers
can recover this structure. In this chapter we introduce two extended
invariants that capture large scale features of selector behavior: the
asymptotic density of exposed digits and the fluctuation index of successive
gaps.

Both invariants depend only on the selector stream and measure its
asymptotic behavior. They are tail dependent and therefore invisible to any
fixed finite observer. Their extreme sensitivity to tail modification is a
consequence of the product topology on $\mathcal{X}$: agreement on long
prefixes imposes no restrictions on long term behavior. As a result, both
invariants are discontinuous everywhere and take all admissible values inside
any nonempty open set.

\section{Asymptotic Density}

Let $G = (M,D,K)$ be a generative identity. Define the indicator
\[
\chi_{M}(n)
=
\begin{cases}
1 & \text{if } M(n)=D,\\[3pt]
0 & \text{if } M(n)=K.
\end{cases}
\]
The asymptotic density (or balance) of $G$ is the lower limit
\[
\eta(G)
=
\liminf_{N\to\infty}
\frac{1}{N}\sum_{n=0}^{N-1}\chi_{M}(n).
\]
This invariant measures the long term frequency of exposed digits. Positive
values correspond to selectors with sustained exposure, while $\eta(G)=0$
indicates arbitrarily long intervals of nonselection.

\subsection{Basic properties}

The quantity $\eta(G)$ has two simple features:
\begin{itemize}
    \item it depends only on the selector stream $M$,
    \item it is invariant under modifications of $M$ beyond any finite prefix.
\end{itemize}
Thus $\eta$ is an extended invariant that captures structure outside the reach
of the collapse map.

\section{Fluctuation Index}

Let
\[
0 \le n_{0} < n_{1} < n_{2} < \dots
\]
be the positions where $M(n)=D$. Define the successive gaps
\[
g_{j} = n_{j+1} - n_{j}.
\]
The fluctuation index of $G$ is the upper limit
\[
\phi(G)
=
\limsup_{j\to\infty}\frac{g_{j}}{n_{j}}.
\]
The ratio $g_{j}/n_{j}$ measures the scale adjusted size of the gap following
the $j$th selected digit. Large values indicate that the selector permits long
intervals of nonselection relative to position.

\subsection{Basic properties}

Like $\eta$, the fluctuation index depends only on the selector stream and is
unchanged by tail modifications that preserve the selected positions. Positive
density selectors often yield small fluctuation. Sparse selectors can produce
arbitrarily large fluctuation.

\section{Asymptotic Sensitivity and Nowhere Continuity}

Asymptotic invariants on symbolic streams are almost never continuous in the
product topology. Agreement on finite prefixes places no constraints on tail
behavior. The invariants $\eta$ and $\phi$ illustrate this phenomenon sharply:
both are discontinuous at every point.

\begin{theorem}[Nowhere continuity of $\eta$]
\label{thm:eta-nowhere}
Let $U$ be a nonempty basic open set in $\mathcal{X}$. For every $\alpha$ in
$[0,1]$ there exists an identity $G$ in $U$ with $\eta(G)=\alpha$.
\end{theorem}

\begin{proof}
Let $U$ be determined by a prefix $w$ of length $N$. Extend $w$ by constructing
a selector tail whose asymptotic density equals $\alpha$. For rational
$\alpha$ one may use a periodic sequence; for irrational $\alpha$ one may use
a Sturmian sequence. The prefix contributes a bounded amount to the average,
which vanishes as $N\to\infty$, so the full selector has asymptotic density
$\alpha$.
\end{proof}

\begin{theorem}[Nowhere continuity of $\phi$]
\label{thm:phi-nowhere}
Let $U$ be a nonempty basic open set in $\mathcal{X}$. For every $\beta$ in
$[0,\infty]$ there exists an identity $G$ in $U$ with $\phi(G)=\beta$.
\end{theorem}

\begin{proof}
Let $U$ be determined by a prefix of length $N$. To obtain $\phi(G)=0$ select
every position beyond $N$. To obtain a finite $\beta>0$ place selected digits
at positions approximately satisfying $n_{j+1}\approx(1+\beta)n_{j}$. To
obtain $\beta=\infty$ let $n_{j}=j!$ for large $j$. In all cases the prefix is
respected and the tail determines the fluctuation index.
\end{proof}

These results show that $\eta$ and $\phi$ vary freely inside any open set.
The invariants are asymptotically sensitive: arbitrarily small changes in the
prefix leave room for arbitrary changes in the tail, and therefore arbitrary
values of the invariants.

\section{Extended Invariants Inside Collapse Fibers}

Collapse fibers contain identities with widely varying selector behavior. Fix
a real number $x$. The fiber $\mathcal{F}(x)$ contains identities with:
\begin{itemize}
    \item every possible asymptotic density in $[0,1]$,
    \item every possible fluctuation index in $[0,\infty]$.
\end{itemize}

This follows by combining tail freedom inside collapse fibers with the
constructions of Theorems \texttt{\ref{thm:eta-nowhere}} and
\texttt{\ref{thm:phi-nowhere}}. Any selector stream with infinitely many
selected positions can be combined with the canonical digits of $x$ to create
a valid identity in the fiber. Thus extended invariant values occur densely
throughout the fiber.

\section{Interpretation}

The extended invariants $\eta$ and $\phi$ quantify aspects of generative
structure invisible to collapse. They are determined entirely by the tail and
are unaffected by finite-prefix perturbations. This aligns with the
indistinguishability principle of Chapter
\texttt{\ref{chap:indistinguishability}}: finite observers detect only finite
prefixes and cannot see the asymptotic features described by $\eta$ and
$\phi$.

The discontinuity of these invariants therefore reflects a deep mismatch
between the product topology and the asymptotic geometry of the selector
stream. The product topology governs finite observation. The extended
invariants measure infinite scale structure.

\section{Summary}

The asymptotic density $\eta$ and the fluctuation index $\phi$ are extended
invariants that describe large scale features of selector behavior. Both are
determined by the tail of the selector stream and are discontinuous at every
point of $\mathcal{X}$. Within any open set, all admissible invariant values
occur.

These invariants illustrate the asymptotic richness that survives inside
collapse fibers and motivate the geometric study of invariant pairs in the
next chapter.
