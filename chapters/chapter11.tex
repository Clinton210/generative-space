\chapter{Extended Invariants and Asymptotic Behavior}
\label{chap:invariants}

\section{Introduction}

Finite observers access only finite prefixes of a generative identity, and their behavior is governed by dependency bounds as described in Part~II. 
Limits and limsup constructions taken along observer towers, however, can define coarse quantities that measure global tail behavior. 
These quantities do not arise from collapse or from any single finite observer. 
Instead, they are derived from the long term behavior of the exposure mechanism inside the fixed collapse representation.

This chapter introduces two extended invariants that represent typical end-stage outputs of observer towers:

\begin{itemize}
    \item the asymptotic exposure density, which measures the long run frequency of exposure events, and
    \item the fluctuation index, which measures the relative scale of successive gaps between exposures.
\end{itemize}

Both invariants depend on tail geometry rather than finite prefixes. 
They are discontinuous everywhere in the product topology and can vary arbitrarily within any nonempty open set. 
Their instability reflects the fact that asymptotic invariants sit at the final tier of the framework’s three-level hierarchy: generative identities, collapse values, and observer-derived invariants.

For background on effective closedness of fibers used in later lemmas, see Appendix~A. 
For prefix stabilization and dependency bounds, see Appendix~B. 
For alignment and sewing constructions used to build examples inside a fiber, see Appendices~C and~D.

\section{Asymptotic Exposure Density}

Let $G$ be a generative identity in the ambient symbolic space. 
The representation includes an exposure mechanism that determines which positions contribute to the collapse coordinate. 
Define an indicator
\[
\chi(n)
=
\begin{cases}
1 & \text{if position $n$ is exposed},\\
0 & \text{otherwise}.
\end{cases}
\]

\begin{definition}[Asymptotic Exposure Density]
The asymptotic exposure density of $G$ is
\[
\eta(G)
=
\liminf_{N\to\infty}
\frac{1}{N}
\sum_{n=0}^{N-1} \chi(n).
\]
\end{definition}

This invariant describes the long term frequency of exposures relative to position. 
Positive values correspond to persistent exposure, while $\eta(G)=0$ indicates sparse exposure. 
Since the definition depends only on the tail of the exposure pattern, the quantity is unchanged by modifying any finite prefix.

\section{Fluctuation Index}

Let
\[
0 \le n_0 < n_1 < n_2 < \cdots
\]
denote the increasing sequence of exposure positions. 
Define the successive gaps
\[
g_j = n_{j+1} - n_j.
\]

\begin{definition}[Fluctuation Index]
The fluctuation index of $G$ is
\[
\phi(G)
=
\limsup_{j\to\infty} \frac{g_j}{n_j}.
\]
\end{definition}

The ratio $g_j / n_j$ measures the size of the next gap relative to the current scale. 
Finite values indicate at most linear gap growth, while $\phi(G)=\infty$ arises when gaps grow faster than linearly. 
As with $\eta(G)$, this invariant depends solely on tail behavior.

\section{Discontinuity of Extended Invariants}

The product topology on the ambient space constrains only finitely many coordinates at a time. 
Within any basic open set determined by a finite prefix, the tail remains unrestricted. 
This full tail freedom implies that both extended invariants are discontinuous everywhere.

\begin{theorem}[Nowhere continuity of exposure density]
\label{thm:density-nowhere}
Let $U$ be a nonempty basic open set in the ambient space. 
For every $\alpha \in [0,1]$ there exists $G \in U$ with $\eta(G)=\alpha$.
\end{theorem}

\begin{proof}
Let $U$ be determined by a finite prefix of length $N$. 
Fix $\alpha \in [0,1]$. 
Extend the prefix with a symbolic tail whose exposure frequency has limiting lower density $\alpha$. 
A periodic pattern gives rational $\alpha$, and balanced or Sturmian sequences produce irrational $\alpha$. 
The contribution of the fixed prefix is negligible in the limit. 
Thus the extended identity lies in $U$ and satisfies $\eta(G)=\alpha$.
\end{proof}

\begin{theorem}[Nowhere continuity of fluctuation index]
\label{thm:fluctuation-nowhere}
Let $U$ be a nonempty basic open set. 
For every $\beta \in [0,\infty]$ there exists $G \in U$ with $\phi(G)=\beta$.
\end{theorem}

\begin{proof}
Let $U$ be specified by a finite prefix of length $N$. 
Fix $\beta$. 
If $\beta=0$, expose all positions beyond $N$. 
If $\beta$ is finite and positive, choose a sequence $(n_j)$ satisfying $n_{j+1}\approx (1+\beta)n_j$ for large $j$. 
If $\beta=\infty$, choose $n_j = j!$ or another superlinear sequence. 
These constructions determine the tail and keep the prefix fixed, so the resulting identity lies in $U$ and has the desired fluctuation index.
\end{proof}

Both results express that asymptotic invariants are unstable under the product topology and that finite-prefix constraints do not restrict their long term values. 
This instability is characteristic of Baire class 1 functions arising as limits of continuous observers, as described in Chapter~\ref{chap:asymptotic-observers}.

\section{Behavior Inside Collapse Fibers}

Fix a real number $x$. 
The collapse fiber $\mathcal{F}(x)$ consists of all identities whose exposed-value coordinate realizes the chosen representation of $x$. 
Fibers possess full tail freedom by construction, and alignment with sewing (Appendices~C and~D) allows arbitrary tail patterns to be grafted onto a fixed prefix without leaving the fiber.

\begin{theorem}[Extended invariant variation inside fibers]
\label{thm:fiber-variation}
Let $x \in [0,1]$. 
For any $\alpha \in [0,1]$ and any $\beta \in [0,\infty]$ there exists $G \in \mathcal{F}(x)$ such that
\[
\eta(G)=\alpha,
\qquad
\phi(G)=\beta.
\]
\end{theorem}

\begin{proof}
Begin with any reference identity $H \in \mathcal{F}(x)$. 
Using alignment (Appendix~C), identify an exposure index sufficiently far along the collapse coordinate so that modifying the tail preserves the observed-value sequence. 
Construct a tail realizing the desired density or fluctuation value using the methods of Theorems~\ref{thm:density-nowhere} and~\ref{thm:fluctuation-nowhere}. 
Sew this tail to the aligned prefix using the method of Appendix~D. 
The resulting identity lies in the fiber and has the specified invariant values.
\end{proof}

This theorem shows that extended invariants do not classify collapse fibers. 
Every fiber contains identities with the full range of asymptotic behavior.

\section{Interpretation}

Extended invariants arise as limits or limsup values of observer towers rather than as quantities accessible to finite observers. 
They sit one level beyond continuous projections in the framework’s hierarchy. 
Their instability reflects this derived nature:

\begin{itemize}
    \item they depend on tail geometry invisible to any finite-prefix observer,
    \item they are not continuous and are sensitive to arbitrarily small changes in the tail, and
    \item they cannot classify collapse fibers because fibers contain full tail freedom.
\end{itemize}

Thus the invariants measure coarse asymptotic structure but do not recover the identity. 
Their behavior illustrates the limitations of representation-based observation and justifies treating them as shadows of the observer layer rather than as intrinsic coordinates.

\section{Summary}

Extended invariants such as asymptotic exposure density and fluctuation index quantify large scale symbolic behavior in the chosen collapse representation. 
They are tail dependent, discontinuous everywhere, and vary freely inside any nonempty basic open set. 
Collapse fibers contain identities with every possible invariant value, illustrating that these quantities cannot classify generative structure.

These observations motivate the study of invariant pairs and other derived summaries developed in the next chapter.
