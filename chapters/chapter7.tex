\chapter{A Menagerie of Secondary Coordinates}

\section{Introduction}

Chapter~6 introduced secondary projections as computable or continuous functionals on the generative space.  
These projections serve as coordinate systems that summarize internal features of a generative identity.  
Although collapse is the only primary invariant, secondary projections reveal structural patterns that classical magnitude ignores.  
This chapter develops several families of such projections, including digit frequencies, meta frequencies, variation statistics, and mixer complexity.  
The examples illustrate how different projections offer distinct and sometimes incompatible perspectives on the same hybrid or ghost identity.

These coordinate systems are not intrinsic to the generative space.  
Each is a choice of measurement, and distinct choices highlight different properties of a generative identity.  
The resulting diversity of views is called the Rashomon effect, which becomes central in the diagonalizer construction of Chapter~8.

\section{Frequency-Based Coordinates}

Digit and meta frequencies provide simple but informative secondary coordinates.  
They record how often each layer contributes to the canonical output.

\begin{definition}[Digit Frequency Vector]
For an identity $G = (M,D,K)$ with digit density $\eta(G) > 0$, the \emph{digit frequency vector} is
\[
\mathbf{f}_D(G) =
\left(
\lim_{n \to \infty}
\frac{1}{n}
\bigl| \{ 0 \le k < n : M(k)=D \text{ and } D(k)=i \} \bigr|
\right)_{i=0}^{b-1},
\]
whenever the limits exist.
\end{definition}

A similar definition applies to the meta layer.

\begin{definition}[Meta Frequency Vector]
The \emph{meta frequency vector} is
\[
\mathbf{f}_K(G) =
\left(
\lim_{n \to \infty}
\frac{1}{n}
\bigl| \{ 0 \le k < n : M(k)=K \text{ and } K(k)=a \} \bigr|
\right)_{a \in \Sigma}.
\]
\end{definition}

These coordinates are continuous and computable whenever the limits are computable, and they provide natural summaries of long-term behavior in the hybrid regime.

\begin{remark}
Ghost identities, introduced in Chapter~5, often have undefined digit frequency vectors because their digit density is zero.  
Their meta frequency vectors are typically well defined and informative.
\end{remark}

\section{Entropy and Local Variation Coordinates}

Entropy-type quantities summarize the complexity of the canonical output or of the mixer pattern.  
They capture variation rather than frequency.

\begin{definition}[Block Entropy Coordinate]
Fix $m \ge 1$.  
The \emph{$m$-block entropy} of $G$ is
\[
H_m(G)
=
-
\sum_{w \in A^m}
p_G(w) \log p_G(w),
\]
where $A$ is the set of possible symbols in the canonical output and $p_G(w)$ is the limiting frequency of the block $w$ of length $m$, when the limit exists.
\end{definition}

Block entropy is continuous on $\mathcal{X}$ when defined and computable on $\mathcal{G}_{\mathrm{eff}}$ when the limiting frequencies are computable.  
These coordinates often distinguish identities that have identical digit or meta frequencies.

A simpler coordinate summarizes local changes.

\begin{definition}[Local Variation Statistic]
For $G = (M,D,K)$ define
\[
V(G) 
=
\limsup_{n \to \infty}
\frac{1}{n}
\bigl| \{ 0 \le k < n : X(G)_k \ne X(G)_{k+1} \} \bigr|.
\]
\end{definition}

This coordinate detects switching patterns and can distinguish many hybrid identities with identical digit densities.

\section{Mixer Complexity}

The mixer itself carries structural information that can be measured through secondary projections.  
These measures quantify how the mixer interleaves the digit and meta layers.

\begin{definition}[Mixer Complexity Coordinate]
Let $\chi$ be an encoding of $M$ as a binary sequence.  
The \emph{mixer complexity} is
\[
C_M(G)
=
\limsup_{n \to \infty}
\frac{1}{n}
K_U(\chi{\upharpoonright}n),
\]
where $K_U$ denotes prefix-free Kolmogorov complexity with respect to a fixed universal machine.
\end{definition}

Although Kolmogorov complexity is not computable, bounded or monotone versions yield computable secondary projections that summarize mixer variation.  
These projections capture algorithmic structure that frequency and entropy coordinates may miss.

\begin{remark}
Mixer complexity is often low for ghost identities and highly variable for hybrid identities.  
This contrast becomes important in Chapter~8, where tail modifications to the mixer are used to neutralize families of projections.
\end{remark}

\section{The Rashomon Effect}

Different secondary projections may give conflicting impressions of the same generative identity.  
This phenomenon reflects the fact that each projection samples a particular aspect of the structure of $(M,D,K)$.

\begin{proposition}[Rashomon Effect]
Let $G$ be any hybrid or ghost identity.  
There exist secondary projections $\Phi_1$ and $\Phi_2$ such that
\[
\Phi_1(G) \ne \Phi_2(G)
\]
and for which $\Phi_1$ and $\Phi_2$ emphasize incompatible aspects of $(M,D,K)$.
\end{proposition}

\begin{proof}
Choose $\Phi_1$ to be a digit frequency vector and $\Phi_2$ to be an entropy coordinate or mixer complexity coordinate.  
Hybrid and ghost identities can be constructed to have identical digit frequencies but distinct entropy statistics or distinct mixer complexity.  
Conversely, identities with identical entropy may be constructed to have distinct digit frequencies.  
Explicit examples are provided in Appendix~C.
\end{proof}

The Rashomon effect emphasizes the inherent incompleteness of any finite coordinate system.  
Different summaries cannot be reconciled into a single invariant without discarding essential internal structure.  
This sets the stage for the structural incompleteness results of Part~IV.

\section{Case Studies and Examples}

Appendix~C contains explicit examples of hybrid and ghost identities that demonstrate the contrasts among digit frequency vectors, meta frequency vectors, entropy coordinates, and mixer complexity.  
These examples illustrate how coordinate choices reflect specific structural features of generative identities.

\section{Outlook}

This chapter illustrates the diversity of secondary coordinate systems and the multiplicity of perspectives they provide on collapse fibers.  
Chapter~8 uses the finite-prefix dependency bounds from Chapter~6 and the diversity of coordinates from this chapter to construct the meta diagonalizer, which evades any finite family of secondary projections.  
The resulting impossibility theorem in Chapter~9 formalizes structural incompleteness in $\mathcal{G}_{\mathrm{eff}}$.
