\chapter{Projective Incompatibility}
\label{chap:projective-incompatibility}

\section{Introduction}

Structural projections extract different aspects of a generative identity.
Some measure digit frequencies, others examine spacing patterns, and others
recover classical information through collapse. Although each projection
depends only on a finite prefix of the generative identity at any prescribed
precision, their finite-prefix requirements may conflict.

This chapter develops a formal notion of such conflicts. Distinct observers
often demand incompatible local structures from the selector or digit streams.
These incompatibilities arise from the way dependency bounds determine the
finite windows that observers examine, and they play a central role in the
structural indistinguishability results of Chapter
\texttt{\ref{chap:indistinguishability}}. They express the basic limitation
that no finite window can satisfy all observers simultaneously.

The phenomenon is analogous to familiar situations in symbolic dynamics, where
different combinatorial or ergodic invariants require incompatible blocks to
appear in a shift space. Here the same principle applies to observational
functionals rather than to subshifts.

\section{Observer Requirements}

Let $\Phi$ and $\Psi$ be two structural projections with dependency bounds
$B_{\Phi}$ and $B_{\Psi}$. Fix a precision $\varepsilon > 0$. To approximate
$\Phi(G)$ and $\Psi(G)$ within $\varepsilon$, we must satisfy
\[
G[0..B_{\Phi}(\varepsilon)] \text{ determines } \Phi(G),
\qquad
G[0..B_{\Psi}(\varepsilon)] \text{ determines } \Psi(G).
\]
If the projections extract unrelated forms of structural information, these
finite windows may demand incompatible patterns.

\subsection*{Example: density versus spacing}

Let $\Phi$ be a digit density projection and let $\Psi$ be a spacing or
fluctuation projection. To approximate $\Phi(G)$ with small error, the initial
prefix must contain many positions where $M(n) = D$. To approximate $\Psi(G)$
with small error, the prefix must contain a long interval in which
$M(n)=K$, so that the gap ratio can be measured accurately. A single prefix
cannot realize both requirements simultaneously.

This type of conflict is common in combinatorics on words. Local constraints
on symbol frequencies and local constraints on block lengths do not always
admit a common finite witness.

\section{Formal Definition of Incompatibility}

\begin{definition}[Projective incompatibility]
Two projections $\Phi$ and $\Psi$ are incompatible at precision $\varepsilon$
if no prefix of length
\[
L = \max\bigl\{ B_{\Phi}(\varepsilon), B_{\Psi}(\varepsilon) \bigr\}
\]
can simultaneously satisfy the finite-prefix requirements needed to approximate
both projections to within $\varepsilon$ at their target values.
\end{definition}

Incompatibility therefore expresses a finite informational impossibility:
within the window $[0..L]$ the observers demand conflicting symbolic patterns.

\section{Concrete Instances}

The density versus spacing example above is representative. Let
$N_{\Phi} = B_{\Phi}(\varepsilon)$ and $N_{\Psi} = B_{\Psi}(\varepsilon)$.
If we examine the prefix $[0..L]$ with
$L = \max(N_{\Phi}, N_{\Psi})$, then a density requirement may force many
selected positions in this interval, whereas a spacing requirement may force a
long unselected block. These demands cannot be met by the same prefix.

This incompatibility is entirely local. It depends only on the finite window
used by each observer, not on the behavior of the generative identity outside
this window.

\section{Finite Families of Observers}

Finite families of projections may also contain internal conflicts.

\begin{proposition}
Let $\mathcal{P}$ be a finite family of projections. If $\mathcal{P}$
contains two projections that are incompatible at precision $\varepsilon$,
then no prefix of length
\[
B_{\mathcal{P}}(\varepsilon)
=
\max_{\Phi \in \mathcal{P}} B_{\Phi}(\varepsilon)
\]
can satisfy all projections in the family at precision $\varepsilon$.
\end{proposition}

\begin{proof}
Any prefix satisfying the family must satisfy each member individually. If two
members of the family impose incompatible requirements on the prefix, there is
no prefix that satisfies all of them.
\end{proof}

Thus incompatibility propagates across finite families of observers.

\section{Lack of Interval Structure in Projective Images}

In earlier versions of this chapter it was natural to expect that the image of
a collapse fiber under a projection forms an interval. This is not the case.
Continuous maps from zero-dimensional compact spaces can have highly
disconnected images (see \cite{Kechris}). Therefore no general structural
statement can be made about $\Phi(\mathcal{F}(x))$ beyond continuity.

This observation reinforces the finite-prefix viewpoint: projection behavior is
governed by dependency bounds, not by global geometric structure of the fiber.

\section{Implications for Indistinguishability}

Prefix incompatibility has a direct consequence for observational limits.
Since different observers require different finite windows, the prefixes needed
to satisfy growing families of observers also grow. This monotonic expansion
of prefix requirements implies that observers reveal only finite structural
information about generative identities.

In Chapter \texttt{\ref{chap:indistinguishability}} we use this observation to
construct identities that agree with a reference identity on all relevant
prefixes for any finite family of observers, while differing in their
tails. This shows that generative structure beyond these prefixes remains
undetectable.

\section{Summary}

Different structural projections impose distinct finite-prefix constraints. When
these constraints cannot be realized by a single prefix, the projections are
incompatible. This incompatibility is a local symbolic phenomenon and reflects
the fact that observers examine finite windows of the generative identity at
finite precision. These finite-prefix effects provide the conceptual foundation
for the indistinguishability results of Chapter
\texttt{\ref{chap:indistinguishability}}.
