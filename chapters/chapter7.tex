\chapter{Projective Incompatibility}

\section{Introduction}

Chapter~6 introduced secondary projections as computable functionals on the generative space $\mathcal{X}$ and established their inherent limitation: each projection can observe only a finite prefix of an effective identity. While the collapse map $\pi$ reduces the entire mechanism to its classical magnitude, secondary projections provide partial coordinate systems that summarize internal structural features of a generative identity.

This chapter develops concrete examples of secondary projections, including frequency statistics, entropy measures, and selector-based complexity. These examples illustrate that different projections highlight different aspects of the same identity, often in mutually incompatible ways. The tension among these coordinate systems is formalized as \emph{projective incompatibility}: no single finite family of computable projections can capture the full structure of even a single collapse fiber.

\begin{remark}
Informally, this phenomenon is sometimes compared to the ``Rashomon effect,'' in which different observers give incompatible accounts. In the generative framework, the incompatibility is objective: it arises from the orthogonality of computable projections and the vast size of collapse fibers.
\end{remark}

\section{Frequency-Based Coordinates}

One of the simplest forms of secondary information is the asymptotic distribution of symbols. Because secondary projections must be continuous and computable, we use limsup versions of frequencies to avoid undefined limits.

\begin{definition}[Digit Frequency Vector]
Let $G = (M,D,K)$ with digit density $\eta(G) > 0$.  
The \emph{digit frequency vector} of $G$ is the vector
\[
\mathbf{f}_D(G)
=
\left(
\limsup_{n\to\infty}
\frac{1}{n}
\left|
\{ 0 \le k < n : M(k)=D,\; D(k)=i \}
\right|
\right)_{i=0}^{b-1}.
\]
If $\eta(G)=0$, the digit frequency vector is defined to be the zero vector.
\end{definition}

\begin{definition}[Meta Frequency Vector]
The \emph{meta frequency vector} of $G$ is
\[
\mathbf{f}_K(G)
=
\left(
\limsup_{n\to\infty}
\frac{1}{n}
\left|
\{ 0 \le k < n : M(k)=K,\; K(k)=a \}
\right|
\right)_{a\in\Sigma}.
\]
\end{definition}

Both vectors are computable whenever $G$ belongs to the effective core. These coordinates measure large-scale symbol prevalence but do not distinguish between sequences with identical symbol counts arranged in different patterns.

\section{Entropy and Local Variation}

To detect structural differences that frequency statistics ignore, we consider entropy and variation-based projections derived from the canonical output.

\begin{definition}[Block Entropy Coordinate]
Fix $m \ge 1$.  
Let $A$ denote the alphabet of the canonical output $X(G)$.  
The $m$-block entropy of $G$ is
\[
H_m(G)
=
-
\sum_{w\in A^m}
p_G(w)\,\log p_G(w),
\]
where $p_G(w)$ is the upper limiting frequency of the block $w$ in $X(G)$.
\end{definition}

Block entropy captures the multiplicity of admissible patterns of length $m$.  
Hybrid and null-density generators may produce canonical outputs with dramatically different entropy profiles, even when they collapse to the same real number.

\begin{definition}[Local Variation Statistic]
The \emph{local variation} of $G$ is
\[
V(G)
=
\limsup_{n\to\infty}
\frac{1}{n}
\bigl|
\{\, 0 \le k < n : X(G)_k \neq X(G)_{k+1} \,\}
\bigr|.
\]
\end{definition}

Variation measures the frequency of symbol changes.  
It is sensitive to how often the selector $M$ switches between layers, making it a natural projection for distinguishing hybrids from sparse patterns.

\section{Selector-Based Complexity}

The selector sequence $M$ is itself a source of structural variation.  
While Kolmogorov complexity is not computable, we can employ computable compression schemes as continuous proxies.

\begin{definition}[Selector Complexity]
Let $\mathcal{C}$ be a fixed computable compression algorithm.  
Let $\chi_n$ be the binary encoding of $M(0),\dots,M(n-1)$.  
The \emph{selector complexity} of $G$ is
\[
C_M(G)
=
\limsup_{n\to\infty}
\frac{|\mathcal{C}(\chi_n)|}{n}.
\]
\end{definition}

This coordinate discriminates between periodic, low-complexity selectors and selectors with pseudo-random or combinatorially rich structure.

\begin{remark}
Null-density generators may have low selector complexity (e.g., $M(k)=D$ only when $k=j^2$), while hybrid identities can achieve arbitrarily high selector complexity.  
This divergence plays a key role in Chapter~8.
\end{remark}

\section{Projective Incompatibility}

We now state the fundamental incompatibility theorem.  
It articulates that although any two effective identities in a fiber can be distinguished by an appropriately chosen projection, no \emph{fixed} finite family of projections can distinguish every pair.

\begin{theorem}[Projective Incompatibility]
\label{thm:projective-incompatibility}
Let $x\in[0,1]$ and consider the effective fiber $\mathcal{F}_{\mathrm{eff}}(x)$.

\begin{enumerate}
\item For any two distinct $G,H \in \mathcal{F}_{\mathrm{eff}}(x)$, there exists a computable secondary projection $\Phi$ such that $\Phi(G)\neq \Phi(H)$.

\item For every finite family of computable secondary projections $\mathcal{P}$, there exist distinct $G',H' \in \mathcal{F}_{\mathrm{eff}}(x)$ such that
\[
\Phi(G') = \Phi(H') \quad\text{for all } \Phi\in\mathcal{P}.
\]
\end{enumerate}
\end{theorem}

The first part asserts the richness of the internal structure of a fiber: every distinction can be captured by some computable projection.  
The second part asserts that no \emph{finite} set of projections can capture all such distinctions.  
Together these statements formalize projective incompatibility.

\begin{example}
Let $G$ be a hybrid generator with $\eta(G)>0$ and let $H$ be a null-density generator with $\eta(H)=0$ collapsing to the same $x$.

\begin{itemize}
    \item Under digit density, $G$ and $H$ are easily distinguished.
    \item Under meta entropy, one may construct $H$ so that $H_m(G)=H_m(H)$ for several block lengths.
    \item Under selector complexity, $G$ may exhibit high complexity while $H$ is nearly low-complexity.
\end{itemize}

These conflicting observations illustrate the structural incompatibility of projections.
\end{example}

\section{Outlook}

This chapter has introduced concrete secondary projections and demonstrated their mutual incompatibility.  
The key structural insight is that collapse fibers are too large and too varied to be fully resolved by any finite set of computable coordinates.

In Chapter~8 we construct the \emph{meta-diagonalizer}, a generative identity that exploits dependency bounds and finite lookahead to evade any finite set of projections.  
This construction yields the Structural Incompleteness Theorem, the central result of Part~IV.
