\chapter{Projective Incompatibility}
\label{chap:projective-incompatibility}

\section{Introduction}

Structural projections extract different aspects of a generative identity.  
Some examine digit statistics in the collapse coordinate, others evaluate
spacing patterns in the selector layer, and others recover classical
information through collapse.  
Although each observer depends only on a finite prefix at any chosen
precision, the finite-prefix requirements of different observers may conflict.
These conflicts express a basic limitation: no finite window can realize all
local structural patterns demanded by all observers simultaneously.

This chapter formalizes this phenomenon.  
Projective incompatibility arises when two observers require incompatible
finite blocks within the prefix windows determined by their dependency bounds.
The idea is analogous to classical situations in symbolic dynamics, where
constraints on block frequencies and constraints on block lengths or gap
patterns cannot always be satisfied by the same word  
\cite{LindMarcus}.  
Here the conflicting requirements arise not from shift-space admissibility
conditions but from the local demands of continuous observers.

Projective incompatibility is a key ingredient for the indistinguishability
results of Chapter~\ref{chap:indistinguishability}.  
It highlights that finite-prefix structure, rather than global geometry,
governs what observers can and cannot distinguish.

\section{Observer Requirements}

Let $\Phi$ and $\Psi$ be structural projections on $\mathcal{G}^*$ with
dependency bounds $B_\Phi$ and $B_\Psi$.  
Fix $\varepsilon>0$.  
To evaluate the observers within error $\varepsilon$, the identity must
satisfy
\[
G[0..B_\Phi(\varepsilon)] \text{ determines } \Phi(G), \qquad
G[0..B_\Psi(\varepsilon)] \text{ determines } \Psi(G).
\]

If $\Phi$ and $\Psi$ extract unrelated structural features, the patterns
required in these finite windows may contradict one another.

\subsection*{Example: density versus spacing}

Let $\Phi$ measure the lower asymptotic density of selected digits and let
$\Psi$ measure a spacing or fluctuation invariant of the selector.  
Approximating $\Phi$ requires many positions with $M(n)=D$ in the initial
window.  
Approximating $\Psi$ requires a long block where $M(n)=K$ so that a large gap
ratio can be observed.  
A single finite prefix cannot meet both requirements simultaneously.

This mirrors classical conflicts between block-frequency and block-length
constraints in combinatorics on words, where no finite word can satisfy two
incompatible local prescriptions.

\section{Formal Definition of Incompatibility}

\begin{definition}[Projective Incompatibility]
Let $\Phi$ and $\Psi$ be structural projections with finite precision
$\varepsilon>0$.  
Set
\[
L = \max\{ B_\Phi(\varepsilon),\, B_\Psi(\varepsilon) \}.
\]
The projections are \emph{incompatible at precision $\varepsilon$} if no
prefix of length $L$ can simultaneously satisfy the finite-prefix conditions
required by both observers to achieve error less than $\varepsilon$.
\end{definition}

Incompatibility is therefore a \emph{local} impossibility: within the window
$[0..L]$ no single block can satisfy the joint structural demands of the two
observers.

The notion depends only on the observers and the chosen precision, not on any
particular generative identity.

\section{Concrete Instances}

Let $N_\Phi=B_\Phi(\varepsilon)$ and $N_\Psi=B_\Psi(\varepsilon)$, and set
$L=\max(N_\Phi,N_\Psi)$.

A density-based observer may require that within $[0..L]$ the selector
contains many exposures.  
A spacing observer may require a long unbroken run of non-exposures.  
If the requirements contradict one another, then the observers are
incompatible at precision $\varepsilon$.

The phenomenon is purely local.  
It depends only on the window required by each observer, not on the behavior
of the identity outside it.

\section{Finite Families of Observers}

Incompatibility extends naturally to finite families.

\begin{proposition}
Let $\mathcal{P}$ be a finite family of structural projections.  
If two members $\Phi,\Psi\in\mathcal{P}$ are incompatible at precision
$\varepsilon$, then no prefix of length
\[
B_{\mathcal{P}}(\varepsilon)
  = \max_{\Theta\in\mathcal{P}} B_\Theta(\varepsilon)
\]
can satisfy all projections in $\mathcal{P}$ at that precision.
\end{proposition}

\begin{proof}
Any prefix that satisfies the family must satisfy each projection individually.
If two projections impose incompatible requirements on the prefix of length
$B_{\mathcal{P}}(\varepsilon)$, no such prefix exists.
\end{proof}

Thus incompatibility propagates across finite observer families, which is
crucial for diagonalization.

\section{Disconnected Projective Images}

One might expect that for a structural projection $\Phi$, the image
$\Phi(\mathcal{F}(x))$ is an interval.  
This is false.  
Since collapse fibers are compact, totally disconnected spaces, continuous
images of such spaces may be highly disconnected  
\cite{Kechris}.  
No general convexity or interval structure can be expected.

This reinforces that projection behavior is determined by finite-prefix
constraints rather than by global geometric features of the fiber.

\section{Implications for Indistinguishability}

As the number of observers grows, their combined dependency bounds require an
ever-longer initial prefix to satisfy all of them simultaneously.  
No finite identity prefix can satisfy infinitely many constraints.  
This limitation drives the indistinguishability phenomenon: for any finite
family of observers, one can construct identities agreeing with a reference
identity on all relevant prefixes while diverging in their tails.

This mechanism underlies the alignment and sewing arguments developed in
Chapter~\ref{chap:indistinguishability} and is central to the
meta-diagonalizer of Part~\ref{part:incompleteness}.

\section{Summary}

Structural projections examine finite windows of generative identities.  
When two observers demand incompatible structural patterns within the same
window, they are projectively incompatible.  
This incompatibility is a finite, local obstruction determined by dependency
bounds.  
It propagates across families of observers and reveals a fundamental
limitation of continuous observation: finite windows cannot encode all
structural information.

These ideas form the conceptual foundation for the indistinguishability
results of the next chapter.
