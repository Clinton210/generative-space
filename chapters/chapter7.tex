\chapter{Secondary Projections and Finite Lookahead}

\section{Introduction}

Chapter~6 introduced structural projections as continuous, prefix-determined
maps from the generative space $\mathcal{X}$ into metric spaces.  
These projections model what an observer can discern from a generative identity
based on finite information at each stage.  
Among all such projections, those that are \emph{effective} or
\emph{computationally realizable} play a central role in the structural
incompleteness phenomenon of Part~IV.

This chapter develops the theory of \emph{secondary projections}: computational
observers that operate under finite lookahead and must decide their outputs on
the basis of bounded prefixes of the generative identity.  
These projections formalize the idea that a computable measurement can only
inspect finitely many coordinates of $(M,D,K)$ before committing to an output
value.

We begin by formalizing dependency bounds, which restrict how far a projection
may look into a mechanism.  
We then show that secondary projections are continuous, prefix-determined maps
that fit naturally into the projection lattice of Chapter~6.  
Finally, we connect finite lookahead to stabilization properties that will be
exploited by the meta-diagonalizer of Chapter~9.

\section{Dependency Bounds}

A computable observer cannot examine an unbounded portion of the mechanism when
deciding its output on a given precision scale.  
The dependency of a projection at resolution $\varepsilon$ must therefore be
limited by a computable function.

\begin{definition}[Dependency Bound]
Let $\Phi : \mathcal{X} \to \mathbb{R}$ be a structural projection.  
A \emph{dependency bound} for $\Phi$ is a function
\[
B_\Phi : (0,\infty) \to \mathbb{N}
\]
such that for all $G,H \in \mathcal{X}$ and all $\varepsilon > 0$, if
\[
(M_G{\upharpoonright}B_\Phi(\varepsilon),\,
 D_G{\upharpoonright}B_\Phi(\varepsilon),\,
 K_G{\upharpoonright}B_\Phi(\varepsilon))
=
(M_H{\upharpoonright}B_\Phi(\varepsilon),\,
 D_H{\upharpoonright}B_\Phi(\varepsilon),\,
 K_H{\upharpoonright}B_\Phi(\varepsilon)),
\]
then
\[
|\Phi(G) - \Phi(H)| < \varepsilon.
\]
\end{definition}

Dependency bounds express the idea that agreement on a finite prefix suffices to
determine the projection’s output to within a prescribed tolerance.

\begin{remark}
For computable projections, the function $B_\Phi$ must itself be computable.
This aligns with the Type--2 framework used to analyze $\mathcal{G}_{\mathrm{eff}}$.
\end{remark}

\section{Computable Structural Projections}

We now restrict attention to projections that can be computed with finite
lookahead at any precision level.

\begin{definition}[Computable Structural Projection]
A projection $\Phi : \mathcal{X} \to \mathbb{R}$ is a \emph{computable structural
projection} if:
\begin{enumerate}
    \item $\Phi$ is continuous (prefix-determined), and
    \item $\Phi$ admits a computable dependency bound $B_\Phi$.
\end{enumerate}
\end{definition}

Examples include:

\begin{itemize}
    \item digit-frequency maps that estimate the proportion of positions where
    $M(k)=D$,
    \item meta-frequency projections based on the limiting behavior of $K(n)$,
    \item pattern detectors that check whether certain finite blocks occur
    infinitely often,
    \item maps associated with effective limit-average or limsup conventions.
\end{itemize}

These observers represent the algorithmic analogue of the general structural
projections introduced in Chapter~6.

\section{Finite-Prefix Stabilization}

A secondary projection must stabilize on any effective mechanism: once the
prefix is long enough, the observer’s output changes only within arbitrarily
small tolerances.

\begin{proposition}[Prefix Stabilization]
\label{prop:prefix-stabilization}
Let $\Phi$ be a computable structural projection with dependency bound
$B_\Phi$.  
For any $G \in \mathcal{X}$ and any sequence $H_n \in \mathcal{X}$ satisfying
\[
H_n{\upharpoonright}B_\Phi(\varepsilon)
=
G{\upharpoonright}B_\Phi(\varepsilon)
\]
for all sufficiently large $n$, we have
\[
\Phi(H_n) \to \Phi(G).
\]
\end{proposition}

\begin{proof}
Fix $\varepsilon>0$.  
By the definition of $B_\Phi$, agreement on the prefix of length
$B_\Phi(\varepsilon)$ forces the projection values to lie within $\varepsilon$.  
Thus $\Phi(H_n)$ lies in the $\varepsilon$-ball around $\Phi(G)$ for all
sufficiently large $n$.  
Since $\varepsilon$ is arbitrary, the claimed convergence follows.
\end{proof}

Prefix stabilization is the key computational constraint that the
meta-diagonalizer exploits in Part~IV.  
If a family of projections shares compatible dependency bounds, then they all
stabilize on sufficiently long prefixes of a given mechanism.

\section{Secondary Coordinates as Projections}

Many quantities of interest in the generative framework arise as secondary
coordinates defined via observables on the selector or meta layer.  
These coordinates are naturally realized as computable structural projections.

\begin{example}[Digit Density Estimates]
Let $F_n(G)$ denote the digit frequency in the first $n$ positions:
\[
F_n(G)
=
\frac{1}{n}|\{\,0 \le k < n : M(k)=D\,\}|.
\]
For fixed $n$, the map $G \mapsto F_n(G)$ depends only on the prefix
$M{\upharpoonright}n$ and is therefore a computable structural projection.
\end{example}

\begin{example}[Meta-Pattern Indicators]
For a fixed block $w \in \Sigma^k$, define $\Phi_w(G)$ to be $1$ if $w$ appears
in $K$ infinitely often and $0$ otherwise.  
Determining $\Phi_w(G)$ requires only checking sufficiently long prefixes to
decide whether occurrences of $w$ continue; it is therefore a secondary
projection with a computable dependency bound.
\end{example}

These examples illustrate that secondary coordinates fit squarely into the
projection lattice of Chapter~6, but occupy the computationally accessible
region of that lattice.

\section{Interaction with the Projection Lattice}

Computable structural projections inherit the lattice operations of
Chapter~6.  
In particular:

\begin{itemize}
    \item the meet $\Phi \wedge \Psi$ is computable if $\Phi$ and $\Psi$ are,
    since its output is the pair $(\Phi(G),\Psi(G))$,
    \item finite joins of computable projections are computable,
    \item dependency bounds combine effectively:
    \[
    B_{\Phi \wedge \Psi}(\varepsilon)
    =
    \max\{\, B_\Phi(\varepsilon), B_\Psi(\varepsilon) \,\}.
    \]
\end{itemize}

However, unlike the full lattice, not every supremum of computable projections
is computable: infinite joins may fail to admit a uniform computable dependency
bound.

This limitation plays an important role in the incompatibility phenomena
developed in Chapter~8 and is a precursor to the diagonalization argument of
Chapter~9.

\section{Outlook}

Secondary projections model the behavior of effective observers in the
generative framework.  
They operate with finite lookahead, stabilize on sufficiently long prefixes, and
fit naturally into the lattice of structural projections.

Chapter~8 develops the phenomenon of \emph{projective incompatibility}: different
projection families can impose incompatible prefix constraints, preventing any
single mechanism from satisfying all of them simultaneously.  
This conflict drives the diagonalizer of Chapter~9, which systematically evades
finite families of computable projections by exploiting the tail freedom inherent
in the generative space.
