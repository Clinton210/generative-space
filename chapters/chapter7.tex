\chapter{Projective Incompatibility}

\section{Introduction}

Structural projections measure different aspects of a generative identity.
Some observe digit frequencies, others observe spacing patterns, and others
extract classical information through collapse.  
Although each projection examines only a finite prefix at a fixed precision,
their requirements may conflict.  
It may be impossible for a single finite prefix of an identity to satisfy the
demands of multiple projections at once.

This chapter formalizes this notion of conflict.  
We show that distinct observers often require incompatible finite structures
from the selector and digit streams.  
This incompatibility is a central ingredient in the construction of the
meta-diagonalizer in Part IV, where controlled divergence from reference
identities is enforced by exploiting these conflicting constraints.

The conceptual roots of this phenomenon appear in symbolic dynamics, where
different ergodic or combinatorial invariants may demand incompatible blocks
to appear in a shift space.  
Here the same idea arises in the generative setting but is applied directly to
observational functionals rather than to subshifts.

\section{Observer Requirements}

Let $\Phi$ and $\Psi$ be two structural projections with dependency bounds
$B_\Phi$ and $B_\Psi$.  
Fix a desired precision $\varepsilon > 0$.  
Then any identity $G$ must satisfy:
\[
\Phi(G) \text{ determined by } G[0..B_\Phi(\varepsilon)], 
\quad
\Psi(G) \text{ determined by } G[0..B_\Psi(\varepsilon)].
\]

If the projections measure unrelated aspects of the identity, these
finite prefixes may be required to contain contradictory patterns.

\subsection*{Example: density versus spacing}

Digit density projections may require the initial prefix to contain many
positions where $M(n) = D$.  
Conversely, spacing projections (such as fluctuation observers) may require
long runs where $M(n) = K$ in order to witness large gaps between selected
positions.  
A single finite block cannot simultaneously exhibit both high density and
large gaps at the same index range.

This tension reflects a basic fact from combinatorics on words: local
constraints on symbol frequencies and local constraints on block lengths need
not be simultaneously satisfiable.

\section{Formal Definition of Incompatibility}

\begin{definition}[Projective Incompatibility]
Two projections $\Phi$ and $\Psi$ are \emph{incompatible at precision}
$\varepsilon$ if no prefix of length
\[
L = \max\{ B_\Phi(\varepsilon), B_\Psi(\varepsilon) \}
\]
can satisfy both of the following:
\begin{enumerate}
    \item the prefix forces $\Phi(G)$ to lie within $\varepsilon$ of its target
          value, and
    \item the prefix forces $\Psi(G)$ to lie within $\varepsilon$ of its target
          value.
\end{enumerate}
\end{definition}

Incompatibility expresses a structural impossibility: the observers demand
conflicting patterns in the same finite window.

\section{Concrete Instances}

Although incompatibility is common in practice, one example illustrates the
idea clearly.

\subsection*{Example: Frequent selection and large gaps}

Let $\Phi$ be the selector density projection and $\Psi$ the fluctuation
projection.  
Fix $\varepsilon = 0.05$.

To force $\Phi(G)$ within $\varepsilon$ of a positive density, the prefix must
contain many instances of $M(n) = D$.  
To force $\Psi(G)$ within $\varepsilon$ of a large index gap ratio, the prefix
must contain a long contiguous run where $M(n) = K$.

Let $N_\Phi = B_\Phi(\varepsilon)$ and $N_\Psi = B_\Psi(\varepsilon)$.  
Consider the interval $[0..L]$ with $L = \max(N_\Phi, N_\Psi)$.  
If the prefix contains the required density of selected positions for
$\Phi$, it cannot contain the long block of unselected positions required by
$\Psi$ within the same interval.  
Thus the two requirements are incompatible at precision $\varepsilon$.

This incompatibility is a finite-informational fact and does not depend on
the behavior of the identity beyond the prefix.

\section{Incompatibility Across a Family}

Finite families of projections may also exhibit internal conflicts.

\begin{proposition}
Let $\mathcal{P}$ be a finite family of projections.  
If $\mathcal{P}$ contains two projections that are incompatible at some
precision $\varepsilon$, then no identity can satisfy the entire family at
precision $\varepsilon$ using a single prefix of length $B_{\mathcal{P}}(\varepsilon)$.
\end{proposition}

\begin{proof}
Since $B_{\mathcal{P}}(\varepsilon)$ is the maximum dependency bound for the
family, any prefix satisfying all observers must satisfy each one
individually.  
If two observers impose incompatible requirements on that prefix, no such
prefix exists.
\end{proof}

This proposition explains why the diagonalizer construction can always force
deviations.  
Whenever a finite family of observers demands a uniform prefix, one can
choose tails that generate divergent structures not simultaneously detectable
by the family.

\section{Implications for Diagonalization}

The diagonalizer of Part IV depends critically on the existence of
projections whose finite precision requirements cannot all be realized in the
same prefix.  
This incompatibility creates space for controlled tail divergence.  
Once an identity has satisfied observers up to the required prefix length, the
unobservable tail may be modified to enforce structural differences from
reference generators.

This mechanism reflects standard arguments in computable analysis, where
different constraints on prefixes of names of real numbers may conflict.  
Here the conflicts arise between observers on generative identities and serve
as the engine of the incompleteness phenomenon.

\section{Summary}

Different projections impose distinct finite structural requirements on
generative identities.  
When these requirements cannot be satisfied simultaneously in a single
prefix, we say the projections are incompatible.  
Such conflicts provide the combinatorial foundation for the diagonalizer,
allowing one to satisfy observers finitely while enforcing divergence beyond
their range of observation.

In the next chapter we develop the alignment and sewing tools needed to
exploit this incompatibility inside the effective collapse fiber.
