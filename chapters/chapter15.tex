\chapter{Orthogonal Extensions and the Complex Analogy}

\section{Introduction}

Collapse extracts a single coordinate from a generative identity: its classical
magnitude.  
Entropy balance and fluctuation index extract structural information that
collapse discards.  
Together, these invariants begin to form a coordinate system on the generative
space, revealing a geometry richer than the one-dimensional continuum obtained
from the collapse quotient.

The purpose of this chapter is to formalize a conceptual analogy:  
\emph{adding a secondary invariant to collapse is analogous to extending the
real line to a plane}.  
This is not an isomorphism of structures, but a geometric metaphor: the
classical real number $\pi(G)$ is one coordinate, and an extended invariant
(such as $\eta(G)$ or $\phi(G)$) provides an orthogonal direction that restores
structure lost under collapse.

We develop this analogy rigorously by constructing two-dimensional embeddings of
the generative space.  
These embeddings highlight how extended invariants enrich the generative
representation without overcoming the fundamental limitations imposed by
structural incompleteness.

\section{Collapse as a One-Dimensional Projection}

The collapse map
\[
\pi : \mathcal{X}^* \to [0,1]
\]
is a structural projection that forgets nearly all internal structure.  
Viewed geometrically, collapse captures only the ``horizontal’’ coordinate of a
generative identity.  
Every collapse fiber is an entire vertical column of mechanisms projecting to
the same point.

\section{Adding a Secondary Coordinate}

Let $I : \mathcal{X}\to\mathbb{R}$ be an extended invariant such as entropy
balance $\eta$ or fluctuation index $\phi$.  
Both are continuous, prefix-determined, and non-collapsing.

We consider the map
\[
G \longmapsto (\pi(G), I(G)) \in \mathbb{R}^2.
\]

\begin{proposition}[Two-Dimensional Embedding]
If $I$ is non-collapsing, then the map
\[
\Theta_I(G) := (\pi(G), I(G))
\]
is an embedding of each collapse fiber into $\mathbb{R}^2$.
\end{proposition}

\begin{proof}
If $G,H\in\mathcal{F}(x)$ with $G\neq H$, then $\pi(G)=\pi(H)=x$ but
$I(G)\neq I(H)$ by non-collapse.  
Thus $\Theta_I$ is injective on the fiber.  
Continuity follows from continuity of $\pi$ and $I$.
\end{proof}

Thus adding a single extended invariant “lifts’’ each collapse fiber into an
interval of vertical values, restoring structure lost in the one-dimensional
collapse.

\section{Orthogonal Extension Analogy}

We now explain the complex-plane analogy carefully and rigorously.

\subsection*{The real line}

In classical mathematics:
\[
\mathbb{R} \quad\text{is one-dimensional.}
\]

\subsection*{The complex plane}

The complex plane arises by adding an orthogonal direction:
\[
\mathbb{C} = \mathbb{R} \oplus i\mathbb{R}.
\]

Geometrically this means:
- same horizontal coordinate (real part),
- second, independent vertical coordinate (imaginary part).

\subsection*{The generative analogy}

In the generative setting:

- $\pi(G)$ plays the role of the “horizontal” coordinate,
- an extended invariant $I(G)$ plays the role of a “vertical” coordinate.

The analogy is:

\[
\text{Collapse-only: } G \mapsto \pi(G)
\quad\leadsto\quad
\text{One-dimensional real axis.}
\]

\[
\text{Extended coordinates: } G \mapsto (\pi(G), I(G))
\quad\leadsto\quad
\text{Plane-like embedding restoring lost structure.}
\]

\begin{remark}
This analogy is conceptual:  
we are not claiming that $(\pi,I)$ forms a field, a vector space, or an
algebraic closure.  
The analogy concerns dimensional enrichment, not algebraic structure.
\end{remark}

\section{Choosing $I=\eta$ or $I=\phi$}

Both entropy balance and fluctuation index provide valid orthogonal extensions.

\subsection*{Entropy balance plane}

The map
\[
G \longmapsto (\pi(G),\eta(G))
\]
produces a plane in which:

- horizontal axis: classical magnitude,
- vertical axis: frequency of digit selections.

\begin{itemize}
    \item Hybrid identities ($\eta>0$) appear in the positive vertical region.
    \item Null-density identities ($\eta=0$) lie on the horizontal axis.
    \item Each classical real $x$ corresponds to the vertical line
    $\{x\}\times[0,1]$.
\end{itemize}

\subsection*{Fluctuation plane}

The embedding
\[
G \longmapsto (\pi(G),\phi(G))
\]
produces a different slice of structure:

- horizontal axis: magnitude,
- vertical axis: irregularity or dispersion.

These two planes emphasize complementary aspects of the generative space.

\section{Higher-Dimensional Embeddings}

We may combine multiple invariants:

\[
G \longmapsto (\pi(G), \eta(G), \phi(G)) \in \mathbb{R}^3.
\]

This embedding distinguishes:

\begin{itemize}
    \item magnitude (collapse),
    \item average digit density,
    \item long-run selector irregularity.
\end{itemize}

Many distinct invariants—meta-frequency statistics, pattern densities, or
computable subshift entropies—can be added as further axes.

\section{Limits of Dimensional Restoration}

The Structural Incompleteness Theorem remains in force.

\begin{proposition}
No finite-dimensional embedding
\[
\mathcal{X} \longrightarrow \mathbb{R}^d
\]
using computable structural projections is injective on any effective fiber.
\end{proposition}

\begin{proof}
Each coordinate of such an embedding is a computable structural projection.
Apply the Structural Incompleteness Theorem to the finite family of these
projections.
\end{proof}

Thus extended invariants enrich generative coordinates but cannot fully recover
the lost structure.  
This limitation mirrors the fact that the complex plane adds only one new axis
to the real line; it does not recover all structure lost in collapsing
$\mathbb{R}^2$ onto $\mathbb{R}$.

\section{Interpretation}

The analogy with the complex plane should be understood as follows:

\begin{quote}
Adding an independent invariant to $\pi$ produces a two-dimensional coordinate
system, just as adding an imaginary coordinate extends the real line to the
complex plane.  
Both enrich the representational landscape, revealing structure invisible to the
original projection.
\end{quote}

This conceptual picture clarifies the role of extended invariants: they provide
orthogonal directions in the generative geometry, expanding the classical
representation into a richer multidimensional framework.

\section{Outlook}

The final chapter, Chapter~16, investigates the geometry of extended
coordinates.  
Even as more invariants are added, the ability to recover structure diminishes
rapidly.  
Part~VI concludes by analyzing this phenomenon and explaining why the
generative framework supports an expanding hierarchy of invariants but no
finite system can fully classify generative identities.
