\chapter{Observer Towers and Derived Limits}
\label{appendix:observer-towers}

\section{Introduction}

This appendix develops the technical background for observer towers and the
derived limits that appear throughout the invariant tier of the framework.
Observer towers consist of computable families of continuous real valued
functions on the ambient generative space. Each function in the family has a
finite prefix dependence bound, and these bounds organize the family into a
scale of finite information tests.

Derived limits, such as limits of prefix based averages or limsup constructions,
are obtained from these towers. They are Baire class one maps that inherit
instability from the tail freedom present in collapse fibers. The main text
uses these facts in the analysis of asymptotic invariants. This appendix gives
the formal statements and proofs.

\section{Observer Towers}

\subsection{Definition of a tower}

A sequence of structural projections
\[
\Phi_{0}, \Phi_{1}, \Phi_{2}, \ldots
\]
is an observer tower if each $\Phi_{k}$ is continuous on the ambient space and
each has a prefix dependence bound $B_{k}$. Such towers arise from finite
prefix statistics, empirical averages, and other prefix based quantities.

\begin{definition}[Observer Tower]
A sequence $(\Phi_{k})_{k\in\mathbb{N}}$ of continuous real valued maps on the
ambient generative space is an observer tower if for each $\varepsilon>0$ and
each $k$ there exists $N$ such that agreement on $[0..N]$ forces
\[
|\Phi_{k}(G)-\Phi_{k}(H)| < \varepsilon.
\]
\end{definition}

The effective content of a tower is controlled by the uniform computability of
the dependence bounds $B_{k}$. Appendix~\ref{appendix:tte} contains background
on moduli of continuity and computable maps in Type 2 Effectivity.

\subsection{Prefix synchronization inside a tower}

For a fixed $k$, the dependence bound $B_{k}$ determines the prefix depth that
controls the observable behavior of $\Phi_{k}$. For all $m\le k$, the joint
prefix depth
\[
N_{k} = \max\{B_{0}(\varepsilon_{0}),\ldots,B_{k}(\varepsilon_{k})\}
\]
synchronizes the first $k$ observers.

Prefix synchronization plays a central role in the finite information arguments
used in the incompleteness tier. The same principle appears in the sewing and
diagonalization results proven in Appendices~\ref{appendix:alignment-sewing}
and~\ref{appendix:mimicry}.

\section{Derived Limit Maps}

Derived limits arise from observing the outputs of a tower and passing to
limits along the index. These limits measure long range behavior of prefixes,
not of the full identity.

\subsection{Pointwise limits}

Let $(\Phi_{k})$ be an observer tower. The pointwise limit
\[
\Lambda(G) = \lim_{k\to\infty} \Phi_{k}(G)
\]
may fail to exist for some identities. When the limit exists, it depends only
on the tail of the selector structure but is determined through finite prefix
tests applied at larger depths. This makes $\Lambda$ a typical Baire class one
map.

\begin{lemma}[Baire Class One]
\label{lem:bc1}
If $(\Phi_{k})$ is an observer tower, then the set of identities where
$\Lambda(G)$ exists is a $G_{\delta}$ set, and the map $\Lambda$ is Baire class
one on its domain.
\end{lemma}

\begin{proof}
For each rational interval $(a,b)$, the preimage
\[
\Lambda^{-1}((a,b))
=
\bigcup_{m} \bigcap_{k\ge m} \Phi_{k}^{-1}((a,b))
\]
is $G_{\delta}$ because each $\Phi_{k}$ is continuous. This shows that
$\Lambda$ is Baire class one.
\end{proof}

This regularity is the strongest continuity property available, since tail
freedom inside fibers creates discontinuity at every point for most derived
limits. See the examples in Appendix~\ref{appendix:extended-invariants}.

\subsection{Limsup and liminf}

Limsup based maps are even less regular. For a tower $(\Phi_{k})$ define
\[
\Lambda^{*}(G) = \limsup_{k\to\infty} \Phi_{k}(G),
\qquad
\Lambda_{*}(G) = \liminf_{k\to\infty} \Phi_{k}(G).
\]

\begin{lemma}[Upper and Lower Semicontinuity]
The map $\Lambda^{*}$ is upper semicontinuous and $\Lambda_{*}$ is lower
semicontinuous.
\end{lemma}

\begin{proof}
If $\Lambda^{*}(G) < r$ then there exists $m$ such that for all $k\ge m$,
$\Phi_{k}(G) < r$. By continuity of each $\Phi_{k}$, this persists on a
neighborhood of $G$, proving upper semicontinuity. The argument for
$\Lambda_{*}$ is similar.
\end{proof}

These semicontinuity properties reflect the prefix based nature of the tower.
They also explain the instability of asymptotic invariants.

\section{Tail Freedom and Instability}

Derived limits inherit discontinuity from the symbolic tail. In collapse fibers
the tail may be changed freely while all finite prefixes are held fixed. This
produces strong instability for any limit that depends on infinitely many
prefix levels.

\begin{lemma}[Instability Under Tail Modification]
\label{lem:tail-instability}
Let $(\Phi_{k})$ be an observer tower. If tail modification leaves all finite
prefixes unchanged, then for any identity $G$ and for any real $r$ in the range
of $\Phi_{k}$ there exists an identity $G'$ agreeing with $G$ on all finite
prefixes but satisfying $\Lambda^{*}(G') = r$.
\end{lemma}

\begin{proof}
Fix $G$ and $r$. For each $k$ choose a tail extension that forces
$\Phi_{k}(G')$ to be within $1/k$ of $r$ using the prefix freedom in the fiber.
This is possible because each $\Phi_{k}$ depends only on a finite prefix. Then
$\Lambda^{*}(G')=r$.
\end{proof}

This shows that derived limits cannot classify identities in a fiber. This
result is an early form of the incompleteness principle proved in full in
Chapter~7 and Appendix~\ref{appendix:mimicry}.

\section{Uniform Bounds and Derived Stability}

In some constructions it is necessary to control derived limits for a finite
family of towers. Uniform prefix bounds provide such control.

\begin{definition}[Uniform Dependence for a Family]
Let $\mathcal{T}=\{(\Phi^{(i)}_{k})_{k\in\mathbb{N}} : i=1,\ldots,m\}$ be a
finite collection of observer towers. A function $B_{\mathcal{T}}$ is a uniform
dependence bound if agreement on $[0..B_{\mathcal{T}}(\varepsilon)]$ forces
\[
|\Phi^{(i)}_{k}(G)-\Phi^{(i)}_{k}(H)| < \varepsilon
\]
for all $i$ and all $k$.
\end{definition}

\begin{lemma}[Uniform Prefix Stabilization]
If $G$ and $H$ agree on $[0..B_{\mathcal{T}}(\varepsilon)]$ then
\[
|\Lambda(G)-\Lambda(H)| < \varepsilon
\]
for any derived limit $\Lambda$ formed from the towers in the family.
\end{lemma}

\begin{proof}
Agreement on the synchronized prefix forces agreement for each $\Phi^{(i)}_{k}$
beyond the dependence bound. Taking limits gives the result.
\end{proof}

Uniform stabilization is used in controlled constructions where multiple limit
quantities must be preserved while manipulating the tail. This appears in the
sewing arguments of Appendix~\ref{appendix:alignment-sewing} and in the mimicry
procedure of Appendix~\ref{appendix:mimicry}.

\section{Summary}

This appendix formalized the structure of observer towers and the derived
limits that arise from them. The key points are:

\begin{itemize}
    \item each observer in a tower has finite prefix dependence,
    \item pointwise limits of observer towers are Baire class one,
    \item limsup and liminf maps are semicontinuous but unstable,
    \item tail freedom inside collapse fibers produces maximal instability,
    \item uniform prefix bounds control derived limits for finite families.
\end{itemize}

These results justify the analysis of finite and asymptotic invariants in the
main text and highlight their fundamental limitations.
