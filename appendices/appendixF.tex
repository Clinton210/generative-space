\chapter{Supplemental Proofs and Case Analyses}

\section{Introduction}

This appendix contains auxiliary results, proofs, and case analyses referenced
in the main text but omitted for clarity.  
These include additional observations about selector regimes, refined arguments
related to projective incompatibility, and special cases of dependency-bound
calculations.  
Nothing in this appendix is required for the logical structure of the
monograph, but these details may be useful for readers seeking a deeper
understanding of the examples or for verifying that edge cases behave as
claimed.

\section{Selector Regimes: Boundary Cases}

\subsection{Dense but Irregular Selectors}

A selector may have full digit-selection density $\eta(G)=1$ while exhibiting
extreme irregularity in its switching pattern.  
For example:
\[
M(n) = 
\begin{cases}
D,&\text{if }n\notin \{2^k : k\in\mathbb{N}\},\\
K,&\text{otherwise}.
\end{cases}
\]
The digit-selection density is $1$, but the fluctuation index satisfies
\[
\phi(G) = 0
\]
because switches occur only at the powers of two.

This shows that entropy balance and fluctuation index are independent.

\subsection{Sparse but Structured Selectors}

Let $M(n)=D$ when $n=j^2$ and $M(n)=K$ otherwise.  
The density is zero, but the selected positions form a highly structured subsequence.  
If $D(j^2)$ is defined by $D(j^2)=f(j)$ for some computable $f$, then the
collapse value depends entirely on the values of $f$ and not on the structure of
the sparse selector itself.  
This highlights that collapse is indifferent to the complexity of the index set
of selected positions.

\section{Supplementary Arguments for Projective Incompatibility}

\subsection{Two Projections Extracting Conflicting Features}

Let $\Phi_1$ measure digit-selection density $\eta(G)$, and let $\Phi_2$ measure
switching frequency $\phi(G)$.  
Construct $G,H\in\mathcal{X}$ as follows:

\begin{itemize}
    \item $G$ alternates $D,K$ at every step (giving $\eta(G)=\tfrac12$ and
          $\phi(G)=1$).
    \item $H$ uses $D$ at positions $j^2$ and $K$ elsewhere (giving
          $\eta(H)=0$ and $\phi(H)=0$).
\end{itemize}

Then $\Phi_2(G)=\Phi_2(H)=0$ but $\Phi_1(G)\neq\Phi_1(H)$.

This shows $\Phi_1$ and $\Phi_2$ cannot be simultaneously minimized, and they
produce incompatible partitions of the generative space.

\subsection{Families of Projections and Finite Precision}

For a finite family $\mathcal{P}=\{\Phi_1,\Phi_2\}$, one may choose $\varepsilon$
so small that the prefix length $N=B_{\mathcal{P}}(\varepsilon)$ covers all
coordinates influencing $\Phi_1$ at precision $\varepsilon$, but not all
coordinates required for $\Phi_2$ at the same precision.  
Thus the frozen prefix may constrain one projection more tightly than the other,
producing effectively incompatible observational windows.

\section{Extended Invariants: Special Cases}

\subsection{Selectors With Infinite Switching but Vanishing Entropy}

Let $M$ be defined by
\[
M = 
\underbrace{D}_1, 
\underbrace{K,K}_2,
\underbrace{D,D,D}_3,
\underbrace{K,K,K,K}_4,
\ldots
\]
Here:
\[
\eta(G) = \liminf_{n\to\infty} \frac{\lfloor \sqrt{2n}\rfloor}{n} = 0,
\]
but the switching index is infinite because switches occur at every block
boundary.  
Thus:
\[
\eta(G)=0,\qquad \phi(G)>0.
\]

\subsection{Selectors With Positive Density but Zero Fluctuation}

Let
\[
M = \underbrace{D,\ldots,D}_{N},\underbrace{K,\ldots,K}_{N},\underbrace{D,\ldots,D}_{N},\underbrace{K,\ldots,K}_{N},\ldots
\]
with $N$ fixed.  
Then
\[
\eta(G)=\tfrac12,\qquad \phi(G)=\frac{1}{N}.
\]

As $N\to\infty$, we obtain hybrid generators with arbitrarily small fluctuation
indices.

\section{Additional Dependency-Bound Calculations}

\subsection{Frequency-Based Projections}

Suppose $\Phi$ computes the first $k$ digits of the empirical frequency of a
symbol in the canonical output.  
Then
\[
B_\Phi(2^{-k}) \asymp C \cdot 2^k
\]
for some constant $C$ depending on the alphabet.  
Thus higher precision requires exponentially larger prefixes.

\subsection{Local Variation Projections}

If $\Phi$ computes the switching frequency of the selector $M$, then
\[
B_\Phi(2^{-k})
\]
depends on controlling the number of switches in the first $N$ steps.  
Explicitly,
\[
B_\Phi(2^{-k}) = O(2^k)
\]
in typical cases.

\section{Supplementary Remarks on Tail Modification}

The construction in Appendix~C ensures divergence occurs at infinitely many
coordinates.  
For readers interested in extreme patterns, note:

- the divergence can be made periodic,
- or block-structured,
- or governed by an external computable sequence (e.g., the Thue–Morse word),
- or randomized (using a computable martingale).

All such variants preserve the guarantees required by the diagonalizer.

\section{Summary}

This appendix provided:

\begin{itemize}
    \item extremal examples of selector behavior,
    \item refined demonstrations of projective incompatibility,
    \item boundary-case calculations for entropy balance and fluctuation index,
    \item supplementary dependency-bound estimates,
    \item expanded remarks on allowed tail-modification patterns.
\end{itemize}

These details complement the main text and provide additional clarity for
readers exploring the edge cases of the generative framework.
