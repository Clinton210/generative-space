\chapter{Dependency Bounds and Tail Modification Lemmas}

\section{Introduction}

The dependency-bound framework plays a central role in the analysis of
computable structural projections and in the construction of the
meta-diagonalizer.  
A computable projection can inspect only finitely many coordinates of a
generative identity to achieve a prescribed precision.  
This appendix formalizes these bounds and establishes a collection of lemmas
allowing controlled modifications of a generator’s tail without affecting the
output of a finite family of projections.

\section{Computable Projections and Moduli of Continuity}

Let $\Phi : \mathcal{X} \to \mathbb{R}$ be a computable structural projection.
By effective continuity (Appendix~A), for each rational precision parameter
$\varepsilon > 0$ there exists a computable function
\[
B_\Phi(\varepsilon) \in \mathbb{N}
\]
such that if two generators $G,H\in\mathcal{X}$ agree on their first
$B_\Phi(\varepsilon)$ coordinates, then
\[
|\Phi(G) - \Phi(H)| < \varepsilon.
\]

\begin{definition}[Dependency Bound]
A function $B_\Phi : \mathbb{Q}^+ \to \mathbb{N}$ satisfying the above property
is called a \emph{dependency bound} for $\Phi$.
\end{definition}

Dependency bounds quantify the finite lookahead available to computable
projections and are foundational for all subsequent constructions.

\section{Uniform Bounds for Finite Families}

For a finite family $\mathcal{P} = \{\Phi_1,\ldots,\Phi_r\}$ of computable
projections, define the family dependency bound
\[
B_{\mathcal{P}}(\varepsilon)
=
\max\{ B_{\Phi_1}(\varepsilon),\ldots,B_{\Phi_r}(\varepsilon) \}.
\]

\begin{proposition}[Uniform Family Bound]
If $G,H\in\mathcal{X}$ agree on their first $B_{\mathcal{P}}(\varepsilon)$
coordinates, then
\[
|\Phi_i(G)-\Phi_i(H)| < \varepsilon
\quad\text{for all } 1\le i\le r.
\]
\end{proposition}

This uniformity allows one to control the behavior of all projections in the
family simultaneously using a single prefix constraint.

\section{Prefix Freezing and Tail Freedom}

Dependency bounds induce a clean separation between the \emph{prefix}, which
fully determines the projections to fixed precision, and the \emph{tail}, which
is unconstrained from the viewpoint of the projections.

\begin{lemma}[Prefix Freezing]
\label{lem:prefix-freezing}
Fix $\varepsilon>0$ and a finite family $\mathcal{P}$.  
If $G$ is any effective generator, then any $H$ satisfying
\[
H|_{B_{\mathcal{P}}(\varepsilon)} = G|_{B_{\mathcal{P}}(\varepsilon)}
\]
obeys
\[
|\Phi(H)-\Phi(G)| < \varepsilon
\]
for all $\Phi\in \mathcal{P}$.
\end{lemma}

\begin{proof}
Immediate from the definition of $B_{\mathcal{P}}$.
\end{proof}

The tail of $H$ may differ arbitrarily from the tail of $G$ without affecting
the projections at precision $\varepsilon$.

\section{Tail Modification Lemmas}

We now develop tail modification tools that allow alterations to the tail of a
generator while preserving:

1. agreement on the prefix up to a dependency threshold, and  
2. membership in a fixed collapse fiber.

These lemmas are essential for the sewing procedure in the meta-diagonalizer.

\subsection{Coordinatewise Tail Replacement}

\begin{lemma}[Digit-Preserving Tail Replacement]
\label{lem:digit-tail}
Let $G=(M,D,K)\in\mathcal{X}^*$ and fix $N\in\mathbb{N}$.  
Let $D'$ be any sequence such that
\[
D'(n) = D(n)\quad\text{for } n < N.
\]
If $M$ continues to select digits infinitely often, then replacing $D$ with $D'$
after position $N$ yields a new generator $H$ with $\pi(H)=\pi(G)$.
\end{lemma}

\begin{proof}
The collapse value depends only on the digits at selected positions.  
Since $D'(n)=D(n)$ for all $n<N$, and since selected indices beyond $N$
contribute only to tail digits of the base-$b$ expansion, their modification
does not affect the value of the infinite series defining $\pi(G)$.
\end{proof}

\subsection{Selector-Preserving Tail Replacement}

\begin{lemma}[Selector Free Tail Modification]
\label{lem:selector-tail}
Fix $G=(M,D,K)$ and $N\in\mathbb{N}$.  
Let $M'$ be any selector satisfying $M'(n)=M(n)$ for $n<N$.  
If the set of selected indices of $M'$ is infinite and agrees with $M$ on the
first $\ell$ selected positions, then $\pi(M',D,K)=\pi(G)$.
\end{lemma}

\begin{proof}
If the first $\ell$ selected indices coincide for $M$ and $M'$, then the first
$\ell$ digits of the base-$b$ expansion of $\pi(G)$ and $\pi(M',D,K)$ coincide.
Differences in later selected positions correspond only to tail digits, which
do not change the represented real number.
\end{proof}

\subsection{Meta-Layer Modification}

\begin{lemma}[Meta-Layer Freedom]
\label{lem:meta-tail}
Let $G=(M,D,K)$ and let $K'$ be any sequence satisfying $K'|_N = K|_N$ for some
prefix length $N$.  
Then for all such $K'$, we have $\pi(M,D,K') = \pi(G)$.
\end{lemma}

\begin{proof}
The meta layer is ignored entirely by collapse.
\end{proof}

\section{Combined Tail Modification}

The previous lemmas can be combined to perform joint modifications across all
three layers.

\begin{lemma}[Fiber-Preserving Prefix Agreement]
\label{lem:fiber-prefix}
Let $G=(M,D,K)\in\mathcal{F}(x)$ and let $N\in\mathbb{N}$.  
For any sequences $M',D',K'$ satisfying
\[
(M',D',K')|_N = (M,D,K)|_N,
\]
and for which $M'$ selects digits infinitely often and agrees with $M$ on the
first $\ell$ selected positions, we have
\[
\pi(M',D',K') = x.
\]
\end{lemma}

\begin{proof}
Combining Lemmas~\ref{lem:digit-tail}, \ref{lem:selector-tail}, and
\ref{lem:meta-tail}.
\end{proof}

\section{Tail Modification with Dependency Control}

We now introduce the core lemma used in the meta-diagonalizer: tail
modifications can be arranged so that, for a finite family of projections, the
projections remain unaffected at prescribed precision.

\begin{lemma}[Precision-Preserving Tail Replacement]
\label{lem:precision-tail}
Let $\mathcal{P}$ be a finite family of computable projections and fix
$\varepsilon>0$.  
Let $N = B_{\mathcal{P}}(\varepsilon)$.  
If $H$ agrees with $G$ on the first $N$ coordinates, then:
\[
|\Phi(H)-\Phi(G)| < \varepsilon
\quad\text{for all } \Phi\in\mathcal{P}.
\]
Moreover, if $H$ satisfies the fiber-preservation conditions of
Lemma~\ref{lem:fiber-prefix}, then $H\in\mathcal{F}(\pi(G))$.
\end{lemma}

\begin{proof}
The first statement follows from Lemma~\ref{lem:prefix-freezing}; the second
from Lemma~\ref{lem:fiber-prefix}.
\end{proof}

\section{Summary}

This appendix provided the complete machinery for controlled tail modifications:

\begin{itemize}
    \item dependency bounds for computable projections,
    \item uniform bounds for finite families,
    \item prefix freezing and tail freedom,
    \item fiber-preserving modifications across all layers,
    \item precision-preserving tail replacement.
\end{itemize}

These tools are used in Part~III to analyze structural projections, and in
Part~IV to construct the meta-diagonalizer and prove the Structural
Incompleteness Theorem.
