\chapter{Technical Lemmas for the Diagonalizer}

\section{Introduction}

This appendix collects the technical results used in Chapter~8 to construct the meta diagonalizer.  
These lemmas describe how computable projections depend on finite prefixes of generative identities and how tail modifications can be used to force disagreement with projections.  
All results assume the notation and framework developed in Chapters~6 and~8.

\section{Prefix Stabilization Lemma}

The first lemma formalizes the idea that computable secondary projections stabilize once a sufficiently long prefix of the input is fixed.

\begin{lemma}[Prefix Stabilization]
\label{lem:prefix-stabilization}
Let $\Phi : \mathcal{G}_{\mathrm{eff}} \to \mathbb{R}$ be a computable projection with dependency bound $B_{\Phi}$.  
If two effective identities $G$ and $H$ satisfy
\[
(M_G,D_G,K_G){\upharpoonright}B_{\Phi}(\varepsilon)
=
(M_H,D_H,K_H){\upharpoonright}B_{\Phi}(\varepsilon),
\]
then
\[
|\Phi(G) - \Phi(H)| < \varepsilon.
\]
\end{lemma}

\begin{proof}
This is immediate from the definition of dependency bound in Chapter~6.  
The proof uses the fact that a computable approximation of $\Phi$ reads only a finite number of input symbols before producing an $\varepsilon$ approximation.
\end{proof}

\section{Uniform Stabilization for Finite Families}

Finite families of projections admit a single bound that controls prefix dependence for all members simultaneously.

\begin{lemma}[Uniform Stabilization]
\label{lem:uniform-stabilization}
Let $\mathcal{F} = \{ \Phi_1, \ldots, \Phi_m \}$ be a finite family of computable secondary projections.  
Let $B(\varepsilon)$ be the uniform dependency bound from Proposition~\ref{prop:uniform-bound}.  
If $G$ and $H$ agree on the first $B(\varepsilon)$ positions, then
\[
|\Phi_i(G) - \Phi_i(H)| < \varepsilon
\quad \text{for all } i=1,\ldots,m.
\]
\end{lemma}

\begin{proof}
Since $B(\varepsilon) = \max_i B_{\Phi_i}(\varepsilon)$, agreement on the first $B(\varepsilon)$ positions implies agreement on the prefix required for each projection in the family.  
Lemma~\ref{lem:prefix-stabilization} completes the argument.
\end{proof}

\section{Tail Sewing Lemma}

The diagonalizer repeatedly modifies a generative identity on intervals that lie beyond the uniform dependency bound.  
The next lemma shows that these tail modifications do not affect projection values within the stabilization margin.

\begin{lemma}[Tail Sewing]
\label{lem:tail-sewing}
Let $G$ and $H$ be effective identities, and let $L \in \mathbb{N}$.  
Define an identity $G'$ by
\[
G'(n) =
\begin{cases}
G(n), & \text{for } n \le L,\\
H(n), & \text{for } n > L.
\end{cases}
\]
If $L \ge B(\varepsilon)$, where $B$ is the uniform dependency bound for a finite family $\mathcal{F}$, then for each $\Phi_i \in \mathcal{F}$,
\[
|\Phi_i(G') - \Phi_i(G)| < \varepsilon.
\]
\end{lemma}

\begin{proof}
Since $G'$ agrees with $G$ on the first $L \ge B(\varepsilon)$ positions, Lemma~\ref{lem:uniform-stabilization} applies and gives the desired inequality.
\end{proof}

This lemma formalizes the principle that projections cannot detect tail modifications until they cross the region on which projections depend.

\section{Adjustment Lemma}

To force disagreement with a projection, the diagonalizer inserts segments of another identity $A$ into the tail.  
The next lemma guarantees that these insertions produce controlled changes in projection values.

\begin{lemma}[Adjustment Lemma]
\label{lem:adjustment}
Let $A$ and $H$ be effective identities.  
Suppose $I = [N_1, N_2]$ is an interval with $N_1 > B(\varepsilon)$, and define $G'$ by
\[
G'(n) =
\begin{cases}
H(n), & n \notin I,\\
A(n), & n \in I.
\end{cases}
\]
Then for each $\Phi_i \in \mathcal{F}$,
\[
|\Phi_i(G') - \Phi_i(H)| \le |\Phi_i(A) - \Phi_i(H)| + \varepsilon.
\]
\end{lemma}

\begin{proof}
By construction, $G'$ agrees with $H$ on the first $B(\varepsilon)$ coordinates.  
Thus Lemma~\ref{lem:uniform-stabilization} gives
\[
|\Phi_i(G') - \Phi_i(H')| < \varepsilon,
\]
where $H'$ is obtained by replacing the tail of $H$ after $B(\varepsilon)$ with itself.  
Since $A$ is used only beyond $N_1 > B(\varepsilon)$, the projections of $G'$ differ from those of $A$ only in the stabilization margin.  
Combining these estimates gives the desired bound.
\end{proof}

\section{Controlled Divergence Lemma}

To guarantee that a projection is forced to disagree with a reference identity, we select adjustment zones and an adjustment identity $A$ so that the difference is large enough.

\begin{lemma}[Controlled Divergence]
\label{lem:controlled-divergence}
Let $\mathcal{F} = \{ \Phi_1, \ldots, \Phi_m \}$ be a finite family of projections.  
For each $i$, choose $\delta_i > 0$.  
There exists an effective identity $A$ such that
\[
|\Phi_i(A) - \Phi_i(H)| > \delta_i
\quad \text{for all } i.
\]
\end{lemma}

\begin{proof}
Since the projections are computable and the generative space contains infinitely many effective identities, one may search effectively through $\mathcal{G}_{\mathrm{eff}}$ for an identity whose projection values exceed the desired thresholds.  
The details follow a standard dovetailing argument, using the fact that deviations in any of the layers $(M,D,K)$ can alter projection values arbitrarily.  
\end{proof}

This lemma provides the adjustment identity used in Chapter~8.

\section{Diagonalization Lemma}

Combining the previous lemmas yields the diagonalizer used to prove structural incompleteness.

\begin{lemma}[Diagonalization]
\label{lem:diagonalization}
Let $\mathcal{F}$ and $H$ be as in Chapter~8.  
There exists an effective identity $G^{*}$ such that
\[
\Phi_i(G^{*}) \ne \Phi_i(H)
\quad \text{for each } \Phi_i \in \mathcal{F}.
\]
\end{lemma}

\begin{proof}
Choose a sequence $(\varepsilon_k)$ with $\varepsilon_k \to 0$, and let $L_k = B(\varepsilon_k)$.  
Choose adjustment zones $I_k$ with $I_k = [N_{k,1}, N_{k,2}]$ satisfying $N_{k,1} > L_k$.  
For each $k$, choose an adjustment identity $A_k$ using Lemma~\ref{lem:controlled-divergence} so that
\[
|\Phi_i(A_k) - \Phi_i(H)| > 3\varepsilon_k.
\]

Define $G^{*}$ by sewing these adjustments into the tail of $H$ using Lemmas~\ref{lem:tail-sewing} and~\ref{lem:adjustment}.  
The resulting identity satisfies the effectiveness and prefix agreement conditions required in Chapter~8, and its projection values differ from those of $H$ by at least $\varepsilon_k$ for sufficiently large $k$.  
Thus $\Phi_i(G^{*}) \ne \Phi_i(H)$ for all $i$.
\end{proof}

\section{Summary}

These lemmas formalize the computational properties and tail manipulation arguments that the diagonalizer construction relies on.  
They ensure that computable projections stabilize on finite prefixes, that tail modifications are undetectable below a dependency threshold, and that controlled adjustments can force projection values to diverge.  
Together they support the Structural Incompleteness Theorem proved in Chapter~9.
