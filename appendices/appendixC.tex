\chapter{Alignment and Sewing: Full Technical Proofs}

\section{Introduction}

This appendix provides full proofs of the technical lemmas used in
Chapter~8 and Chapter~9.  
These results justify the alignment of selected digits, the sewing of tails,
and the preservation of collapse fibers under controlled concatenation of
prefixes and tails.

The purpose of this appendix is to present these arguments in their natural
level of detail while keeping the main text focused on conceptual structure.

\section{Canonical Output and Selection Indices}

For a generative identity $G = (M,D,K)$, define its sequence of selected
positions by
\[
n_0 < n_1 < n_2 < \cdots,
\]
where $n_j$ is the $j$th index with $M(n_j) = D$.  
The canonical output of $G$ is the sequence
\[
d_0, d_1, d_2, \ldots,
\qquad\text{where } d_j = D(n_j).
\]
For identities in $\mathcal{X}^{*}$, this output yields a valid digit
expansion.

Two identities $H$ and $A$ lie in the same collapse fiber if and only if
\[
D_H(n_j^{H}) = D_A(n_j^{A}) = x_j
\]
for all $j$, where $n_j^{H}$ and $n_j^{A}$ are their respective selection
indices.

\section{Alignment of Selected Digits}

The first lemma states that identities in the same collapse fiber expose the
same canonical digits at potentially different positions.  
This basic fact allows us to use selection indices as alignment points.

\begin{lemma}[Alignment of Selection Indices]
\label{lem:alignment}
Let $H$ and $A$ be identities in the same collapse fiber
$\mathcal{F}(x)$.  
Let $n_j^{H}$ and $n_j^{A}$ be their respective $j$th selection indices.  
Then the digits exposed at these positions coincide:
\[
D_H(n_j^{H}) = D_A(n_j^{A}) = x_j.
\]
\end{lemma}

\begin{proof}
By definition of the collapse fiber,
\[
\pi(H) = \pi(A) = x.
\]
The canonical output of $\pi(H)$ is the sequence of digits
\[
x_0, x_1, x_2, \ldots,
\]
and the same holds for $\pi(A)$.  
Since $n_j^{H}$ and $n_j^{A}$ denote the $j$th positions where $H$ and $A$
expose their digits, the exposed digits must coincide with the $j$th digit of
$x$.  
Therefore
\[
D_H(n_j^{H}) = x_j = D_A(n_j^{A}),
\]
as required.
\end{proof}

This lemma provides the foundation for sewing: two identities in the same
collapse fiber may disagree on the positions where selected digits occur, but
their canonical output digits occur in the same order.

\section{Prefix Completion and Tail Extraction}

The following definition formalizes the process of replacing the tail of one
identity with the tail of another, starting at aligned selection indices.

\begin{definition}[Prefix Completion and Tail Extraction]
Let $H$ and $A$ be identities in $\mathcal{X}^{*}$ and let $j \in \mathbb{N}$.  
Define
\[
h_j = n_j^{H},
\qquad
a_j = n_j^{A}.
\]
We define the identity $G = H \,\widehat{\ }_{\,j}\, A$ by
\[
G(n) =
\begin{cases}
H(n) & n \le h_j, \\
A(n - h_j + a_j) & n > h_j.
\end{cases}
\]
\end{definition}

This construction preserves all symbols of $H$ up to the $j$th selected
position and then reproduces the symbolic behavior of $A$ starting at the
corresponding selected digit.

\section{Sewing Preserves the Collapse Fiber}

The next lemma shows that prefix completion and tail extraction preserve the
collapsed value when the identities lie in the same fiber.

\begin{lemma}[Tail Sewing Preserves Collapse]
\label{lem:sewing-preserves-collapse}
Let $H$ and $A$ lie in the collapse fiber $\mathcal{F}(x)$ and let $G =
H \,\widehat{\ }_{\,j}\, A$.  
Then $G \in \mathcal{F}(x)$.
\end{lemma}

\begin{proof}
Let $h_j$ and $a_j$ denote the $j$th selected positions of $H$ and $A$.  
The identity $G$ agrees with $H$ on every position $n \le h_j$.  
In particular, the first $j$ selected digits of $G$ occur at the same indices
as in $H$ and have the same values.

For $n > h_j$, the identity $G$ reproduces the behavior of $A$ starting at
index $a_j$.  
The $(j+1)$st selected digit in $G$ appears at the first position $m > h_j$
with $A(m - h_j + a_j) = D$, which corresponds to the $(j+1)$st selection
index of $A$.

Thus $G$ exposes the same canonical digit sequence as $A$, namely the digit
expansion of $x$.  
Hence $\pi(G) = x$ and $G \in \mathcal{F}(x)$.
\end{proof}

This result holds for every $j$ and for any choice of $A$ in the collapse
fiber.

\section{Dependency Bounds and Controlled Sewing}

The next lemma shows how dependency bounds combine with sewing to preserve the
values of structural projections.

\begin{lemma}[Controlled Sewing]
\label{lem:controlled-sewing}
Let $\mathcal{P}$ be a finite family of structural projections with uniform
dependency bound
\[
N = B_{\mathcal{P}}(\varepsilon).
\]
Let $H$ and $A$ lie in the collapse fiber $\mathcal{F}(x)$.  
Let $j$ satisfy $h_{j} \ge N$.  
Define $G = H \,\widehat{\ }_{\,j}\, A$.  
Then
\[
|\Phi(G) - \Phi(H)| < \varepsilon
\quad\text{for all } \Phi \in \mathcal{P}.
\]
\end{lemma}

\begin{proof}
Since $G$ and $H$ agree on all coordinates $n \le h_j$ and $h_j \ge N$, the
prefix agreement condition of the structural projections implies
\[
|\Phi(G) - \Phi(H)| < \varepsilon
\]
for each $\Phi \in \mathcal{P}$.  
The tail of $G$ beyond $h_j$ is irrelevant, since dependency bounds imply
that only the prefix of length $N$ influences the value of $\Phi$ to
precision $\varepsilon$.
\end{proof}

This lemma shows that sewing changes structure only beyond the observational
reach of the projections.

\section{Sewing with Dependency Bounds: A Technical Refinement}

In the diagonalizer construction, we need an explicit estimate relating $j$,
$N$, and the positions of selected digits.  
The following lemma provides this relationship.

\begin{lemma}[Selection Index Lower Bound]
\label{lem:selection-index-lower-bound}
Let $H$ be a generative identity with infinitely many selected digits.  
For any $N \in \mathbb{N}$, there exists a $j$ such that $h_j \ge N$.  
Moreover, if $H$ has positive selector density $\eta(H) > 0$, then
\[
h_j \le \frac{j}{\eta(H)}.
\]
\end{lemma}

\begin{proof}
Since $H$ exposes infinitely many digits, the sequence
\[
h_0 < h_1 < h_2 < \cdots
\]
is strictly increasing and unbounded.  
Thus for any $N$ there exists $j$ with $h_j \ge N$.

If $\eta(H) > 0$, then by definition of lower density,
\[
\frac{j}{h_j} \to \eta(H)
\quad\text{along a subsequence}.
\]
Equivalently,
\[
h_j \le \frac{j}{\eta(H)}
\]
for all sufficiently large $j$.
\end{proof}

This lemma ensures that we can always find an alignment index beyond the range
required by the dependency bounds.

\section{Full Sewing Lemma and Its Consequences}

We now combine the previous results into a single statement that is used in
the diagonalizer construction.

\begin{lemma}[Full Sewing Lemma]
\label{lem:full-sewing}
Let $\mathcal{P}$ be a finite family of structural projections with uniform
dependency bound $B_{\mathcal{P}}(\varepsilon) = N$.  
Let $H$ and $A$ lie in the collapse fiber $\mathcal{F}(x)$.  
Let $j$ satisfy $h_j \ge N$.  
Then the sewed identity $G = H \,\widehat{\ }_{\,j}\, A$ satisfies:
\begin{enumerate}
    \item $G \in \mathcal{F}(x)$,
    \item $|\Phi(G) - \Phi(H)| < \varepsilon$ for all $\Phi \in \mathcal{P}$.
\end{enumerate}
\end{lemma}

\begin{proof}
The first part follows from Lemma~\ref{lem:sewing-preserves-collapse}.  
The second part follows from Lemma~\ref{lem:controlled-sewing}.  
\end{proof}

The Full Sewing Lemma provides the key finite information control needed for
diagonalization: observers remain stable under changes to the identity beyond
a sufficiently long prefix.

\section{Computability of the Sewn Identity}

We finish with the computability properties of the sewing operation.

\begin{lemma}[Computability of Sewing]
\label{lem:computable-sewing}
If $H$ and $A$ are computable identities in $\mathcal{F}(x)$ and $j$ is
computable from $H$, then $H \,\widehat{\ }_{\,j}\, A$ is a computable identity.
\end{lemma}

\begin{proof}
Computable identities have computable selector, digit, and meta streams.
Given $j$ and the selection indices $h_j$ and $a_j$, which are computable from
$H$ and $A$, the definition of the sewed identity provides an explicit
algorithm to compute $G(n)$ for each $n$.  
Thus $G$ is computable.
\end{proof}

This lemma ensures that the diagonalizer constructed in the main text is
computable.

\section{Summary}

This appendix provided full proofs of the alignment and sewing lemmas that
support the diagonalizer construction.  
These results show that:

\begin{itemize}
    \item identities in the same collapse fiber expose the same canonical
          digits in the same order,
    \item tails may be replaced freely once alignment indices are chosen,
    \item dependency bounds ensure that observers are unaffected by tail
          modification,
    \item computable identities remain computable under sewing.
\end{itemize}

Together, these tools form the core technical machinery used to establish the
Structural Incompleteness Theorem.
