\chapter{Alignment and Sewing: Full Technical Proofs}
\label{appendix:alignment-sewing}

\section{Introduction}

This appendix gives detailed proofs of the alignment and sewing principles that
underlie the finite–information constructions in the incompleteness tier of the
framework. These results verify that:

\begin{itemize}
    \item identities in the same collapse fiber expose the same canonical digits,
    \item tails may be replaced once digits are aligned,
    \item dependency bounds ensure observers are unaffected by tail changes,
    \item computable sewing preserves membership in effective fibers.
\end{itemize}

The proofs formalize the finite–prefix reasoning used in the alignment and sewing
chapter and supply the technical foundation for the mimicry diagonalizer.

\section{Canonical Output and Selection Indices}

For a generative identity $G = (M,D,K)$, the selected positions form an
increasing sequence
\[
n_0 < n_1 < n_2 < \cdots,
\]
where $n_j$ is the $j$th index such that $M(n)=D$.

The canonical output of $G$ is the sequence
\[
d_0, d_1, d_2, \ldots,
\qquad d_j = D(n_j).
\]
For identities in $\mathcal{X}^{*}$ this is a valid digit expansion of a point
in $[0,1]$.

Two identities $H$ and $A$ lie in the same collapse fiber if and only if their
canonical outputs agree digit by digit. Thus for all $j$,
\[
D_H(n_j^{H}) = D_A(n_j^{A}) = x_j,
\]
where $(x_j)$ is the expansion of the collapsed value.

\section{Alignment of Selected Digits}

\begin{lemma}[Alignment of Selection Indices]
\label{lem:alignment}
If $H$ and $A$ lie in the collapse fiber $\mathcal{F}(x)$ and
$n_j^{H}$, $n_j^{A}$ are their $j$th selection indices, then
\[
D_H(n_j^{H}) = D_A(n_j^{A}) = x_j.
\]
\end{lemma}

\begin{proof}
Membership in the fiber means $\pi(H)=\pi(A)=x$. The canonical output of $x$ is
the sequence $(x_j)_{j\ge 0}$. Since $n_j^{H}$ and $n_j^{A}$ pick out the $j$th
selected digit of $H$ and $A$, the exposed digits must equal $x_j$.
\end{proof}

Alignment ensures that although selector positions may differ, the symbolic
content of the selected digits is synchronized across the fiber. This allows
tails to be replaced after matching selected digits.

\section{Prefix Completion and Tail Extraction}

\begin{definition}[Sewing at the $j$th Selection Index]
Let $H$ and $A$ be identities in $\mathcal{X}^{*}$ and let
\[
h_j = n_j^{H}, \qquad a_j = n_j^{A}.
\]
The sewed identity $G = H \,\widehat{\ }_{j}\, A$ is defined by
\[
G(n) =
\begin{cases}
H(n) & n \le h_j,\\
A(n - h_j + a_j) & n > h_j.
\end{cases}
\]
\end{definition}

This construction keeps the prefix of $H$ up to its $j$th selected digit and then
continues with the symbolic pattern of $A$ beginning at its corresponding
selection index.

\section{Sewing Preserves Collapse}

\begin{lemma}[Tail Sewing Preserves Collapse]
\label{lem:sewing-preserves-collapse}
If $H,A \in \mathcal{F}(x)$ and $G = H \,\widehat{\ }_{j}\, A$, then
$G \in \mathcal{F}(x)$.
\end{lemma}

\begin{proof}
Up to index $h_j$ the identity $G$ agrees with $H$, so their first $j$ selected
digits coincide. For $n>h_j$, the definition of $G$ reproduces the behavior of
$A$ starting at $a_j$, so the $(j+1)$st and subsequent selected digits appear in
the same order and with the same values as in $A$. Thus $G$ and $A$ share the
same canonical output, namely the expansion of $x$.
\end{proof}

Thus sewing modifies the symbolic tail without affecting the collapsed real.

\section{Dependency Bounds and Controlled Sewing}

\begin{lemma}[Controlled Sewing]
\label{lem:controlled-sewing}
Let $\mathcal{P}$ be a finite family of structural projections with uniform
dependency bound $N = B_{\mathcal{P}}(\varepsilon)$. If $H,A \in \mathcal{F}(x)$
and $j$ is such that $h_j \ge N$, then the identity $G = H \,\widehat{\ }_{j}\, A$
satisfies
\[
|\Phi(G) - \Phi(H)| < \varepsilon
\qquad\text{for all } \Phi \in \mathcal{P}.
\]
\end{lemma}

\begin{proof}
Since $G$ and $H$ agree on all coordinates up to $h_j$ and $h_j \ge N$, the
prefix stabilization condition for the uniform dependency bound yields the
desired inequality for each $\Phi\in\mathcal{P}$.
\end{proof}

Thus sewing modifies only the unobserved tail and leaves all observers stable at
the required precision.

\section{A Selection Index Lower Bound}

\begin{lemma}[Selection Index Growth]
\label{lem:selection-index-lower-bound}
Let $H$ expose infinitely many digits with selection indices $h_j$. Then:

\begin{enumerate}
    \item For every $N$ there exists $j$ such that $h_j \ge N$.
    \item If $\eta(H)>0$, then for all sufficiently large $j$,
    \[
    h_j \le \frac{j}{\eta(H)}.
    \]
\end{enumerate}
\end{lemma}

\begin{proof}
The first claim follows because $(h_j)$ is strictly increasing without bound.
The second follows from the definition of lower density:
\[
\liminf_{N\to\infty} \frac{1}{N} \bigl|\{0\le n < N : M(n)=D\}\bigr| = \eta(H),
\]
which implies $\frac{j}{h_j}\to \eta(H)$ along a subsequence.
\end{proof}

This lemma guarantees the existence of alignment indices beyond any required
dependency bound.

\section{Full Sewing Lemma}

\begin{lemma}[Full Sewing Lemma]
\label{lem:full-sewing}
Let $\mathcal{P}$ be a finite family of structural projections with uniform
dependency bound $N = B_{\mathcal{P}}(\varepsilon)$. Let $H,A \in
\mathcal{F}(x)$ and choose $j$ such that $h_j \ge N$. Then the sewed identity
$G = H \,\widehat{\ }_{j}\, A$ satisfies:

\begin{enumerate}
    \item $G \in \mathcal{F}(x)$,
    \item $|\Phi(G) - \Phi(H)| < \varepsilon$ for every $\Phi\in\mathcal{P}$.
\end{enumerate}
\end{lemma}

\begin{proof}
Collapse preservation follows from Lemma~\ref{lem:sewing-preserves-collapse}.
Observer stability follows from Lemma~\ref{lem:controlled-sewing}.
\end{proof}

This is the key finite–information lemma underlying the mimicry construction.

\section{Computability of Sewing}

\begin{lemma}[Computable Sewing]
\label{lem:computable-sewing}
If $H$ and $A$ are computable identities in $\mathcal{F}(x)$ and $j$ is
computable from $H$, then $H \,\widehat{\ }_{j}\, A$ is computable.
\end{lemma}

\begin{proof}
The selection indices $(h_j)$ and $(a_j)$ are computable from $H$ and $A$. Given
$j$, the definition of sewing is an explicit rule computing $G(n)$ from $H$ and
$A$. Thus $G$ is computable as a coordinatewise effective sequence.
\end{proof}

\section{Summary}

This appendix formalized the technical facts used in the alignment and sewing
arguments:

\begin{itemize}
    \item alignment synchronizes selected digits of identities in the same fiber,
    \item sewing replaces tails while preserving collapse,
    \item dependency bounds ensure observers see only prefixes,
    \item sewing is computable and preserves effective fibers.
\end{itemize}

These results supply the structural and computational backbone of the diagonal
mimicry method used to establish structural indistinguishability.
