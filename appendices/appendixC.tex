\chapter{Alignment and Sewing: Full Technical Proofs}
\label{appendix:alignment-sewing}

\section{Introduction}

This appendix provides complete proofs of the alignment and sewing principles
that support the finite information constructions used in the
indistinguishability and diagonalization results. The main goals are:

\begin{itemize}
    \item to show that identities in the same collapse fiber expose the same
          canonical digit sequence,
    \item to verify that tails may be replaced once selected digits are aligned,
    \item to establish that dependency bounds guarantee observer stability under
          tail modification,
    \item to demonstrate that sewing operations preserve computability and
          effective fiber membership.
\end{itemize}

These results formalize the finite prefix reasoning used in the alignment and
sewing chapter and justify the mimicry construction that produces
indistinguishable identities.

\section{Canonical Output and Selection Indices}

Let $G = (M,D,K)$ be a generative identity. The selected positions form the
increasing sequence
\[
n_0 < n_1 < n_2 < \cdots,
\]
where $n_j$ is the $j$th index for which $M(n) = D$.

The canonical output of $G$ is the sequence of exposed digits
\[
d_0, d_1, d_2, \ldots,
\qquad
d_j = D(n_j).
\]
If $G$ exposes digits infinitely often, this defines a valid base expansion of
a point in $[0,1]$.

Two identities $H$ and $A$ lie in the same collapse fiber exactly when their
canonical outputs agree. If the expansion of $x$ is $(x_j)$, then for every $j$,
\[
D_H(n_j^{H}) = D_A(n_j^{A}) = x_j.
\]

\section{Alignment of Selected Digits}

\begin{lemma}[Alignment of Selection Indices]
\label{lem:alignment}
If $H$ and $A$ lie in the collapse fiber $\mathcal{F}(x)$ and
$n_j^{H}$, $n_j^{A}$ denote their $j$th selection indices, then
\[
D_H(n_j^{H}) = D_A(n_j^{A}) = x_j.
\]
\end{lemma}

\begin{proof}
Membership in the fiber means $\pi(H)=\pi(A)=x$. The canonical output of $x$ is
$(x_j)$. The definition of selection indices ensures that $n_j^{H}$ and
$n_j^{A}$ correspond to the $j$th exposed digit in $H$ and $A$. Therefore each of
these digits equals $x_j$.
\end{proof}

Alignment identifies matching symbolic content across the fiber despite
differences in exposure positions. This makes it possible to splice identities
after aligned selections.

\section{Prefix Preservation and Tail Extraction}

\begin{definition}[Sewing at the $j$th Selection Index]
Let $H$ and $A$ be identities in $\mathcal{X}^{*}$ with selection indices
\[
h_j = n_j^{H},
\qquad
a_j = n_j^{A}.
\]
Define the sewed identity $G = H \widehat{\ }_{j} A$ by
\[
G(n) =
\begin{cases}
H(n) & n \le h_j,\\[4pt]
A(n - h_j + a_j) & n > h_j.
\end{cases}
\]
\end{definition}

This construction keeps the prefix of $H$ through its $j$th selected digit and
then continues with the tail of $A$ starting from its $j$th selected digit.

\section{Sewing Preserves Collapse}

\begin{lemma}[Tail Sewing Preserves Collapse]
\label{lem:sewing-preserves-collapse}
If $H,A \in \mathcal{F}(x)$ and $G = H \widehat{\ }_{j} A$, then
$G \in \mathcal{F}(x)$.
\end{lemma}

\begin{proof}
Up to index $h_j$, $G$ agrees with $H$ and therefore shares the first $j$
exposed digits with $H$. For $n > h_j$, the identity $G$ reproduces the symbolic
pattern of $A$ beginning at $a_j$. Since $A$ exposes the $(j+1)$st and later
digits of $x$ at these indices, $G$ exposes the same values in the same order.
Therefore $\pi(G)=x$.
\end{proof}

Thus sewing does not change the collapsed value.

\section{Dependency Bounds and Controlled Sewing}

Let $\mathcal{P}$ be a finite family of structural projections with uniform
dependency bound $N = B_{\mathcal{P}}(\varepsilon)$.

\begin{lemma}[Controlled Sewing]
\label{lem:controlled-sewing}
Let $H,A \in \mathcal{F}(x)$ and choose $j$ with $h_j \ge N$. Then for the sewed
identity $G = H \widehat{\ }_{j} A$,
\[
|\Phi(G) - \Phi(H)| < \varepsilon
\qquad
\text{for all } \Phi \in \mathcal{P}.
\]
\end{lemma}

\begin{proof}
The identities $G$ and $H$ agree on all coordinates up to $h_j$, and
$h_j \ge N$. Thus they agree on the prefix required by the uniform dependency
bound. By prefix stabilization, every $\Phi \in \mathcal{P}$ evaluates them
within $\varepsilon$.
\end{proof}

This lemma shows that observers see no difference between $G$ and $H$ once the
prefix requirement is met.

\section{Growth of Selection Indices}

\begin{lemma}[Selection Index Growth]
\label{lem:selection-index-lower-bound}
Let $H$ expose infinitely many digits with selection indices $h_j$. Then:

\begin{enumerate}
    \item For every $N$ there exists $j$ such that $h_j \ge N$.
    \item If $H$ has positive lower exposure density $\eta(H) > 0$, then for all
          sufficiently large $j$,
          \[
          h_j \le \frac{j}{\eta(H)}.
          \]
\end{enumerate}
\end{lemma}

\begin{proof}
The sequence $(h_j)$ is strictly increasing and unbounded, which proves the
first claim.

For the second, the definition of lower density gives
\[
\liminf_{N\to\infty}
\frac{1}{N}
\bigl|\{\, n < N : M(n)=D \,\}\bigr|
=
\eta(H).
\]
Thus $j / h_j \to \eta(H)$ along a subsequence, which implies
$h_j \le j / \eta(H)$ for all sufficiently large $j$.
\end{proof}

This guarantees that selection indices eventually exceed any required prefix
bound.

\section{Full Sewing Lemma}

\begin{lemma}[Full Sewing Lemma]
\label{lem:full-sewing}
Let $\mathcal{P}$ be a finite family of structural projections with uniform
dependency bound $N = B_{\mathcal{P}}(\varepsilon)$. If $H,A \in \mathcal{F}(x)$
and $j$ satisfies $h_j \ge N$, then $G = H \widehat{\ }_{j} A$ satisfies:

\begin{enumerate}
    \item $G \in \mathcal{F}(x)$,
    \item $|\Phi(G) - \Phi(H)| < \varepsilon$ for every $\Phi \in \mathcal{P}$.
\end{enumerate}
\end{lemma}

\begin{proof}
Collapse preservation follows from Lemma~\ref{lem:sewing-preserves-collapse}.
Observer stability follows from Lemma~\ref{lem:controlled-sewing}.
\end{proof}

This is the finite information tool that supports the mimicry diagonalizer.

\section{Computability of Sewing}

\begin{lemma}[Computable Sewing]
\label{lem:computable-sewing}
If $H$ and $A$ are computable identities in $\mathcal{F}(x)$ and $j$ is
computable from $H$, then $G = H \widehat{\ }_{j} A$ is computable.
\end{lemma}

\begin{proof}
The selection indices $(h_j)$ and $(a_j)$ are computable from $H$ and $A$. Given
$j$, the definition of $G$ specifies an explicit coordinatewise procedure for
computing $G(n)$ from $H$ and $A$. Thus $G$ is computable.
\end{proof}

\section{Summary}

This appendix established the technical foundations needed for alignment,
sewing, and diagonalization. The key facts are:

\begin{itemize}
    \item identities in the same fiber expose identical canonical digits,
    \item aligned sewing preserves collapse,
    \item dependency bounds ensure observer stability under tail replacement,
    \item sewing operations preserve computability and effective fiber
          membership.
\end{itemize}

These results form the structural backbone of the indistinguishability
construction in the main text.
