\chapter{Effective Closed Sets and Pi Zero One Classes}
\label{appendix:closed-sets}

\section{Introduction}

This appendix provides the computability background needed for the effective
fiber arguments in Parts II and III. The material comes from classical
computable analysis and represented space theory as developed by Weihrauch and
Pauly. We focus on three tools used repeatedly in the main text:

\begin{enumerate}
    \item effective open and closed subsets of sequence spaces,
    \item prefix based failure conditions,
    \item Pi zero one classes as effective closed sets.
\end{enumerate}

Throughout, the ambient space is a product of discrete symbolic coordinates
equipped with the product topology, and names are viewed as infinite sequences
in Baire space.

\section{Effective Open and Closed Sets}

Let $\mathbb{N}^{\mathbb{N}}$ denote Baire space with the product topology
generated by basic cylinders
\[
[u] = \{\, p \in \mathbb{N}^{\mathbb{N}} : p[0..|u|-1] = u \,\}.
\]

\begin{definition}
A set $U \subseteq \mathbb{N}^{\mathbb{N}}$ is effectively open if it is a
computably enumerable union of basic cylinders. Its complement is an
effectively closed set.
\end{definition}

Membership in an effectively closed set can be disproved by finding a finite
prefix that forces the sequence outside the set. This prefix based
falsifiability matches the finite information principles used in structural
projections and dependency bounds in Appendix~B.

\subsection*{Prefix tests}

If $C$ is effectively closed and $p \notin C$, there exists a finite word $u$
such that $p \in [u]$ and $[u]$ is disjoint from $C$. This fact underlies
arguments about collapse fibers, where deviation from a required output digit
appears at a specific finite index.

\section{Effective Fibers}

Fix a collapse representation with output in $[0,1]$. For a computable real
number $x$, its effective fiber is
\[
\mathcal{F}_{\mathrm{eff}}(x)
=
\{\, G \in \mathcal{G}_{\mathrm{eff}} : \pi(G)=x \,\}.
\]

Here $\mathcal{G}_{\mathrm{eff}}$ consists of identities whose symbolic
coordinates are computable sequences, which can be interleaved into a single
computable name in Baire space.

\begin{proposition}
For every computable real number $x$, the effective fiber
$\mathcal{F}_{\mathrm{eff}}(x)$ is an effectively closed subset of Baire
space.
\end{proposition}

\begin{proof}
Let $(x_{j})_{j\ge 0}$ be the canonical output expansion of $x$ under the fixed
collapse representation. If $G$ fails to lie in the fiber then at some finite
index a selected coordinate produces a value different from $x_{j}$. This
finite deviation determines a basic cylinder disjoint from the fiber.
Enumerating all such cylinders enumerates the complement, so the complement is
effectively open. Therefore the fiber is effectively closed.
\end{proof}

Effective closedness explains why fibers support tail replacement: any finite
prefix consistent with the fiber can be extended without violating the defining
condition.

\section{Pi Zero One Classes}

\begin{definition}
A Pi zero one class is an effectively closed subset of Baire space.
\end{definition}

Pi zero one classes arise naturally as solution sets to prefix determined
constraints. Effective fibers fit this pattern: a generative identity belongs
to $\mathcal{F}_{\mathrm{eff}}(x)$ exactly when no finite prefix violates the
required output pattern.

\subsection*{Properties}

Pi zero one classes are closed under finite prefix extension and contain
computable points whenever they are nonempty. These properties are central in
diagonalization arguments:

\begin{itemize}
    \item finite prefix agreement ensures that dependency bounds can be met,
    \item tail freedom allows divergence beyond any fixed prefix,
    \item membership is preserved during sewing constructions.
\end{itemize}

Appendix~C uses these facts for the lemmas leading to the diagonalizer. 
Appendix~D relies on them for tail replacement inside collapse fibers.

\section{Application to Collapse Geometry}

This viewpoint supports several structural features of collapse geometry:

\begin{itemize}
    \item Effective fibers are closed under tail modification because membership
          is enforced by finite failures.
    \item Observers act on finite prefixes, so indistinguishability arises when
          two identities agree on all prefixes of interest.
    \item The diagonalizer in Chapter~\ref{chap:indistinguishability} succeeds
          because each stage extends a prefix consistent with the Pi zero one
          definition of the fiber.
\end{itemize}

\section{Summary}

Effective openness and closedness encode prefix based computability conditions.
Effective fibers are Pi zero one classes and therefore exhibit the tail freedom
used throughout Parts II and III. These structural properties explain why
finite observers cannot recover hidden generative structure and why collapse
fibers contain rich computably accessible families of identities.
