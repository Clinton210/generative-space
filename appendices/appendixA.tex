\appendix
\chapter{Computable Analysis Background}

\section{Introduction}

This appendix summarizes the background from computable analysis and Type--2 computability that is used in Parts~I to~IV of the monograph.  
The purpose is not to provide a comprehensive treatment, but to gather the definitions and standard results that justify the computability assumptions underlying the effective core $\mathcal{G}_{\mathrm{eff}}$ and the dependency bounds for secondary projections.

Classical references include the work of Turing~\cite{turing1936computablenumbers}, Pour-El and Richards, and the monograph of Weihrauch~\cite{weihrauch2000computable}.  
Only the material needed to support the generative framework is presented here.

\section{Computable Sequences and Names}

A sequence $s : \mathbb{N} \to A$ over a finite alphabet $A$ is \emph{computable} if there exists a total Turing machine that outputs $s(n)$ on input $n$.  
This definition applies to each layer of a generative identity:
\[
M : \mathbb{N} \to \{D,K\}, \qquad
D : \mathbb{N} \to \{0,1,\ldots,b-1\}, \qquad
K : \mathbb{N} \to \Sigma.
\]

The product topology on the sequence space $A^{\mathbb{N}}$ is induced by finite-prefix agreement.  
A finite prefix of $s$ is written $s{\upharpoonright}n = (s(0), \ldots, s(n-1))$.  
All computability notions respect this topology.

In computable analysis, elements of function spaces or metric spaces are often represented by infinite sequences called \emph{names}.  
In the generative framework, elements of $\mathcal{X}$ play the role of names.  
Collapse identifies classical real numbers with equivalence classes of these names.

\section{Computable Reals}

A real number $x$ is \emph{computable} if there is a Turing machine that outputs a rational approximation of $x$ to arbitrary precision.  
Equivalently, $x$ is computable if it has a computable base-$b$ expansion.

The set of computable reals is denoted $\mathbb{R}_c$.  
It is countable and dense in $[0,1]$.  
Chapter~2 shows that
\[
\pi(\mathcal{G}_{\mathrm{eff}}) = \mathbb{R}_c,
\]
where $\pi$ is the collapse map.

\section{Computable Functionals on Sequence Spaces}

Let $A^{\mathbb{N}}$ be a sequence space with the product topology.  
A function
\[
\Phi : A^{\mathbb{N}} \to \mathbb{R}
\]
is \emph{computable} if, given a name of the input sequence, there is a Turing machine that produces rational approximations of $\Phi(s)$ to arbitrary precision.

A standard result in Type--2 computability states that computable functionals depend only on finite prefixes of their input to produce an approximation within any prescribed accuracy.  
This property is reflected in the dependency bounds defined in Chapter~6.

\begin{proposition}[Finite Prefix Dependence]
\label{prop:finite-prefix}
Let $\Phi : A^{\mathbb{N}} \to \mathbb{R}$ be a computable functional.  
For each rational $\varepsilon > 0$ there exists an integer $N$ such that if $s{\upharpoonright}N = t{\upharpoonright}N$ then
\[
|\Phi(s) - \Phi(t)| < \varepsilon.
\]
\end{proposition}

\begin{proof}
This is a standard consequence of the fact that a Type--2 Turing machine performing an $\varepsilon$ approximation reads only finitely many input symbols before halting.  
See Weihrauch~\cite{weihrauch2000computable} for details.
\end{proof}

This proposition is the foundation for the dependency bounds and uniform dependency bounds used in Part~III and Part~IV.

\section{Computable Projections on the Generative Space}

Given $G = (M,D,K) \in \mathcal{G}_{\mathrm{eff}}$, each layer is a computable sequence.  
A secondary projection
\[
\Phi : \mathcal{G}_{\mathrm{eff}} \to \mathbb{R}^k
\]
is computable if each of its coordinate functions is computable in the sense above.  
The finite-prefix property applies to the concatenation of $(M,D,K)$ viewed as a single sequence over a larger finite alphabet.

\begin{proposition}
Every computable secondary projection on $\mathcal{G}_{\mathrm{eff}}$ has a dependency bound in the sense of Chapter~6.
\end{proposition}

\begin{proof}
Combine Proposition~\ref{prop:finite-prefix} with the fact that $(M,D,K)$ can be encoded as a single computable sequence.
\end{proof}

This validates the prefix-dependent nature of all computable measurements of generative identities and supports the diagonalizer construction of Chapter~8.

\section{Collapse in the TTE Framework}

The collapse map
\[
\pi : \mathcal{X} \to [0,1]
\]
decodes the digit subsequence selected by the mixer.  
In the language of computable analysis, collapse is a \emph{representation} of real numbers by names in the generative space.  
Computability of collapse on $\mathcal{G}_{\mathrm{eff}}$ follows from the computability of digit selection and base-$b$ decoding.

\begin{proposition}
If $G \in \mathcal{G}_{\mathrm{eff}}$, then $\pi(G)$ is a computable real number.
\end{proposition}

\begin{proof}
Both the digit subsequence and the positional evaluation are computable.
\end{proof}

This connection ensures that collapse behaves as expected under composition with computable functions on $[0,1]$, a fact used in Chapter~10.

\section{Summary}

This appendix provides the formal background for the computability assumptions used throughout the monograph.  
Effective generative identities correspond to computable names in the sense of Type--2 computability.  
Secondary projections are computable functionals on these names with finite prefix dependence.  
Collapse extracts the classical real numbers represented by these names.  
These definitions and results support the structural analysis developed in Parts~I to~IV.
