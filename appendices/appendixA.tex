\chapter{Computable Analysis Background}

\section{Introduction}

This appendix summarizes the basic tools from computable analysis and
Type--2 computability that are used throughout Parts~I--IV of the monograph.
The purpose is not to survey the full subject, but to gather the definitions
and standard results needed to justify the effective core
$\mathcal{G}_{\mathrm{eff}}$ and the constructions involving dependency bounds,
secondary projections, and computable collapse.

Classical references include the foundational work of Turing
\cite{turing1936computablenumbers} and the standard development in computable
analysis presented by Weihrauch \cite{weihrauch2000computable}.

\section{Computable Sequences}

Let $A$ be a finite alphabet.  
A sequence $s : \mathbb{N} \to A$ is \emph{computable} if there exists a Turing
machine that, on input $n$, outputs $s(n)$.  
This definition applies directly to the three layers of a generative identity:
\[
M : \mathbb{N} \to \{D,K\}, \qquad
D : \mathbb{N} \to \{0,1,\ldots,b-1\}, \qquad
K : \mathbb{N} \to \Sigma,
\]
where $\Sigma$ is the finite meta alphabet.

The sequence space $A^{\mathbb{N}}$ carries the product topology generated by
finite-prefix agreement.  
For $s \in A^{\mathbb{N}}$, the prefix of length $n$ is written
\[
s{\upharpoonright}n = (s(0),\ldots,s(n-1)).
\]
All computability notions respect this topology: a Turing machine accessing an
input sequence reads only finitely many symbols before producing a finite
portion of the output.

\section{Names and Represented Spaces}

In computable analysis, elements of a represented space are given by
\emph{names}---infinite sequences encoding potentially infinite information.
In the generative framework, a triple $G=(M,D,K)\in \mathcal{X}$ acts as a name
for the classical real number obtained through collapse.  
Chapter~2 shows that collapse respects computability and that
$\mathcal{G}_{\mathrm{eff}}$ corresponds exactly to the computable names of
reals.

\section{Computable Real Numbers}

A real number $x$ is \emph{computable} if a Turing machine can produce a
sequence of rational approximations that converge to $x$ at a computable rate.
Equivalently, $x$ is computable if it has a computable base-$b$ expansion.
The set of computable reals is denoted $\mathbb{R}_c$.

In Chapter~2 it is shown that
\[
\pi(\mathcal{G}_{\mathrm{eff}}) = \mathbb{R}_c,
\]
where $\pi$ is the collapse map that interprets the selected digits of an
identity as a base-$b$ expansion.  
Thus every computable real has an effective generative representation.

\section{Computable Functionals on Sequence Spaces}

Let $A^{\mathbb{N}}$ be a sequence space and consider a functional
\[
\Phi : A^{\mathbb{N}} \to \mathbb{R}.
\]
The functional $\Phi$ is \emph{computable} if there exists a Type--2 Turing
machine which, given oracle access to a name of $s \in A^{\mathbb{N}}$, outputs
rational approximations of $\Phi(s)$ to arbitrary precision.

A core principle of the Type--2 setting is that computable functionals depend
only on a finite prefix of their input when producing any specified
approximation.  
This property underlies the dependency bounds used in Chapters~6--8.

\begin{proposition}[Finite Prefix Dependence]
\label{prop:finite-prefix}
Let $\Phi : A^{\mathbb{N}} \to \mathbb{R}$ be computable.  
For every rational $\varepsilon > 0$ there exists $N \in \mathbb{N}$ such that
\[
s{\upharpoonright}N = t{\upharpoonright}N
\quad\Longrightarrow\quad
|\Phi(s) - \Phi(t)| < \varepsilon.
\]
\end{proposition}

\begin{proof}
A Type--2 computation of an $\varepsilon$-approximation of $\Phi(s)$ inspects
only finitely many input symbols before halting.  
See \cite{weihrauch2000computable}.
\end{proof}

This is the foundation for the dependency bounds and uniform bounds developed
in Part~III.

\section{Secondary Projections on the Generative Space}

Let $G=(M,D,K)\in\mathcal{G}_{\mathrm{eff}}$.  
Since each coordinate sequence is computable, the triple can be encoded as a
single sequence over a larger finite alphabet.  
A secondary projection
\[
\Phi : \mathcal{G}_{\mathrm{eff}} \to \mathbb{R}^k
\]
is \emph{computable} if each coordinate function is computable in the sense
above.  
The finite-prefix dependence of $\Phi$ then follows by applying
Proposition~\ref{prop:finite-prefix} to the encoded triple.

\begin{proposition}
Every computable secondary projection on $\mathcal{G}_{\mathrm{eff}}$ admits a
computable dependency bound in the sense of Chapter~6.
\end{proposition}

\begin{proof}
Encode $(M,D,K)$ as a single computable sequence and apply
Proposition~\ref{prop:finite-prefix}.
\end{proof}

This validates the prefix-limited nature of all computable observations of
generative identities.

\section{Computability of Collapse}

The collapse map
\[
\pi : \mathcal{X} \to [0,1]
\]
extracts the subsequence of digits selected by the mixer and interprets these
digits as a base-$b$ expansion.  
In the TTE framework, collapse is a representation transforming a name in
$\mathcal{X}$ into the real it encodes.

\begin{proposition}
If $G \in \mathcal{G}_{\mathrm{eff}}$ then $\pi(G)$ is a computable real.
\end{proposition}

\begin{proof}
The digit subsequence is computable from $(M,D,K)$, and the base-$b$
evaluation is a computable real-valued function.
\end{proof}

This ensures that collapse interacts correctly with computable real-valued
functions, as used in Chapters~9 and~10.

\section{Summary}

This appendix provides the minimal background from computable analysis required
in the monograph.  
Effective generative identities are precisely computable names in the sense of
Type--2 computability.  
Secondary projections are computable functionals with finite-prefix
dependence.  
Collapse is a computable representation that extracts the classical magnitude
encoded by a generative identity.  
These principles support the structural analysis and diagonalization
constructions developed in Parts~I--IV.
