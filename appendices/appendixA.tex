\chapter{Type--2 Effectivity Essentials}

\section{Introduction}

This appendix summarizes the basic notions from Type--2 Effectivity (TTE) and
computable analysis that are used implicitly in the main text.  
The aim is not to give a complete treatment, but to explain the background
needed for structural projections, dependency bounds, effective fibers, and
the diagonalizer.  
Standard references include Weihrauch's monograph on computable analysis, the
work of Brattka, Hertling, and Weihrauch on represented spaces, and Pauly's
surveys on synthetic descriptive set theory.

We focus on three themes:

\begin{enumerate}
    \item names for real numbers and elements of product spaces,
    \item computable functionals on sequence spaces and their moduli of continuity,
    \item effective closed sets and $\Pi^{0}_{1}$ classes.
\end{enumerate}

Throughout, $\mathbb{N} = \{0,1,2,\ldots\}$ and sequences are indexed from
zero.

\section{Names and Represented Spaces}

\subsection{Baire space and Cantor space}

\emph{Baire space} is the set $\mathbb{N}^{\mathbb{N}}$ of all infinite
sequences of natural numbers.  
\emph{Cantor space} is the set $\{0,1\}^{\mathbb{N}}$ of infinite binary
sequences.  
Both spaces carry the product topology generated by basic open sets determined
by finite prefixes.

Baire and Cantor space are the standard domains for TTE.  
Elements of more complicated spaces, such as real numbers or continuous
functions, are represented by infinite sequences in these spaces.

\subsection{Represented spaces}

A \emph{represented space} is a pair $(X,\delta_X)$ where $X$ is a set and
\[
\delta_X : \subseteq \mathbb{N}^{\mathbb{N}} \to X
\]
is a partial surjective map.  
Elements $p \in \mathbb{N}^{\mathbb{N}}$ with $\delta_X(p) = x$ are called
\emph{names} of $x$.

Different representation maps encode different ways of describing objects in
$X$.  
In computable analysis, the usual Cauchy representation of real numbers is
obtained by interpreting $p \in \mathbb{N}^{\mathbb{N}}$ as a rapidly
converging sequence of rational approximations to a real number.

\subsection{Computable points}

A point $x \in X$ is \emph{computable} if it has a computable name, that is,
there is a computable sequence $p \in \mathbb{N}^{\mathbb{N}}$ such that
$\delta_X(p) = x$.

In the Generative Identity Framework, the effective core
$\mathcal{G}_{\mathrm{eff}}$ of the generative space plays the role of
computable elements.  
Each generative identity $G = (M,D,K)$ corresponds to a name that interleaves
the symbols of the three streams in a computable way, and vice versa.

\section{Type--2 Machines and Computable Maps}

\subsection{Type--2 Turing machines}

A Type--2 Turing machine is an abstract device that reads and writes infinite
sequences.  
It has:

\begin{itemize}
    \item a read only input tape containing $p \in \mathbb{N}^{\mathbb{N}}$,
    \item a write only output tape on which it produces $q \in \mathbb{N}^{\mathbb{N}}$,
    \item a work tape for finite internal computation.
\end{itemize}

The machine is required to produce each output symbol $q(n)$ after reading
only finitely many symbols of the input.  
This finite use condition ensures that the induced map on Baire space is
continuous in the product topology.

\subsection{Computable maps between represented spaces}

Let $(X,\delta_X)$ and $(Y,\delta_Y)$ be represented spaces.  
A (partial) function $f : X \to Y$ is \emph{computable} if there exists a
Type--2 Turing machine $M$ such that, for every $p$ with $\delta_X(p) = x$,
the output sequence $M(p)$ is defined and satisfies
\[
\delta_Y(M(p)) = f(x).
\]

Intuitively, given infinite access to a name of $x$, the machine produces a
name of $f(x)$ using only finite prefixes of the input at each step.

\subsection{Continuity and computability}

A basic theorem of TTE states that every computable function between
represented spaces is continuous with respect to the induced topologies.  
In many natural representations, the converse holds as well: any continuous
function with an effective modulus of continuity is computable.

In this monograph, structural projections are continuous real valued
functionals on a product space of symbolic sequences.  
When such functionals are computable, they admit effective moduli of
continuity that appear as dependency bounds.

\section{Moduli of Continuity and Dependency Bounds}

\subsection{Moduli of continuity on sequence spaces}

Let $f : \mathbb{N}^{\mathbb{N}} \to \mathbb{R}$ be continuous in the product
topology.  
For each $\varepsilon > 0$ there exists an $N$ such that any two sequences
that agree on their first $N$ terms have $f$ values within $\varepsilon$.

A \emph{modulus of continuity} is a function
\[
\mu : (0,1] \to \mathbb{N}
\]
such that agreement on the first $\mu(\varepsilon)$ indices implies
\[
|f(p) - f(q)| < \varepsilon.
\]
If $f$ is computable, then $\mu$ can be chosen to be computable as well.

\subsection{Dependency bounds in the generative space}

The generative space $\mathcal{X}^{*}$ is a product of discrete alphabets,
equipped with the product topology.  
A structural projection
\[
\Phi : \mathcal{X}^{*} \to \mathbb{R}
\]
is continuous if and only if there exists a function $B_{\Phi}$ such that
agreement of generative identities $G$ and $H$ on their first
$B_{\Phi}(\varepsilon)$ coordinates implies
\[
|\Phi(G) - \Phi(H)| < \varepsilon.
\]

In the main text, $B_{\Phi}$ is called a \emph{dependency bound}.  
This is exactly a modulus of continuity for $\Phi$ in the sense of TTE,
presented in a way that emphasizes its combinatorial meaning: the value of
$\Phi(G)$ to precision $\varepsilon$ depends only on a finite prefix of the
identity.

For a finite family of projections, a common bound is obtained by taking the
maximum of the individual bounds.  
This yields a uniform dependency bound that controls the entire family.

\subsection{Prefix stabilization and tail invariance}

If $B_{\Phi}$ is a dependency bound for $\Phi$, then for any $\varepsilon$,
agreement on the prefix of length $B_{\Phi}(\varepsilon)$ guarantees that
changes to the tail beyond this prefix cannot alter the value of $\Phi$ by
more than $\varepsilon$.

This prefix stabilization property is used repeatedly in Part~III and
Part~IV.  
It shows that continuous observers consume only finitely much information at
any fixed precision, which in turn permits tail modifications that preserve
all observations in a given finite family.

\subsection{Effective fibers as \texorpdfstring{$\Pi^{0}_{1}$}{Pi01} classes}

\subsection{Effective open and closed sets}

A subset $U \subseteq \mathbb{N}^{\mathbb{N}}$ is \emph{effectively open} (or
$\Sigma^{0}_{1}$) if it is a union of a computably enumerable family of basic
open sets.  
The complement of an effectively open set is \emph{effectively closed}
(or $\Pi^{0}_{1}$).

In Cantor or Baire space, basic open sets are specified by finite prefixes.
Thus an effectively closed set $F$ is one for which membership can be disproved
by finite evidence.  
To see that $p \notin F$, it suffices to find a finite prefix that forces
$p$ into the complement.

\subsection{Effective fibers as \texorpdfstring{$\Pi^{0}_{1}$}{Pi01} classes}

The effective core of the generative space,
\[
\mathcal{G}_{\mathrm{eff}} \subseteq \mathcal{X}^{*},
\]
inherits a natural representation from its presentation as a subspace of a
finite alphabet product.  
The effective fiber associated with a computable real $x$ is
\[
\mathcal{F}_{\mathrm{eff}}(x)
  = \{ G \in \mathcal{G}_{\mathrm{eff}} : \pi(G) = x \}.
\]

This set is a $\Pi^{0}_{1}$ class in the sense that $G \notin
\mathcal{F}_{\mathrm{eff}}(x)$ can be witnessed by a finite prefix where the
canonical output deviates from the prescribed digit sequence of $x$.

In the main text, this perspective is used to justify the existence of
computable identities inside the fiber and to support constructions that
modify tails while preserving membership in the fiber.

\section{Application to the Generative Identity Framework}

The TTE machinery summarized above arises in the monograph in the following
ways.

\begin{itemize}
    \item The generative space $\mathcal{X}^{*}$ is a represented space whose
          elements are generative identities.  
          The effective core $\mathcal{G}_{\mathrm{eff}}$ corresponds to those
          identities that admit computable names.

    \item Structural projections are continuous real valued functionals on
          $\mathcal{X}^{*}$.  
          When such projections are computable, they admit computable
          dependency bounds, which are moduli of continuity in the TTE sense.

    \item Effective fibers $\mathcal{F}_{\mathrm{eff}}(x)$ are $\Pi^{0}_{1}$
          classes of identities that collapse to a fixed real number.  
          The nonemptiness and internal structure of these classes are used in
          the construction of the meta diagonalizer.

    \item Prefix stabilization and tail invariance are direct consequences of
          continuity and the existence of dependency bounds.  
          These properties express the finite information content of
          observations and allow the diagonalizer to introduce divergence
          beyond the reach of any finite family of observers.
\end{itemize}

The framework of represented spaces and Type--2 computability therefore
provides a conceptual foundation for the generative identity framework.  
It explains why structural projections necessarily depend on finite prefixes,
why effective fibers admit rich internal structure, and why diagonalization
against continuous observers is possible.
