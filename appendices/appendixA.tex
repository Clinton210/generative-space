\chapter{Type 2 Effectivity and Computable Structure}
\label{appendix:tte}

\section{Introduction}

This appendix summarizes the background from Type 2 Effectivity (TTE) and
computable analysis that underlies collapse geometry, structural projections,
dependency bounds, and effective fibers.  
The aim is not to develop a full theory but to present the essential tools
needed throughout the monograph.  
Standard references include the monographs of Weihrauch, Brattka, Hertling, and
Pauly on represented spaces and computable analysis.

Three themes appear repeatedly in the main text:
\begin{enumerate}
    \item names for elements of sequence spaces and real numbers,
    \item computable functionals with effective moduli of continuity,
    \item effective closed sets and $\Pi^{0}_{1}$ classes in product spaces.
\end{enumerate}

Throughout, sequences are indexed from zero and $\mathbb{N}$ denotes the set
$\{0,1,2,\ldots\}$.

\section{Represented Spaces and Names}

\subsection{Baire and Cantor space}

Baire space is the space $\mathbb{N}^{\mathbb{N}}$ of infinite sequences of
natural numbers.  
Cantor space is the space $\{0,1\}^{\mathbb{N}}$ of infinite binary sequences.
Both carry the product topology generated by basic open sets determined by
finite prefixes.  
They serve as standard domains for describing names of mathematical objects in
TTE.

\subsection{Represented spaces}

A represented space is a pair $(X,\delta_{X})$ where $X$ is a set and
\[
\delta_{X} : \subseteq \mathbb{N}^{\mathbb{N}} \to X
\]
is a partial surjection.  
The sequence $p$ is called a \emph{name} of $x$ when $\delta_{X}(p)=x$.
Different representation maps give different codings of the same underlying
objects.

Real numbers are typically represented by rapidly converging sequences of
rational approximations, which form computable names under the standard Cauchy
representation.

\subsection{Computable points}

A point $x \in X$ is computable if it has a computable name.  
In the Generative Identity Framework, the effective core
$\mathcal{G}_{\mathrm{eff}}$ consists of generative identities whose selector,
digit, and meta streams are computable.  
These streams can be interleaved into a single sequence in Baire space to form
a computable name.

\section{Type 2 Machines and Computable Maps}

\subsection{Type 2 Turing machines}

A Type 2 Turing machine reads an infinite input sequence and produces an
infinite output sequence.  
To produce the output symbol $q(n)$, it may inspect only finitely many input
symbols.  
This finite use property implies that the induced map on Baire space is
continuous with respect to the product topology.

\subsection{Computable maps between represented spaces}

Let $(X,\delta_{X})$ and $(Y,\delta_{Y})$ be represented spaces.  
A function $f : X \to Y$ is computable if there exists a Type 2 Turing machine
that converts any name of $x$ into a name of $f(x)$.

Intuitively, the machine computes approximations to $f(x)$ using only finite
information from the approximations to $x$.

\subsection{Continuity and computability}

A foundational theorem of TTE states that every computable function between
represented spaces is continuous.  
Conversely, many continuous maps that admit effective moduli of continuity are
computable.

In the monograph, structural projections are continuous real valued maps on a
symbolic product space.  
When such projections are computable, their effective moduli of continuity
appear directly as dependency bounds.

\section{Moduli of Continuity and Dependency Bounds}

\subsection{Moduli of continuity}

Let $f : \mathbb{N}^{\mathbb{N}} \to \mathbb{R}$ be continuous.  
For each $\varepsilon > 0$ there exists $N$ such that agreement on the first
$N$ coordinates forces
\[
|f(p) - f(q)| < \varepsilon.
\]
A function
\[
\mu : (0,1] \to \mathbb{N}
\]
with this property is called a modulus of continuity.  
If $f$ is computable, $\mu$ can be chosen computable.

\subsection{Structural projections and dependency bounds}

The ambient generative space $\mathcal{G}$ is a product of discrete alphabets.
A structural projection
\[
\Phi : \mathcal{G} \to \mathbb{R}
\]
is continuous exactly when there exists a dependency bound $B_{\Phi}$ such that
agreement of generative identities on their first $B_{\Phi}(\varepsilon)$
coordinates implies
\[
|\Phi(G) - \Phi(H)| < \varepsilon.
\]

Dependency bounds express the finite information principle: observers examine
only finitely many coordinates at any fixed precision.  
For finite families of projections, a uniform dependency bound is obtained by
taking the maximum over individual bounds.

These bounds are central to prefix stabilization and to the construction of the
diagonalizer.  
Once observers are stabilized on a long finite prefix, the tail may be modified
freely without affecting their outputs.

\section{Effective Closed Sets and \texorpdfstring{$\Pi^{0}_{1}$}{Pi$^0_1$} Classes}


\subsection{Effective open and closed sets}

A set $U \subseteq \mathbb{N}^{\mathbb{N}}$ is \emph{effectively open} (or
$\Sigma^{0}_{1}$) if it is a computably enumerable union of basic open sets.
Its complement is \emph{effectively closed} (or $\Pi^{0}_{1}$).

Membership in a $\Pi^{0}_{1}$ set is falsifiable by finite evidence: observing
a finite prefix that forces the sequence outside the set.

\subsection{Effective fibers as \texorpdfstring{$\Pi^{0}_{1}$}{Pi$^0_1$} classes}


For a computable real number $x$, the effective fiber
\[
\mathcal{F}_{\mathrm{eff}}(x)
=
\{\, G \in \mathcal{G}_{\mathrm{eff}} : \pi(G) = x \,\}
\]
is a $\Pi^{0}_{1}$ class.  
Any deviation from the canonical digit sequence of $x$ can be detected by a
finite prefix, which yields an effective enumeration of the complement.

This perspective explains:
\begin{itemize}
    \item why computable identities exist inside each fiber,
    \item why fibers are closed under tail modification,
    \item why observer indistinguishability arises naturally from finite
          prefix tests.
\end{itemize}

\section{Application to the Generative Identity Framework}

The TTE concepts summarized above support the framework in several ways.

\begin{itemize}
    \item The generative space $\mathcal{G}$ is a represented space, and
          $\mathcal{G}_{\mathrm{eff}}$ consists of identities with computable
          names.

    \item Structural projections are continuous real valued maps with computable
          dependency bounds.  
          These bounds govern prefix stabilization and determine the finite
          windows that observers examine.

    \item Effective fibers are $\Pi^{0}_{1}$ classes.  
          Their structure supplies the tail freedom used in alignment, sewing,
          and diagonalization.

    \item The indistinguishability results follow from the finite informational
          nature of computable observers and the effective closedness of
          collapse fibers.
\end{itemize}

Type 2 Effectivity therefore provides the computational and topological
foundation for the generative identity framework.  
It clarifies why collapse fibers admit rich internal structure, why observers
access only finite prefixes, and why generative information cannot be
recovered from classical magnitude alone.
