\chapter{Type 2 Effectivity and Computable Structure}
\label{appendix:tte}

\section{Introduction}

This appendix summarizes the basic notions from Type 2 Effectivity (TTE) and
computable analysis that appear throughout the monograph. The purpose is not
to provide a complete treatment but to outline the background needed for
structural projections, dependency bounds, effective fibers, and prefix
indistinguishability. Standard references include the works of Weihrauch,
Brattka, Hertling, and Pauly on represented spaces and computable analysis.

Three themes recur in the main text:

\begin{enumerate}
    \item names for real numbers and elements of product spaces,
    \item computable functionals and effective moduli of continuity,
    \item effective closed sets and $\Pi^{0}_{1}$ classes.
\end{enumerate}

Throughout, $\mathbb{N}=\{0,1,2,\ldots\}$ and sequences are indexed from zero.

\section{Represented Spaces and Names}

\subsection{Baire space and Cantor space}

Baire space is the sequence space $\mathbb{N}^{\mathbb{N}}$. Cantor space is
the binary sequence space $\{0,1\}^{\mathbb{N}}$. Both carry the product
topology generated by basic open sets determined by finite prefixes. These
spaces serve as standard domains for TTE, and more complicated mathematical
objects are represented by infinite sequences in them.

\subsection{Represented spaces}

A represented space is a pair $(X,\delta_{X})$ where $X$ is a set and
\[
\delta_{X} : \subseteq \mathbb{N}^{\mathbb{N}} \to X
\]
is a partial surjection. A sequence $p$ with $\delta_{X}(p)=x$ is called a
\emph{name} of $x$. Different representation maps correspond to different
codings of objects in $X$.

For real numbers, the standard Cauchy representation interprets $p$ as a
rapidly converging sequence of rational approximations to a real value.

\subsection{Computable points}

A point $x\in X$ is \emph{computable} if it has a computable name. In the
Generative Identity Framework, the effective core
$\mathcal{G}_{\mathrm{eff}}$ consists of identities whose mixer, digit, and
meta streams can be encoded by computable sequences. Names can be obtained by
interleaving these streams into a single sequence in Baire space.

\section{Type 2 Machines and Computable Maps}

\subsection{Type 2 Turing machines}

A Type 2 Turing machine reads an infinite input sequence, writes an infinite
output sequence, and performs finite internal computation. The output symbol
$q(n)$ must be produced after inspecting only finitely many input symbols.
This finite use condition ensures the induced map on Baire space is continuous
in the product topology.

\subsection{Computable maps between represented spaces}

Let $(X,\delta_{X})$ and $(Y,\delta_{Y})$ be represented spaces. A function
$f:X\to Y$ is computable if there exists a Type 2 Turing machine that converts
any name of $x$ into a name of $f(x)$, respecting the representation maps.
Intuitively, the machine computes the value of $f(x)$ to any fixed precision
using only finite information from the name of $x$.

\subsection{Continuity and computability}

A foundational theorem of TTE states that every computable function between
represented spaces is continuous with respect to the induced topologies.
Conversely, whenever a continuous map admits an effective modulus of
continuity, it is computable.

In the monograph, structural projections are continuous real valued maps on a
product space of symbolic sequences. When these projections are computable,
they have computable moduli of continuity, which appear as dependency bounds.

\section{Moduli of Continuity and Dependency Bounds}

\subsection{Moduli of continuity}

Let $f:\mathbb{N}^{\mathbb{N}}\to\mathbb{R}$ be continuous. For each
$\varepsilon>0$ there exists $N$ such that agreement on the first $N$
coordinates guarantees the values differ by less than $\varepsilon$. A
function
\[
\mu : (0,1]\to \mathbb{N}
\]
with this property is called a modulus of continuity. If $f$ is computable,
$\mu$ may be chosen computable.

\subsection{Structural projections and dependency bounds}

The ambient generative space $\mathcal{X}$ is a product of discrete alphabets.
A structural projection
\[
\Phi:\mathcal{X}\to\mathbb{R}
\]
is continuous precisely when there exists a dependency bound $B_{\Phi}$ such
that agreement of two identities on their first $B_{\Phi}(\varepsilon)$
coordinates implies
\[
|\Phi(G)-\Phi(H)| < \varepsilon.
\]

Dependency bounds express the finite informational nature of continuous
observation. They guarantee prefix stabilization: changes to the tail beyond a
certain index do not affect the value of the observer beyond a specified
tolerance.

For a finite family of projections, a uniform bound is obtained by taking the
maximum of the individual bounds.

\section{Effective Closed Sets and \texorpdfstring{$\Pi^{0}_{1}$}{Pi01} Classes}

\subsection{Effective open and closed sets}

A set $U\subseteq\mathbb{N}^{\mathbb{N}}$ is \emph{effectively open} (or
$\Sigma^{0}_{1}$) if it is a computably enumerable union of basic open sets.
Its complement is \emph{effectively closed} (or $\Pi^{0}_{1}$). Membership in
a $\Pi^{0}_{1}$ set can be disproved by finite evidence: observing a finite
prefix that forces the sequence into the complement.

\subsection{Effective fibers as \texorpdfstring{$\Pi^{0}_{1}$}{Pi01} classes}

The effective fiber associated with a computable real $x$ is
\[
\mathcal{F}_{\mathrm{eff}}(x)
=
\{ G\in\mathcal{G}_{\mathrm{eff}} : \pi(G)=x \}.
\]
Since deviations from the canonical digit sequence of $x$ can be detected by
finite prefixes, the set of valid identities is effectively closed. Thus
$\mathcal{F}_{\mathrm{eff}}(x)$ is a $\Pi^{0}_{1}$ class.

This perspective explains why computable identities exist inside each fiber
and why tails can be modified without leaving the fiber, provided the selected
digits remain consistent with the canonical expansion of $x$.

\section{Application to the Generative Identity Framework}

The concepts summarized above support the technical development of the
framework in several ways.

\begin{itemize}
    \item The ambient generative space $\mathcal{X}$ is a represented space.
          The effective core $\mathcal{G}_{\mathrm{eff}}$ corresponds to those
          elements with computable names.

    \item Structural projections are continuous real valued maps with computable
          dependency bounds. These bounds capture the prefix dependence of
          observers and underlie the principle of prefix indistinguishability.

    \item Effective fibers $\mathcal{F}_{\mathrm{eff}}(x)$ are $\Pi^{0}_{1}$
          classes. Their nonemptiness and internal flexibility permit tail
          modification constructions inside collapse fibers.

    \item Prefix stabilization and the finiteness of dependency bounds explain
          why observers cannot distinguish identities that agree on observable
          prefixes. These properties enable the indistinguishability
          construction of Chapter \texttt{\ref{chap:indistinguishability}}.
\end{itemize}

The machinery of represented spaces and Type 2 computability therefore
provides the formal foundation for the generative identity framework. It
clarifies why continuous observation accesses only finite information, why
collapse fibers admit rich internal variation, and why structural
indistinguishability is an inherent feature of generative structure.
