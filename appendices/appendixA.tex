\chapter{Computable Structures and Represented Spaces}

\section{Introduction}

The generative framework relies on Type--2 computability to formalize effective
generative identities, computable collapse, dependency bounds, and effective
fibers.  
This appendix summarizes the foundational concepts of represented spaces,
computable functions, and effectively closed sets used throughout the monograph.
Standard references include the books of Weihrauch and of Pour-El and Richards.

\section{Baire Space and Names}

Let $\mathbb{N}^{\mathbb{N}}$ denote Baire space equipped with the product
topology and the standard notion of computability: a sequence $p \in
\mathbb{N}^{\mathbb{N}}$ is computable if $p(n)$ is uniformly computable in $n$.

A \emph{name} is a Baire-space element that encodes an object in some intended
mathematical space.  
The central idea is that computability of objects and functions is determined by
computability of their names.

\begin{definition}[Representation]
A \emph{representation} of a set $X$ is a partial surjection
\[
\delta : \subseteq \mathbb{N}^{\mathbb{N}} \to X.
\]
If $p$ satisfies $\delta(p)=x$, then $p$ is called a \emph{name} of $x$.
\end{definition}

Representations allow us to transfer computability from sequences of natural
numbers to arbitrary mathematical spaces.

\section{Computable Points and Computable Functions}

\begin{definition}[Computable Point]
Let $\delta$ be a representation of $X$.  
A point $x \in X$ is \emph{computable} if it has a computable name:
there exists a computable $p\in\mathbb{N}^{\mathbb{N}}$ with $\delta(p)=x$.
\end{definition}

\begin{definition}[Computable Function]
Let $\delta_X$ and $\delta_Y$ be representations of $X$ and $Y$.  
A function $f : X \to Y$ is \emph{computable} if there exists a computable map
$F : \mathbb{N}^{\mathbb{N}} \to \mathbb{N}^{\mathbb{N}}$ such that
\[
\delta_Y(F(p)) = f(\delta_X(p))
\]
for all $p$ naming elements in the domain of $f$.
\end{definition}

Computable functions between represented spaces generalize the notion of Type--2
Turing computability for real functions and infinite sequences.

\section{Representing Sequence Spaces}

For any discrete alphabet $A$, the sequence space $A^{\mathbb{N}}$ is naturally
represented by coding each letter using a natural number and concatenating these
codes into a Baire-space name.  
This yields an admissible representation under which:

\begin{itemize}
    \item continuity corresponds to prefix continuity,
    \item computability corresponds to prefix-computable selection of coordinates.
\end{itemize}

In particular:

\[
G = (M,D,K)\in \mathcal{X}
\]
is \emph{effective} exactly when $M$, $D$, and $K$ are computable sequences.

\section{Computable Closed Sets and \texorpdfstring{$\Pi^0_1$}{Pi-0-1} Classes}

A set is effectively closed if its complement is a union of uniformly
computably open sets.

\begin{definition}[$\Pi^0_1$ Subset]
For a represented space $X$, a set $C\subseteq X$ is a \emph{$\Pi^0_1$ set} if
there is a computable relation $R$ such that
\[
x\in C
\quad\Longleftrightarrow\quad
\forall n\, R(x,n).
\]
\end{definition}

In the sequence spaces considered in this monograph:

- A subset is $\Pi^0_1$ iff it is defined by forbidding a computably enumerable family of finite prefixes.
- Effective fibers $\mathcal{F}_{\mathrm{eff}}(x)$ are $\Pi^0_1$ classes because membership can be disproved by a finite prefix violation but never confirmed finitely.

This characterization underlies the diagonalization arguments of Part~IV.

\section{Effective Continuity and Dependency Bounds}

Every computable function between represented spaces is effectively continuous:

\begin{proposition}
Let $f : X \to Y$ be a computable function between represented spaces.  
For each precision parameter $k$, there exists a computable \emph{dependency
bound} $B(k)$ such that for all $x,x'\in X$,
\[
x|_{B(k)} = x'|_{B(k)} \;\Longrightarrow\; 
f(x)\ \text{and}\ f(x') \text{ agree to precision } 2^{-k}.
\]
\end{proposition}

This dependency bound expresses the finite observational horizon of computable
maps.  
It is essential for:

- finite-lookahead theory (Chapter~7),
- projective incompatibility (Chapter~8),
- construction of the meta-diagonalizer (Chapter~9).

\section{Computable Operations on Generative Space}

Under the standard representations:

\begin{itemize}
    \item the canonical output map $X(G)$ is computable,
    \item the collapse map $\pi$ is computable on $\mathcal{X}^*$,
    \item projection observables (digit frequency, block statistics, selector
          density) are computable,
    \item all structural projections used in Parts~III and~IV are computable maps
          in the sense above.
\end{itemize}

This guarantees that the projection-lattice theory and the incompleteness
results apply directly to the effective core $\mathcal{G}_{\mathrm{eff}}$.

\section{Summary}

This appendix provided:

\begin{itemize}
    \item the basic machinery of represented spaces,
    \item computability for points and functions,
    \item effective closed sets and $\Pi^0_1$ classes,
    \item dependency bounds for computable projections,
    \item effective continuity of canonical maps in the generative framework.
\end{itemize}

These tools support the dependency-bound arguments of Part~III and the
diagonalization arguments of Part~IV.  
Together, they form the computability-theoretic foundation for the effective
subspace of generative identities.
