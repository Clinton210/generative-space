\chapter{Mimicry Construction Details}
\label{appendix:mimicry}

\section{Introduction}

This appendix provides full technical details for the mimicry construction
used in Chapter~9 to prove the Structural Indistinguishability Theorem. The
goal is to construct a computable identity inside a collapse fiber that is
distinct from a given reference identity but indistinguishable from it by any
computable structural projection.

The construction uses three ingredients developed earlier in the text:

\begin{itemize}
    \item dependency bounds for structural projections,
    \item alignment and sewing tools from Appendix~\ref{appendix:alignment-sewing},
    \item the perfectness of effective collapse fibers.
\end{itemize}

Throughout, $x$ is a computable real with canonical digit expansion
$(x_j)_{j \ge 0}$, and $H$ is a fixed computable identity in the effective
fiber $\mathcal{F}_{\mathrm{eff}}(x)$. We assume a computable enumeration of
all computable structural projections
\[
\Phi_0, \Phi_1, \Phi_2, \ldots,
\]
each equipped with a computable dependency bound $B_k(\varepsilon)$.

\section{Preliminaries}

\subsection{Effective fibers}

The effective fiber $\mathcal{F}_{\mathrm{eff}}(x)$ is a nonempty
$\Pi^{0}_{1}$ class containing all computable identities that collapse to $x$.
It is perfect, so for any identity $H$ in the fiber and any finite prefix
length $N$ there exists another computable identity $A$ in the fiber that
agrees with $H$ on $[0..N]$ but differs at some later coordinate.

This property supplies the tail variation needed for the mimicry construction.

\subsection{Selection indices}

For any identity $G$, let
\[
n_0^{G} < n_1^{G} < n_2^{G} < \cdots
\]
list the indices where $M(n)=D$. Sewing operations from
Appendix~\ref{appendix:alignment-sewing} use selection indices to align the
tails of two identities inside the same collapse fiber.

\section{Overview of the Mimicry Construction}

We construct a sequence
\[
G_0, G_1, G_2, \ldots
\]
with the following properties:

\begin{enumerate}
    \item $G_0 = H$,
    \item $G_k \in \mathcal{F}_{\mathrm{eff}}(x)$ for all $k$,
    \item $G_{k+1}$ extends $G_k$ on a prefix of length $N_{k+1}$,
    \item for each $k$,
    \[
    |\Phi_k(G_{k+1}) - \Phi_k(H)| < \varepsilon_k,
    \]
    \item $G_{k+1}$ is chosen to be distinct from $H$ by varying the tail.
\end{enumerate}

The limit identity $G^{\sharp}$ will agree with $H$ on arbitrarily long
prefixes and therefore be indistinguishable from $H$ by every computable
projection, while still being symbolically distinct.

\subsection{Tolerances}

Define the error tolerances
\[
\varepsilon_k = 2^{-(k+2)}.
\]
These form a computable, strictly decreasing sequence tending to zero.

\subsection{Prefix stabilization lengths}

Define
\[
N_0 = 0,
\qquad
N_{k+1} = \max \left( N_k,\, B_k(\varepsilon_k) \right) + 1.
\]
Thus $N_{k+1}$ strictly increases and is computable. Agreement on
$[0..N_{k+1}]$ guarantees agreement of $\Phi_k$ to within $\varepsilon_k$.

\section{Inductive Step}

Assume $G_k$ has been defined.

\subsection{Step 1: Preserving observer accuracy}

The next identity $G_{k+1}$ must agree with $G_k$ (and therefore with $H$) on
$[0..N_{k+1}]$. This ensures
\[
|\Phi_k(G_{k+1}) - \Phi_k(H)| < \varepsilon_k.
\]

\subsection{Step 2: Selecting a distinct tail inside the fiber}

By perfectness of $\mathcal{F}_{\mathrm{eff}}(x)$, there exists a computable
identity
\[
A_k \in \mathcal{F}_{\mathrm{eff}}(x)
\]
such that
\[
A_k[0..N_{k+1}] = H[0..N_{k+1}]
\quad\text{and}\quad
A_k \neq H.
\]

This identity provides controlled tail variation while preserving collapse.

\subsection{Step 3: Locating an alignment index}

Let
\[
n_0^{G_k} < n_1^{G_k} < \cdots,
\qquad
n_0^{A_k} < n_1^{A_k} < \cdots
\]
denote the selection indices. Since both identities have infinitely many
exposed digits, there exists $j_k$ such that
\[
n_{j_k}^{G_k} \ge N_{k+1}.
\]

By the alignment lemma in Appendix~\ref{appendix:alignment-sewing}, the digits
exposed at these aligned positions coincide.

\subsection{Step 4: Sewing the tail}

Define
\[
G_{k+1}(n)=
\begin{cases}
G_k(n) & n \le n_{j_k}^{G_k}, \\
A_k(n - n_{j_k}^{G_k} + n_{j_k}^{A_k}) & n > n_{j_k}^{G_k}.
\end{cases}
\]

By the Full Sewing Lemma:

\begin{itemize}
    \item $G_{k+1} \in \mathcal{F}_{\mathrm{eff}}(x)$,
    \item $G_{k+1}[0..N_{k+1}] = G_k[0..N_{k+1}]$,
    \item $\Phi_k(G_{k+1})$ lies within $\varepsilon_k$ of $\Phi_k(H)$,
    \item $G_{k+1}$ differs from $H$ at some tail coordinate.
\end{itemize}

\section{Existence of the Limit Identity}

For each index $n$, choose $k$ such that $N_k > n$. For all $m \ge k$,
\[
G_m(n) = G_k(n),
\]
so the coordinate stabilizes. Define
\[
G^{\sharp}(n) = \lim_{k\to\infty} G_k(n).
\]

\subsection{Membership in the effective fiber}

Each $G_k$ collapses to $x$, and sewing preserves the canonical output. Thus
\[
G^{\sharp} \in \mathcal{F}_{\mathrm{eff}}(x).
\]

\subsection{Indistinguishability}

For any computable projection $\Phi_m$, agreement of $G^\sharp$ and $H$ on
prefixes of length at least $B_m(\varepsilon_k)$ ensures
\[
|\Phi_m(G^{\sharp}) - \Phi_m(H)| < \varepsilon_k
\quad\text{for all } k \ge m.
\]
Therefore
\[
\Phi_m(G^{\sharp}) = \Phi_m(H)
\quad\text{for all } m.
\]

\subsection{Distinctness}

Since at each stage the tail is chosen to differ from $H$, one can ensure that
$G^{\sharp} \neq H$ by avoiding a fixed forbidden coordinate or pattern.

\section{Computability}

\subsection{Computability of prefix bounds}

The values $N_k$ are computable because the dependency bounds $B_k$ and the
tolerances $\varepsilon_k$ are computable.

\subsection{Computability of alignment}

Selection indices of computable identities are computable by scanning the
selector until the $j$th exposure is found.

\subsection{Computability of sewing}

The sewing operation is computable coordinatewise using the computable indices
$n_{j_k}^{G_k}$ and $n_{j_k}^{A_k}$.

\subsection{Computability of the limit}

To compute $G^{\sharp}(n)$, find $k$ with $N_k > n$ and output $G_k(n)$.
Since $N_k$ grows without bound and is computable, this procedure yields a
computable name for $G^{\sharp}$.

\section{Summary}

This appendix provided the full technical development of the mimicry
construction:

\begin{itemize}
    \item computation of stabilization lengths,
    \item selection of distinct tails inside the collapse fiber,
    \item alignment at selection indices,
    \item sewing of tails while preserving collapse,
    \item preservation of observer values via dependency bounds,
    \item coordinatewise convergence,
    \item computability of the limit identity.
\end{itemize}

These tools establish the existence of a computable identity in
$\mathcal{F}_{\mathrm{eff}}(x)$ that is observationally indistinguishable from
a given identity but symbolically distinct, proving the Structural
Indistinguishability Theorem.
