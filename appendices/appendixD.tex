\chapter{Diagonalizer Construction Details}

\section{Introduction}

This appendix contains the full technical details of the meta diagonalizer
constructed in the main text.  
The purpose is to present the inductive machinery underlying the
construction in a complete and self contained form.  
We give precise definitions of stabilization indices, prefix bounds, and
alignment points, and we verify the computability of the final identity.

Throughout, $x$ denotes a computable real number with canonical digit
expansion $(x_j)_{j \ge 0}$, and $H$ is a fixed computable reference identity
in the effective collapse fiber $\mathcal{F}_{\mathrm{eff}}(x)$.

We assume an enumeration of all computable structural projections
\[
\Phi_0, \Phi_1, \Phi_2, \ldots,
\]
together with computable dependency bounds $B_k$ for each $\Phi_k$.

\section{Preliminaries}

\subsection{Effective fibers}

The set $\mathcal{F}_{\mathrm{eff}}(x)$ of computable identities that collapse
to $x$ is a $\Pi^{0}_{1}$ class.  
Elements of this class are represented by computable selector, digit, and meta
streams whose canonical outputs agree with the expansion of $x$.

\subsection{Divergent identities}

For each $k$ and each rational $\varepsilon > 0$, the Divergence Lemma from
Chapter~9 guarantees the existence of a computable identity $A \in
\mathcal{F}_{\mathrm{eff}}(x)$ such that
\[
|\Phi_k(A) - \Phi_k(H)| > 3\varepsilon.
\]
This provides the source of divergence at stage $k$.

\subsection{Selection indices}

For any identity $G$, let
\[
n_0^{G} < n_1^{G} < n_2^{G} < \cdots
\]
denote the selection indices corresponding to the positions where $M(n) = D$.

\section{Inductive Construction Overview}

We construct a sequence
\[
G_0, G_1, G_2, \ldots
\]
of computable identities satisfying:

\begin{enumerate}
    \item $G_0 = H$,
    \item $G_k \in \mathcal{F}_{\mathrm{eff}}(x)$ for all $k$,
    \item $G_{k+1}$ extends $G_k$ on a prefix of computable length $N_{k+1}$,
    \item $G_{k+1}$ introduces divergence on projection $\Phi_k$,
    \item the limit identity $G^{\sharp}$ defined coordinatewise is computable.
\end{enumerate}

\subsection{Tolerances}

Define the sequence of tolerances
\[
\varepsilon_k = 2^{-(k+2)},
\]
which is computable, strictly decreasing, and tends to zero.

\subsection{Prefix stabilization lengths}

Define
\[
N_0 = 0,
\]
and inductively
\[
N_{k+1} = \max \left( N_k,\ B_k(\varepsilon_k) \right).
\]
This ensures that agreement on the first $N_{k+1}$ symbols forces
agreement of $\Phi_k$ to within $\varepsilon_k$.

\section{Inductive Step}

Assume $G_k$ has been defined.  
We construct $G_{k+1}$ in several stages.

\subsection{Stage 1: Choosing the divergent identity}

By the Divergence Lemma, choose a computable identity
\[
A_k \in \mathcal{F}_{\mathrm{eff}}(x)
\]
such that
\[
|\Phi_k(A_k) - \Phi_k(H)| > 3\varepsilon_k.
\]

The identity $A_k$ will supply the tail divergence at stage $k$.

\subsection{Stage 2: Locating an alignment index}

Let
\[
n_0^{G_k} < n_1^{G_k} < n_2^{G_k} < \cdots
\]
and
\[
n_0^{A_k} < n_1^{A_k} < n_2^{A_k} < \cdots
\]
be the respective selection indices.

Since $G_k$ exposes infinitely many digits, there exists $j_k$ such that
\[
n_{j_k}^{G_k} \ge N_{k+1}.
\]

By alignment (Lemma C.1), the digit exposed at $n_{j_k}^{G_k}$ in $G_k$ is the
same as the digit exposed at $n_{j_k}^{A_k}$ in $A_k$, and both coincide with
the digit $x_{j_k}$.

\subsection{Stage 3: Sewing the tail}

Define $G_{k+1}$ by
\[
G_{k+1}(n) =
\begin{cases}
G_k(n) & n \le n_{j_k}^{G_k}, \\
A_k(n - n_{j_k}^{G_k} + n_{j_k}^{A_k}) & n > n_{j_k}^{G_k}.
\end{cases}
\]

\begin{itemize}
    \item By construction, $G_{k+1}$ matches $G_k$ on all indices
          $\le n_{j_k}^{G_k} \ge N_{k+1}$.
    \item By the sewing lemma, $G_{k+1} \in \mathcal{F}_{\mathrm{eff}}(x)$.
    \item By dependency bounds, $\Phi_k(G_{k+1})$ differs from $\Phi_k(G_k)$
          by less than $\varepsilon_k$.
    \item Since $A_k$ diverges by more than $3\varepsilon_k$ from $H$, the
          final divergence of $G_{k+1}$ from $H$ remains at least
          $\varepsilon_k$.
\end{itemize}

Thus $G_{k+1}$ satisfies all inductive requirements.

\section{Existence of the Limit Identity}

\subsection{Coordinate stabilization}

For each index $n$, there exists a stage $k$ such that
\[
N_k > n.
\]
Since $G_{k+1}$ and $G_k$ agree on all indices up to $N_{k+1} \ge N_k$, it
follows that for all $m \ge k$,
\[
G_m(n) = G_k(n).
\]
Thus every coordinate stabilizes.

\subsection{Definition of the limit}

Define $G^{\sharp} \in \mathcal{X}^{*}$ by
\[
G^{\sharp}(n) = \lim_{k\to\infty} G_k(n),
\]
where the limit is understood as coordinatewise stabilization.

The limit exists by the preceding argument.

\subsection{Membership in the effective fiber}

Every $G_k$ lies in $\mathcal{F}_{\mathrm{eff}}(x)$, and sewing preserves
membership.  
Since each $G_k$ exposes the canonical digits of $x$ at aligned positions, the
same holds for the limit.  
Thus
\[
G^{\sharp} \in \mathcal{F}_{\mathrm{eff}}(x).
\]

\section{Computability of the Diagonalizer}

\subsection{Computability of stabilization indices}

The sequence $(N_k)$ is computable because:

\begin{itemize}
    \item $B_k$ are computable dependency bounds,  
    \item $\varepsilon_k$ is computable,  
    \item $N_{k+1}$ depends only on $N_k$, $B_k(\varepsilon_k)$, and basic
          arithmetic.
\end{itemize}

Thus $N_k$ is uniformly computable.

\subsection{Computing selection indices}

For each computable identity, the selector stream is computable, so the
selection indices are computable by scanning until the required number of
$D$ symbols have been seen.

Thus the indices $n_j^{G_k}$ and $n_j^{A_k}$ are computable.

\subsection{Computability of the sewing operation}

Given $n$ and $k$, to compute $G_{k+1}(n)$:

\begin{itemize}
    \item check whether $n \le n_{j_k}^{G_k}$,  
    \item if so, return $G_k(n)$,  
    \item otherwise, return the aligned symbol from $A_k$.
\end{itemize}

All needed values are computable, so $G_{k+1}$ is computable.

\subsection{Computability of the limit identity}

For any fixed $n$, to compute $G^{\sharp}(n)$:

\begin{itemize}
    \item find $k$ such that $N_k > n$,  
    \item output $G_k(n)$.
\end{itemize}

Since $N_k$ is computable and increasing without bound, this procedure is
effective.

Thus $G^{\sharp}$ is computable.

\section{Summary}

This appendix presented the full details of the diagonalizer construction:

\begin{itemize}
    \item the inductive construction of $(G_k)$,
    \item the stabilization lengths $N_k$,
    \item alignment indices $j_k$,
    \item controlled sewing of tails,
    \item preservation of collapse fibers,
    \item computability of each $G_k$,
    \item computability of the limit identity $G^{\sharp}$.
\end{itemize}

These tools establish the existence of a computable identity in the collapse
fiber of $x$ that agrees with a reference identity on all observed prefixes
yet diverges along every computable structural projection.
