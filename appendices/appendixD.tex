\chapter{Mimicry Construction Details}
\label{appendix:mimicry}

\section{Introduction}

This appendix provides the full technical development of the mimicry procedure
used in the incompleteness tier of the framework. The goal is to construct a
computable identity inside a collapse fiber that agrees with a given reference
identity on arbitrarily long prefixes while differing in its tail. Continuous
observers, which depend only on finite prefixes, cannot distinguish the two.

The construction relies on three structural ingredients:

\begin{itemize}
    \item dependency bounds for computable structural projections,
    \item alignment and sewing tools from Appendix~\ref{appendix:alignment-sewing},
    \item the perfectness of effective collapse fibers.
\end{itemize}

Together, these tools allow us to build a limit identity that is computationally
indistinguishable from a reference while symbolically distinct.

\section{Preliminaries}

\subsection{Effective collapse fibers}

Let $x$ be a computable real number. The effective fiber
\[
\mathcal{F}_{\mathrm{eff}}(x)
=
\{ G \in \mathcal{G}_{\mathrm{eff}} : \pi(G)=x \}
\]
is a nonempty $\Pi^{0}_{1}$ class. It is perfect: for every identity $H$ in the
fiber and every $N$ there exists a distinct computable identity $A$ in the fiber
with
\[
A[0..N]=H[0..N].
\]
This provides the controlled tail variation required for mimicry.

\subsection{Selection indices}

For any identity $G$ with infinitely many exposed digits, write
\[
n_0^{G} < n_1^{G} < n_2^{G} < \cdots
\]
for the indices where $M(n)=D$. Alignment and sewing rely on the fact that if
$H$ and $A$ lie in the same fiber, then their $j$th selected digits coincide and
occur at respective indices $n_j^{H}$ and $n_j^{A}$.

This alignment permits tail replacement at matched selection points.

\subsection{Computable observers}

Fix an effective enumeration of the computable structural projections,
\[
\Phi_0, \Phi_1, \Phi_2, \ldots,
\]
each with computable dependency bound $B_k(\varepsilon)$. Dependency bounds
express prefix stabilization:
agreement beyond $B_k(\varepsilon)$ guarantees
\[
|\Phi_k(G)-\Phi_k(H)| < \varepsilon.
\]

\section{Construction Outline}

We inductively build identities
\[
G_0, G_1, G_2, \ldots \in \mathcal{F}_{\mathrm{eff}}(x)
\]
with increasing prefix agreement lengths that ensure stability with respect to
the first $k$ observers.

\begin{enumerate}
    \item $G_0 = H$,
    \item $G_{k+1}$ agrees with $H$ on a prefix of length $N_{k+1}$,
    \item $G_{k+1}$ differs from $H$ at some later coordinate,
    \item $\Phi_k(G_{k+1})$ remains within $\varepsilon_k$ of $\Phi_k(H)$.
\end{enumerate}

The limit identity $G^{\sharp}$ inherits agreement on all finite prefixes
required by observers and therefore mimics $H$ from every observational
perspective.

\section{Prefix Stabilization Lengths}

Define the error tolerances
\[
\varepsilon_k = 2^{-(k+2)}.
\]

Define stabilization lengths recursively by
\[
N_0 = 0,
\qquad
N_{k+1} = \max \left( N_k, \, B_k(\varepsilon_k) \right ) + 1.
\]

These values are computable and strictly increasing. Agreement on
$[0..N_{k+1}]$ ensures that observer $\Phi_k$ sees $G_{k+1}$ and $H$
within tolerance $\varepsilon_k$.

\section{Inductive Step}

Assume $G_k$ has been constructed.

\subsection{Step 1: Enforce observer agreement}

To ensure accuracy for observer $\Phi_k$, the next identity must satisfy
\[
G_{k+1}[0..N_{k+1}] = H[0..N_{k+1}].
\]

\subsection{Step 2: Select a distinct computable tail}

By perfectness of $\mathcal{F}_{\mathrm{eff}}(x)$, choose a computable
\[
A_k \in \mathcal{F}_{\mathrm{eff}}(x)
\]
such that
\[
A_k[0..N_{k+1}] = H[0..N_{k+1}]
\quad\text{and}\quad
A_k \neq H.
\]
This identity provides a distinct tail compatible with collapse.

\subsection{Step 3: Find an alignment index}

Let $(n_j^{G_k})$ and $(n_j^{A_k})$ be the selection indices. Since both
identities expose infinitely many digits, there exists $j_k$ such that
\[
n_{j_k}^{G_k} \ge N_{k+1}.
\]
Alignment ensures that the digits exposed at the positions
$n_{j_k}^{G_k}$ and $n_{j_k}^{A_k}$ coincide.

\subsection{Step 4: Sew the tail}

Define
\[
G_{k+1}(n)=
\begin{cases}
G_k(n) & n \le n_{j_k}^{G_k}, \\
A_k(n - n_{j_k}^{G_k} + n_{j_k}^{A_k}) & n > n_{j_k}^{G_k}.
\end{cases}
\]

By the Full Sewing Lemma:

\begin{itemize}
    \item $G_{k+1} \in \mathcal{F}_{\mathrm{eff}}(x)$,
    \item $G_{k+1}[0..N_{k+1}] = H[0..N_{k+1}]$,
    \item $|\Phi_k(G_{k+1}) - \Phi_k(H)| < \varepsilon_k$,
    \item $G_{k+1}$ differs from $H$ on some tail coordinate.
\end{itemize}

\section{Existence of the Limit}

Since the prefixes $[0..N_k]$ stabilize and $N_k \to \infty$, the sequence
$(G_k)$ converges coordinatewise to a limit identity
\[
G^{\sharp}(n) = \lim_{k\to\infty} G_k(n).
\]

\subsection{Membership in the fiber}

Every $G_k$ collapses to $x$, and the fiber is closed, hence
\[
G^{\sharp} \in \mathcal{F}_{\mathrm{eff}}(x).
\]

\subsection{Indistinguishability}

Fix any computable observer $\Phi_m$.  
For all $k \ge m$, the construction guarantees agreement on a prefix of length
at least $B_m(\varepsilon_k)$, so
\[
|\Phi_m(G_k) - \Phi_m(H)| < \varepsilon_k.
\]
Taking limits gives
\[
\Phi_m(G^{\sharp}) = \Phi_m(H).
\]

Thus $G^{\sharp}$ is observationally indistinguishable from $H$.

\subsection{Distinctness}

Because each stage modifies the tail in a way that prevents eventual agreement
with $H$, the limit identity satisfies
\[
G^{\sharp} \neq H.
\]

\section{Computability}

\subsection{Computability of stabilization lengths}

$N_k$ is computable because $B_k$ and $\varepsilon_k$ are computable.

\subsection{Computability of alignment}

Given computable identities, the selection indices $n_j^{G_k}$ and
$n_j^{A_k}$ are computable by scanning the selector streams.

\subsection{Computability of sewing}

The sewed identity is computable coordinatewise from the computable indices
and the computable streams of $G_k$ and $A_k$.

\subsection{Computing the limit identity}

To compute $G^{\sharp}(n)$, find $k$ with $N_k > n$ and output $G_k(n)$.
This yields a computable name for $G^{\sharp}$.

\section{Summary}

This appendix developed the full machinery behind the mimicry construction.
The key ingredients are:

\begin{itemize}
    \item dependency bounds that enforce finite observation windows,
    \item perfectness of the effective collapse fiber,
    \item alignment and sewing along selection indices,
    \item controlled tail modification without disturbing collapse,
    \item convergence of a stabilizing sequence inside the fiber,
    \item computability of every step.
\end{itemize}

These tools produce a computable identity that agrees with a reference on all
observationally relevant prefixes while differing in its tail, establishing
the Structural Indistinguishability Theorem.
