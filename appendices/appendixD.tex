\chapter{Mimicry Construction Details}
\label{appendix:mimicry}

\section{Introduction}

This appendix develops the full technical foundations of the mimicry procedure used in the incompleteness tier of the framework. The purpose is to build a computable generative identity that matches a given reference identity on arbitrarily long finite prefixes while differing in its tail. Such a construction shows that continuous observers, which depend on finite prefixes through dependency bounds, are unable to recover the underlying generative structure.

The proofs rely on three structural components developed in earlier parts of the monograph:

\begin{itemize}
    \item dependency bounds for computable structural projections,
    \item alignment and sewing results from Appendix~\ref{appendix:alignment-sewing},
    \item perfectness and effective closedness of collapse fibers from Appendix~\ref{appendix:tte}.
\end{itemize}

Together these tools yield a computable identity that is observationally indistinguishable from a reference identity yet symbolically distinct.

\section{Effective Collapse Fibers}

Let $x$ be a computable real number. The effective fiber is the set
\[
\mathcal{F}_{\mathrm{eff}}(x)
=
\{\, G \in \mathcal{G}_{\mathrm{eff}} : \pi(G) = x \,\}.
\]
Appendix~\ref{appendix:tte} shows that this set is a nonempty $\Pi^{0}_{1}$ class. It is perfect and therefore closed under finite-prefix extension. In particular, for every identity $H$ in the fiber and every integer $N$, there exists a distinct computable identity $A$ in the fiber with
\[
A[0..N] = H[0..N].
\]

This property provides the tail freedom required for mimicry.

\section{Selection Indices and Alignment}

For any identity $G$ that exposes infinitely many observed values, write
\[
n_0^{G} < n_1^{G} < n_2^{G} < \cdots
\]
for the indices at which observed positions occur. If $H$ and $A$ lie in the same collapse fiber, then Appendix~\ref{appendix:alignment-sewing} shows that they expose the same collapsed values at their respective selection indices:
\[
D_{H}(n_j^{H}) = D_{A}(n_j^{A})
\quad\text{for all } j.
\]

This alignment permits tail replacement at matched selection positions without altering the collapsed value.

\section{Computable Observers and Dependency Bounds}

Fix an effective enumeration of the computable structural projections,
\[
\Phi_0, \Phi_1, \Phi_2, \ldots,
\]
each with computable dependency bound $B_k(\varepsilon)$ as described in Chapter~\ref{chap:prefix-stabilization}. A dependency bound guarantees that agreement of two identities on their prefixes of length $B_k(\varepsilon)$ implies
\[
|\Phi_k(G) - \Phi_k(H)| < \varepsilon.
\]

Observers therefore inspect only finitely many coordinates at any fixed precision.

\section{Outline of the Construction}

We construct identities
\[
G_0, G_1, G_2, \ldots \in \mathcal{F}_{\mathrm{eff}}(x)
\]
that stabilize coordinatewise. The sequence satisfies:

\begin{enumerate}
    \item $G_0 = H$,
    \item $G_{k+1}[0..N_{k+1}] = H[0..N_{k+1}]$,
    \item $G_{k+1}$ differs from $H$ at some coordinate beyond $N_{k+1}$,
    \item $\Phi_k(G_{k+1})$ lies within $\varepsilon_k$ of $\Phi_k(H)$.
\end{enumerate}

The limit identity $G^{\sharp}$ will match $H$ on every observationally relevant finite prefix while differing in its tail.

\section{Stabilization Lengths}

Define tolerances
\[
\varepsilon_k = 2^{-(k+2)}.
\]
Define stabilization lengths recursively by
\[
N_0 = 0,
\qquad
N_{k+1}
=
\max\bigl(N_k,\, B_k(\varepsilon_k)\bigr) + 1.
\]

These values are computable and strictly increasing. Agreement on $[0..N_{k+1}]$ ensures that observer $\Phi_k$ evaluates $G_{k+1}$ and $H$ within the required accuracy.

\section{The Inductive Step}

Assume $G_k$ has been constructed.

\subsection{Observer Agreement}

To ensure agreement for observer $\Phi_k$, the next identity must satisfy
\[
G_{k+1}[0..N_{k+1}] = H[0..N_{k+1}].
\]

\subsection{Selecting a Distinct Tail}

By perfectness of $\mathcal{F}_{\mathrm{eff}}(x)$, select a computable identity
\[
A_k \in \mathcal{F}_{\mathrm{eff}}(x)
\]
such that
\[
A_k[0..N_{k+1}] = H[0..N_{k+1}]
\quad\text{and}\quad
A_k \ne H.
\]

\subsection{Alignment Index}

Let $(n_j^{G_k})$ and $(n_j^{A_k})$ be the respective selection indices. Since both sequences are unbounded, choose $j_k$ with
\[
n_{j_k}^{G_k} \ge N_{k+1}.
\]
Alignment ensures that the value exposed at $n_{j_k}^{G_k}$ matches the value exposed at $n_{j_k}^{A_k}$.

\subsection{Sewing the Tail}

Define
\[
G_{k+1}(n)
=
\begin{cases}
G_k(n), & n \le n_{j_k}^{G_k},\\
A_k(n - n_{j_k}^{G_k} + n_{j_k}^{A_k}), & n > n_{j_k}^{G_k}.
\end{cases}
\]

Appendix~\ref{appendix:alignment-sewing} ensures:

\begin{itemize}
    \item $G_{k+1}$ remains in $\mathcal{F}_{\mathrm{eff}}(x)$,
    \item $G_{k+1}[0..N_{k+1}] = H[0..N_{k+1}]$,
    \item $|\Phi_k(G_{k+1}) - \Phi_k(H)| < \varepsilon_k$,
    \item $G_{k+1}$ differs from $H$ on its tail.
\end{itemize}

\section{Existence of the Limit Identity}

Since $G_{k+1}$ and $G_k$ agree on $[0..N_k]$ and $N_k \to \infty$, the sequence stabilizes coordinatewise. Define
\[
G^{\sharp}(n) = \lim_{k\to\infty} G_k(n).
\]

\subsection{Membership in the Fiber}

Each $G_k$ collapses to $x$, the fiber is closed, and therefore
\[
G^{\sharp} \in \mathcal{F}_{\mathrm{eff}}(x).
\]

\subsection{Observer Indistinguishability}

Fix a computable observer $\Phi_m$. For all $k \ge m$,
\[
|\Phi_m(G_k) - \Phi_m(H)| < \varepsilon_k.
\]
Since $\varepsilon_k \to 0$, continuity implies
\[
\Phi_m(G^{\sharp}) = \Phi_m(H).
\]

\subsection{Distinctness}

Because each stage introduces a tail difference, the identities never stabilize to $H$. Hence
\[
G^{\sharp} \ne H.
\]

\section{Computability}

\subsection{Stabilization Lengths}

Each $N_k$ is computable because $B_k$ and $\varepsilon_k$ are computable.

\subsection{Alignment Computation}

Selection indices are computable by scanning the exposure coordinate.

\subsection{Sewing Computation}

The sewn identity is computed pointwise from the streams of $G_k$ and $A_k$ and the computable alignment index.

\subsection{Computability of the Limit Identity}

To compute $G^{\sharp}(n)$, find $k$ with $N_k > n$ and output $G_k(n)$. This yields a computable name for $G^{\sharp}$.

\section{Summary}

This appendix supplied the full technical foundation for the mimicry construction. The key components are:

\begin{itemize}
    \item dependency bounds that quantify finite visibility of observers,
    \item perfectness and effective closedness of collapse fibers,
    \item alignment and sewing along selection indices,
    \item stability of observers under tail replacement,
    \item coordinatewise convergence of the constructed sequence,
    \item uniform computability at every stage.
\end{itemize}

These elements establish that finite continuous observation cannot recover the generative identity, even within an effective collapse fiber.
