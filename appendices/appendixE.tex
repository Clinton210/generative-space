\chapter{Extended Invariants and Selector Geometry}
\label{appendix:extended-invariants}

\section{Introduction}

This appendix presents examples, geometric interpretations, and structural
properties of the extended invariants used throughout the invariant tier of the
framework. These invariants describe long range behavior of the exposure
coordinate of a generative identity and illustrate how finite prefix structure,
asymptotic behavior, and collapse fibers interact.

For an identity
\[
G = (X_{1}, X_{2}, X_{3}) \in \mathcal{G},
\]
the exposure pattern in $X_{1}$ determines the two extended invariants:
\[
\eta(G)
  =
  \liminf_{N\to\infty}
  \frac{1}{N}\sum_{n<N} \mathbf{1}[\, X_{1}(n)
  \text{ is an exposure event} \,],
\]
\[
\phi(G)
  =
  \limsup_{j\to\infty}
  \frac{g_{j}}{n_{j}},
\]
where $(n_{j})$ lists the exposure positions in increasing order and
$g_{j} = n_{j+1} - n_{j}$. The invariant $\eta$ measures lower asymptotic
exposure frequency. The invariant $\phi$ measures relative growth of exposure
gaps.

The goals of the appendix are:

\begin{itemize}
    \item to describe geometric slice interpretations of these invariants,
    \item to present explicit examples spanning their full admissible ranges,
    \item to prove robustness under tail changes and discontinuity in the
          product topology.
\end{itemize}

Extended invariants are derived objects rather than structural features. They
are limits of observer towers as described in Chapters~8 and~9 of the outline
:contentReference[oaicite:2]{index=2}. Their role is illustrative rather than classificatory.

\section{Vertical, Horizontal, and Fiber Slices}

Extended invariants fit naturally into a slice based geometric description of
the generative space. These slices highlight the independence of finite prefix
structure, asymptotic behavior, and collapse value.

\subsection{Vertical slices}

For a finite word $u$ of length $N$ in the product alphabet of $\mathcal{G}$,
define the cylinder
\[
\mathcal{C}(u)
  =
  \{\, G \in \mathcal{G} : G[0..N-1] = u \,\}.
\]
These sets form the basic open sets of the product topology. Observers with a
dependency bound at precision $\varepsilon$ examine exactly one such slice of
depth $B(\varepsilon)$, as described in Chapter~4 and Appendix~B
:contentReference[oaicite:3]{index=3}.

Vertical slices impose no restriction on the values of $\eta$ or $\phi$. All
asymptotic behaviors compatible with symbolic constraints appear densely inside
every such slice.

\subsection{Horizontal slices}

For fixed $\alpha \in [0,1]$ and $\beta \in [0,\infty]$, define
\[
\mathcal{H}_{\alpha}
  =
  \{\, G : \eta(G) = \alpha \,\},
\qquad
\mathcal{H}^{\beta}
  =
  \{\, G : \phi(G) = \beta \,\}.
\]

These sets group identities by long range exposure behavior. They cut across
all finite prefix classes and all collapse fibers. When viewed in the invariant
plane, horizontal slices appear as straight lines. They reflect asymptotic
structure that cannot be detected by any continuous observer because every
observer has finite prefix dependence.

\subsection{Fiber slices}

For a real value $x \in [0,1]$ and its associated fiber
\[
\mathcal{F}(x)
  =
  \{\, G \in \mathcal{G}^{*} : \pi(G) = x \,\},
\]
the extended invariants depend only on the exposure pattern of $G$ and not on
the observed-value coordinate. As shown in Appendix~C, fibers are compact and
perfect. Using the sewing results of Appendix~D, any desired exposure tail may
be combined with the fixed collapsed value. Consequently, the projection
\[
G \longmapsto (\eta(G), \phi(G))
\]
typically sends $\mathcal{F}(x)$ onto a large region of the invariant plane.

\section{Worked Examples}

The following examples illustrate the variety of invariant behavior allowed in
the generative space. For each example the observed-value and auxiliary
coordinates can be chosen freely to place the identity in any collapse fiber.

\subsection{Periodic positive frequency}

Consider an exposure pattern that alternates exposure and non-exposure in a
periodic manner. Then
\[
\eta(G) = \frac{1}{2}, \qquad \phi(G) = 0.
\]
This is the simplest positive frequency pattern with bounded exposure gaps.

\subsection{Positive frequency with irregularity}

Let the exposure coordinate repeat a nonuniform block with two exposure events
and one non-exposure event. Then
\[
\eta(G) = \frac{2}{3}, \qquad \phi(G) = 0.
\]
Although the pattern is not uniform, gap growth remains bounded, and relative
gap fluctuation is zero.

\subsection{Zero frequency with bounded gaps}

If exposure events occur at prime indices, then
\[
\eta(G) = 0.
\]
Because prime gaps grow sublinearly, the relative gap growth satisfies
\[
\phi(G) = 0.
\]

\subsection{Zero frequency with unbounded fluctuation}

Let exposure events occur at factorial indices $n_{j} = j!$. Then
\[
\eta(G) = 0,
\qquad
\frac{g_{j}}{n_{j}} = j,
\qquad
\phi(G) = \infty.
\]
This gives zero frequency with arbitrarily large fluctuations.

\subsection{Oscillating behavior}

Consider exposure blocks of lengths
\[
2^{0}, 2^{0}, 2^{1}, 2^{1}, 2^{2}, 2^{2}, \ldots
\]
with alternating exposure and non-exposure. Then
\[
\eta(G) = 0,
\qquad
\phi(G) = \infty.
\]
This example exhibits density oscillation combined with large gap growth.

\section{Robustness Under Tail Modification}

\subsection{Tail robustness}

If two identities agree on all indices beyond some fixed point, then
\[
\eta(G) = \eta(G'),
\qquad
\phi(G) = \phi(G').
\]
Thus the invariants depend only on the tail behavior of the exposure
coordinate.

\subsection{Discontinuity of density}

Let $G$ satisfy $\eta(G) = 0$ and define $G_{k}$ to match $G$ on $[0..k]$ and
expose all indices beyond $k$. Then $G_{k} \to G$ in the product topology, but
\[
\eta(G_{k}) = 1.
\]
Hence the density invariant is discontinuous at every point.

\subsection{Discontinuity of relative gap growth}

If $G$ has bounded gaps so that $\phi(G)=0$, and $G_{k}$ is obtained by
introducing a single large gap of length $\ell_{k} \to \infty$ beyond index
$k$, then $G_{k} \to G$, yet
\[
\phi(G_{k}) = \infty.
\]

\subsection{No possibility of continuity}

If $G_{k}$ has a single exposure event at position $k$ and $G$ has none, then
$G_{k} \to G$, while
\[
\eta(G_{k}) = \frac{1}{k}, \qquad \eta(G) = 0.
\]
Similar discontinuities occur for $\phi$. Thus both invariants are nowhere
continuous with respect to the product topology.

These discontinuity phenomena reflect the fact that the product topology is
governed purely by finite prefix structure, while extended invariants depend on
the entire infinite tail.

\section{Behavior Inside Collapse Fibers}

Since extended invariants depend only on the exposure coordinate, every
collapse fiber contains identities realizing the full range of admissible
invariant values.

\subsection{Arbitrary frequency inside a fiber}

Let $\alpha \in [0,1]$. Construct an exposure tail with
$\eta(G) = \alpha$. By alignment and sewing (Appendix~D), one may assign the
observed-value coordinate to match any desired collapsed value $x$ while
preserving the tail, yielding an identity in $\mathcal{F}(x)$ with frequency
$\alpha$.

\subsection{Arbitrary fluctuation inside a fiber}

Let $\beta \in [0,\infty]$. Construct an exposure tail with
$\phi(G) = \beta$ and combine it with the collapsed value $x$ using the sewing
tools of Appendix~D. This gives an identity in $\mathcal{F}(x)$ with relative
gap growth $\beta$.

\subsection{Simultaneous control}

Given any pair $(\alpha,\beta)$, one may construct a symbolic tail achieving
both invariants simultaneously and combine it with the observed-value
coordinate of any fixed $x$. Thus collapse fibers project onto significant
regions of the invariant plane.

\section{Summary}

This appendix described the geometric roles of the extended invariants
$\eta$ and $\phi$, which quantify long range behavior of the exposure
coordinate:

\begin{itemize}
    \item Vertical slices describe finite prefix structure and the portion of a
          generative identity visible to observers.

    \item Horizontal slices describe asymptotic exposure behavior and cut across
          all collapse fibers.

    \item Fiber slices show that collapse restricts only observed values and
          leaves the exposure pattern unconstrained on asymptotic scales.
\end{itemize}

Extended invariants are robust under tail changes yet discontinuous in the
product topology. They illustrate the substantial symbolic freedom that remains
invisible to finite observation and collapse, reinforcing the structural
incompleteness results developed earlier in the monograph.
