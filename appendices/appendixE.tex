\chapter{Examples and Calculations for Extended Invariants}

\section{Introduction}

Part~VI develops extended generative invariants such as entropy balance,
fluctuation index, and orthogonal coordinate embeddings.  
These invariants enrich the representation of a generative identity beyond its
classical magnitude and partially restore structure lost under collapse.

This appendix collects explicit examples illustrating:
\begin{itemize}
    \item computation of entropy balance $\eta(G)$,
    \item computation of fluctuation index $\phi(G)$,
    \item comparison of generators with identical collapse values but distinct
          extended coordinates,
    \item two- and three-dimensional embeddings using $(\pi,\eta)$ and
          $(\pi,\eta,\phi)$,
    \item cases where extended invariants reveal structure invisible to collapse.
\end{itemize}

These examples are not required for the main results, but clarify the geometric
interpretation of extended coordinates.

\section{Entropy Balance Examples}

Recall that the entropy balance of a generator is
\[
\eta(G)
=
\liminf_{n\to\infty}
\frac{1}{n}\bigl|\{\,k<n : M(k)=D\,\}\bigr|.
\]

\subsection{Example 1: Pure Digit Selector}

Let $M(n)=D$ for all $n$.  
Then
\[
\eta(G)=1
\]
for every choice of $D$ and $K$.  
The canonical output is the digit sequence $D$ itself.

\subsection{Example 2: Alternating Selector}

Let
\[
M(n) = 
\begin{cases}
D, & n\text{ even},\\
K, & n\text{ odd}.
\end{cases}
\]
Then
\[
\eta(G)=\tfrac12.
\]

This example shows that entropy balance can be tuned easily by patterning the
selector.

\subsection{Example 3: Polynomial Sparsity}

Let $M(n)=D$ exactly when $n=j^2$ for some $j\ge0$.  
Then the number of squares $\le n$ is approximately $\sqrt{n}$, so
\[
\eta(G) = 
\lim_{n\to\infty} \frac{\sqrt{n}}{n} = 0.
\]

This produces a null-density generator.

\section{Fluctuation Index Examples}

The fluctuation index $\phi(G)$ measures selector irregularity by quantifying
local variation of the selector or canonical output.  
For illustration, consider a version based on empirical switching frequency:
\[
\phi(G)
=
\limsup_{n\to\infty}
\frac{1}{n}\bigl|\{\,k<n : M(k)\neq M(k+1)\,\}\bigr|.
\]

\subsection{Example 4: Smooth Hybrid}

Let $M(n)=D$ for $n$ even and $M(n)=K$ for odd $n$.  
Then switching occurs at every step, so
\[
\phi(G)=1.
\]

\subsection{Example 5: Block-Constant Selector}

Define $M$ by alternating blocks of growing lengths:
\[
\begin{aligned}
& \underbrace{D,D,\ldots,D}_{2^1},
\underbrace{K,K,\ldots,K}_{2^1},
\\
& \underbrace{D,D,\ldots,D}_{2^2},
\underbrace{K,K,\ldots,K}_{2^2},
\\
& \cdots
\end{aligned}
\]
Switching occurs only at the block boundaries, producing
\[
\phi(G) = 0.
\]
Yet the entropy balance is $\eta(G)=\tfrac12$.

This example shows $\eta$ and $\phi$ are independent invariants.

\section{Generators Sharing Collapse but Not Extended Coordinates}

\subsection{Example 6: Same Magnitude, Different Density}

Let $x$ have base-$b$ expansion $(x_j)$.  
Define two generators:

\begin{itemize}
    \item $G$ uses $M_G(n)=D$ for all $n$ and $D_G(n)=x_n$,
    \item $H$ uses $M_H(n)=D$ only when $n=j^2$, and chooses $D_H(j^2)=x_j$.
\end{itemize}

Both satisfy $\pi(G)=\pi(H)=x$,  
but
\[
\eta(G)=1,\qquad \eta(H)=0.
\]

Thus entropy balance separates these points in the extended coordinate plane.

\subsection{Example 7: Same Magnitude and Same Density, Different Fluctuation}

Let $M_G$ alternate $D,K$ at every step, while $M_H$ alternates in growing
blocks.  
Both have $\eta(G)=\eta(H)=\tfrac12$,  
but
\[
\phi(G)=1,\qquad \phi(H)=0.
\]

Thus the pair $(\eta,\phi)$ distinguishes generators that collapse to the same
real number and share the same digit-selection density.

\section{Embedding Examples}

\subsection{Two-Dimensional Embedding}

Consider the map
\[
G \mapsto (\pi(G),\eta(G)).
\]

Each classical real $x$ gives a vertical line
\[
\{x\} \times [0,1] \subseteq \mathbb{R}^2.
\]

Generators from the same collapse fiber occupy distinct points on this line.

\subsection{Three-Dimensional Embedding}

The map
\[
G \mapsto (\pi(G),\eta(G),\phi(G))
\]
produces a three-dimensional coordinate system.  
Generators from the same collapse fiber form a curve or surface inside the
vertical plane $\{x\}\times \mathbb{R}^2$.

This embedding emphasizes complementary structural dimensions of the generative
space.

\section{Interpretation}

These examples illustrate the following themes:

\begin{itemize}
    \item Extended invariants map collapse fibers into higher-dimensional spaces.
    \item Entropy balance and fluctuation index are independent dimensions.
    \item Generators with identical collapse values can differ dramatically in
          extended coordinates.
    \item Extended invariants partially restore structure lost under collapse.
\end{itemize}

While extended invariants do not defeat structural incompleteness, they enrich
the generative representation and reveal systematic internal behaviors that
collapse alone cannot express.
