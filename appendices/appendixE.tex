\chapter{Extended Invariants and Selector Geometry}

\section{Introduction}

This appendix provides explicit examples and geometric interpretations of the
robust asymptotic invariants introduced in Part~VI. These invariants measure
large scale properties of the selector stream and illustrate how generative
identities distribute across symbolic space. The appendix also develops a
slice based geometric viewpoint that clarifies how finite prefix structure,
asymptotic selector statistics, and collapse fibers interact.

Throughout, $G = (M,D,K)$ is a generative identity with selector
$M \in \{D,K\}^{\mathbb{N}}$. The robust asymptotic invariants are

\[
\eta(G)
    = \liminf_{N\to\infty}
        \frac{1}{N}\sum_{n<N} \chi_M(n),
\qquad
\phi(G)
    = \limsup_{j\to\infty} \frac{g_j}{n_j},
\]

where $n_j$ lists the selected positions and $g_j = n_{j+1} - n_j$.

\section{Vertical, Horizontal, and Fiber Slices}

Extended invariants give rise to three natural types of slices through the
generative space. These slices provide conceptual maps of selector behavior
and explain why finite observation cannot constrain asymptotic structure.

\subsection{Vertical slices: fixing a prefix}

A vertical slice is a cylinder set

\[
\mathcal{C}(u)
  = \{ G \in \mathcal{X}^{*} : G[0..N-1] = u \},
\]

where $u$ is a finite prefix. Vertical slices represent the observable region
available to any continuous projection at fixed precision. Dependency bounds
ensure that each observer samples only a single vertical slice at a time, and
so vertical slices encode the finite information geometry underlying
indistinguishability.

Vertical slices impose no restriction on $\eta$ or $\phi$. Invariant values
can vary freely inside any cylinder.

\subsection{Horizontal slices: fixing invariant values}

Fix $\alpha \in [0,1]$ or $\beta \in [0,\infty]$. The level sets

\[
\mathcal{H}_{\alpha}
  = \{ G : \eta(G) = \alpha \},
\qquad
\mathcal{H}^{\beta}
  = \{ G : \phi(G) = \beta \}
\]

group identities by large scale selector behavior. These slices cut across
collapse fibers and across all vertical slices. When drawn in the
$(\eta,\phi)$ plane, horizontal slices appear as straight lines and illustrate
how asymptotic structure is decoupled from the finite structure observed by
continuous projections.

\subsection{Fiber slices: fixing the collapsed value}

Fix a real number $x$. The fiber slice

\[
\mathcal{F}(x)
    = \{ G \in \mathcal{X}^{*} : \pi(G) = x \}
\]

contains all identities whose selected digits give $x$. Extended invariants
depend only on the selector, not on collapse. Consequently, the image of
$\mathcal{F}(x)$ under the map

\[
G \longmapsto (\eta(G), \phi(G))
\]

typically occupies a large subset of the invariant plane. This illustrates
why the classical value of a real number constrains only a small portion of
its generative structure.

\section{Worked Examples of Extended Invariants}

\subsection{Periodic positive density example}

Let

\[
M(n) =
\begin{cases}
D & \text{if } n \text{ is even},\\
K & \text{otherwise}.
\end{cases}
\]

Then $\eta(G) = 1/2$. The indices are $n_j = 2j$, so $g_j = 2$ and

\[
\phi(G) = 0.
\]

This selector has positive density and perfectly regular spacing.

\subsection{Positive density with mild irregularity}

Let $M$ repeat the pattern $DDK$. Then

\[
\eta(G) = 2/3,
\qquad
\phi(G) = 0.
\]

The pattern is not uniform but its gap growth is bounded.

\subsection{Zero density with bounded gaps}

Let $M(n)=D$ when $n$ is prime. The density of primes is zero, so
$\eta(G)=0$, but since gaps grow sublinearly,

\[
\phi(G) = 0.
\]

\subsection{Zero density with large fluctuations}

Select digits at factorial indices:

\[
n_j = j!.
\]

Then $\eta(G)=0$ and

\[
\frac{g_j}{n_j} = j,
\qquad
\phi(G) = \infty.
\]

\subsection{Oscillating density example}

Expose digits in blocks

\[
D^{2^0}K^{2^0}D^{2^1}K^{2^1}D^{2^2}K^{2^2}\cdots
\]

Then the density oscillates between near zero and near one. One obtains

\[
\eta(G) = 0,
\qquad
\phi(G) = \infty.
\]

These examples show that $\eta$ and $\phi$ capture independent aspects of
asymptotic selector behavior.

\section{Robustness and Discontinuity}

The invariants $\eta$ and $\phi$ are robust under finite tail modifications
beyond any fixed prefix, but they are maximally discontinuous in the product
topology. The following examples demonstrate this behavior.

\subsection{Lower asymptotic density}

Let $G$ satisfy $\eta(G)=0$. Define $G_k$ by copying the first $k$ symbols of
$M$ and then exposing digits at all later positions. Then

\[
\eta(G_k) = 1
\quad\text{for all } k,
\]

yet $G_k \to G$ in the product topology. This shows that $\eta$ is not upper
semicontinuous and exhibits strong discontinuity.

\subsection{Relative gap growth}

Let $G$ have evenly spaced selected digits, so $\phi(G)=0$. Modify the tail of
$M$ in $G_k$ by inserting a single gap of length $\ell_k \to \infty$. Then

\[
\phi(G_k) = \infty
\quad\text{for all } k,
\]

and again $G_k \to G$ in the product topology. This demonstrates
discontinuity of $\phi$.

\subsection{No possibility of continuity}

Consider $G_k$ selecting $D$ at a single position $k$ and $G$ selecting $D$ at
no positions. Then $G_k \to G$, yet $\eta(G_k) = 1/k$ while
$\eta(G)=0$, and similarly for fluctuation. Both invariants fail continuity in
the strongest possible sense.

\section{Extended Invariants Inside Collapse Fibers}

Since invariants depend only on the selector, each fiber contains identities
with all allowable invariant values.

\subsection{Arbitrary density in a fiber}

Fix $\alpha \in [0,1]$. Construct a selector $M$ with $\eta(M)=\alpha$ and
place the digits of $x$ at selected positions. This yields a generative
identity in $\mathcal{F}(x)$ with invariant $\eta=\alpha$.

\subsection{Arbitrary fluctuation in a fiber}

Fix $\beta \in [0,\infty]$. Construct a selector with $\phi=\beta$ and again
place the digits of $x$ at selected positions. This produces an identity with
the desired fluctuation inside $\mathcal{F}(x)$.

\subsection{Simultaneous control}

Given any $(\alpha,\beta)$ in the invariant plane, build a selector realizing
both values and place the digits of $x$ accordingly. Thus each collapse fiber
maps to a substantial region in $(\eta,\phi)$ space.

\section{Summary}

This appendix presented explicit examples illustrating the full range of
possible values for the robust asymptotic invariants $\eta$ and $\phi$, as
well as slice based geometric interpretations that clarify how selectors
populate the generative space. Vertical slices represent finite symbolic
prefixes, horizontal slices represent long term invariant values, and fiber
slices reveal the symbolic variety compatible with a fixed collapsed value.
These perspectives support the broader conclusion that collapse conceals the
vast majority of generative structure.
