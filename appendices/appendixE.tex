\chapter{Extended Invariants and Selector Geometry}
\label{appendix:extended-invariants}

\section{Introduction}

This appendix develops worked examples and geometric interpretations of the
extended asymptotic invariants used in the selector geometry tier. These
invariants describe long range behavior of the selector stream and clarify how
finite prefixes, asymptotic statistics, and collapse fibers interact.

For a generative identity $G = (M,D,K)$ with selector $M \in \{D,K\}^{\mathbb{N}}$,
the extended invariants are
\[
\eta(G)
= \liminf_{N\to\infty}
\frac{1}{N}\sum_{n<N} \chi_{M}(n),
\qquad
\phi(G)
= \limsup_{j\to\infty} \frac{g_j}{n_j},
\]
where $(n_j)$ lists the selected positions and $g_j = n_{j+1} - n_j$.
The invariant $\eta$ records the lower asymptotic density of exposed digits,
while $\phi$ records the relative gap growth and thus the scale of fluctuation.

The aim of this appendix is threefold:

\begin{itemize}
    \item to give a slice based geometric interpretation of the invariants,
    \item to present examples spanning the full range of selector behaviors,
    \item to explain why invariants are robust under tail changes but
          discontinuous in the product topology.
\end{itemize}

These ideas support the structural projection results and the
indistinguishability constructions of later chapters.

\section{Vertical, Horizontal, and Fiber Slices}

The invariants $\eta$ and $\phi$ determine natural geometric slices through
the generative space. These slices illustrate how finite and infinite scale
structures relate to each other.

\subsection{Vertical slices: fixing a prefix}

For a finite word $u$ of length $N$, the cylinder
\[
\mathcal{C}(u)
= \{ G \in \mathcal{X}^{*} : G[0..N-1] = u \}
\]
is a vertical slice of the generative space. Dependency bounds for structural
projections imply that any continuous observer can inspect only a single
vertical slice at a given precision. Thus vertical slices encode the finite
information geometry that drives prefix indistinguishability.

Vertical slices impose no restriction on $\eta$ or $\phi$. Any invariant
values compatible with symbolic structure can occur inside any cylinder.

\subsection{Horizontal slices: fixing invariant values}

For $\alpha \in [0,1]$ and $\beta \in [0,\infty]$, the horizontal slices
\[
\mathcal{H}_{\alpha}
= \{ G : \eta(G) = \alpha \},
\qquad
\mathcal{H}^{\beta}
= \{ G : \phi(G) = \beta \}
\]
sort identities according to large scale selector behavior.
These slices cut across all vertical slices and all collapse fibers. When
projected to the invariant plane, horizontal slices appear as straight lines
that reflect asymptotic behavior independent of any finite prefix.

This illustrates a central theme of selector geometry: continuous observation
captures only local structure, while $\eta$ and $\phi$ record global structure.

\subsection{Fiber slices: fixing the collapsed value}

Fix a real number $x$. The fiber
\[
\mathcal{F}(x)
= \{ G \in \mathcal{X}^{*} : \pi(G) = x \}
\]
records all generative identities that produce $x$ under collapse.

Extended invariants depend only on the selector stream and not on the exposed
digits. The map
\[
G \longmapsto (\eta(G), \phi(G))
\]
thus sends $\mathcal{F}(x)$ to a large subset of the invariant plane.
This shows that the classical real number $x$ constrains only the symbolic
content of selected digits and leaves the selector free to vary widely.

\section{Worked Examples of Invariants}

The following examples illustrate the full range of behaviors of $\eta$ and
$\phi$. Each example describes a selector stream; the digit stream may be
assigned arbitrarily to place the identity in any desired collapse fiber.

\subsection{Periodic positive density}

Let $M(n)=D$ for even $n$ and $M(n)=K$ for odd $n$. Then
\[
\eta(G)=\frac{1}{2},
\qquad
n_j = 2j,
\qquad
g_j = 2,
\]
so
\[
\phi(G)=0.
\]
This is an example of positive density with bounded gaps.

\subsection{Positive density with mild irregularity}

Let $M$ repeat the block $DDK$. Then
\[
\eta(G)=\frac{2}{3},
\qquad
\phi(G)=0.
\]
Although the pattern is nonuniform, gap growth remains bounded.

\subsection{Zero density with bounded gaps}

Let $M(n)=D$ when $n$ is prime. The prime density is zero, so
\[
\eta(G)=0.
\]
Since prime gaps grow sublinearly,
\[
\phi(G)=0.
\]

\subsection{Zero density with large fluctuations}

Select positions at factorial indices $n_j=j!$. Then $\eta(G)=0$ and
\[
\frac{g_j}{n_j} = j,
\qquad
\phi(G)=\infty.
\]
This is the canonical example of zero density with unbounded relative gaps.

\subsection{Oscillating density}

Select digits in alternating blocks of growing size:
\[
D^{2^0} K^{2^0} D^{2^1} K^{2^1} D^{2^2} K^{2^2} \cdots.
\]
The density oscillates between values near $0$ and $1$, yielding
\[
\eta(G)=0,
\qquad
\phi(G)=\infty.
\]
These examples demonstrate that $\eta$ and $\phi$ capture fundamentally
different aspects of long range selector behavior.

\section{Robustness and Discontinuity}

The invariants $\eta$ and $\phi$ are invariant under tail changes that occur
beyond any fixed prefix. This robustness follows from their asymptotic
definitions. At the same time, both invariants are maximally discontinuous
with respect to the product topology.

\subsection{Robustness}

If $G$ and $G'$ agree on all sufficiently large indices, then
\[
\eta(G)=\eta(G'),
\qquad
\phi(G)=\phi(G').
\]
Thus both invariants depend only on the tail of the selector.

\subsection{Discontinuity of density}

Let $G$ satisfy $\eta(G)=0$. Define $G_k$ by matching $G$ on $[0..k]$ and
setting $M(n)=D$ for all $n>k$. Then $G_k \to G$ in the product topology, yet
\[
\eta(G_k)=1.
\]
Thus $\eta$ is not upper semicontinuous and is discontinuous at every point.

\subsection{Discontinuity of relative gap growth}

Let $G$ have bounded gaps so that $\phi(G)=0$. Modify $G_k$ by inserting a
single gap of length $\ell_k \to \infty$ beyond index $k$. Then $G_k \to G$
but
\[
\phi(G_k)=\infty.
\]
Thus $\phi$ is also maximally discontinuous.

\subsection{No possibility of continuity}

If $G_k$ selects $D$ at a single position $k$ and $G$ selects $D$ nowhere, then
$G_k\to G$ yet
\[
\eta(G_k)=\frac{1}{k},
\qquad
\eta(G)=0.
\]
Similar discontinuities occur for $\phi$. Both invariants therefore exhibit the
strongest possible failure of continuity in the product topology.

\section{Extended Invariants Inside Collapse Fibers}

Because invariants depend only on the selector, every collapse fiber contains
identities with the full range of invariant values permitted by symbolic
constraints.

\subsection{Arbitrary density inside a fiber}

For any $\alpha \in [0,1]$, construct a selector with $\eta(G)=\alpha$ and
place the canonical digits of $x$ at selected positions. The resulting identity
lies in $\mathcal{F}(x)$ with density $\alpha$.

\subsection{Arbitrary fluctuation inside a fiber}

For any $\beta \in [0,\infty]$, construct a selector with $\phi(G)=\beta$ and
assign the digits of $x$ at selected indices. This yields an identity in
$\mathcal{F}(x)$ with fluctuation $\beta$.

\subsection{Simultaneous control}

Given any pair $(\alpha,\beta)$ in the invariant plane, build a selector
realizing both $\eta=\alpha$ and $\phi=\beta$, and place the digits of $x$
accordingly. Collapse fibers therefore project onto substantial regions of the
invariant plane.

\section{Summary}

This appendix illustrated the geometry of the extended invariants
$\eta$ and $\phi$ and their interaction with vertical slices, horizontal
slices, and collapse fibers. Vertical slices capture finite prefix structure,
horizontal slices capture asymptotic selector behavior, and fiber slices reveal
the symbolic variety compatible with a fixed collapsed value. These viewpoints
explain why collapse reveals only a small part of a generative identity and
why finite observation cannot constrain its asymptotic structure.
