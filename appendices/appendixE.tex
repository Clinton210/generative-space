\chapter{Extended Invariants and Selector Geometry}

\section{Introduction}

This appendix provides explicit examples and geometric interpretations of the
extended invariants introduced in Part~VI.  
These invariants measure large scale properties of the selector stream and
reveal how generative identities distribute across the symbolic space.  
The appendix also develops a slice based geometric viewpoint of selector
behavior, which generalizes and formalizes earlier three dimensional intuition
using the modern terminology of the monograph.

Throughout, $G = (M,D,K)$ denotes a generative identity with selector stream
$M \in \{D,K\}^{\mathbb{N}}$.  
The entropy balance is
\[
\eta(G)
  = \liminf_{N\to\infty}
      \frac{1}{N} \sum_{n<N} \chi_M(n),
\]
and the fluctuation index is
\[
\phi(G)
  = \limsup_{j\to\infty} \frac{g_j}{n_j},
\]
where $n_j$ are the selection indices and $g_j = n_{j+1} - n_j$.

\section{Vertical, Horizontal, and Fiber Slices}

Extended invariants give rise to natural geometric slices through the
generative space.  
These slices provide useful conceptual pictures of selector behavior and
clarify how collapse fibers intersect large scale invariant structure.

\subsection{Vertical slices: fixing a prefix}

A vertical slice is a cylinder set of the form
\[
\mathcal{C}(u)
  = \{ G \in \mathcal{X}^{*} : G[0..N-1] = u \},
\]
where $u$ is a finite prefix of length $N$.  
Vertical slices represent the set of identities that agree on a finite
segment of the selector, digit, and meta streams.

Vertical slices are the regions that structural projections inspect.  
Dependency bounds imply that a projection samples only a vertical slice, and
prefix stabilization ensures that observers ignore all tails beyond the slice
depth.  
This interpretation connects directly to the incompleteness phenomena of
Part~IV.

\subsection{Horizontal slices: fixing invariant values}

Fix $\alpha \in [0,1]$ or $\beta \in [0,\infty]$.  
The horizontal slices
\[
\mathcal{H}_{\alpha}
  = \{ G : \eta(G) = \alpha \},
\qquad
\mathcal{H}^{\beta}
  = \{ G : \phi(G) = \beta \}
\]
represent level sets of extended invariants.  
These sets group identities by long term selector behavior and cut across many
collapse fibers.

When plotted in the $(\eta,\phi)$ plane, horizontal slices correspond to
vertical or horizontal lines.  
They reveal the large scale organization of selector patterns and illustrate
the diversity of possible behaviors.

\subsection{Fiber slices: fixing the collapsed value}

Fix a real number $x$.  
The fiber slice
\[
\mathcal{F}(x)
  = \{ G \in \mathcal{X}^{*} : \pi(G) = x \}
\]
represents all identities that collapse to $x$.  
This slice is a symbolic sheet that cuts through the invariant geometry of
the selector space.

Because extended invariants depend only on the selector, not on collapse, the
image of a fiber under the embedding
\[
G \mapsto (\eta(G), \phi(G))
\]
typically fills a large region of the $(\eta,\phi)$ plane.  
This geometric picture illustrates how collapse conceals nearly all selector
structure.

\section{Worked Examples of Extended Invariants}

\subsection{Periodic positive density example}

Let
\[
M(n) =
\begin{cases}
D & \text{if } n \text{ is even},\\
K & \text{otherwise}.
\end{cases}
\]
Then half of all positions expose digits:
\[
\eta(G) = \frac{1}{2}.
\]
The selection indices are $n_j = 2j$, so $g_j = 2$ and
\[
\phi(G) = 0.
\]
This identity represents regular, evenly spaced digit exposure.

\subsection{Positive density with mild irregularity}

Let $M$ be the periodic sequence obtained by repeating $DDK$.  
Then
\[
\eta(G) = \frac{2}{3},
\qquad
\phi(G) = 0,
\]
even though the exposure pattern is not evenly spaced.

\subsection{Zero density with bounded gaps}

Let $M(n) = D$ when $n$ is prime and $K$ otherwise.  
The density of primes is zero, so
\[
\eta(G) = 0.
\]
However, gaps grow only logarithmically, so
\[
\phi(G) = 0.
\]

\subsection{Zero density with large fluctuations}

Select digits at factorial indices
\[
n_j = j!.
\]
Then $\eta(G) = 0$, but
\[
\frac{g_j}{n_j} = j,
\quad
\phi(G) = \infty.
\]

\subsection{Oscillating density example}

Expose digits in blocks:
\[
D^{2^0}K^{2^0}D^{2^1}K^{2^1}D^{2^2}K^{2^2}\cdots
\]
Then the density oscillates between values near $0$ and near $1$, and
\[
\eta(G) = 0,
\qquad
\phi(G) = \infty.
\]

\section{Semicontinuity Demonstrations}

\subsection{Lower semicontinuity of \texorpdfstring{$\eta$}{eta}}


Let $G$ satisfy $\eta(G) = 0$.  
Define $G_k$ by copying the first $k$ bits of $M$ and then exposing digits
forever.  
Then $\eta(G_k) = 1$ for all $k$ and
\[
\eta(G)
  \le \liminf_{k\to\infty} \eta(G_k).
\]

\subsection{Upper semicontinuity of \texorpdfstring{$\phi$}{phi}}

Let $G$ have evenly spaced selected digits so $\phi(G) = 0$.  
Modify the tail of $M$ in $G_k$ by inserting a single gap of length
$\ell_k \to \infty$.  
Then
\[
\limsup_{k\to\infty} \phi(G_k) = \infty,
\qquad
\phi(G) \ge \limsup_{k\to\infty} \phi(G_k).
\]

\subsection{Failure of continuity}

Let $G_k$ select digit $D$ only at position $k$ and set $G$ to have no
selected digits.  
Then $G_k \to G$, but the invariants behave discontinuously.  
This illustrates that neither $\eta$ nor $\phi$ can be continuous in the
product topology.

\section{Extended Invariants Inside Collapse Fibers}

Since extended invariants depend only on the selector, each collapse fiber
contains identities with arbitrary invariant values.

\subsection{Arbitrary balance in a fiber}

Given any $\alpha \in [0,1]$, construct a selector $M$ with
\[
\eta(G) = \alpha,
\]
assign the canonical digits of $x$ at selected positions, and choose any meta
stream.  
The resulting identity lies in $\mathcal{F}(x)$.

\subsection{Arbitrary fluctuation in a fiber}

Given $\beta \in [0,\infty]$, construct a selector with
\[
\phi(G) = \beta
\]
by adjusting the growth of the gap sequence.  
Placing the digits of $x$ at selected indices yields an identity in
$\mathcal{F}(x)$.

\subsection{Simultaneous control}

Given any $(\alpha,\beta)$, selectors can be built to realize
\[
\eta(G) = \alpha,
\qquad
\phi(G) = \beta,
\]
and the corresponding identities lie in the collapse fiber of $x$.  
Thus fibers map to substantial regions of the invariant plane.

\section{Summary}

This appendix presented explicit examples illustrating the full range of
behavior for $\eta$ and $\phi$, as well as slice based geometric
interpretations that clarify how selectors populate the generative space.  
Vertical slices represent finite symbolic prefixes, horizontal slices represent
invariant level sets, and fiber slices show the richness of selector behavior
compatible with a fixed collapsed value.  
Together, these examples and geometric perspectives support the broader
conclusion that collapse conceals substantial symbolic structure.
