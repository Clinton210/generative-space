\chapter{Case Studies in Invariant Behavior}
\label{appendix:case-studies}

\section{Introduction}

This appendix presents detailed case studies illustrating the behavior of
finite and asymptotic invariants under a fixed collapse representation.
These examples complement the material in
Chapters~\ref{chap:invariants-eta-phi} through
\ref{chap:slice-geometry}, and they demonstrate how observer-derived limits
can fluctuate, diverge, or stabilize in unexpected ways.

The focus is on the following principles:

\begin{itemize}
    \item finite observers depend on finite prefixes,
    \item extended invariants arise as limits or limsups of observer towers,
    \item invariants are Baire class one functions and may fail to converge,
    \item collapse fibers do not constrain asymptotic selector behavior.
\end{itemize}

All examples are representation specific. They illustrate the behavior of
observer-derived limits but do not describe intrinsic structure.
See Appendix~\ref{appendix:extended-invariants} for further geometric
discussion.

\section{Density Instabilities}

This section shows that frequency based invariants can fluctuate even inside
fixed collapse fibers. All constructions refer only to the exposure
coordinate; digits can be assigned to place each identity inside any desired
collapse fiber.

\subsection{Oscillating lower density}

Let the exposure pattern consist of alternating blocks
\[
D^{2^{0}}, K^{2^{0}},
D^{2^{1}}, K^{2^{1}},
D^{2^{2}}, K^{2^{2}}, \ldots
\]
The lower density is zero and the upper density is one. The finite prefix
frequencies oscillate without approaching a limit. This illustrates that
density as a limit need not exist. This mirrors classical constructions in
symbolic dynamics where lower and upper block densities diverge.

\subsection{Slow convergence of density approximants}

Let the exposure stream be
\[
D^{1}, K^{1}, D^{2}, K^{2}, D^{3}, K^{3}, \ldots
\]
The finite approximants satisfy
\[
\frac{1 + 2 + \cdots + m}{2(1 + 2 + \cdots + m)}
= \frac{1}{2},
\]
but convergence is slow. The density invariant is insensitive to prefix
structure, and continuous observation cannot stabilize it in finite time.

\subsection{Density instability inside a fiber}

Fix a real number $x$ and place its canonical digits at the exposed
positions. Any pattern of exposures may be used. All density behaviors remain
available in the fiber because the exposures do not affect the collapse
value.

These examples support the general principle that density invariants cannot
classify collapse fibers.

\section{Entropy Case Studies}

Although entropy is not a central invariant in the main text, finite
approximate entropies behave analogously to frequency invariants. They serve
as instructive examples of observer-derived irregularity.

\subsection{Highly predictable exposure patterns}

Consider a periodic pattern such as $DDK$. The block entropy is zero. The
finite approximants converge rapidly because the symbolic behavior is highly
regular. This shows that simple patterns induce stable observer values.

\subsection{Increasing-block patterns}

Consider blocks of the form
\[
D^{m}, K^{m}, \qquad m = 1,2,3,\ldots
\]
Finite entropies fluctuate because long deterministic blocks suppress
diversity. As $m$ grows, longer prefixes reflect more deterministic structure.
Finite observers cannot stabilize entropy quickly.

\subsection{Entropy in sparse selectors}

Selectors of zero density may have arbitrarily low empirical entropy because
exposures become increasingly rare. This behavior, too, is compatible with any
collapsed value.

\section{Fluctuation Case Studies}

The fluctuation index measures the relative size of successive gaps in the
exposure coordinate. The following examples span its natural range.

\subsection{Bounded gap growth}

If exposure occurs at every second position, then the gaps are all equal to
two. The fluctuation index is zero. This is the simplest example of bounded
fluctuation.

\subsection{Linear gap growth}

Let $n_{j} = j + \lfloor \sqrt{j} \rfloor$. Then
\[
\frac{g_{j}}{n_{j}} \to 0,
\]
but the gaps increase slowly. The fluctuation index remains zero, yet the
behavior is not bounded. This shows that bounded gap growth is not necessary
for small fluctuation index.

\subsection{Superlinear growth}

Selectors with $n_{j} = j!$ satisfy
\[
\frac{g_{j}}{n_{j}} = j,
\]
and therefore have infinite fluctuation index. These identities exist in every
collapse fiber and demonstrate that collapse magnitude places no restriction
on the fluctuation index.

\section{Mixed Invariant Behavior}

This section presents examples combining density and fluctuation features to
illustrate the independence of the two invariants.

\subsection{Positive density and unbounded fluctuation}

Let $M$ place exposures at positions
\[
n_{j} =
\begin{cases}
j, & \text{if $j$ is not a square},\\
j^{2}, & \text{if $j$ is a square}.
\end{cases}
\]
Most exposures occur at positions $j$, giving density one, but occasional
square indexed exposures create very large gaps. Then
\[
\eta(G) = 1,
\qquad
\phi(G) = \infty.
\]

\subsection{Zero density and bounded fluctuation}

Let exposures occur at the prime numbers. This yields zero density and zero
fluctuation index. This shows that zero density does not imply large relative
gaps.

\subsection{Arbitrary pairs inside a fiber}

Given any $(\alpha,\beta)$ in the invariant plane, selectors can be constructed
to realize those values. Collapse freedom ensures that for any real number
$x$, the fiber $\mathcal{F}(x)$ contains identities exhibiting that pair.

\section{Invariant Non-Classifiability}

These examples reinforce the main conclusions of
Chapters~\ref{chap:invariants-eta-phi} through \ref{chap:slice-geometry}:

\begin{itemize}
    \item asymptotic invariants cannot classify collapse fibers,
    \item every invariant value is compatible with every classical value,
    \item invariants are derived from observer towers and inherit their
          limitations,
    \item fibers contain selector behavior of every asymptotic type.
\end{itemize}

Finite observation and collapse magnitude jointly constrain finite-prefix
structure but leave asymptotic behavior free to vary.

\section{Summary}

The case studies presented here demonstrate that observer-derived invariants
exhibit diverse and unstable behaviors. They may converge, oscillate, or
diverge, and any of these behaviors can occur inside every collapse fiber.
These examples support the structural incompleteness results of
Part~\ref{part:incompleteness} and illustrate the independence of finite
prefix geometry from asymptotic selector structure.

