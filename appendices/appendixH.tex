\chapter{Mixed Exposure Patterns Under a Specific Representation}
\label{appendix:mixed-exposure}

\section{Introduction}

This appendix presents examples of mixed exposure patterns that arise under a
fixed collapse representation. These patterns combine regular and irregular
behavior in the exposure coordinate and illustrate how the representation
permits a wide range of asymptotic features inside every collapse fiber.
The purpose is illustrative. None of the phenomena described here are
structural. They reflect the chosen naming system rather than intrinsic
properties of generative identities.

The examples support the results of
Chapters~\ref{chap:invariants-eta-phi} through \ref{chap:slice-geometry} and
demonstrate the following principles:

\begin{itemize}
    \item mixed exposure patterns can create complex finite prefix behavior,
    \item asymptotic invariant values remain representation dependent,
    \item collapse does not restrict mixed patterns,
    \item continuous observers cannot classify mixed behavior.
\end{itemize}

Additional case studies appear in Appendix~\ref{appendix:case-studies}.

\section{Representation Setup}

Throughout this appendix we fix a collapse representation in which one symbolic
coordinate determines exposure positions and another determines the observed
values. The remaining coordinates are auxiliary and do not influence the
collapsed real number. This setup aligns with the representation framework
described in Chapters~\ref{chap:ambient-generative-space} and
\ref{chap:collapse-map}.

Under this representation, exposure patterns determine where the collapse
mechanism reads values. By modifying the exposure coordinate while preserving
the observed-value coordinate, we obtain families of generative identities that
collapse to the same real number but exhibit different asymptotic and finite
prefix behavior.

\section{Mixed Exposure Examples}

The following examples illustrate how mixed patterns combine regularity and
irregularity in the exposure coordinate. These behaviors are representation
dependent and serve as demonstrations of the breadth of collapse fibers.

\subsection{Alternating deterministic and sparse regimes}

Define the exposure sequence by alternating long deterministic blocks with
increasingly sparse blocks:
\[
\underbrace{D D D \cdots D}_{m}
\quad
\underbrace{K K K \cdots K}_{m}
\quad
\underbrace{D D D \cdots D}_{m^{2}}
\quad
\underbrace{K K K \cdots K}_{m^{2}}
\quad
\cdots
\]
The deterministic blocks induce strong finite prefix structure, while the
sparser blocks create large gaps. The resulting behavior oscillates between
high exposure frequency and near absence of exposure. This produces strong
finite prefix variability and significant irregularity in asymptotic
statistics.

\subsection{Coupled window patterns}

Let each exposure decision depend on a preceding window of length
\[
w_{m} = 2^{m}.
\]
Within each window the pattern is deterministic, but windows grow so quickly
that the global exposure pattern shows substantial variation. The exposure
frequency within each window may be fixed, yet the transition between windows
creates sharp jumps in finite prefix characteristics. This yields examples in
which window based structure dominates short scale behavior while long scale
behavior appears sparse or irregular.

\subsection{Periodic base with irregular inserts}

Let the exposure coordinate be periodic except for irregular inserts at rapidly
growing indices:
\[
D K D K D K \cdots
\quad\text{with inserts at positions}\quad
n_{j} = j^{3}.
\]
Each insert may be a short block of either exposures or non-exposures. These
inserts disrupt periodicity but only at widely separated scales. The mixed
pattern creates stable short-range statistics and unstable long-range
statistics. Such examples demonstrate that finite observers may register strong
regularity even when asymptotic invariants vary widely.

\section{Invariant Behavior in Mixed Patterns}

Mixed exposure patterns produce a range of behaviors for finite and asymptotic
invariants. The examples in this section illustrate how these behaviors depend
on the representation and how invariants respond to changes in mixed patterns.
See Appendix~\ref{appendix:extended-invariants} for further examples of
density and fluctuation behavior.

\subsection{Density behavior}

In alternating block patterns, the lower density can approach any value in
\([0,1]\) depending on how block lengths grow. For instance, in the mixed
alternating example, if the deterministic blocks grow faster than the sparse
blocks, the density tends to one. If the sparse blocks dominate, the density
tends to zero. Balanced growth yields oscillation with no limit.

\subsection{Fluctuation behavior}

Mixed patterns often create large relative gaps. Window based patterns produce
gaps of size approximately \(2^{m}\) at the end of each window, which push the
fluctuation index toward infinity. Periodic base patterns with sparse inserts
tend to have bounded fluctuation on short scales but unbounded fluctuation on
large scales.

\subsection{Finite observer instability}

Finite observers detect prefix structure only within fixed windows. In a mixed
pattern, windows may stabilize early behavior but fail to predict behavior in
later windows. This leads to instability in observer derived finite invariants.
Mixed exposure patterns therefore serve as examples of prefix dependent
complexity that finite observers cannot summarize.

\section{Why Mixed Patterns Do Not Classify Fibers}

Mixed exposure patterns illustrate the limitations of classification by
asymptotic invariants or finite observers.

\begin{itemize}
    \item Collapse fixes only the observed values at exposure positions and does
          not restrict the exposure schedule.
    \item Asymptotic invariants such as density and fluctuation depend only on
          the exposure coordinate and therefore vary freely within each collapse
          fiber.
    \item Mixed patterns demonstrate that finite prefix structure does not imply
          asymptotic structure, and asymptotic structure does not imply finite
          prefix behavior.
    \item Continuous observers have finite prefix dependence and therefore
          cannot detect long range features created by mixed patterns.
\end{itemize}

Thus mixed exposure patterns reinforce the Structural Incompleteness Theorem of
Chapter~\ref{chap:indistinguishability}. No set of finite observers, and no
derived asymptotic invariant, can classify identities in a collapse fiber.

\section{Summary}

Mixed exposure patterns provide non-structural case studies that demonstrate
the complexity permitted by collapse representations. These patterns combine
deterministic and irregular regimes, create sharp contrasts between finite
prefix and asymptotic behavior, and exhibit invariant values spanning the full
admissible range. They highlight the representation dependence of invariants,
the freedom present inside collapse fibers, and the inherent limitations of
finite observers.

These examples complement the invariant case studies of
Appendix~\ref{appendix:case-studies} and help clarify the scope of the
incompleteness principles developed in the main text.
