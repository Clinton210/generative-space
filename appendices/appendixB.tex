\chapter{Uniform Bounds and Technical Lemmas}

\section{Introduction}

This appendix collects the technical results that support the development of
finite lookahead, projective incompatibility, and the meta-diagonalizer
construction.  
While Appendix~A summarized background from computable analysis, the results in
this appendix are specific to the generative framework and are used throughout
Chapters~6--9.

The lemmas presented here provide uniform stabilization bounds for computable
secondary projections, prefix agreement principles, and the key sewing and
divergence tools used in the construction of diagonalizers.

\section{Prefix Stabilization}

The first result formalizes the idea that computable observers stabilize on the
basis of a finite prefix of the input identity.  This is a more explicit form
of finite-prefix dependence (Appendix~A) tailored to the layered structure of
generative identities.

\begin{lemma}[Prefix Stabilization]
\label{lem:prefix-stabilization}
Let $\Phi : \mathcal{G}_{\mathrm{eff}} \to \mathbb{R}^k$ be a computable
secondary projection.  
For every rational $\varepsilon > 0$ there exists an $N$ such that if two
effective identities $G$ and $H$ satisfy
\[
(M_G,D_G,K_G){\upharpoonright}N
=
(M_H,D_H,K_H){\upharpoonright}N,
\]
then
\[
\|\Phi(G) - \Phi(H)\| < \varepsilon.
\]
\end{lemma}

\begin{proof}
Encode the triple $(M,D,K)$ as a single computable sequence and apply
Proposition~\ref{prop:finite-prefix} of Appendix~A.
\end{proof}

\section{Uniform Bounds for Finite Families}

A central ingredient in Chapter~6 is the existence of a common horizon for a
finite family of projections.  This guarantees that any observer chosen from a
fixed list inspects at most a bounded prefix of the generator.

\begin{proposition}[Uniform Dependency Bound]
\label{prop:uniform-bound}
Let $\mathcal{P} = \{\Phi_1,\ldots,\Phi_m\}$ be a finite family of computable
secondary projections.  
For any rational $\varepsilon > 0$ there exists an integer $L$ such that for
all $G,H \in \mathcal{G}_{\mathrm{eff}}$,
\[
(M_G,D_G,K_G){\upharpoonright}L
=
(M_H,D_H,K_H){\upharpoonright}L
\quad\Longrightarrow\quad
\|\Phi_i(G) - \Phi_i(H)\| < \varepsilon
\]
for every $1 \le i \le m$.
\end{proposition}

\begin{proof}
Let $B_{\Phi_i}(\varepsilon)$ be the dependency bound for $\Phi_i$.  
Define
\[
L = \max_{1 \le i \le m} B_{\Phi_i}(\varepsilon).
\]
The claim follows immediately from Lemma~\ref{lem:prefix-stabilization}.
\end{proof}

\section{Controlled Divergence Inside Fibers}

The next lemma shows that collapse fibers contain identities with arbitrarily
different secondary structure.  This is the technical backbone of the existence
of adjustment identities used in Chapter~8.

\begin{lemma}[Controlled Divergence]
\label{lem:controlled-divergence}
Let $x \in \mathbb{R}_c$, let $H \in \mathcal{F}_{\mathrm{eff}}(x)$, and let
$\Phi$ be any computable secondary projection.  
For every rational $\delta > 0$ there exists an identity
$A \in \mathcal{F}_{\mathrm{eff}}(x)$ such that
\[
\|\Phi(A) - \Phi(H)\| > \delta.
\]
\end{lemma}

\begin{proof}
Since collapse fibers are infinite (Chapter~3) and $\Phi$ is not injective on
any effective fiber (Chapter~7), we may enumerate effective elements of
$\mathcal{F}_{\mathrm{eff}}(x)$ and search for one whose projection differs
from that of $H$ by more than $\delta$.  
This enumeration is computable because the set of effective generators with
collapse value $x$ is a $\Pi^0_1$ class.
\end{proof}

\section{Tail Sewing and Index Alignment}

The meta-diagonalizer construction requires splicing two identities together
while preserving the digit subsequence of the collapse.  
The following lemma formalizes this sewing operation.

\begin{lemma}[Tail Sewing]
\label{lem:tail-sewing}
Let $H,A \in \mathcal{F}_{\mathrm{eff}}(x)$ and let $L \in \mathbb{N}$.  
There exists an effective identity $G^* \in \mathcal{F}_{\mathrm{eff}}(x)$ such
that:

\begin{enumerate}
\item $G^*{\upharpoonright}L = H{\upharpoonright}L$, and
\item the tail of $G^*$ agrees with a time-shifted tail of $A$ chosen so that
the selected digit indices align.
\end{enumerate}

\end{lemma}

\begin{proof}
Let $k_H$ be the number of digit selections made by $H$ before index $L$.  
Search in $A$ for the least index $L'$ such that $A$ has made $k_H$ digit
selections before $L'$.  
Define $G^*$ to copy the prefix of $H$ up to $L$ and to copy the tail of $A$
starting at $L'$.  
Because $A \in \mathcal{F}_{\mathrm{eff}}(x)$, the shifted tail produces the
correct remaining digit sequence.  
Effectiveness follows from the computability of $H$ and $A$.
\end{proof}

\section{Adjustment Lemma}

The final tool combines uniform bounds, controlled divergence, and tail sewing
to guarantee the existence of a generator that evades any finite family of
secondary projections.

\begin{lemma}[Adjustment Lemma]
\label{lem:adjustment}
Let $\mathcal{P} = \{\Phi_1,\ldots,\Phi_m\}$ be a finite family of computable
secondary projections and let $H \in \mathcal{F}_{\mathrm{eff}}(x)$.  
For every rational $\varepsilon > 0$ there exists an identity
$G^* \in \mathcal{F}_{\mathrm{eff}}(x)$ such that:

\[
\|\Phi_i(G^*) - \Phi_i(H)\| > \varepsilon
\quad\text{for all } i.
\]

\end{lemma}

\begin{proof}
Let $L$ be the uniform dependency bound of
Proposition~\ref{prop:uniform-bound} for precision $\varepsilon/3$.  
By Lemma~\ref{lem:controlled-divergence}, choose an identity $A$ whose
projections differ from those of $H$ by more than $\varepsilon$.  
Apply tail sewing (Lemma~\ref{lem:tail-sewing}) beyond index $L$ to obtain
$G^*$ that agrees with $H$ on the observable prefix and follows $A$ on the
tail.  
Since the switch happens beyond the uniform bound, the projections of $H$ and
$G^*$ differ by more than $\varepsilon$.
\end{proof}

\section{Summary}

The results in this appendix supply the technical scaffolding for the
meta-diagonalizer and the Structural Incompleteness Theorem.  
Prefix stabilization and uniform bounds describe the observational limits of
computable projections, while the controlled divergence and sewing lemmas show
that these limits can be exploited to construct identities that evade any
finite family of observers.

