\chapter{Symbolic Dynamics Essentials}
\label{appendix:symbolic-dynamics}

\section{Introduction}

This appendix summarizes the symbolic dynamics concepts that appear implicitly
throughout the monograph. Although the generative identity framework uses
selector streams rather than general symbolic blocks, many structural features
of selector behavior are naturally expressed in symbolic terms. The purpose of
this appendix is to outline these tools and indicate how they interact with
the generative space and with the asymptotic invariants developed in
Part~VI.

We begin with full shift spaces and the product topology. We then describe
densities, gap statistics, and block structures, and conclude with a brief
discussion of residual sets and typicality in symbolic dynamics.

\section{Shift Spaces and the Product Topology}

\subsection{Full shifts}

Let $\mathcal{A}$ be a finite alphabet. The full shift is the space
\[
\mathcal{A}^{\mathbb{N}}
=
\{\, x_0 x_1 x_2 \cdots : x_n \in \mathcal{A} \,\}.
\]
Basic open sets are cylinders
\[
[x_0 x_1 \cdots x_{k-1}]
=
\{\, y : y_i = x_i \text{ for } 0 \le i < k \,\}.
\]

The product topology makes $\mathcal{A}^{\mathbb{N}}$ compact, totally
disconnected, and metrizable. These properties hold for the ambient generative
space $\mathcal{X}$, which is also a full shift on a finite alphabet. The
subspace $\mathcal{X}^{*}$, which requires infinitely many digit exposures, is
dense but not compact. This distinction is important in the analysis of
collapse fibers and convergence.

\subsection{The shift map}

The shift map
\[
\sigma(x)_n = x_{n+1}
\]
is continuous, surjective, and preserves cylinder sets. Although the shift is
not used as a dynamical map in the monograph, it provides structural
intuition. Properties such as densities, recurrence, and gap growth are
shift-invariant features.

Selectors are sequences in $\{D,K\}^{\mathbb{N}}$, and shifting corresponds
to advancing the decision of which positions expose digits. This perspective
connects the generative setting to classical symbolic tools.

\subsection{Subshifts}

A subshift is a closed, shift invariant subset of $\mathcal{A}^{\mathbb{N}}$.
Such sets arise by forbidding finite blocks. Families of selectors with
additional constraints (such as prescribed asymptotic density or regular gap
control) form natural subshifts. These subshifts offer a symbolic framework
for describing structured selector behavior.

\section{Densities and Gap Structure}

\subsection{Lower and upper densities}

For $x \in \mathcal{A}^{\mathbb{N}}$ and $a \in \mathcal{A}$, define
\[
\underline{d}_{a}(x)
=
\liminf_{N \to \infty}
\frac{1}{N}
\left|\{ 0 \le n < N : x_n = a \}\right|,
\]
\[
\overline{d}_{a}(x)
=
\limsup_{N \to \infty}
\frac{1}{N}
\left|\{ 0 \le n < N : x_n = a \}\right|.
\]

For selectors $M \in \{D,K\}^{\mathbb{N}}$, the asymptotic density
\[
\eta(G) = \underline{d}_{D}(M)
\]
plays the role of the lower frequency of digit exposures. Chapter
\texttt{\ref{chap:invariants-eta-phi}} analyzes $\eta$ as an asymptotic
invariant and shows that it is discontinuous at every point of the generative
space.

\subsection{Gap sequences}

List the indices at which $x_n=a$ as
\[
n_0 < n_1 < n_2 < \cdots.
\]
The gap sequence is $g_j = n_{j+1} - n_j$. The relative gap growth is
\[
\phi(x)
=
\limsup_{j \to \infty}
\frac{g_j}{n_j},
\]
which is the fluctuation index in the main text. Large values of $\phi$
correspond to sporadic occurrences of $a$ relative to scale.

Gap sequences and related statistics are classical tools for studying sparse
occurrences in symbolic sequences. In the generative framework, they describe
the long range behavior of selectors.

\subsection{Recurrence}

A symbol $a$ is recurrent in $x$ if it appears infinitely often. For selectors
in $\mathcal{X}^{*}$, the requirement that $D$ be recurrent corresponds to the
obligation that digits be exposed infinitely many times in order to produce a
complete canonical expansion. Selectors of zero density admit arbitrarily long
gaps, while positive density selectors exhibit more regular spacing.

\section{Block Structures and Empirical Measures}

\subsection{Blocks}

A block of length $k$ is an element of $\mathcal{A}^{k}$. The set of blocks
appearing in a symbolic sequence $x$ is
\[
\mathcal{L}(x)
=
\bigcup_{k \ge 0}
\{\, x_n x_{n+1} \cdots x_{n+k-1} : n \in \mathbb{N} \,\}.
\]

Selector block structures in $\{D,K\}^{k}$ describe finite patterns of
exposures and suppressions. These local patterns determine fine scale features
that are not captured by invariants such as $\eta$ or $\phi$.

\subsection{Empirical measures}

Given $u \in \mathcal{A}^{k}$, the empirical frequency is
\[
\mathrm{freq}_{N}(u,x)
=
\frac{1}{N}
\left|
\{\, 0 \le n < N-k+1 : x_n x_{n+1} \cdots x_{n+k-1}=u \,\}
\right|.
\]

Empirical frequency ideas motivate possible higher order invariants, such as
block frequencies or empirical measures on selector streams. These quantities
extend the geometric framework of extended invariants discussed in Part~VI.

\section{Residual Structure and Irregularity}

Residual sets, or dense $G_{\delta}$ subsets, describe typical behavior in the
sense of Baire category. In classical symbolic dynamics, many forms of
irregularity are residual:

\begin{itemize}
    \item unbounded fluctuations in gap growth,
    \item oscillating symbol frequencies,
    \item absence of limiting densities.
\end{itemize}

Although the generative identity framework does not rely directly on residual
genericity, the prevalence of irregular symbolic behavior demonstrates that
collapse fibers contain identities with extreme or pathological selector
patterns. This supports the structural indistinguishability results.

\section{Interaction with the Generative Identity Framework}

Symbolic dynamics interacts with the generative framework in several ways.

\begin{itemize}
    \item Selector streams are symbolic sequences in a full shift space, and
          their long term behavior determines the asymptotic invariants
          $\eta$ and $\phi$.

    \item Density and gap statistics describe tail behavior of selectors,
          which is central to invariant analysis and prefix indistinguishability.

    \item Block structures and empirical frequency ideas motivate refined
          invariants beyond those studied in the monograph and suggest future
          directions for generative geometry.

    \item The product topology on symbolic sequences is the same topology used
          to define continuity of structural projections and to obtain
          dependency bounds.

    \item The irregularity typical of symbolic sequences helps explain why
          collapse fibers contain identities with many distinct selector
          patterns.
\end{itemize}

Symbolic dynamics therefore provides a natural mathematical language for
describing selector behavior, asymptotic invariants, and the geometry of the
generative space.
