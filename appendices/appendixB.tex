\chapter{Symbolic Dynamics Essentials}

\section{Introduction}

This appendix summarizes the symbolic dynamics concepts that appear implicitly
throughout the monograph.  
Although the generative identity framework uses selector streams rather than
traditional symbol blocks, many of its structural properties are naturally
expressed using tools from symbolic dynamics.  
The purpose of this appendix is to describe these tools and explain how they
interact with the generative space.

We begin with the shift map and the product topology on sequences.  
We then describe notions of frequency, recurrence, gap statistics, and
residual sets, all of which are used in the analysis of selector behavior.  
The appendix concludes with a discussion of block structures and how they
relate to extended invariants.

\section{Shift Spaces and the Product Topology}

\subsection{Full shift spaces}

Let $\mathcal{A}$ be a finite alphabet.  
The full shift over $\mathcal{A}$ is the space
\[
\mathcal{A}^{\mathbb{N}} = \{ x_0 x_1 x_2 \ldots : x_n \in \mathcal{A} \}.
\]
This space is equipped with the product topology generated by basic open sets
of the form
\[
[x_0 x_1 \ldots x_{k-1}]
  = \{ y \in \mathcal{A}^{\mathbb{N}} : y_i = x_i \text{ for } 0 \le i < k \},
\]
also called cylinder sets.

The product topology makes $\mathcal{A}^{\mathbb{N}}$ compact, totally
disconnected, and metrizable.  
These topological properties are inherited by the generative space
$\mathcal{X}$ and the digit selecting subspace $\mathcal{X}^{*}$.

\subsection{The shift map}

The shift map
\[
\sigma : \mathcal{A}^{\mathbb{N}} \to \mathcal{A}^{\mathbb{N}}
\]
is defined by $(\sigma(x))_n = x_{n+1}$.  
It is continuous, surjective, and preserves the cylinder structure.

In the generative identity framework, one may shift the selector, digit, or
meta stream independently.  
The shift is not used as a dynamical map in the main text, but the structural
intuition provided by shifting plays an important role.  
For example, asymptotic densities and gap statistics are shift invariant
properties.

\subsection{Subshifts}

A \emph{subshift} is a closed, shift invariant subset of $\mathcal{A}^{\mathbb{N}}$.  
Such sets are determined by specifying which finite blocks of symbols are
allowed or forbidden.

Selectors may be viewed informally as elements of the subshift
\[
\{D,K\}^{\mathbb{N}},
\]
and families of selectors with additional structural constraints form natural
subshifts within this space.

\section{Density, Frequencies, and Gap Structure}

\subsection{Lower and upper densities}

For $x \in \mathcal{A}^{\mathbb{N}}$ and $a \in \mathcal{A}$, the lower and
upper densities of the symbol $a$ are given by
\[
\underline{d}_{a}(x)
  = \liminf_{N\to\infty}
      \frac{1}{N} \bigl| \{ 0 \le n < N : x_n = a \} \bigr|,
\]
\[
\overline{d}_{a}(x)
  = \limsup_{N\to\infty}
      \frac{1}{N} \bigl| \{ 0 \le n < N : x_n = a \} \bigr|.
\]

In selector analysis, these become:
\[
\eta(G) = \underline{d}_{D}(M),
\]
the lower asymptotic density of digit exposures.

\subsection{Gap sequences}

Let $x \in \mathcal{A}^{\mathbb{N}}$ and fix a symbol $a \in \mathcal{A}$.  
List the positions at which $x_n = a$ as
\[
n_0 < n_1 < n_2 < \ldots.
\]
Define the gap sequence
\[
g_j = n_{j+1} - n_j.
\]

The \emph{relative gap growth} is the limit superior
\[
\phi(x)
  = \limsup_{j\to\infty} \frac{g_j}{n_j},
\]
which is the definition of the fluctuation index in the main text.

Gap statistics are classical objects in symbolic dynamics and informally
describe how sparsely a symbol may occur.

\subsection{Recurrence and regularity}

A symbol $a$ is \emph{recurrent} in $x$ if it appears infinitely often.  
Every selector in $\mathcal{X}^{*}$ has $D$ recurrent, since otherwise the
canonical output would be finite.

Selectors with positive density of $D$ are regular in the sense that their gap
sequence is bounded above by a linear function.  
Selectors of zero density admit much larger fluctuations.

\section{Block Structures and Empirical Measures}

\subsection{Blocks and patterns}

A block (or word) of length $k$ over $\mathcal{A}$ is an element of
$\mathcal{A}^{k}$.  
The set of blocks appearing in $x \in \mathcal{A}^{\mathbb{N}}$ is
\[
\mathcal{L}(x)
  = \bigcup_{k\ge0}
      \{ x_n x_{n+1} \ldots x_{n+k-1} : n \in \mathbb{N} \}.
\]

Selectors have block structures in $\{D,K\}^{k}$ that reflect which positions
expose digits and which do not.  
These blocks determine fine scale properties of selectors that are not
captured by density or fluctuation alone.

\subsection{Empirical frequency measures}

For a block $u \in \mathcal{A}^{k}$, the empirical frequency up to index $N$
is
\[
\mathrm{freq}_{N}(u,x)
  = \frac{1}{N}
    \bigl| \{ 0 \le n < N-k+1 : x_{n} \ldots x_{n+k-1} = u \} \bigr|.
\]

Empirical frequencies provide refined statistical information about symbolic
sequences.  
Although not used directly in the main text, these measures motivate the
extended invariants discussed in Part VI.

\section{Selector Behavior in Symbolic Terms}

Selectors are symbolic sequences in $\{D,K\}^{\mathbb{N}}$ with the additional
constraint that $D$ must occur infinitely often.  
Many properties of selectors are classical:

\begin{itemize}
    \item positive density selectors correspond to sequences in which $D$
          occurs with positive lower density,
    \item zero density selectors correspond to sparse symbol occurrences,
    \item large fluctuation selectors correspond to sequences with large gaps.
\end{itemize}

These behaviors are well studied in the context of return times, symbolic
recurrence, and sparse subshifts.  
The extended invariants $\eta$ and $\phi$ adapt classical notions to the
selector setting.

\section{Residual Structure and Typicality}

In classical symbolic dynamics, residual sets (dense $G_\delta$ sets) describe
typical behavior under the Baire category notion of genericity.  
Many selector-related properties are generic in the space $\{D,K\}^{\mathbb{N}}$:

\begin{itemize}
    \item irregular gap growth,
    \item oscillating frequencies,
    \item absence of limiting densities.
\end{itemize}

Although the generative identity framework does not rely directly on generic
properties, the irregularity of symbolic sequences supports the observation
that collapse fibers contain many identities with extreme or pathological
selector patterns.

\section{Interaction with the Generative Identity Framework}

The symbolic tools summarized above enter the monograph in the following ways.

\begin{itemize}
    \item Selector streams are symbolic sequences in a full shift
          $\{D,K\}^{\mathbb{N}}$, and their asymptotic properties determine the
          extended invariants $\eta$ and $\phi$.

    \item Lower densities, gap sequences, and limsup statistics are used to
          analyze large scale selector structure.

    \item Block structures and empirical frequency ideas motivate possible
          higher dimensional invariants that extend the geometric picture in
          Part VI.

    \item The product topology on symbolic sequences is the same topology used
          to define continuity of structural projections and to derive
          dependency bounds.

    \item Residual irregularity of symbolic sequences illustrates why collapse
          fibers contain identities with many distinct selector patterns,
          reinforcing the incompleteness phenomena of Part IV.
\end{itemize}

Symbolic dynamics therefore provides a conceptual and mathematical foundation
for understanding selector behavior, extended invariants, and the geometry of
the generative space.
