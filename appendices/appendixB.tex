\chapter{Product Topology and Cylinder Continuity}

\section{Introduction}

The generative space $\mathcal{X}$ is a product of sequence spaces equipped with
the product topology.  
Continuity in this setting is determined entirely by finite prefixes, and this
simple topological structure underlies the continuity of collapse, canonical
output, structural projections, and the embeddings developed in Part~VI.

This appendix summarizes the background on product topology and cylinder sets
used throughout the monograph.

\section{Sequence Spaces and Basic Open Sets}

Let $A$ be a finite discrete alphabet.  
The sequence space $A^{\mathbb{N}}$ is equipped with the product topology, whose
basis consists of \emph{cylinder sets} determined by finite prefixes.

\begin{definition}[Cylinder Set]
For $u \in A^k$, the cylinder determined by $u$ is
\[
[u] = \{\, x \in A^{\mathbb{N}} : x(0)=u_0,\ldots,x(k-1)=u_{k-1} \,\}.
\]
\end{definition}

Cylinder sets form the basis for the topology, and a map between sequence spaces
is continuous iff agreement on sufficiently long prefixes of the input implies
agreement on prefixes of the output.

\section{The Generative Space}

Recall that the generative space is the product
\[
\mathcal{X}
  = \{D,K\}^{\mathbb{N}}
  \times \{0,1,\ldots,b-1\}^{\mathbb{N}}
  \times \Sigma^{\mathbb{N}}.
\]

Each component space carries the cylinder topology, and $\mathcal{X}$ is given
the product topology.  
A basic neighborhood of $G=(M,D,K)$ is determined by finite prefixes of each
component:

\[
[M|_k] \times [D|_{\ell}] \times [K|_m].
\]

\section{Continuity of the Canonical Output}

The canonical output map
\[
G \mapsto X(G)
\]
is continuous because it is defined coordinatewise by reading either $D(n)$ or
$K(n)$ depending on $M(n)$.  
If two generators agree on a finite prefix of $(M,D,K)$, they agree on the
corresponding prefix of $X(G)$.

\begin{proposition}
The canonical output map $X : \mathcal{X} \to A^{\mathbb{N}}$ is continuous in
the product topology.
\end{proposition}

\section{Continuity of Collapse}

The collapse map
\[
\pi : \mathcal{X}^* \to [0,1]
\]
is continuous because:

\begin{itemize}
    \item it depends on the selected digit subsequence,
    \item every finite approximation of $\pi(G)$ is determined by finitely many
          selected digit values,
    \item those values depend on finitely many coordinates of $M$ and $D$.
\end{itemize}

\begin{proposition}
The collapse map $\pi$ is continuous with respect to the product topology on
$\mathcal{X}^*$ and the usual topology on $[0,1]$.
\end{proposition}

\section{Closedness of Collapse Fibers}

A key fact used in Part~I is that collapse fibers
\[
\mathcal{F}(x) = \pi^{-1}(\{x\})
\]
are closed subsets of $\mathcal{X}$.

\begin{proposition}
For each $x\in[0,1]$, the fiber $\mathcal{F}(x)$ is closed in $\mathcal{X}$.
\end{proposition}

\begin{proof}
Since $\pi$ is continuous and $\{x\}$ is closed in the Hausdorff space
$[0,1]$, the preimage $\pi^{-1}(\{x\})$ is closed.
\end{proof}

\section{Prefix Sensitivity and Structural Projections}

Every structural projection $\Phi : \mathcal{X} \to \mathbb{R}^k$ used in Parts
III–VI is continuous in the product topology.  
This ensures:

\begin{itemize}
    \item projections respect finite-prefix neighborhoods,
    \item projection lattice operations preserve continuity,
    \item dependency bounds for computable projections rely on topological prefix
          sensitivity.
\end{itemize}

In particular:

\[
G|_k = H|_k
\quad\Longrightarrow\quad
\Phi(G)|_m = \Phi(H)|_m
\]
for suitably chosen $k$ depending on $m$.

\section{Continuity of Extended Coordinate Maps}

Secondary and tertiary invariants, such as entropy balance
\[
\eta(G)
\]
and fluctuation index
\[
\phi(G),
\]
are limits of empirical statistics computed from prefixes of the selector or
canonical output.  
Such invariants are continuous because:

- long-run averages depend on finite prefixes for any finite precision,
- small changes in finite prefixes change empirical frequencies only slightly.

\begin{proposition}
Entropy balance, fluctuation index, and all extended coordinate maps developed
in Part~VI are continuous in the product topology on $\mathcal{X}$.
\end{proposition}

\section{Summary}

This appendix summarized the topological structure underlying the generative
framework:

\begin{itemize}
    \item sequence spaces carry the cylinder topology,
    \item $\mathcal{X}$ is the product of three such spaces,
    \item collapse and structural projections are continuous,
    \item collapse fibers are closed sets,
    \item extended invariants are continuous limit maps.
\end{itemize}

These properties support the analysis of collapse fibers (Part~I), selector
regimes (Part~II), the projection lattice (Part~III), and the embedding theory
of extended coordinates (Part~VI).
