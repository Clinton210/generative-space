\chapter*{Part VI Summary}

Part VI introduces extended invariants that measure large scale features of
selector behavior and provide coarse geometric perspectives on the generative
space. These invariants capture asymptotic properties of the selector stream
and therefore reveal structural features that survive tail modification but
remain invisible to continuous projections.

The first chapter presents the entropy balance $\eta$ and the fluctuation
index $\phi$. The entropy balance measures the lower asymptotic density of
digit exposures, while the fluctuation index measures the relative growth of
gaps between successive selected positions. These quantities are tail
dependent and discontinuous at every point of the product space, yet they
satisfy natural semicontinuity properties that allow controlled analysis of
their behavior. Their full range of values appears inside every collapse
fiber, which illustrates the symbolic diversity hidden beneath classical
magnitude.

The second chapter develops geometric embeddings based on these invariants.
Plotting identities in the $(\eta,\phi)$ plane reveals large scale structure
in selector behavior. Positive density selectors and sparse selectors occupy
very different regions, and identities with large fluctuation index lie along
extreme geometric directions. Additional coordinates may be introduced using
block statistics, gap growth patterns, or meta stream features, suggesting a
higher dimensional geometric organization of the generative space.

The final chapter synthesizes the framework and outlines future directions.
Extended invariants and geometric embeddings provide new ways to understand
generative representations of real numbers and suggest further study of higher
order invariants, symbolic dynamical methods, and connections to computability
and randomness.

Part VI therefore shows how generative identities can be analyzed using
structural, asymptotic, and geometric coordinates that lie beyond collapse and
beyond the reach of finite observers.
