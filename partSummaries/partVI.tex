\chapter*{Part VI Summary}

Part VI examines the behavior of extended invariants derived from observer
towers and studies how these invariants interact with finite prefix structure
and collapse fibers. These invariants describe long range patterns in symbolic
coordinates but are not themselves structural parameters. They arise as limits
or limsup values of families of continuous observers and therefore inherit the
same representation dependence and the same finite information restrictions as
the observers from which they originate.

The primary examples are the asymptotic density of exposures and the
fluctuation index that measures relative gap growth. Both invariants depend
only on the tail of the symbolic coordinate that governs exposure events. Every
finite prefix can be modified without affecting their values. This tail
dependence implies that they are Baire class 1 functions obtained as pointwise
limits of continuous approximants. Consequently they are discontinuous at every
point of the ambient space and vary freely within every nonempty cylinder set.

Part VI shows that extended invariants do not classify collapse fibers and
cannot be used to recover generative information. Fibers contain identities
with every admissible combination of invariant values. This follows from the
perfectness of fibers together with the freedom to replace tails while
preserving the collapsed value. The invariants therefore describe asymptotic
regimes that coexist densely inside each fiber, independent of classical
magnitude.

The slice perspective clarifies these relationships. Vertical slices fix finite
prefixes and represent the region visible to continuous observers. Horizontal
slices fix invariant values and cut across every vertical slice and every
fiber. Fiber slices fix collapsed magnitude and admit every compatible
asymptotic behavior. Together these slices show that finite observation,
asymptotic behavior, and classical value form three independent geometric
dimensions within the generative space.

Part VI concludes with case studies illustrating instability and oscillation in
invariant behavior. These examples demonstrate that invariant values can
diverge, oscillate, or fail to converge, even while the underlying identities
remain indistinguishable to all continuous observers. They emphasize that
extended invariants serve only as coarse, derived descriptions of asymptotic
behavior rather than intrinsic coordinates of generative structure.
