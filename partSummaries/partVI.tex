\chapter*{Part VI Summary}

Part VI introduces extended invariants that measure large scale features of
selector behavior and provide coarse geometric perspectives on the generative
space.  
Unlike collapse or finite observers, these invariants capture asymptotic
properties of the selector stream and therefore reveal structural features
that survive tail modification but remain invisible to continuous projections.

The first chapter presents the entropy balance $\eta$ and the fluctuation
index $\phi$.  
The balance measures the lower asymptotic density of digit exposures, while
the fluctuation index measures the growth of relative gaps between selected
positions.  
These invariants are discontinuous but satisfy natural semicontinuity
properties.  
Their values vary widely inside collapse fibers, which illustrates the
symbolic diversity hidden beneath classical magnitude.

The second chapter introduces geometric embeddings based on these invariants.
Plotting generative identities in the $(\eta, \phi)$ plane reveals large scale
structure in selector behavior.  
Hybrid and null density selectors occupy distinct regions, and identities with
high or low fluctuation index appear at very different geometric scales.
Higher dimensional embeddings are also possible using block statistics, gap
growth rates, or meta stream behavior.

The final chapter synthesizes the framework and outlines future directions.
Extended invariants and geometric embeddings provide new perspectives on the
generative representation of real numbers and suggest further investigation of
higher order invariants, connections to symbolic dynamics, and interactions
with computability and randomness.

Part VI therefore shows how generative identities can be analyzed using
structural, asymptotic, and geometric coordinates that lie beyond collapse and
beyond the reach of finite observers.
