\chapter*{Part III Summary}

Part III develops the theory of structural projections, which formalize how
continuous observers extract information from generative identities. A
structural projection is a continuous real valued functional on the generative
space. Such observers depend only on finite symbolic information at any fixed
precision, and this finite information principle is expressed through
computable dependency bounds.

A dependency bound specifies how long a prefix must be inspected in order to
guarantee a desired accuracy. This yields prefix stabilization. Whenever two
identities agree on a prefix that is long enough for a family of observers,
their values under all observers in that family differ by less than the
prescribed tolerance. Changes made beyond that prefix are invisible to every
observer in the family.

Different projections impose different prefix requirements. These requirements
may be incompatible when combined. One observer may force exposure of digits
at regular intervals, while another may require periods of suppression to
achieve its value. Such incompatibilities demonstrate that finite prefix
constraints cannot fully control the internal structure of identities inside a
collapse fiber. The symbolic freedom that remains outside these constraints is
what allows controlled divergence of identities while preserving classical
value.

Part III establishes the fundamental limits of continuous observation. It
presents the dependency bound machinery, develops prefix stabilization, and
identifies the sources of projective incompatibility. These tools form the
technical foundation for the mimicry and diagonalization arguments carried out
in Part IV.
