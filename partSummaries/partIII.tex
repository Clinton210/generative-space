\chapter*{Part III Summary}

Part III establishes the incompleteness and indistinguishability phenomena that
lie at the heart of the Generative Identity Framework. The central theme is
that finite continuous observation cannot recover the symbolic structure of a
generative identity. Continuous observers depend only on finite prefixes, while
collapse conceals all tail structure. These limitations combine to produce a
strong form of structural indistinguishability inside every effective collapse
fiber.

The technical backbone of Part III consists of the alignment and sewing
constructions. Identities in the same collapse fiber expose the same canonical
digits, though at different selector positions. Alignment identifies matching
selection indices across identities, and sewing replaces the tail of one
identity with the tail of another while preserving the collapsed value. When
combined with dependency bounds, these tools allow observers to be frozen on
any finite window while the tail is modified freely.

Using these ideas, Part III constructs a computable identity that mimics a
reference identity on every prefix required by a finite family of observers yet
diverges on infinitely many coordinates. Taking limits across an effective
enumeration of observers yields the Structural Indistinguishability Theorem:
for any computable identity in the effective fiber of a computable real number,
there exists a distinct computable identity that is observationally
indistinguishable from it for all computable structural projections.

This result shows that no finite or computable family of continuous observers,
even when combined with collapsed magnitude, can reconstruct the generative
identity that produced a given real number. The incompleteness of observation
is thus intrinsic to the topology of the generative space and to the finite
information constraints governing continuous functionals.
