\chapter*{Part III Summary}

Part III develops the theory of structural projections, which formalize how
continuous observers extract information from generative identities.  
A structural projection is any continuous real valued functional on the
generative space.  
Such observers depend only on finite prefixes of an identity at any fixed
precision, and this finite information principle is captured by computable
dependency bounds.

Dependency bounds provide explicit control over the amount of symbolic data
required to determine the value of an observer within a given error.  
This leads to prefix stabilization, which states that once two identities
agree on a sufficiently long prefix, all observers in a finite family must
agree on their values to within any chosen tolerance.  
Tail modification beyond this prefix has no effect on the output of the
observers.

Different observers impose different finite constraints on generative
identities.  
These constraints may conflict in a single prefix, producing projective
incompatibility.  
For example, one observer may require frequent digit exposures, while another
requires long gaps.  
Such conflicts show that no finite prefix can simultaneously satisfy all
structural demands and provide the combinatorial mechanism that allows
controlled divergence inside collapse fibers.

Part III therefore establishes the observational limits imposed by
continuity, provides the finite information tools that govern the behavior of
observers, and sets the stage for the diagonalizer construction in Part IV.
