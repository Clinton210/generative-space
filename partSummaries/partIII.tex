\part*{Summary of Part III: Secondary Coordinates and Observational Limits}

Part~III examines the role of secondary projections, which provide coordinate systems for describing internal features of generative identities beyond classical magnitude.  
Chapter~6 defines secondary projections as continuous or computable functionals on the generative space and establishes their finite-prefix dependency bounds.  
These bounds reflect the fundamental limitation that each computable projection can inspect only a finite initial segment of an effective identity to produce any fixed-precision approximation.

Chapter~7 develops concrete examples of secondary coordinate systems, including digit and meta frequency vectors, entropy-type statistics, local variation measures, and mixer complexity.  
These projections reveal distinct and sometimes incompatible aspects of hybrid and ghost identities.  
This mismatch of perspectives is called the Rashomon effect: different projections give different views of the same generative mechanism.

Together, these results show that secondary projections capture only partial information about internal structure.  
Their finite dependence properties set the stage for the diagonalizer construction and the impossibility results developed in Part~IV.
