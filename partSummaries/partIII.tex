\part{Structural Projection Theory}

\chapter*{Summary of Part III: Structural Projection Theory}
\addcontentsline{toc}{chapter}{Summary of Part III: Structural Projection Theory}

Part~III develops the mathematical theory of \emph{structural projections}:
maps that extract partial numerical information from generative identities.
These projections formalize the act of “observing’’ internal structure beyond
classical magnitude.  The goal of this part is twofold: to establish the
topological and order-theoretic foundations of projection-based measurement, and
to determine the limits of what can be recovered from such observations.

\medskip

\textbf{Chapter~6 introduces the projection lattice.}
A structural projection is defined as a continuous map
\[
\Phi : \mathcal{X} \to \mathbb{R}^k
\]
whose value depends on symbolic structure but is invariant under all
irrelevant coordinate changes.  Classical collapse $\pi$ appears as the
minimal projection that preserves classical magnitude and simultaneously as the
maximally lossy projection in the entire lattice.  This lattice perspective
establishes the conceptual separation between the \emph{geometry of values}
encoded by $\pi$ and the \emph{geometry of structure} encoded by other
projections.

\medskip

\textbf{Chapter~7 introduces computable secondary projections} by imposing
Type--2 computability constraints.  The key result is the finite-lookahead
principle: any computable projection can inspect only a finite prefix of an
effective generator when producing approximations of fixed precision.  This
leads to explicit \emph{dependency bounds} that quantify the observational
horizon of each computable coordinate system.

\medskip

\textbf{Chapter~8 develops projective incompatibility.}
Concrete families of projections—digit frequencies, meta-layer statistics,
local-variation measures, and selector-based complexity—are shown to highlight
mutually incompatible aspects of generative structure.  Distinct identities
within a single collapse fiber can be simultaneously distinguished by one
projection and indistinguishable under another.  This structural misalignment
across projections is the first sign that generative structure cannot be
compressed into a finite coordinate system.

\medskip

Together, these chapters establish the mathematical theory of structural
projections: their lattice, their topological constraints, their computational
limitations, and the ways in which their perspectives clash.  This prepares the
ground for the diagonalization arguments of Part~IV, where finite families of
computable projections are shown to be inherently incapable of classifying
effective generative identities.
