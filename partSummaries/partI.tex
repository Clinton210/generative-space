\part*{Summary of Part I: The Generative Ontology}

Part~I introduces the foundational objects of the generative framework.  
The generative space $\mathcal{X}$ is defined as a layered product space of mixer, digit, and meta sequences equipped with the product topology.  
Its effective core $\mathcal{G}_{\mathrm{eff}}$ consists of computable generative identities and parallels the classical distinction between Baire space and its computable elements in Type--2 computability.

The collapse map $\pi$ is introduced as the primary invariant.  
It extracts the digit subsequence selected by the mixer and decodes it into classical magnitude.  
Collapse maps the full space onto the classical continuum and maps the effective core onto the computable real numbers.  
Fibers of the collapse map, studied in Chapter~3, reveal the internal structure underlying each real number.  
These fibers are uncountable in the full space and infinite in the effective core.  
They form the geometric setting for the hybrid and ghost behaviors that appear in Part~II.
