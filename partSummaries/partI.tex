\chapter*{Part I Summary}

Part I develops the symbolic and topological foundations of the Generative
Identity Framework. A generative identity is defined as a triple of infinite
streams $(M,D,K)$ consisting of a selector, a digit source, and a
meta-information source. These streams form a compact product space
$\mathcal{X}$ equipped with the product topology. The subspace
$\mathcal{X}^{*}$ contains those identities whose selectors expose infinitely
many digits and therefore produce complete canonical outputs.

The collapse map assigns a classical real number to each identity by reading
the digits exposed by the selector and interpreting them as a base $b$
expansion. This map is continuous and surjective. Its fibers are compact,
perfect, and totally disconnected subsets of the generative space. Two
identities lie in the same fiber exactly when they expose the same digit
sequence, even if their selectors and meta streams differ extensively.

This geometry shows that collapse conceals a large amount of symbolic
structure. A single real number has a collapse fiber that contains identities
with positive or zero selector density, regular or highly irregular gap
patterns, and arbitrary meta-information content. None of these features
affect the collapsed value. The fiber therefore represents the symbolic
variation that classical magnitude does not capture.

Part I establishes the ambient generative space, the collapse map, and the
structure of collapse fibers. These foundations prepare the way for the study
of continuous observers in Part II and for the incompleteness results proved
in Part III and Part IV. The symbolic richness of the fibers motivates the
central question of the monograph. How much of this hidden structure can be
recovered by observers that access only finite prefixes of a generative
identity?
