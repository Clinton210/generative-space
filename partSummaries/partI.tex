\chapter*{Part I Summary}

Part I introduces the symbolic foundations of the Generative Identity
Framework.  
A generative identity is defined as a triple of infinite sequences
$(M, D, K)$ consisting of a selector stream, a digit stream, and a
meta-information stream.  
These sequences form a product space $\mathcal{X}$ equipped with the product
topology, and the digit selecting subspace $\mathcal{X}^*$ contains those
identities that expose infinitely many digits.

The collapse map extracts the classical real value associated with a
generative identity by selecting the digits exposed by $M$ and interpreting
them in base $b$.  
This map is continuous and surjective.  
Its fibers are compact, perfect, and totally disconnected, and they contain
many identities that differ dramatically in their internal structure while
producing the same collapsed value.

The geometry of collapse fibers is the first indication that classical
magnitude conceals substantial symbolic structure.  
Fibers contain identities with dense or sparse selectors, identities with
regular or irregular spacing patterns, and identities with arbitrary
meta-information streams.  
These degrees of freedom motivate the study of how much structure can be
detected by continuous observers, which becomes the focus of Part III and the
incompleteness phenomena of Part IV.

Part I therefore provides the symbolic setting, the collapse mechanism, and
the foundational geometry that underpins the entire monograph.
