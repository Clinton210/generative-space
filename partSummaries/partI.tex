\part{The Generative Ontology}

\chapter*{Summary of Part I: The Generative Ontology}
\addcontentsline{toc}{chapter}{Summary of Part I: The Generative Ontology}

Part~I establishes the foundational setting of the generative framework.
The central objective is to replace the classical viewpoint—where real numbers
arise as magnitudes—with a mechanism-oriented perspective in which real
numbers are the collapsed images of richer symbolic processes.

A generative identity is a triple
\[
G = (M, D, K),
\]
where the selector $M$ determines which layer contributes each symbol of the
canonical output, the sequence $D$ supplies classical digit information, and the
sequence $K$ carries additional meta-structure.  
The generative space $\mathcal{X}$ is the full product of these three layers,
equipped with the product topology.  This topology makes finite-prefix
agreement the basic notion of nearness and places the theory squarely in the
context of symbolic dynamics and represented spaces.

Chapter~1 develops this ontology.  
It defines $\mathcal{X}$, introduces the canonical output associated with each
identity, and isolates the effective core $\mathcal{G}_{\mathrm{eff}}$, the
subset consisting of computable generative identities.  This core plays the
role of a computable analogue of Baire space and forms the computational
foundation for all subsequent results.

Chapter~2 introduces the collapse map $\pi : \mathcal{X}^* \to [0,1]$, the
primary invariant of the framework.  Collapse discards nearly all of the
symbolic structure of $G$ and preserves only the selected digits, which it
interprets as a base-$b$ expansion.  The map is continuous, surjective, and
computably well-behaved: it maps the full generative space onto the unit
interval and maps the effective core precisely onto the computable real
numbers.

Chapter~3 analyzes the structure of collapse fibers
\[
\mathcal{F}(x) = \{\, G \in \mathcal{X}^* : \pi(G) = x \,\}.
\]
These fibers are closed, uncountable subsets of $\mathcal{X}$, and their
effective counterparts $\mathcal{F}_{\mathrm{eff}}(x)$ are nontrivial
$\Pi^0_1$ classes whenever $x$ is computable.  Each fiber contains a wide range
of internal behaviors: selectors of varying density, digit streams with
different unused coordinates, and a full spectrum of meta-layer structures.
This internal abundance forms the foundational motivation for the projection
theory and incompleteness results developed in later parts.

Part~I therefore establishes the ontological core of the generative framework:
the space of symbolic mechanisms, the collapse map that connects this space to
the classical continuum, and the geometric and effective properties of the
collapse fibers.  These ideas underpin the study of selector dynamics in
Part~II, the theory of structural projections in Part~III, and the diagonal
arguments leading to structural incompleteness in Part~IV.
