\part*{Summary of Part I: The Generative Ontology}

Part~I develops the foundational ontology of the generative framework.
The generative space $\mathcal{X}$ is introduced as a layered product space of
mixer, digit, and meta sequences, equipped with the product topology.  
Its effective subspace $\mathcal{G}_{\mathrm{eff}}$ consists of computable
generative identities and plays the role of a computable analogue of Baire
space, in line with the standard representation methods of Type--2
computability.

The central invariant of the framework is the collapse map $\pi$.  
Collapse extracts the subsequence of digits selected by the mixer and interprets
it as a base-$b$ expansion, thereby assigning a classical magnitude to a
generative identity.  
Chapter~2 shows that collapse is surjective onto the full unit interval and that
its restriction to the effective core is surjective onto the computable reals.

Chapter~3 examines the structure of collapse fibers
\[
\mathcal{F}(x)=\{G\in\mathcal{X}^* : \pi(G)=x\}.
\]
These fibers reveal the internal variability hidden beneath classical
magnitude: each real number corresponds to an uncountable family of mechanisms
in the full space and an infinite, effectively closed family in the effective
core.  
This geometric and descriptive complexity forms the backdrop for the hybrid and
null-density behaviors analyzed in Part~II.
