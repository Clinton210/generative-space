\chapter*{Part I Summary}

Part I develops the geometric foundations of the Generative Identity Framework.
The ambient generative space is defined as a compact product of discrete
symbolic coordinates. Each generative identity consists of three infinite
streams that evolve in parallel. The product topology equips this space with a
Cantor type structure that is totally disconnected and contains no isolated
points.

The collapse map reads a single coordinate of the generative identity and
produces a classical real number. The map is continuous and surjective. Each
real number corresponds to a collapse fiber that contains all identities that
produce the same canonical output. These fibers are compact, perfect, and
closed under finite modification. They contain identities with a wide range of
selector densities, gap growth patterns, and meta coordinate behavior, even
though all identities in the fiber agree on the exposed digits.

The properties of collapse fibers reveal the difference between generative
structure and classical magnitude. The collapsed value depends only on the
order of the exposed digits. Every remaining symbolic coordinate can vary
freely beyond any finite prefix without altering the collapsed value. This tail
freedom explains why many distinct generative identities correspond to the same
real number.

Part I also introduces coarse features of selector behavior, such as positive
density and null density. These coarse regimes appear densely in the ambient
space and inside every collapse fiber. They show that collapse imposes almost
no restriction on long range exposure behavior.

Together, these results establish the collapse geometry that supports the later
development of finite observers and structural incompleteness. Part I
demonstrates that the continuum is the projection of a richer symbolic space,
and that collapse conceals a large amount of structure that becomes central in
the next parts of the monograph.
