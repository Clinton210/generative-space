\chapter*{Part IV Summary}

Part IV establishes the central incompleteness phenomenon of the Generative
Identity Framework. The collapse map determines the classical real value of a
generative identity, yet it reveals only a small fraction of the symbolic
structure encoded in the selector, digit, and meta streams. This part shows
that no finite collection of continuous observers can recover the hidden
identity from its collapsed value.

The first chapter develops the alignment and sewing tools that operate inside
collapse fibers. Identities in the same fiber expose the canonical digits of
their collapsed value in the same order, even when the positions of exposure
differ. This shared output makes it possible to align selected digits and then
replace the tail of one identity with the tail of another without altering the
collapsed value. These methods ensure that agreement on a finite prefix can be
preserved while symbolic differences are introduced beyond the reach of all
observers.

The second chapter presents the mimicry construction. Given an effective
enumeration of computable structural projections, the construction builds a
computable identity that matches a reference identity on every prefix required
by the observers, while differing in its tail. Dependency bounds guarantee
that this prefix agreement forces all observers to assign nearly identical
values. The resulting identity is therefore indistinguishable from the
reference for every computable projection, yet it is symbolically distinct.

The final chapter proves the Structural Incompleteness Theorem. For any
computable real number $x$ and any finite family of computable observers, there
exist distinct identities in the effective collapse fiber
$\mathcal{F}_{\mathrm{eff}}(x)$ that produce identical observations for every
observer in the family. Observers fail to distinguish these identities because
their values depend only on finite prefixes, while the symbolic differences lie
entirely beyond those prefixes.

Part IV shows that generative structure is invisible to finite continuous
observation. This incompleteness follows from the topology of the generative
space and from the finite information inherent in continuous functionals. It
does not rely on probabilistic effects or approximation. The results establish
a precise and unavoidable limit on the reconstructive power of finite
observation.
