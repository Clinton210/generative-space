\chapter*{Part IV Summary}

Part IV interprets the classical continuum as a quotient of the generative
space under the collapse map. This viewpoint clarifies how real numbers arise
from generative identities and why collapse conceals the vast symbolic freedom
present in the ambient space.

The collapse equivalence relation identifies two identities whenever they
produce the same sequence of exposed digits. Each equivalence class is the
collapse fiber $\mathcal{F}(x)$ of a real number $x$. Earlier parts established
that these fibers are compact, perfect, and totally disconnected, with
unrestricted variation in selector and meta-information structure. Part IV
shows that when the entire generative space is quotiented by this relation, the
resulting space is homeomorphic to the real interval $[0,1]$.

This quotient perspective aligns naturally with the theory of represented
spaces in computable analysis. A computable identity in the effective fiber of
a computable real number serves as a computable name for that real. Conversely,
no member of the effective fiber of a noncomputable real can be computable.
Thus collapse fibers generalize the classical notion of names while revealing
the symbolic richness that is ignored by classical magnitude.

Part IV emphasizes that the continuum is the shadow of a much larger symbolic
space. Collapse extracts only the canonical digit sequence, leaving selector
behavior, gap geometry, and meta-information invisible. This interpretation
frames the limitations of finite observation in geometric terms and motivates
the extended invariants developed in Part V.
