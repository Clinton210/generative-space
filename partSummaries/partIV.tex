\part*{Summary of Part IV: Structural Incompleteness}

Part~IV develops the central impossibility results of the generative framework.
Chapter~8 constructs the meta-diagonalizer, an effective generative identity
designed to evade any finite family of computable secondary projections.  
The construction relies on the finite-prefix dependency bounds established in
Part~III: by matching a reference generator on all positions that observers can
inspect, and altering only the unobserved tail, the diagonalizer introduces
structural changes that no projection in the family can detect until it is too
late.  
Index alignment ensures that these tail modifications preserve the digit
subsequence encoding the classical value.

Chapter~9 applies this mechanism to prove the \emph{Structural Incompleteness
Theorem}.  
The theorem shows that no finite collection of computable secondary projections
can classify an effective collapse fiber or distinguish all effective
generators that represent the same real number.  
Collapse preserves classical magnitude, but every other computable invariant is
limited by finite lookahead.  
As a result, the internal structure of generative identities cannot be recovered
or fully described by any finite coordinate system.

Together, these chapters establish structural incompleteness as an intrinsic
feature of the generative space and reveal the profound information loss caused
by the collapse map.
