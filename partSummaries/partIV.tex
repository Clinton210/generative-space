\chapter*{Part IV Summary}

Part IV establishes the central incompleteness phenomenon of the Generative
Identity Framework. The collapse map determines the classical real value
associated with a generative identity, but it reveals only a small portion of
the symbolic structure encoded by the selector, digit, and meta streams. This
part shows that no finite collection of continuous observers can recover the
hidden generative identity from its collapsed value.

The first chapter develops the alignment and sewing tools that operate inside
collapse fibers. Identities in the same fiber expose the canonical digits of
their collapsed value in the same order, even when the positions of those
exposures differ. This shared output allows selected digits to be aligned,
after which the tail of one identity may be replaced with the tail of another
without affecting the collapsed value. These operations ensure that finite
prefix agreement can always be preserved while symbolic differences are
introduced beyond the reach of observers.

The second chapter presents the mimicry construction. Given an enumeration of
computable structural projections, the construction builds a computable
identity that agrees with a reference identity on all prefixes required by the
observers, yet differs from the reference in its tail. Dependency bounds
guarantee that this agreement on finite prefixes forces observers to assign
identical values, even though the two identities are structurally distinct.
This yields a computable identity that is observationally indistinguishable
from the reference while not being equal to it.

The final chapter proves the Structural Incompleteness Theorem. For any
computable real number $x$ and any finite family of computable continuous
observers, there exist distinct identities in the effective collapse fiber
$\mathcal{F}_{\mathrm{eff}}(x)$ that produce the same observations for every
observer in the family. Observers cannot distinguish these identities because
their measurements depend only on finite prefixes, while the symbolic
differences lie entirely beyond those prefixes.

Part IV therefore shows that generative structure is fundamentally invisible
to finite continuous observation. This incompleteness arises from the topology
of the generative space and the finite information inherent in continuous
functionals, not from randomness or approximation.
