\part{Structural Incompleteness}

\chapter*{Summary of Part IV: Structural Incompleteness}
\addcontentsline{toc}{chapter}{Summary of Part IV: Structural Incompleteness}

Part~IV establishes the central impossibility results of the generative
framework.  The preceding part developed a rich theory of structural
projections—continuous and computable maps that extract partial information from
generative identities.  This part shows that such observations, no matter how
carefully designed or combined, are fundamentally insufficient for recovering
the full internal structure of an effective generator.  Collapse fibers are
simply too large, and computable observers too limited, for any finite family of
projections to classify them.

\medskip

\textbf{Chapter~9 constructs the meta-diagonalizer.}
Given a computable real $x$ and a reference identity $H \in
\mathcal{F}_{\mathrm{eff}}(x)$, the meta-diagonalizer produces an effective
identity $G^*$ that matches $H$ on every prefix that any projection in a finite
family can observe, while diverging in its tail structure in a controlled,
fiber-preserving way.  
The construction uses three core ingredients:
\begin{itemize}
    \item the finite lookahead of computable projections (Part~III),
    \item uniform dependency bounds for finite families of observers,
    \item a sewing procedure that aligns digit-selection indices to preserve
          classical magnitude.
\end{itemize}
The result is a generator that is observationally identical to $H$ for all
computable observers in the family but structurally distinct in ways they
cannot detect.

\medskip

\textbf{Chapter~10 proves the Structural Incompleteness Theorem.}
For any computable real $x$ and any finite collection of computable structural
projections, there exist two distinct effective identities in the collapse fiber
$\mathcal{F}_{\mathrm{eff}}(x)$ that no projection in the family can
distinguish.  
Equivalently, no finite computable coordinate system can classify the effective
fiber of any computable real number.  
Magnitude $\pi(G)$ is therefore the only invariant that fully survives collapse;
every other computable invariant is necessarily partial.

\medskip

Together, Chapters~9 and~10 show that collapse fibers cannot be compressed into
finite lists of structural coordinates.  The internal geometry of an effective
generator always contains infinitely many degrees of freedom invisible to any
finite observational horizon.  
This result forms the conceptual bridge to Part~V, where the real line is
reinterpreted as a quotient space arising from collapse, and the continuum is
understood as a lossy image of a much richer symbolic manifold.
