\chapter*{Part V Summary}

Part V develops the quotient perspective that connects the infinite
dimensional generative space to the classical continuum. The collapse map
reads the digits exposed by the selector stream of a generative identity and
interprets them as a real number in base $b$. Although the generative space
contains extensive symbolic structure, the collapse map identifies many
distinct identities and assigns them the same classical value.

The equivalence classes of the collapse map are the collapse fibers. Each
fiber is a closed subset of the ambient compact product space $\mathcal{X}$,
and is therefore compact, perfect, and totally disconnected. These fibers
contain identities with a wide range of selector behaviors, including positive
density, zero density, regular spacing, and large irregular gaps. These forms
of variation are invisible to the collapse mechanism but play central roles in
observer behavior and in the incompleteness phenomena established in Part IV.

The quotient of $\mathcal{X}^{*}$ by collapse equivalence is homeomorphic to
the interval $[0,1]$. This parallels the viewpoint of represented spaces in
computable analysis, where classical mathematical objects are understood as
equivalence classes of symbolic descriptions. In this framework, each real
number corresponds not to a single canonical identity but to an entire fiber
of generative representations.

Part V shows that the classical continuum is a coarse image of a symbolic
space with substantial internal structure. The geometry of collapse fibers, and
the freedom of selector behavior within them, prepares the way for Part VI,
where extended invariants are used to analyze generative identities through
large scale numerical and geometric coordinates.
