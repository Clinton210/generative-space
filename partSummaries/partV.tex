\chapter*{Part V Summary}

Part V develops the quotient perspective that connects the infinite
dimensional generative space to the classical continuum. The collapse map
reads the digits exposed by the selector stream of a generative identity and
interprets them as a real number in base $b$. Although the generative space
contains extensive symbolic structure, the collapse map identifies many
distinct identities and assigns them the same classical value.

The equivalence classes of the collapse map are the collapse fibers. Each
fiber is a closed subset of the ambient compact product space $\mathcal{X}$,
hence compact, perfect, and totally disconnected. These fibers contain
identities with a wide range of selector behaviors, including positive
density, zero density, regular spacing, and large irregular gaps. These
structural differences are invisible to collapse but play essential roles in
the behavior of observers and in the incompleteness phenomena established in
Part IV.

The quotient of $\mathcal{X}^{*}$ by collapse equivalence is naturally
homeomorphic to the interval $[0,1]$. This parallels the viewpoint of
represented spaces in computable analysis, where classical objects are treated
as equivalence classes of symbolic descriptions. In this setting, each real
number corresponds to the entire fiber of its generative representations, not
to a single canonical identity.

Part V therefore shows that the classical continuum is a coarse image of a
rich symbolic space. The structure of collapse fibers, together with the
freedom of selector behavior within them, prepares the ground for Part VI,
where extended invariants are used to organize and analyze generative
identities through large scale numerical and geometric coordinates.
