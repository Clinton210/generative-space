\chapter*{Part V Summary}

Part V develops the extended invariants that describe large scale selector
behavior beyond the reach of collapse and continuous observation. These
invariants capture asymptotic structure that is invisible to any finite prefix
and therefore remain undetected by all structural projections introduced in
Part II.

Two invariants form the core of the analysis. The asymptotic density
\[
\eta(G)
=
\liminf_{N\to\infty}
\frac{1}{N}
\sum_{n<N} \chi_M(n)
\]
measures the long term frequency of digit exposure, while the fluctuation index
\[
\phi(G)
=
\limsup_{j\to\infty}
\frac{g_j}{n_j}
\]
describes the relative scale of successive gaps between selected positions.
Both invariants depend only on the tail of the selector stream and are
unchanged by any finite modification. Their behavior reflects global patterns
of sparsity, regularity, growth, and fluctuation.

A key finding of Part V is that both $\eta$ and $\phi$ are everywhere
discontinuous in the product topology. Every nonempty open set contains
identities realizing every admissible value of both invariants. This maximal
discontinuity arises because finite-prefix topology constrains only limited
local information, while the invariants measure long range behavior. The result
illustrates a fundamental separation between continuity and asymptotic
structure.

Part V also shows that collapse fibers contain identities with arbitrary
asymptotic behavior. Given any real number $x$, the fiber $\mathcal{F}(x)$
contains identities with every density value $\alpha \in [0,1]$ and every
fluctuation value $\beta \in [0,\infty]$. Tail freedom inside fibers enables
selectors to vary without affecting collapsed magnitude.

Finally, the slice geometry developed in Part V positions vertical slices
(finite prefixes), horizontal slices (invariant level sets), and fiber slices
(collapsed magnitude) within a unified geometric picture. These slices
demonstrate that finite-prefix structure, asymptotic selector behavior, and
collapsed value act as independent coordinates in the generative space.

Part V therefore reveals the asymptotic richness hidden by collapse and
illustrates why classical magnitude cannot recover the global geometry of the
selector. This prepares the ground for the broader geometric and analytic
interpretation developed in Part VI.
