\part*{Summary of Part V: Synthesis and Outlook}

Part~V situates the generative framework within the landscape of classical
analysis and outlines several directions for further development.  
Chapter~10 presents the continuum as a collapse quotient.  
The collapse map identifies classical magnitude while discarding the internal
structure of each generative identity; classical real numbers correspond to
equivalence classes of mechanisms, and classical analysis operates entirely on
this quotient.  
This viewpoint clarifies the relationship between generative and classical
representations and makes precise the information loss inherent in collapse.

Chapter~11 explores potential extensions of the theory.  
These include generative measure theory, shift-invariant and fiber measures,
operator-theoretic approaches to layer transformations, and higher-order mixer
architectures.  
Connections to computable analysis and symbolic dynamics suggest further
directions involving computability, complexity, and dynamical invariants.  
These outlooks indicate how the generative viewpoint may interact with, and
possibly enrich, broader areas of mathematics.

Together, Part~V positions the generative framework as a foundation for future
work on symbolic mechanisms, measurement, and structural representation beneath
classical magnitude.
