\chapter*{Part V Summary}

Part V develops the quotient perspective that connects the generative space to
the classical continuum.  
The collapse map sends each generative identity to a real number by selecting
and interpreting the digits exposed by its selector stream.  
Although the generative space is infinite dimensional, the collapse map
identifies many distinct identities and assigns them a single classical value.

The resulting equivalence classes are the collapse fibers.  
Each fiber is a compact, perfect, and totally disconnected symbolic set.  
The fibers vary widely in their internal structure, containing identities with
selector streams of positive density, zero density, regular spacing, or large
irregular gaps.  
These structural differences are invisible to collapse but central to the
behavior of observers.

The quotient space $\mathcal{X}^*/{\sim}$ obtained by identifying identities
with the same collapsed value is homeomorphic to the interval $[0,1]$.  
This interpretation parallels the role of names in computable analysis and the
theory of represented spaces, where classical objects are obtained as
equivalence classes of symbolic descriptions.  
From this viewpoint, each real number corresponds to the entire fiber of its
generative representations.

Part V therefore shows that the classical continuum is a coarse image of a
much richer symbolic structure.  
The relation between fibers and extended invariants prepares the way for Part
VI, where selector behavior is examined through large scale numerical and
geometric coordinates.
