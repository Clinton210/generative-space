\part*{Summary of Part V: Synthesis and Outlook}

Part~V relates the generative framework to classical analysis and outlines directions for future research.  
Chapter~10 presents the continuum as a collapse quotient.  
The collapse map identifies classical magnitude while discarding the internal structure of each generative identity.  
Classical real numbers correspond to equivalence classes of mechanisms, and classical analysis operates entirely on this quotient.  
This perspective highlights the information loss induced by collapse and clarifies the relationship between generative and classical viewpoints.

Chapter~11 discusses extensions of the theory.  
Possible developments include generative measure theory, shift-invariant and fiber measures, operator-theoretic perspectives on layer transformations, and higher-order mixers and meta layers.  
Connections to computable analysis and symbolic dynamics suggest several computability-theoretic and dynamical directions.  
These outlooks illustrate how the generative viewpoint may interact with broader areas of mathematics.

Together, Part~V positions the generative framework as a foundation for future work on symbolic mechanisms, measurement, and structural representation beneath classical magnitude.
