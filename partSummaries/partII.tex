\chapter*{Part II Summary}

Part II analyzes the behavior of selector streams, which determine when digits
of the digit stream are exposed and thus shape the symbolic structure of a
generative identity. The selector governs both the internal geometry of an
identity and the finite-information view available to continuous observers.

Two broad regimes of selector behavior are examined. Hybrid selectors expose
digits with positive asymptotic frequency, while null-density selectors expose
digits only sporadically, yet still infinitely often. Both regimes occur
densely in the generative space and in every collapse fiber. This shows that
collapse places essentially no constraint on how rapidly or irregularly digits
may be revealed.

The analysis in Part II emphasizes that selector patterns vary widely even
among identities sharing the same collapsed value. Within a single fiber one
finds identities with regular, evenly spaced exposures, as well as identities
with extreme sparsity or highly irregular gap growth. These differences are
structural: they persist regardless of how the digit or meta streams behave,
and they are invisible to classical magnitude.

This structural diversity motivates the central themes of Parts III and IV.
Continuous observers examine only finite prefixes, and selector behavior
demonstrates how much long-term structure can lie beyond finite
observational reach. Part II therefore lays the groundwork for projection
theory and for the incompleteness phenomena that arise when observers attempt
to measure identities using only finite information.
