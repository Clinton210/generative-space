\chapter*{Part II Summary}

Part II develops the observational layer of the Generative Identity Framework.
Where Part I described the collapse geometry of the generative space, Part II
examines what continuous observers can detect about the internal structure of a
generative identity.

A structural projection is a continuous real valued functional on the
digit–producing generative space. Results from Type~2 Effectivity show that
every such observer has a computable dependency bound. This bound specifies how
many initial coordinates the observer must read to approximate its value to a
given precision. Dependency bounds express the finite information principle: at
each precision level, an observer sees only a finite prefix of the generative
identity.

From dependency bounds arise two key consequences. The first is prefix
stabilization: once two identities agree on a sufficiently long prefix, every
observer evaluates them within the desired error. The second is stability under
tail modification: observers are unaffected by any changes to the tail once the
required prefix has been fixed.

Part II also introduces projective incompatibility. Different observers may
require incompatible patterns inside their finite dependency windows. No single
finite prefix can satisfy all incompatible finite demands at once. This
finite-prefix obstruction plays a fundamental role in later alignment and
diagonalization arguments.

Together, these results define the geometry of continuous observation. Finite
prefixes determine all observable behavior, while the symbolic tail remains
invisible to every structural projection. This observational asymmetry is the
foundation for the incompleteness and indistinguishability phenomena developed
in Part~III.
