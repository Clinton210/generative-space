\chapter*{Part II Summary}

Part II examines the behavior of selector streams, which play a central role
in the generative identity.  
The selector determines which digits of the digit stream contribute to the
canonical output and therefore shapes both the internal structure of an
identity and its interaction with observers.

Two fundamental regimes of selector behavior are analyzed.  
Hybrid selectors expose digits with positive asymptotic density, while
null density selectors expose digits at vanishing density but still do so
infinitely often.  
Both regimes occur densely in the generative space, and both appear inside
every collapse fiber.  
This shows that the collapse operation imposes almost no constraint on the
rate at which digits are revealed.

Selector diversity inside collapse fibers underscores a key theme of the
framework.  
Identities that collapse to the same real number may differ widely in how
their canonical digits are exposed.  
Some identities reveal digits frequently and regularly, while others reveal
them sparsely or with large irregular gaps.  
This structural variability motivates the question of how much information can
be extracted from a generative identity by continuous observers.

Part II provides a detailed description of selector regimes and prepares the
ground for projection theory in Part III, where continuous observers are used
to measure and compare generative identities.
