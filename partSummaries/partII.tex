\chapter*{Part II Summary}

Part II examines the geometry of selector streams, which determine when digits
of the digit stream are exposed and which therefore shape the symbolic
structure of every generative identity. The selector controls both the internal
pattern of revealed digits and the finite information available to continuous
observers.

Two principal regimes of selector behavior appear throughout the generative
space. Hybrid selectors expose digits with positive asymptotic frequency.
Null-density selectors expose digits only sparsely, yet still infinitely many
times. Both regimes occur densely in the ambient space and in every collapse
fiber. Collapse therefore imposes almost no restrictions on the long term
behavior of selectors. Even identities that represent the same real number may
expose their canonical digits at very different rates and with sharply
different patterns.

Part II emphasizes that selector variation is substantial and structurally
meaningful. Some identities have regular spacing and mild fluctuation. Others
exhibit rapid bursts of exposure followed by long silent intervals. Still
others mix dense and sparse behavior in alternating blocks. These differences
are invisible to classical magnitude, but they play a central role in the
operation of observers that depend only on finite prefixes.

This diversity of selector behavior prepares the ground for Parts III and IV.
Continuous observers have computable dependency bounds and therefore read only
finitely many symbols at any fixed precision. Selector geometry illustrates
how much long term structure lies beyond this finite observational reach. Part
II provides the symbolic and geometric context needed for projection theory
and for the incompleteness results that follow.
