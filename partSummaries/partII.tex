\part{Selector Dynamics}

\chapter*{Summary of Part II: Selector Dynamics}
\addcontentsline{toc}{chapter}{Summary of Part II: Selector Dynamics}

Part~II investigates the internal behaviors exhibited by generative identities
through the long-term structure of their selectors.  The selector $M$ determines
which layer contributes each symbol to the canonical output.  Its asymptotic
pattern governs how information from the digit and meta layers is interwoven
and thereby shapes the mechanism underlying a classical real number.

The central theme of this part is the classification of selector regimes.
Two extremes illustrate the range of behaviors supported within a single collapse
fiber.

\begin{itemize}
    \item \textbf{Hybrid identities} have positive digit-selection density.
    Their canonical outputs draw substantially from both the digit and meta
    layers.  Hybrids form a dense subset of the generative space and are
    algorithmically universal: every computable real number can be generated by
    an effective hybrid identity.

    \item \textbf{Null-density identities} select digits infinitely often but
    with asymptotic density zero.  Their canonical outputs are dominated by the
    meta layer, yet the sparse digit positions still encode the exact classical
    magnitude.  Null-density mechanisms demonstrate that collapse does not
    depend on selector density and that magnitude can be embedded in extremely
    thin symbolic structures.
\end{itemize}

Chapter~4 develops the hybrid regime.  
It introduces digit-selection density, proves that hybrid identities are
topologically generic in $\mathcal{X}$, and shows that every effective collapse
fiber contains infinitely many hybrid generators.  This density of hybrid
structure clarifies how classical magnitude can arise from mechanisms with
substantial internal interaction between layers.

Chapter~5 develops the complementary null-density regime.  
It formalizes sparse selection patterns, proves the existence of effective
null-density generators for every computable real number, and examines their
dynamical stability under the shift.  These generators reveal a contrasting,
meta-dominated geometry within each collapse fiber.

Together, hybrid and null-density identities provide the fundamental selector
geometries that anchor the study of internal structure.  
Their asymptotic behaviors motivate the development of structural projections
in Part~III, where we analyze how much of this internal variation can be
observed through computable coordinate systems, and how much necessarily
remains invisible.
