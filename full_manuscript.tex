\documentclass[11pt,openany]{book}


\usepackage{geometry}
\geometry{margin=1in}

\usepackage{amsmath, amssymb, amsthm, mathtools}

\usepackage{graphicx}
\usepackage{bm}
\usepackage{enumerate}

\usepackage[hidelinks]{hyperref}

\usepackage{silence}
\WarningFilter{latex}{Reference}
\WarningFilter{latex}{There were undefined references}


\usepackage{cite}


\theoremstyle{plain}
\newtheorem{theorem}{Theorem}[chapter]
\newtheorem{proposition}{Proposition}[chapter]
\newtheorem{corollary}{Corollary}[chapter]
\newtheorem{lemma}{Lemma}[chapter]

\theoremstyle{definition}
\newtheorem{definition}{Definition}[chapter]
\newtheorem{remark}{Remark}[chapter]
\newtheorem{example}{Example}[chapter]


\begin{document}

\frontmatter

\begin{titlepage}
    \centering
    \vspace*{2cm}
    {\Huge\bfseries The Generative Identity Framework\par}
    \vspace{1.5cm}
    {\Large Clinton Potter\par}
    \vfill
    {\large \today\par}
\end{titlepage}

\newenvironment{abstract}{
    \cleardoublepage
    \thispagestyle{plain}
    \begin{center}
        {\Large\bfseries Abstract}
    \end{center}
    \begingroup
}{\endgroup
    \cleardoublepage
}


\clearpage{}\begin{abstract}
This monograph develops the Generative Identity Framework, a structural approach
to real numbers based on symbolic generative mechanisms.  
A generative identity is a triple $(M, D, K)$ of infinite sequences: a selector
stream, a digit stream, and a meta-information stream.  
The classical real number associated with an identity is obtained by a
continuous collapse map that reads only the digits exposed by the selector.
Collapse is surjective and highly non-injective, and each collapsed value $x$
corresponds to a large symbolic fiber $\mathcal{F}(x)$ containing many
generative identities.

We study the internal structure of these fibers through continuous
observers.  
A structural projection is any continuous real-valued functional on the
generative space, and its dependence on finite prefixes is controlled by
computable dependency bounds in the sense of Type-2 Effectivity.  
These bounds imply prefix stabilization and tail invariance, which together
give a finite-information description of all continuous observers.

Using these tools, we construct a computable identity inside the effective
fiber of a computable real $x$ that agrees with a reference identity on
arbitrarily long prefixes, yet diverges along every computable structural
projection.  
This diagonalizer yields the Structural Incompleteness Theorem: no finite
family of continuous observers, even when combined with the collapsed value,
can recover the generative identity.  
Finite observation cannot capture the symbolic structure hidden beneath
collapse.

Finally, we introduce extended invariants that measure large-scale selector
behavior.  
The entropy balance $\eta$ (lower asymptotic density of digit exposures) is
lower semicontinuous, and the fluctuation index $\phi$ (relative gap growth)
is upper semicontinuous.  
Although discontinuous, these invariants provide coarse geometric embeddings
of generative identities and illustrate the diversity that persists inside
each collapse fiber.

The framework offers a unified structural, computational, and geometric view
of real numbers, revealing the continuum as a quotient of a rich symbolic
space and exposing intrinsic limits on what any finite process can observe.
\end{abstract}
\clearpage{}
\clearpage{}\chapter*{Acknowledgments}
\addcontentsline{toc}{chapter}{Acknowledgments}

The ideas developed in this monograph grew out of long periods of independent study and reflection that predate my formal training in mathematics.  
My academic background is in Industrial and Organizational Psychology, and I am completing an undergraduate degree in mathematics.  
The earliest versions of the concepts that eventually became the generative framework arose from efforts to understand how symbolic sequences can combine ordered and stochastic behavior.  
These intuitions matured into the program-based architecture presented here.

I made extensive use of contemporary AI systems during the preparation of this manuscript.  
These systems assisted with drafting, restructuring, and checking the exposition, and they helped convert informal ideas and partial sketches into precise mathematical statements.  
All conceptual advances, definitions, and theorems in this work originate with the author, and the responsibility for correctness lies entirely with me.

I am grateful to my family and friends for their patience, encouragement, and support during the development of this project.  
Their confidence made this work possible.
\clearpage{}

\tableofcontents

\chapter*{Prelude}

Real numbers are usually described by their magnitudes and by the symbolic
expansions that represent them.  
This monograph develops a different perspective.  
Instead of viewing a real number as a static point on the continuum, we regard
it as the collapsed output of a symbolic generative mechanism.  
Such a mechanism consists of a selector stream, a digit stream, and a
meta-information stream, all evolving in parallel.  
Only a small portion of this structure survives the collapse to a classical
real value.  
The remainder forms an extensive symbolic landscape that is invisible to
classical analysis.

The guiding idea of the Generative Identity Framework is that classical
magnitude hides substantial internal structure.  
A real number can have many generative identities, all of which produce the
same digit sequence under collapse but differ in how those digits are exposed,
how gaps are distributed, and what symbolic information is carried in
unobserved layers.  
These differences do not affect the collapsed value, yet they play a central
role in the behavior of observers that act on the generative representation.

The first part of the monograph introduces the generative space, the collapse
map, and the geometry of collapse fibers.  
Each fiber contains many identities that collapse to the same real number.
These identities may have selector streams of positive density, zero density,
or highly irregular structure.  
The fiber therefore records a large amount of structure that collapse cannot
recover.

The second and third parts develop projection theory.  
A structural projection is a continuous observer that assigns a real value to a
generative identity based only on finite symbolic information.  
Dependency bounds formalize this finite information principle, and prefix
stabilization shows that observers eventually ignore the tail of the identity
at any fixed precision.  
These tools provide a precise way to analyze the limits of finite observation.

Part IV presents the central technical result.  
A meta-diagonalizer is constructed inside any effective collapse fiber.  
This identity agrees with a reference identity on arbitrarily long prefixes,
yet diverges from it along every computable structural projection.  
The resulting Structural Incompleteness Theorem states that no finite family
of observers, even when combined with the collapsed value, can recover the
generative identity.  
Finite observation is inherently limited by the topology of the generative
space.

Part V describes the continuum as a quotient of the generative space under
collapse.  
This quotient view clarifies why collapse conceals nearly all symbolic
structure.  
It also connects the framework to classical ideas in computable analysis and
represented spaces, where names of real numbers form equivalence classes under
continuous maps.

Part VI introduces extended invariants that measure large scale features of
the selector stream.  
The entropy balance and fluctuation index detect long term frequency and gap
behavior.  
These invariants are discontinuous but semicontinuous, and they give rise to
natural geometric embeddings of generative identities.  
Such embeddings reveal the diversity of selector behavior inside each collapse
fiber and illustrate the breadth of structure hidden beneath classical
magnitude.

The Generative Identity Framework unifies symbolic, computational, and
geometric viewpoints on real numbers.  
It shows that real numbers are not merely magnitudes but are shadows of
complex symbolic identities.  
This perspective opens many directions for further research, including higher
order invariants, geometric embeddings, connections to symbolic dynamics, and
interactions with randomness and computability.

The chapters that follow develop these ideas systematically, beginning with
the foundations of the generative space and culminating in the structural
incompleteness of finite observation.
 
\mainmatter

\chapter*{Part I Summary}

Part I introduces the symbolic foundations of the Generative Identity
Framework.  
A generative identity is defined as a triple of infinite sequences
$(M, D, K)$ consisting of a selector stream, a digit stream, and a
meta-information stream.  
These sequences form a product space $\mathcal{X}$ equipped with the product
topology, and the digit selecting subspace $\mathcal{X}^*$ contains those
identities that expose infinitely many digits.

The collapse map extracts the classical real value associated with a
generative identity by selecting the digits exposed by $M$ and interpreting
them in base $b$.  
This map is continuous and surjective.  
Its fibers are compact, perfect, and totally disconnected, and they contain
many identities that differ dramatically in their internal structure while
producing the same collapsed value.

The geometry of collapse fibers is the first indication that classical
magnitude conceals substantial symbolic structure.  
Fibers contain identities with dense or sparse selectors, identities with
regular or irregular spacing patterns, and identities with arbitrary
meta-information streams.  
These degrees of freedom motivate the study of how much structure can be
detected by continuous observers, which becomes the focus of Part III and the
incompleteness phenomena of Part IV.

Part I therefore provides the symbolic setting, the collapse mechanism, and
the foundational geometry that underpins the entire monograph.
 \clearpage{}\chapter{The Generative Space}

\section{Introduction}

The Generative Identity Framework begins by treating real numbers not as
primitive points on the continuum, but as the collapsed shadows of richer
symbolic mechanisms.  
A \emph{generative identity} consists of three infinite sequences working in
parallel: a selector stream, a digit stream, and a meta-information stream.
Only fragments of these sequences determine the classical real number; the
remainder encode additional structure that becomes invisible after collapse.

The purpose of this chapter is to formally describe the ambient space in
which these identities live.  
We define the generative space as a Cantor-like product of symbolic layers,
introduce its effective (computable) core, and establish the topological
principles that underlie collapse, reconstruction, and structural
incompleteness.

Throughout, we fix a base $b \ge 2$ for numeral expansion, and we assume
$\Sigma$ is a finite meta-alphabet.

\section{Definition of the Generative Space}

A generative identity is a triple
\[
G = (M, D, K),
\]
where:
\begin{itemize}
    \item $M \in \{D,K\}^{\mathbb{N}}$ is the \emph{selector stream},
          indicating at each position whether the mechanism exposes a digit
          or a meta-symbol;

    \item $D \in \{0,1,\ldots,b-1\}^{\mathbb{N}}$ is the \emph{digit stream},
          an infinite reservoir from which classical digits are selected when
          $M(n) = D$;

    \item $K \in \Sigma^{\mathbb{N}}$ is the \emph{meta-information stream},
          carrying auxiliary symbolic structure not visible to the classical
          collapse.
\end{itemize}

Each coordinate is a sequence over a finite alphabet equipped with the
discrete topology.  
The generative space is the product
\[
\mathcal{X}
  = \{D,K\}^{\mathbb{N}}
    \times \{0,1,\ldots,b-1\}^{\mathbb{N}}
    \times \Sigma^{\mathbb{N}},
\]
endowed with the product (Cantor) topology.  
Basic open sets are determined by finite prefixes of the three streams.

This topology reflects the principle that every observation of a generative
identity accesses only finitely many symbols from each layer.

\section{The Canonical Output}

Although a generative identity contains three infinite sequences, only the
selector and digit layers contribute to the production of the classical digit
sequence.  
Define the \emph{canonical output} of $G$ as the infinite sequence
\[
X(G) = (d_G(j))_{j=0}^\infty,
\]
where $d_G(j)$ is the $j$th digit encountered among the positions $n$ with
$M(n) = D$, read in order.

Formally, let
\[
n_0 < n_1 < n_2 < \cdots
\]
be the increasing sequence of indices at which $M(n_k) = D$.  
Then
\[
d_G(j) = D(n_j).
\]

If $M$ selects digits only finitely often, the canonical output is finite.
Since classical real numbers require infinite expansions, we restrict our
attention to a natural subspace.

\section{The Digit-Selecting Subspace}

Define the \emph{digit-selecting subspace}
\[
\mathcal{X}^*
  = \{\, G \in \mathcal{X} : M \text{ selects } D \text{ infinitely often}\,\}.
\]

This subspace is closed under finite modifications and is topologically large
within $\mathcal{X}$.  
Every element of $\mathcal{X}^*$ yields an infinite canonical output sequence
and therefore a well-defined classical real number after collapse.

\section{The Effective Core}

The framework distinguishes between arbitrary symbolic identities and those
that are computably generated.  
A generative identity $G = (M,D,K)$ is \emph{computable} if each of the
streams $M$, $D$, and $K$ is a computable function
$\mathbb{N} \to \{D,K\}$, $\mathbb{N} \to \{0,\ldots,b-1\}$,
and $\mathbb{N} \to \Sigma$, respectively.

The \emph{effective core} of the generative space is the set
\[
\mathcal{G}_{\mathrm{eff}}
  = \{\, G \in \mathcal{X} : M,D,K \text{ are computable}\,\}.
\]

This subset plays a central role in the diagonalization and incompleteness
results developed later.  
It forms the computational analogue of the ambient space $\mathcal{X}$ and is
countable in contrast to the uncountable full product.

\section{Worked Examples}

Although the space $\mathcal{X}$ is infinite-dimensional, simple examples
illustrate the fundamental ideas.

\subsection*{Example 1: Alternating Selector}

Let $M$ alternate deterministically:
\[
M = D,K,D,K,D,K,\ldots,
\]
and let $D$ be the digit expansion of a real number $x$ in base $b$ repeated
infinitely, while $K$ carries arbitrary meta-symbols.

Then:
\begin{itemize}
    \item the canonical output $X(G)$ contains every other digit of $D$,  
    \item the collapse $\pi(G)$ produces a real number whose expansion consists
          of the even-indexed digits of $x$.
\end{itemize}

Different choices of the meta-layer $K$ yield distinct generative identities,
all collapsing to the same classical value.

\subsection*{Example 2: Null-Density Selector}

Fix a sequence of perfect squares $1,4,9,16,\ldots$ and define
\[
M(n) =
\begin{cases}
D & \text{if } n \text{ is a perfect square},\\
K & \text{otherwise}.
\end{cases}
\]

The selector exposes digit positions with asymptotic density~$0$.  
The canonical output still produces an infinite digit sequence, but only at a
slowly growing rate.  
This identity collapses to the same real number as the sequence of selected
digits, despite its extremely sparse structure.

\section{Summary}

The generative space $\mathcal{X}$ is a symbolic product space rich enough to
encode both the visible and invisible structure of real numbers.  
Its effective core $\mathcal{G}_{\mathrm{eff}}$ provides a computationally
tractable subspace with deep descriptive complexity.  
Every generative identity in $\mathcal{X}^*$ yields a canonical output and,
through it, a classical real number.

In the next chapter, we define the collapse map that translates these
identities into points of the continuum, initiating the central dichotomy
between internal structure and classical magnitude.
\clearpage{}
\clearpage{}\chapter{The Collapse Map}

\section{Introduction}

A generative identity contains far more symbolic structure than is visible in
its classical magnitude.  
The collapse map extracts a real number from a generative identity by reading
only the digits exposed by the selector stream.  
This operation forgets almost all of the internal generative behavior,
producing a single value in $[0,1]$ while leaving behind a large fiber of
distinct identities sharing the same classical output.

This chapter defines the collapse map, establishes its continuity, and shows
that every real number---computable or otherwise---arises as the collapse of
many different generative identities.

\section{Digit Selection}

Let $G = (M, D, K) \in \mathcal{X}^*$ be a digit-selecting generative identity.
Recall that the canonical output sequence is defined by enumerating the digits
appearing at positions where $M(n) = D$.

Let
\[
n_0 < n_1 < n_2 < \cdots
\]
be the increasing sequence of indices with $M(n_j) = D$, and define
\[
d_G(j) = D(n_j).
\]

The sequence $(d_G(j))_{j \ge 0}$ is an infinite sequence in
$\{0,1,\ldots,b-1\}^{\mathbb{N}}$ and will serve as the base-$b$ expansion of
the collapsed value.

\section{Definition of the Collapse Map}

For every $G \in \mathcal{X}^*$, define the \emph{collapse map}
\[
\pi(G)
  = \sum_{j=0}^{\infty} \frac{d_G(j)}{b^{j+1}}.
\]

When a real number has two base-$b$ expansions (a terminating expansion and a
repeating one), we adopt the standard convention of using the non-terminating
representation with trailing $(b-1)$s avoided.  
This ensures the collapse map is well defined.

The collapse map is the primary projection from the generative space to the
unit interval.  
It depends only on the canonical output and therefore only on the portions of
the digit stream selected by $M$.

\section{Continuity of Collapse}

The topology on $\mathcal{X}^*$ makes $\pi$ a continuous function onto
$[0,1]$.  
Given $\varepsilon > 0$, choosing $N$ large enough so that
$b^{-(N+1)} < \varepsilon$ shows that the first $N$ selected digits determine
$\pi(G)$ to within $\varepsilon$.

Since selected digits appear infinitely often, the first $N$ of them arise
within some initial prefix of $G$.  
Thus, for every $\varepsilon > 0$, there exists an integer $L$ such that any
two identities agreeing on their first $L$ symbols in each stream have
collapsed values within $\varepsilon$.

Therefore \( \pi : \mathcal{X}^* \to [0,1] \) is continuous.

\section{Surjectivity}

Every real number in $[0,1]$ arises as the collapse of many generative
identities.  
Fix any real number $x$ with base-$b$ expansion
\[
x = \sum_{j=0}^{\infty} \frac{x_j}{b^{j+1}}.
\]

Choose a selector $M$ that always selects digits:
\[
M(n) = D \quad \text{for all } n.
\]
Define the digit stream $D$ by $D(n) = x_n$ for all $n$, and let $K$ be any
meta-information sequence.

Then $G = (M, D, K) \in \mathcal{X}^*$ satisfies $\pi(G) = x$.  
Varying $K$ freely shows that the fiber $\pi^{-1}(\{x\})$ is uncountable.

\section{Effective Surjectivity}

The collapse map behaves correctly on the effective core.  
A real number $x \in [0,1]$ is computable if and only if it has a computable
base-$b$ expansion.  
Given such an expansion, the construction above produces a computable
generative identity $G \in \mathcal{G}_{\mathrm{eff}}$ satisfying
$\pi(G) = x$.

Conversely, if $G \in \mathcal{G}_{\mathrm{eff}} \cap \mathcal{X}^*$, then the
canonical output sequence $d_G(j)$ is computable, and so $\pi(G)$ is a
computable real.

Thus
\[
\pi(\mathcal{G}_{\mathrm{eff}} \cap \mathcal{X}^*)
  = \mathbb{R}_c,
\]
the set of computable reals.

\section{Fibers and Structural Redundancy}

The collapse map is many-to-one.  
For any $x \in [0,1]$, the fiber
\[
\mathcal{F}(x) = \pi^{-1}(\{x\})
\]
contains identities that may share no structural similarity beyond producing
the same output digits.

Two identities may:
\begin{itemize}
    \item select digits at completely different positions,
    \item carry unrelated meta-information streams,
    \item differ arbitrarily on unselected digits,
\end{itemize}
while still collapsing to the same real $x$.

This structural redundancy is essential for the development of the projection
theory and the incompleteness results of later parts.

\section{Summary}

The collapse map converts the symbolic structure of a generative identity into
a classical real number by selecting and aggregating digits according to the
selector stream.  
It is continuous, surjective, effectively surjective on computable identities,
and massively non-injective.  
The fibers of $\pi$ form the central objects of study in the Generative
Identity Framework.

The next chapter analyzes the internal geometry of these fibers and the
degrees of freedom that remain invisible after collapse.
\clearpage{}
\clearpage{}\chapter{Fiber Geometry and the Effective Core}

\section{Introduction}

The collapse map $\pi : \mathcal{X}^* \to [0,1]$ associates to each
digit-selecting generative identity a classical real number.  
The purpose of this chapter is to study the structure of the fibers
\[
\mathcal{F}(x) = \pi^{-1}(\{x\})
\]
and, in particular, their topological and effective properties.

A fiber contains all identities that produce the same classical value.  
The collapse map obliterates most of the internal generative information,
leaving behind a large space of identities sharing a single output.  
Understanding this space is essential: the size and shape of fibers drive the
projection theory of Part~III and the incompleteness phenomena of Part~IV.

We show that each fiber is closed, uncountable, highly redundant, and
topologically rich.  
We then identify the \emph{effective fiber} and prove that it forms a
nonempty $\Pi^0_1$ class.  
This formalizes the computational structure of the identities surviving
collapse.

\section{Fibers of the Collapse Map}

Fix $x \in [0,1]$ and consider the set
\[
\mathcal{F}(x) = \{\, G \in \mathcal{X}^* : \pi(G) = x \,\}.
\]
Two identities $G = (M,D,K)$ and $H = (M',D',K')$ lie in the same fiber if
their canonical outputs agree:
\[
X(G) = X(H) = (x_j)_{j \ge 0},
\]
the base-$b$ expansion of $x$ we have fixed by convention.

Beyond these selected digits, the identities may differ arbitrarily.

\subsection*{Closedness}

Because $\pi$ is continuous and singletons $\{x\}$ are closed in $[0,1]$, each
fiber is a closed subset of $\mathcal{X}^*$.  
Fibers are therefore compact with respect to the product topology.

\subsection*{Degrees of freedom}

Let $(x_j)_{j \ge 0}$ denote the chosen expansion of $x$.  
A generative identity $G$ belongs to $\mathcal{F}(x)$ precisely when:
\begin{enumerate}
    \item[(i)] the selector $M$ chooses infinitely many positions;
    \item[(ii)] the digit stream $D$ satisfies
          $D(n_j) = x_j$ at the selected positions;
    \item[(iii)] the meta stream $K$ is arbitrary.
\end{enumerate}

Between successive selected positions, the identity may place any digit.  
Similarly, at each index where $M(n) = K$, the meta-symbol $K(n)$ is free.
Thus
\[
\mathcal{F}(x)
  \cong \{D,K\}^{\mathbb{N}} \times \{0,\ldots,b-1\}^{\mathbb{N}}
        \times \Sigma^{\mathbb{N}}
\]
subject to a countable collection of fixed coordinates enforcing
$D(n_j) = x_j$.  
The remaining coordinates form a product of Cantor spaces.
Hence every fiber is uncountable, totally disconnected, and perfect.

\section{The Effective Fiber}

We now restrict attention to the effective core.  
For $x \in [0,1]$ define the \emph{effective fiber}
\[
\mathcal{F}_{\mathrm{eff}}(x)
  = \mathcal{F}(x) \cap \mathcal{G}_{\mathrm{eff}}.
\]

In this section we show that $\mathcal{F}_{\mathrm{eff}}(x)$ is a nonempty
$\Pi^0_1$ class.  
This places the effective fibers squarely within the classical hierarchy of
computably presented closed sets in Cantor space.

\subsection{Nonemptiness}

If $x$ is computable, it has a computable expansion $(x_j)$, and the identity
that selects all digits and sets $D(j)=x_j$ lies in $\mathcal{F}_{\mathrm{eff}}(x)$.  
More generally, any computable selector that exposes infinitely many positions
can be combined with the given expansion to produce an element of the
effective fiber.

Thus $\mathcal{F}_{\mathrm{eff}}(x)$ is nonempty whenever $x$ is computable.

\subsection{Computable closedness}

To show that $\mathcal{F}_{\mathrm{eff}}(x)$ is a $\Pi^0_1$ class, we verify
that membership can be disproved by finite evidence.

A computable identity $G = (M,D,K)$ belongs to $\mathcal{F}_{\mathrm{eff}}(x)$
if and only if:
\begin{enumerate}
    \item $M$ selects digits infinitely often, and
    \item for each $j$, the selected digit $d_G(j)$ equals $x_j$.
\end{enumerate}

A violation of membership occurs precisely when:
\[
d_G(j) \ne x_j
\]
for some $j$.  
Because $M$ and $D$ are computable, this disagreement is detected by some
finite prefix of $(M,D)$.

Therefore:
\[
G \notin \mathcal{F}_{\mathrm{eff}}(x)
\quad\Longleftrightarrow\quad
\exists j\, [\, d_G(j) \ne x_j\,],
\]
and the right-hand condition is semidecidable (it is the disjunction of
computable finite checks).  
Hence $\mathcal{F}_{\mathrm{eff}}(x)$ is $\Pi^0_1$.

\section{Geometry of Effective Fibers}

Viewing $\mathcal{G}_{\mathrm{eff}}$ as a subset of Baire space, the effective
fiber inherits a rich internal structure:
\begin{itemize}
    \item it is infinite and compact in the subspace topology,
    \item it contains elements of both hybrid and null-density type,
    \item it admits arbitrary meta-information streams within the effective
          alphabet,
    \item it supports the tail-sewing and alignment constructions needed for
          diagonalization.
\end{itemize}

Crucially, no effective fiber collapses to a single computable identity.
Even after fixing the entire classical expansion of $x$, the selector stream
may choose digits at arbitrarily sparse or dense positions, and the
meta-information stream remains unrestricted.

This degree of freedom is the foundation for the observation and
incompatibility phenomena developed in the next part.

\section{Summary}

Each fiber of the collapse map is a large symbolic object containing every
identity that produces a given real number.  
The effective fiber $\mathcal{F}_{\mathrm{eff}}(x)$ forms a nonempty
$\Pi^0_1$ class and possesses significant internal structure.  
Understanding this geometry reveals why finite observers cannot capture all
the generative information that survives within a fiber.

In Part~II we turn to the dynamics of selector patterns, illustrating the
range of behaviors present among identities collapsing to the same value.
\clearpage{}

\chapter*{Part II Summary}

Part II examines the behavior of selector streams, which play a central role
in the generative identity.  
The selector determines which digits of the digit stream contribute to the
canonical output and therefore shapes both the internal structure of an
identity and its interaction with observers.

Two fundamental regimes of selector behavior are analyzed.  
Hybrid selectors expose digits with positive asymptotic density, while
null density selectors expose digits at vanishing density but still do so
infinitely often.  
Both regimes occur densely in the generative space, and both appear inside
every collapse fiber.  
This shows that the collapse operation imposes almost no constraint on the
rate at which digits are revealed.

Selector diversity inside collapse fibers underscores a key theme of the
framework.  
Identities that collapse to the same real number may differ widely in how
their canonical digits are exposed.  
Some identities reveal digits frequently and regularly, while others reveal
them sparsely or with large irregular gaps.  
This structural variability motivates the question of how much information can
be extracted from a generative identity by continuous observers.

Part II provides a detailed description of selector regimes and prepares the
ground for projection theory in Part III, where continuous observers are used
to measure and compare generative identities.
 \clearpage{}\chapter{Selector Patterns and Density Regimes}

\section{Introduction}

Generative identities differ not only in the symbols they carry but also in
the \emph{rate} at which their selectors expose digits from the underlying
digit stream.  
This rate---the asymptotic density of positions where $M(n) = D$---governs both
the structure of the canonical output and the degree of freedom present inside
the collapse fiber.

This chapter analyzes two fundamental regimes of selector behavior:
\begin{itemize}
    \item \emph{Hybrid selectors}, which expose digits with positive
          asymptotic density, and
    \item \emph{Null-density selectors}, which expose digits at vanishing
          density.
\end{itemize}

Although these two extremes lie on opposite ends of a broad spectrum, both
occur densely in the generative space.  
Understanding these regimes clarifies how generative identities with sharply
different internal behaviors can collapse to the same real number.

\section{Selector Density}

For a selector stream $M \in \{D,K\}^{\mathbb{N}}$, define the indicator
function
\[
\chi_M(n) =
\begin{cases}
1 & M(n) = D,\\[4pt]
0 & M(n) = K.
\end{cases}
\]

The \emph{selector density} of $M$ is the lower asymptotic density
\[
\eta(M)
  = \liminf_{N \to \infty}
      \frac{1}{N}
      \sum_{n=0}^{N-1} \chi_M(n).
\]

If $\eta(M) > 0$, the selector exposes digits at a positive rate; if
$\eta(M)=0$, exposure becomes increasingly sparse.

This density measures only the frequency of digit selections, not their
spacing; a selector may have dense clusters of selections followed by long
voids while still having positive or zero density.

\section{Hybrid Selectors}

\subsection{Definition}

A generative identity $G = (M,D,K)$ is \emph{hybrid} if $\eta(M) > 0$.
Equivalently, the indices $n$ with $M(n)=D$ have positive asymptotic density.

Hybrid identities expose digits regularly enough that, in the long run, a
non-negligible portion of the total stream contributes to the classical
output.

\subsection{Topological density}

Hybrid selectors occur densely in the generative space.

\begin{proposition}
For every nonempty basic open set in $\mathcal{X}$, there exists a hybrid
identity contained in it.
\end{proposition}

\begin{proof}
Let the open set be determined by finite prefixes of $(M, D, K)$.  
Extend these prefixes by placing $M(n) = D$ for all $n$ beyond the given
prefix.  
Then the extended identity is hybrid and remains inside the open set.  
Thus hybrid selectors form a dense subset of $\mathcal{X}$.
\end{proof}

\subsection{Interpretation}

Hybrid identities distribute their observed digits steadily throughout the
total stream.  
They represent the ``typical'' behavior of selectors when little is known
about their structure.

\section{Null-Density Selectors}

\subsection{Definition}

A generative identity is \emph{null-density} if its selector satisfies
$\eta(M) = 0$.

These selectors still expose infinitely many digits (since $G \in
\mathcal{X}^*$), but they do so with asymptotically negligible frequency.

\subsection{Examples}

A standard example uses the perfect squares:
\[
M(n) =
\begin{cases}
D & \text{if } n = k^2 \text{ for some } k,\\
K & \text{otherwise}.
\end{cases}
\]
Since the number of squares below $N$ is $\lfloor \sqrt{N} \rfloor$, the
density of $D$-positions is $N^{-1/2} \to 0$.

More intricate examples use rapidly growing computable sequences such as
$n_k = k!$, $n_k = 2^{2^k}$, or sparse polynomial-time patterns.

\subsection{Existence in every fiber}

Null-density selectors appear in every collapse fiber.

\begin{proposition}
For every $x \in [0,1]$, there exists a null-density generative identity
$G \in \mathcal{F}_{\mathrm{eff}}(x)$.
\end{proposition}

\begin{proof}
Fix the canonical expansion $(x_j)$ of $x$ and define a selector that exposes
digits only at perfect-square positions.  
At each such position $n_j$, set $D(n_j) = x_j$; elsewhere set $D$ arbitrarily.
Let $K$ be any computable meta-stream.  
This identity lies in $\mathcal{F}_{\mathrm{eff}}(x)$ and has density zero by
construction.
\end{proof}

\subsection{Interpretation}

Null-density selectors exhibit extreme sparsity.  
They expose infinitely many digits but at a rate too small to influence the
asymptotic distribution of symbols in the overall generative space.  
Such identities show that collapse fibers contain elements of dramatically
different structural complexity.

\section{Selector Diversity Inside a Fiber}

Hybrid and null-density identities coexist inside the same collapse fiber,
demonstrating that the classical output $x$ places almost no restrictions on
the internal rate of digit revelation.

Given any $x$, the effective fiber $\mathcal{F}_{\mathrm{eff}}(x)$ contains:
\begin{itemize}
    \item identities selecting digits frequently,
    \item identities selecting digits sparsely,
    \item identities with periodic or chaotic selection patterns,
    \item identities with arbitrary meta-information streams.
\end{itemize}

This freedom underscores the essential distinction between internal
generative structure and classical magnitude.

\section{Summary}

Selector density provides the first structural coordinate for generative
identities.  
Hybrid selectors expose digits with positive asymptotic density, whereas
null-density selectors do so sparsely.  
Both behaviors occur densely in the generative space and both appear in every
effective collapse fiber.  
The coexistence of such radically different regimes within a single fiber
illustrates the vast internal variability hidden beneath the collapse.

The next chapter introduces structural projections, continuous observers that
measure generative properties without disrupting the underlying identity.
\clearpage{}
\clearpage{}\chapter{Structural Projections and the Projection Lattice}

\section{Introduction}

The collapse map extracts the classical value of a generative identity while
discarding most of its internal structure.  
To understand which aspects of this structure can be detected by continuous
observers, we introduce the general notion of a \emph{structural projection}.
These projections form a lattice under pointwise comparison and represent
effective measurements that respect the topology of the generative space.

The framework developed in this chapter draws on ideas from Type-2
Effectivity, where continuous functionals on sequence spaces are understood
through their finite information content.  
This finite information principle, central in the work of Weihrauch and
Pauly on represented spaces, appears here in an explicit combinatorial form.
It allows projections to be analyzed through their dependency on finite
prefixes and serves as the foundation for the incompleteness results proved
later.

\section{Structural Projections}

A \emph{structural projection} is any continuous function
\[
\Phi : \mathcal{X}^* \to \mathbb{R},
\]
where $\mathcal{X}^*$ carries the product topology defined in Part I.  
Continuity ensures that the value $\Phi(G)$ is determined to any fixed
precision by a finite prefix of $G$.

More precisely, for every $\varepsilon > 0$, continuity provides an integer
$B_\Phi(\varepsilon)$ such that
\[
G[0..B_\Phi(\varepsilon)]
  = H[0..B_\Phi(\varepsilon)]
\quad\Longrightarrow\quad
|\Phi(G) - \Phi(H)| < \varepsilon.
\]

The function $B_\Phi$ plays the role of a computable modulus of continuity in
the sense of Type-2 computability, which is the standard framework for
analyzing real-valued functionals on symbolic spaces.

\section{Basic Examples}

Several projections arise naturally from the structure of a generative
identity.

\subsection*{Collapse}

The collapse $\pi$ is the foundational projection.  
Its continuity was established in Chapter 2 and follows from the classical
theory of real number representations.

\subsection*{Digit statistics}

Fix a digit $a \in \{0,\ldots,b-1\}$.  
Define
\[
\Phi_a(G)
  = \liminf_{N \to \infty}
      \frac{1}{N}
      \sum_{j=0}^{N-1} \mathbf{1}[\, d_G(j) = a \,].
\]
This projection measures the lower asymptotic frequency of the digit $a$ in
the canonical output.  
Other variants include limsup frequency or empirical block frequencies.

Such projections resemble classical invariants in symbolic dynamics, where
frequency statistics determine measure-theoretic properties of subshifts.
The exposition of Lind and Marcus provides many examples of these quantities
in the context of shift spaces.

\subsection*{Selector statistics}

Define
\[
\Psi(G)
  = \liminf_{N \to \infty}
      \frac{1}{N}
      \sum_{n=0}^{N-1} \mathbf{1}[\, M(n) = D \,].
\]
This projection measures the asymptotic density with which the selector
exposes digits.  
It coincides with the selector density studied in Chapter 4 but now viewed as
an observer on $\mathcal{X}^*$.

\section{Dependency Bounds}

Dependency bounds measure the amount of information an observer requires to
determine its output to a given precision.

\begin{definition}[Dependency Bound]
Let $\Phi : \mathcal{X}^* \to \mathbb{R}$ be continuous.  
A function $B_\Phi : (0,1] \to \mathbb{N}$ is a \emph{dependency bound} for
$\Phi$ if
\[
G[0..B_\Phi(\varepsilon)]
  = H[0..B_\Phi(\varepsilon)]
\quad\Longrightarrow\quad
|\Phi(G) - \Phi(H)| < \varepsilon
\]
for all $\varepsilon > 0$.
\end{definition}

If $\Phi$ is computable, classical results from Type-2 Effectivity imply that
$B_\Phi$ can be chosen to be computable as well.  
This follows from the fact that computable functionals on Baire space admit
computable moduli of continuity.

Dependency bounds quantify the finite information content of observers and
provide the mechanism by which projections can be frozen at finite stages in
the diagonalizer construction of Part IV.

\section{The Projection Lattice}

Given two projections $\Phi$ and $\Psi$, define
\[
\Phi \le \Psi
\quad\Longleftrightarrow\quad
\Phi(G) = \Phi(H) \text{ whenever } \Psi(G) = \Psi(H).
\]
This relation expresses that $\Psi$ distinguishes at least as much structure
as $\Phi$.

\begin{proposition}
The set of structural projections on $\mathcal{X}^*$ ordered by $\le$ forms a
complete lattice.
\end{proposition}

\begin{proof}
For any family of projections $(\Phi_i)$, the pointwise supremum
\[
\Phi(G) = \sup_i \Phi_i(G)
\]
is still continuous and therefore a structural projection.  
This projection is the least upper bound with respect to $\le$.  
Similarly, pointwise infima provide greatest lower bounds.
\end{proof}

This algebraic structure parallels the lattice of continuous real-valued
functionals on represented spaces and has been extensively studied in the
context of Weihrauch degrees.  
Here it provides the organizational framework for understanding how different
projections capture different aspects of generative structure.

\section{Summary}

Structural projections are continuous observers on the generative space.  
Their finite dependency on prefixes gives rise to computable dependency
bounds, and their collective structure forms a complete lattice.  
These properties reflect classical results from Type-2 computability and
symbolic dynamics but are here adapted to the generative identity setting.

In the next chapter we formalize prefix stabilization and show how the finite
dependency of observers enables the controlled constructions that drive the
incompleteness phenomena in Part IV.
\clearpage{}

\chapter*{Part III Summary}

Part III develops the theory of structural projections, which formalize how
continuous observers extract information from generative identities.  
A structural projection is any continuous real valued functional on the
generative space.  
Such observers depend only on finite prefixes of an identity at any fixed
precision, and this finite information principle is captured by computable
dependency bounds.

Dependency bounds provide explicit control over the amount of symbolic data
required to determine the value of an observer within a given error.  
This leads to prefix stabilization, which states that once two identities
agree on a sufficiently long prefix, all observers in a finite family must
agree on their values to within any chosen tolerance.  
Tail modification beyond this prefix has no effect on the output of the
observers.

Different observers impose different finite constraints on generative
identities.  
These constraints may conflict in a single prefix, producing projective
incompatibility.  
For example, one observer may require frequent digit exposures, while another
requires long gaps.  
Such conflicts show that no finite prefix can simultaneously satisfy all
structural demands and provide the combinatorial mechanism that allows
controlled divergence inside collapse fibers.

Part III therefore establishes the observational limits imposed by
continuity, provides the finite information tools that govern the behavior of
observers, and sets the stage for the diagonalizer construction in Part IV.
 \clearpage{}\chapter{Dependency Bounds and Prefix Stabilization}

\section{Introduction}

Structural projections evaluate generative identities using only finitely many
symbols at any fixed precision.  
This finite information principle is central to Type-2 computability, where
continuous functionals on Baire space are understood through their moduli of
continuity.  
In the generative setting, these moduli appear naturally as \emph{dependency
bounds}.  

This chapter develops the machinery that allows observers to be controlled at
finite stages.  
We formalize prefix stabilization, show how dependency bounds govern
finite-stage agreement, and explain how these properties prepare the ground
for the construction of the meta-diagonalizer in Part IV.

\section{Finite Information and Dependency Bounds}

Let $\Phi : \mathcal{X}^* \to \mathbb{R}$ be a structural projection.  
Continuity implies that for every $\varepsilon > 0$ there exists an integer
$B_\Phi(\varepsilon)$ such that agreement on the first $B_\Phi(\varepsilon)$
symbols of the identity forces agreement of the projections within
$\varepsilon$:
\[
G[0..B_\Phi(\varepsilon)]
  = H[0..B_\Phi(\varepsilon)]
\quad\Longrightarrow\quad
|\Phi(G) - \Phi(H)| < \varepsilon.
\]

When $\Phi$ is computable, the classical results of Pour-El, Richards,
Weihrauch, and Pauly guarantee that the map $\varepsilon \mapsto
B_\Phi(\varepsilon)$ may be chosen computably.  
This computability requirement is essential for the effective diagonalization
argument, where observers must be controlled by explicit finite parameters.

\section{Uniform Bounds for Finite Families}

Many arguments involve finite families of projections that must be handled
simultaneously.

\begin{definition}[Uniform Dependency Bound]
Given a finite family
\[
\mathcal{P} = \{ \Phi_1, \ldots, \Phi_k \}
\]
of projections, a function $B_{\mathcal{P}} : (0,1] \to \mathbb{N}$ is a
\emph{uniform dependency bound} if
\[
G[0..B_{\mathcal{P}}(\varepsilon)]
  = H[0..B_{\mathcal{P}}(\varepsilon)]
\quad\Longrightarrow\quad
|\Phi_i(G) - \Phi_i(H)| < \varepsilon
\]
for all $i$.
\end{definition}

Since the family is finite, we may take
\[
B_{\mathcal{P}}(\varepsilon)
  = \max_i B_{\Phi_i}(\varepsilon),
\]
which is computable if each $\Phi_i$ is.

Uniform bounds allow us to freeze a finite family of observers at a single
precision parameter.  
This operation is repeated at increasing precision in the diagonalizer
construction.

\section{Prefix Stabilization}

The key structural property of projections is that agreement beyond the
dependency bound is irrelevant to their evaluation.

\begin{proposition}[Prefix Stabilization]
Let $\Phi$ be a structural projection.  
Fix $\varepsilon > 0$ and set $N = B_\Phi(\varepsilon)$.  
If $G$ and $H$ agree on their first $N$ symbols, then their projections differ
by less than $\varepsilon$:
\[
G[0..N] = H[0..N]
\quad\Longrightarrow\quad
|\Phi(G) - \Phi(H)| < \varepsilon.
\]
\end{proposition}

\begin{proof}
This is exactly the definition of continuity in the product topology.  
The basic open neighborhoods of $G$ are determined by finite prefixes.
Choosing $N$ as the length of such a prefix gives the desired result.
\end{proof}

Prefix stabilization encodes the idea that projections observe only a finite
window of the identity at any fixed resolution.  
The unobserved tail may contain arbitrary structure without being detected by
the observer.

\section{Stability Under Tail Modification}

Tail modification is the process of replacing the portion of a generative
identity beyond some index $N$ with an arbitrary tail.

\begin{proposition}
Let $\Phi$ be a structural projection, let $\varepsilon > 0$, and let
$N = B_\Phi(\varepsilon)$.  
If $G$ and $H$ agree on $[0..N]$, then replacing the tail of $G$ by the tail
of $H$ beyond $N$ produces a new identity $\tilde{G}$ that satisfies
\[
|\Phi(\tilde{G}) - \Phi(G)| < \varepsilon.
\]
\end{proposition}

\begin{proof}
Since $\tilde{G}$ and $G$ agree on their first $N$ symbols, the conclusion
follows from prefix stabilization.
\end{proof}

This invariance under tail modification is one of the central structural
properties of projections.  
It ensures that observers can be satisfied at finite stages, while the tail
remains available for divergence, which is essential for diagonalization.

\section{Interaction with Selector Density}

Many structural projections depend only on selected digits.  
For such projections, the relevant prefixes are determined by the positions
where $M(n) = D$, not by the raw index $n$.  
This leads to selector-dependent versions of dependency bounds, which appear
later when controlling density and fluctuation observers.

The general principle remains unchanged: agreement on the relevant finite
prefix of the canonical output determines agreement of the projection at the
corresponding precision.

\section{Summary}

Dependency bounds capture the finite information content of observers on the
generative space.  
Prefix stabilization and tail invariance show that structural projections
depend only on finite prefixes at any fixed precision.  
These properties enable finite-stage control of observers and are the key
technical tools for the alignment and sewing constructions that begin in the
next chapter and culminate in the meta-diagonalizer of Part IV.
\clearpage{}
\clearpage{}\chapter{Projective Incompatibility}

\section{Introduction}

Structural projections measure different aspects of a generative identity.
Some observe digit frequencies, others observe spacing patterns, and others
extract classical information through collapse.  
Although each projection examines only a finite prefix at a fixed precision,
their requirements may conflict.  
It may be impossible for a single finite prefix of an identity to satisfy the
demands of multiple projections at once.

This chapter formalizes this notion of conflict.  
We show that distinct observers often require incompatible finite structures
from the selector and digit streams.  
This incompatibility is a central ingredient in the construction of the
meta-diagonalizer in Part IV, where controlled divergence from reference
identities is enforced by exploiting these conflicting constraints.

The conceptual roots of this phenomenon appear in symbolic dynamics, where
different ergodic or combinatorial invariants may demand incompatible blocks
to appear in a shift space.  
Here the same idea arises in the generative setting but is applied directly to
observational functionals rather than to subshifts.

\section{Observer Requirements}

Let $\Phi$ and $\Psi$ be two structural projections with dependency bounds
$B_\Phi$ and $B_\Psi$.  
Fix a desired precision $\varepsilon > 0$.  
Then any identity $G$ must satisfy:
\[
\Phi(G) \text{ determined by } G[0..B_\Phi(\varepsilon)], 
\quad
\Psi(G) \text{ determined by } G[0..B_\Psi(\varepsilon)].
\]

If the projections measure unrelated aspects of the identity, these
finite prefixes may be required to contain contradictory patterns.

\subsection*{Example: density versus spacing}

Digit density projections may require the initial prefix to contain many
positions where $M(n) = D$.  
Conversely, spacing projections (such as fluctuation observers) may require
long runs where $M(n) = K$ in order to witness large gaps between selected
positions.  
A single finite block cannot simultaneously exhibit both high density and
large gaps at the same index range.

This tension reflects a basic fact from combinatorics on words: local
constraints on symbol frequencies and local constraints on block lengths need
not be simultaneously satisfiable.

\section{Formal Definition of Incompatibility}

\begin{definition}[Projective Incompatibility]
Two projections $\Phi$ and $\Psi$ are \emph{incompatible at precision}
$\varepsilon$ if no prefix of length
\[
L = \max\{ B_\Phi(\varepsilon), B_\Psi(\varepsilon) \}
\]
can satisfy both of the following:
\begin{enumerate}
    \item the prefix forces $\Phi(G)$ to lie within $\varepsilon$ of its target
          value, and
    \item the prefix forces $\Psi(G)$ to lie within $\varepsilon$ of its target
          value.
\end{enumerate}
\end{definition}

Incompatibility expresses a structural impossibility: the observers demand
conflicting patterns in the same finite window.

\section{Concrete Instances}

Although incompatibility is common in practice, one example illustrates the
idea clearly.

\subsection*{Example: Frequent selection and large gaps}

Let $\Phi$ be the selector density projection and $\Psi$ the fluctuation
projection.  
Fix $\varepsilon = 0.05$.

To force $\Phi(G)$ within $\varepsilon$ of a positive density, the prefix must
contain many instances of $M(n) = D$.  
To force $\Psi(G)$ within $\varepsilon$ of a large index gap ratio, the prefix
must contain a long contiguous run where $M(n) = K$.

Let $N_\Phi = B_\Phi(\varepsilon)$ and $N_\Psi = B_\Psi(\varepsilon)$.  
Consider the interval $[0..L]$ with $L = \max(N_\Phi, N_\Psi)$.  
If the prefix contains the required density of selected positions for
$\Phi$, it cannot contain the long block of unselected positions required by
$\Psi$ within the same interval.  
Thus the two requirements are incompatible at precision $\varepsilon$.

This incompatibility is a finite-informational fact and does not depend on
the behavior of the identity beyond the prefix.

\section{Incompatibility Across a Family}

Finite families of projections may also exhibit internal conflicts.

\begin{proposition}
Let $\mathcal{P}$ be a finite family of projections.  
If $\mathcal{P}$ contains two projections that are incompatible at some
precision $\varepsilon$, then no identity can satisfy the entire family at
precision $\varepsilon$ using a single prefix of length $B_{\mathcal{P}}(\varepsilon)$.
\end{proposition}

\begin{proof}
Since $B_{\mathcal{P}}(\varepsilon)$ is the maximum dependency bound for the
family, any prefix satisfying all observers must satisfy each one
individually.  
If two observers impose incompatible requirements on that prefix, no such
prefix exists.
\end{proof}

This proposition explains why the diagonalizer construction can always force
deviations.  
Whenever a finite family of observers demands a uniform prefix, one can
choose tails that generate divergent structures not simultaneously detectable
by the family.

\section{Implications for Diagonalization}

The diagonalizer of Part IV depends critically on the existence of
projections whose finite precision requirements cannot all be realized in the
same prefix.  
This incompatibility creates space for controlled tail divergence.  
Once an identity has satisfied observers up to the required prefix length, the
unobservable tail may be modified to enforce structural differences from
reference generators.

This mechanism reflects standard arguments in computable analysis, where
different constraints on prefixes of names of real numbers may conflict.  
Here the conflicts arise between observers on generative identities and serve
as the engine of the incompleteness phenomenon.

\section{Summary}

Different projections impose distinct finite structural requirements on
generative identities.  
When these requirements cannot be satisfied simultaneously in a single
prefix, we say the projections are incompatible.  
Such conflicts provide the combinatorial foundation for the diagonalizer,
allowing one to satisfy observers finitely while enforcing divergence beyond
their range of observation.

In the next chapter we develop the alignment and sewing tools needed to
exploit this incompatibility inside the effective collapse fiber.
\clearpage{}
\clearpage{}\chapter{Alignment and Tail Sewing Inside Fibers}

\section{Introduction}

The collapse fiber $\mathcal{F}(x)$ contains a vast collection of generative
identities that all yield the same classical real number.  
The diagonalizer developed in the next chapter constructs a new identity
inside the effective fiber that matches a reference identity on all observed
prefixes while diverging arbitrarily in its unobserved tail.  
To carry out this construction, we need two technical tools.

The first tool is an alignment procedure.  
Since the collapse depends only on the sequence of selected digits in the
order they appear, we must ensure that when we splice the tail of one
identity onto the prefix of another, the resulting identity produces the same
canonical output.  
The second tool is a sewing procedure, which replaces the tail of one
identity with the tail of another while retaining membership in the same
collapse fiber.

These constructions rely on the fact that identities in a fiber agree on
their selected digits when listed in order, even though the positions of
these digits in the raw sequence may differ.  
This kind of alignment appears in various areas of symbolic dynamics, in
particular in the study of synchronized shift spaces, but here it plays a
more basic role.  
The alignment and sewing tools allow us to replace long tails without
changing the collapsed value.

\section{Alignment of Selected Digits}

Let $H$ and $A$ be two identities in the fiber $\mathcal{F}(x)$, and let
\[
d_H(0), d_H(1), d_H(2), \ldots
\quad\text{and}\quad
d_A(0), d_A(1), d_A(2), \ldots
\]
be their canonical output sequences.  
Since $H$ and $A$ lie in the same fiber, these sequences are identical and
represent the expansion of $x$.

Let
\[
h_0 < h_1 < h_2 < \cdots
\quad\text{and}\quad
a_0 < a_1 < a_2 < \cdots
\]
be the indices at which $H$ and $A$ select digits.  
For any $k$, both identities expose the $k$th digit of $x$ at their
respective indices $h_k$ and $a_k$.

\begin{proposition}[Index Alignment]
For any $k$, there exist positions in $H$ and $A$ at which the $k$th canonical
digit is selected, namely $h_k$ and $a_k$.  
Thus an identity obtained by taking the prefix of $H$ up to $h_k$ and the tail
of $A$ beginning at $a_k$ produces the same canonical output as $H$.
\end{proposition}

\begin{proof}
Since both identities lie in $\mathcal{F}(x)$, the value of the $k$th selected
digit in each must be $x_k$.  
Therefore the alignment indices $h_k$ and $a_k$ exist by definition of the
canonical output.
\end{proof}

This proposition ensures that splicing the two identities at matching digit
indices preserves the canonical output sequence.

\section{Sewing of Tails}

Given two identities $H$ and $A$ in the same fiber, consider the identity
$\tilde{G}$ that agrees with $H$ up to $h_k$ and with $A$ beyond $a_k$.  
Alignment ensures that the canonical output of $\tilde{G}$ equals that of
$H$, so $\tilde{G}$ lies in $\mathcal{F}(x)$.

\begin{proposition}[Tail Sewing]
Fix $k \in \mathbb{N}$.  
Let $G$ be the identity defined by
\[
G(n) =
\begin{cases}
H(n) & n \le h_k,\\
A(n - h_k + a_k) & n > h_k.
\end{cases}
\]
Then $G \in \mathcal{F}(x)$.
\end{proposition}

\begin{proof}
The identity $G$ agrees with $H$ on the prefix containing the first $k$ selected
digits.  
Beyond that prefix it reproduces the $(k+1)$st, $(k+2)$nd, and all later
selected digits of $A$ in order.  
Since $A$ and $H$ have the same canonical output, $G$ reproduces this same
sequence.  
Therefore $\pi(G) = x$.
\end{proof}

This construction replaces the tail of one identity with that of another
without altering the canonical output.  
The ability to modify the tail freely inside the fiber is one of the key
structural freedoms used in the diagonalizer.

\section{Controlled Tail Replacement}

In diagonalization, we do not splice tails arbitrarily.  
Instead, we choose $A$ to satisfy a specific structural property that we want
the final identity to inherit, and we sew its tail onto a reference identity
$H$ after a sufficiently long prefix.

Let $\mathcal{P}$ be a finite family of projections that we wish to match up
to precision $\varepsilon$.  
Let $N = B_{\mathcal{P}}(\varepsilon)$ be the uniform dependency bound.  
If $H$ and $A$ agree on their first $N$ symbols, then sewing the tail of $A$
onto the prefix of $H$ at any alignment point beyond $N$ preserves the
projections to within $\varepsilon$.

\begin{proposition}[Controlled Tail Sewing]
Let $\mathcal{P}$ be a finite family of projections with uniform dependency
bound $B_{\mathcal{P}}$.  
Fix $\varepsilon > 0$ and set $N = B_{\mathcal{P}}(\varepsilon)$.  
Let $h_k$ be the $k$th selection index for $H$, and choose $k$ such that
$h_k \ge N$.  
Similarly, let $a_k$ be the $k$th selection index for $A$.  
Define $G$ by sewing the prefix of $H$ up to $h_k$ to the tail of $A$ from
$a_k$ onward.  
Then for every $\Phi \in \mathcal{P}$,
\[
|\Phi(G) - \Phi(H)| < \varepsilon.
\]
\end{proposition}

\begin{proof}
Since $G$ and $H$ agree on their first $h_k$ symbols and $h_k \ge N$, we have
agreement on the first $N$ symbols.  
By definition of $B_{\mathcal{P}}$, agreement on the first $N$ symbols ensures
agreement of all projections in the family to within $\varepsilon$.
\end{proof}

This shows that once observers are satisfied on the prefix of length $N$, the
tail may be replaced freely without altering their outputs at the chosen
precision.  
This powerful freedom is the main technical ingredient of the diagonalizer.

\section{Summary}

Alignment of selected digits ensures that identities in the same fiber expose
their canonical digits in a coherent order.  
Tail sewing uses this alignment to replace the entire tail of one identity
with the tail of another while remaining inside the collapse fiber.

When combined with dependency bounds and prefix stabilization, these tools
allow us to construct identities that satisfy any finite family of observers
on arbitrarily long prefixes while diverging freely in the unobserved tail.
The next chapter uses these tools to build the meta-diagonalizer, which
demonstrates the impossibility of recovering generative structure from any
finite collection of continuous observers.
\clearpage{}

\chapter*{Part IV Summary}

Part IV establishes the central incompleteness phenomenon of the Generative
Identity Framework.  
Although the collapse map determines the classical real number associated with
a generative identity, it discards most of the symbolic structure carried by
the selector, digit, and meta streams.  
This part shows that no finite collection of continuous observers can recover
that structure.

The first chapter develops the alignment and sewing tools that operate inside
collapse fibers.  
Identities in the same fiber expose the same digits in the same order, even
when their selector positions differ.  
This permits alignment of selected digits and controlled replacement of the
tail of one identity by the tail of another without changing the collapsed
value.  
These symbolic operations make it possible to preserve observer agreement on
finite prefixes while introducing divergence in unobserved tails.

The second chapter constructs the meta diagonalizer.  
Given an enumeration of all computable structural projections, the
construction produces a computable identity that agrees with a reference
identity on all observed prefixes but diverges from it along each projection
at a scale that eventually exceeds the tolerance assigned to that projection.
The diagonalizer is built inductively, using dependency bounds to satisfy
observer constraints at each stage and using tail sewing to introduce the
desired divergence.

The final chapter of the part proves the Structural Incompleteness Theorem.
For any computable real number $x$ and any finite family of computable
observers, there exist distinct identities in the collapse fiber
$\mathcal{F}_{\mathrm{eff}}(x)$ that agree on the values of all observers in
the family.  
Finite observation cannot recover the generative identity.  
This establishes an inherent limit on what any finite analytical process can
detect about the internal structure of generative representations.

Part IV therefore demonstrates that generative structure is fundamentally
incomplete with respect to finite continuous observation.  
This incompleteness is not a consequence of randomness or approximation, but
arises from the topology and symbolic architecture of the generative space.
 \clearpage{}\chapter{The Meta-Diagonalizer}

\section{Introduction}

The collapse fiber $\mathcal{F}(x)$ contains many generative identities that
produce the same classical real number.  
In earlier chapters we saw that continuous projections cannot observe the full
internal structure of a generative identity and depend only on finite prefixes
at any fixed precision.  
This chapter constructs a new identity inside the effective fiber that agrees
with a given reference identity on every observed prefix, yet diverges from it
in structural properties that no fixed finite collection of projections can
detect.

The construction is a form of diagonalization.  
Given a countable sequence of projections, we build an identity that avoids
agreement with a prescribed family of reference structures by introducing
divergence at stages beyond their dependency bounds.  
The internal freedom available inside collapse fibers ensures that these
divergent tails do not change the classical value of the identity.

The diagonalizer demonstrates the fundamental incompleteness of finite
observers and prepares the ground for the Incompleteness Theorem of the next
chapter.

\section{Setup and Notation}

Fix a computable real number $x$ and let
\[
(x_j)_{j \ge 0}
\]
denote its canonical base $b$ expansion.  
Let $H$ be a computable identity in $\mathcal{F}_{\mathrm{eff}}(x)$ that we
will use as a reference.  
Let
\[
\Phi_0, \Phi_1, \Phi_2, \ldots
\]
be a computable enumeration of all computable projections on
$\mathcal{G}_{\mathrm{eff}}$.  
For each $k$, let $B_k$ denote a computable dependency bound for $\Phi_k$.

Our goal is to construct an identity
\[
G^{\sharp} \in \mathcal{F}_{\mathrm{eff}}(x)
\]
that diverges from $H$ on every projection $\Phi_k$ by more than a prescribed
amount at stage $k$, despite agreeing with $H$ on prefixes long enough to
satisfy all earlier projections.

\section{Divergent Identities Inside the Fiber}

At each stage $k$ we will require an identity $A_k$ inside the effective fiber
$\mathcal{F}_{\mathrm{eff}}(x)$ that differs from $H$ at scale $\varepsilon_k$
with respect to $\Phi_k$, where
\[
\varepsilon_0 > \varepsilon_1 > \varepsilon_2 > \cdots \to 0
\]
is a fixed computable sequence of positive tolerances.

The following lemma ensures that such identities always exist.

\begin{lemma}[Divergence Inside the Effective Fiber]
\label{lem:divergence-fiber}
For any computable projection $\Phi$ and any computable $\varepsilon > 0$,
there exists a computable identity $A \in \mathcal{F}_{\mathrm{eff}}(x)$ such
that
\[
|\Phi(A) - \Phi(H)| > 3\varepsilon.
\]
\end{lemma}

\begin{proof}
The effective fiber $\mathcal{F}_{\mathrm{eff}}(x)$ is a nonempty
$\Pi^0_1$ class.  
If $\Phi$ were constant on this class, then $\Phi$ would depend only on the
canonical output and would assign the same value to all identities in the
fiber.  
However, by the definition of a structural projection, $\Phi$ may depend on
all components of the generative identity, including the selector and
meta-information streams.

Because the fiber allows arbitrary choices on unselected digits and on the
meta stream, we may modify these components computably without altering the
selected digits.  
By continuity of $\Phi$, local changes to the selector or meta stream beyond
any prefix can move the value of $\Phi$ through an interval.  
Thus the image of $\mathcal{F}_{\mathrm{eff}}(x)$ under $\Phi$ is an interval,
and the value $\Phi(H)$ cannot lie at the boundary of this interval for all
such modifications.

We may therefore choose a computable identity $A$ in the fiber and a
computable prefix for which the subsequent tail guarantees a divergence of at
least $3\varepsilon$.  
This identity satisfies the desired inequality.
\end{proof}

This lemma encapsulates the essential freedom inside the effective fiber: one
may force meaningful structural differences without changing the collapsed
value.

\section{The Diagonal Construction}

We now build the diagonalizer
\[
G^{\sharp} = \lim_{k \to \infty} G_k
\]
as a limit of identities that stabilize on longer and longer prefixes while
introducing controlled divergence at each stage.

Let $G_0 = H$.  
Assume inductively that $G_k$ has been defined and that $G_k$ agrees with $H$
on its first $N_k$ symbols for some computable $N_k$.

\subsection*{Stage $k$: divergence}

Choose
\[
\varepsilon_k = 2^{-(k+2)}.
\]
By Lemma \ref{lem:divergence-fiber}, choose
\[
A_k \in \mathcal{F}_{\mathrm{eff}}(x)
\]
such that
\[
|\Phi_k(A_k) - \Phi_k(H)| > 3\varepsilon_k.
\]

This identity will serve as the source of structural divergence at stage $k$.

\subsection*{Stage $k$: prefix agreement}

To preserve earlier projective agreements, we require that the first $N_k$
symbols of $G_{k+1}$ agree with $G_k$ and with $H$.  
Let
\[
N_{k+1}
  = \max\bigl( N_k,\ B_k(\varepsilon_k) \bigr).
\]
This ensures that agreement on the first $N_{k+1}$ symbols forces agreement of
$\Phi_k$ up to error $\varepsilon_k$.

\subsection*{Stage $k$: alignment}

Let $h_{k}$ be the position in $H$ corresponding to the $k$th selected digit.
Let $a_k$ be the corresponding position in $A_k$.  
Choose $j$ sufficiently large that $h_j \ge N_{k+1}$.  
The alignment lemma guarantees that the indices $h_j$ and $a_j$ correspond to
the same selected digit.

\subsection*{Stage $k$: tail sewing}

Define $G_{k+1}$ by sewing:
\[
G_{k+1}(n)
  =
\begin{cases}
G_k(n) & n \le h_j,\\
A_k(n - h_j + a_j) & n > h_j.
\end{cases}
\]

By the tail sewing proposition from the previous chapter,
\[
G_{k+1} \in \mathcal{F}_{\mathrm{eff}}(x).
\]

Moreover, $G_{k+1}$ agrees with $G_k$ on their first $N_{k+1}$ symbols, so all
projections $\Phi_0,\ldots,\Phi_k$ agree with $H$ to within $\varepsilon_k$ on
$G_{k+1}$.

\section{Existence of the Limit Identity}

The sequence $(G_k)$ stabilizes on longer and longer prefixes.  
For each index $n$, there exists a stage $k$ such that $n \le N_k$, and from
that point onward, the $n$th symbol of $G_m$ remains constant for all
$m \ge k$.

Define $G^{\sharp}$ to be the identity whose $n$th symbol is this eventual
value.  
Then the limit exists and is computable.

\begin{proposition}
The identity $G^{\sharp}$ lies in the effective fiber
$\mathcal{F}_{\mathrm{eff}}(x)$.
\end{proposition}

\begin{proof}
Each $G_k$ lies in the fiber, and the canonical output is preserved at every
stage by the alignment and sewing procedure.  
Thus $G^{\sharp}$ also collapses to $x$.  
Computability follows because each coordinate stabilizes at a computable
stage.
\end{proof}

\section{Diagonalization}

Finally, we verify that $G^{\sharp}$ diverges from $H$ along every projection
in the sequence.

\begin{proposition}
For each $k$,  
\[
|\Phi_k(G^{\sharp}) - \Phi_k(H)| \ge \varepsilon_k.
\]
\end{proposition}

\begin{proof}
At stage $k$ we chose $A_k$ so that
\[
|\Phi_k(A_k) - \Phi_k(H)| > 3\varepsilon_k.
\]
The sewing step ensures that the tail of $G_{k+1}$ beyond index $h_j$ agrees
with the tail of $A_k$ from index $a_j$ onward.  
Since the tail lies entirely beyond $B_k(\varepsilon_k)$, any effect on
$\Phi_k$ caused by the tail persists in $G_{k+1}$ and therefore in
$G^{\sharp}$.

Agreement of prefixes up to $N_{k+1}$ introduces at most $\varepsilon_k$ of
error.  
Because the divergence was chosen to exceed $3\varepsilon_k$, the final
difference remains at least $\varepsilon_k$.
\end{proof}

Thus $G^{\sharp}$ avoids finite approximation by the projections
$\Phi_k$ in the same way that classical diagonalization avoids uniform
compression of information.  
This completes the construction.

\section{Summary}

This chapter constructed a computable identity in the collapse fiber of $x$
that diverges from a reference identity on every computable projection while
agreeing with the reference on arbitrarily long prefixes.  
The construction relies on alignment and sewing tools, the existence of
divergent identities inside the fiber, and finite dependency bounds for
projections.  
In the next chapter we apply the diagonalizer to prove the Structural
Incompleteness Theorem, which states that no finite family of continuous
observers can capture the generative structure of a real number.
\clearpage{}
\clearpage{}\chapter{The Continuum as a Collapse Quotient}

\section{Introduction}

The collapse map sends every generative identity to a real number by selecting
and interpreting the digits exposed by its selector stream.  
This chapter examines the relationship between the generative space
$\mathcal{X}^*$ and the classical continuum $[0,1]$, viewed as the image of
the collapse map.  
We present a quotient perspective in which real numbers arise by identifying
all identities in the same collapse fiber.  
This perspective reveals that the continuum is a coarse shadow of a much
richer symbolic space.

The quotient interpretation is familiar in computable analysis and in the
theory of represented spaces, where classical objects are obtained as
equivalence classes of names.  
Here the equivalence relation is induced by the canonical output mechanism,
and the resulting quotient map is continuous, surjective, and highly
non-injective.

\section{The Collapse Equivalence Relation}

The collapse map $\pi : \mathcal{X}^* \to [0,1]$ induces an equivalence
relation
\[
G \sim H
\quad\Longleftrightarrow\quad
\pi(G) = \pi(H).
\]
The equivalence class of $G$ under this relation is its collapse fiber
$\mathcal{F}(\pi(G))$.

Thus the classical real number $\pi(G)$ may be viewed as the equivalence class
\[
[\![G]\!] = \mathcal{F}(\pi(G)).
\]

Two identities lie in the same class precisely when they produce the same
canonical digit sequence, and this sequence determines the real number under
the usual base $b$ interpretation.

\section{The Quotient Map}

Endow $\mathcal{X}^*$ with the product topology and $[0,1]$ with the usual
Euclidean topology.  
Then the collapse map is continuous and surjective.  
The induced quotient map
\[
\mathcal{X}^* \longrightarrow \mathcal{X}^*/{\sim}
\]
is continuous in the quotient topology, and the space $\mathcal{X}^*/{\sim}$
is homeomorphic to $[0,1]$.

\begin{proposition}
The quotient $\mathcal{X}^*/{\sim}$ is compact, totally disconnected, and
metrizable, and it is homeomorphic to the closed interval $[0,1]$.
\end{proposition}

\begin{proof}
The space $\mathcal{X}^*$ is compact and totally disconnected since it is a
product of compact discrete spaces.  
The quotient of a compact space by a closed equivalence relation is compact.
The equivalence classes are closed because the collapse map is continuous and
singletons in $[0,1]$ are closed.  
Standard results from general topology imply that the quotient is compact and
metrizable.  
Finally, the collapse map is continuous and surjective, and the usual
base $b$ representation provides an explicit homeomorphism between the
quotient and $[0,1]$.
\end{proof}

Although the quotient is topologically simple, the structure of individual
fibers is highly complex.  
The quotient space collapses intricate symbolic data into a one dimensional
object.

\section{Topological Interpretation of Collapse Fibers}

Each fiber is a compact, perfect, totally disconnected subspace of
$\mathcal{X}^*$, and typically a product of Cantor sets with additional
structure imposed by the selected digit constraints.  
This rich internal structure contrasts with the simplicity of the collapsed
value.

The diagonalizer constructed in Part IV exploits this complexity by using
symbolic freedom inside the fiber to generate structural divergence while
keeping the classical value fixed.  
The quotient viewpoint therefore provides natural language for describing why
finite observers cannot reconstruct a generative identity from its collapsed
value.

\section{Computability Perspective}

From the viewpoint of computable analysis, the collapse equivalence classes
correspond to sets of names for real numbers.  
Every computable real number $x$ has a computable identity in the fiber
$\mathcal{F}_{\mathrm{eff}}(x)$, and this identity serves as a computable name
in the sense of Type-2 Effectivity.

Conversely, noncomputable reals correspond to fibers with no computable
elements.  
Such fibers may still have rich internal structure, but none of their
identities can serve as effective names.

This perspective aligns the generative identity framework with classical
represented space theory while emphasizing that the generative space contains
significantly more structure than a conventional naming system.

\section{Summary}

Classical real numbers arise as equivalence classes of generative identities
under the collapse map.  
The quotient map from the generative space to the continuum is continuous and
surjective, and it identifies identities that agree on their canonical output.
Each fiber is a large symbolic set containing many identities with the same
classical value.  
This quotient interpretation explains why generative structure cannot be
recovered from magnitude and prepares the groundwork for the study of extended
invariants in the next part.
\clearpage{}

\chapter*{Part V Summary}

Part V develops the quotient perspective that connects the generative space to
the classical continuum.  
The collapse map sends each generative identity to a real number by selecting
and interpreting the digits exposed by its selector stream.  
Although the generative space is infinite dimensional, the collapse map
identifies many distinct identities and assigns them a single classical value.

The resulting equivalence classes are the collapse fibers.  
Each fiber is a compact, perfect, and totally disconnected symbolic set.  
The fibers vary widely in their internal structure, containing identities with
selector streams of positive density, zero density, regular spacing, or large
irregular gaps.  
These structural differences are invisible to collapse but central to the
behavior of observers.

The quotient space $\mathcal{X}^*/{\sim}$ obtained by identifying identities
with the same collapsed value is homeomorphic to the interval $[0,1]$.  
This interpretation parallels the role of names in computable analysis and the
theory of represented spaces, where classical objects are obtained as
equivalence classes of symbolic descriptions.  
From this viewpoint, each real number corresponds to the entire fiber of its
generative representations.

Part V therefore shows that the classical continuum is a coarse image of a
much richer symbolic structure.  
The relation between fibers and extended invariants prepares the way for Part
VI, where selector behavior is examined through large scale numerical and
geometric coordinates.
 \clearpage{}\chapter{Extended Invariants: Entropy Balance and Fluctuation}

\section{Introduction}

The collapse map sends a generative identity to its classical real value, but
two identities that collapse to the same number may differ significantly in
their internal structure.  
Part IV showed that no finite collection of continuous observers can recover
this structure.  
In this chapter we introduce two extended invariants that capture broad
features of selector behavior: the entropy balance and the fluctuation index.

Both invariants measure long term properties of the selector stream.  
The entropy balance describes how frequently digits are exposed, while the
fluctuation index measures the relative size of gaps between successive
selected positions.  
Neither invariant is continuous in the product topology, and this reflects a
deeper fact: asymptotic statistical quantities on symbolic streams are rarely
continuous unless they depend only on finite windows.  
What can be proved, and what is sufficient for our purposes, is that these
quantities satisfy natural semicontinuity properties.

\section{Entropy Balance}

Let $G = (M, D, K)$ be a generative identity.  
Define the indicator function
\[
\chi_M(n)
  =
\begin{cases}
1 & M(n) = D,\\
0 & M(n) = K.
\end{cases}
\]

The entropy balance (or simply the balance) of $G$ is the limit inferior
\[
\eta(G)
  = \liminf_{N \to \infty}
      \frac{1}{N} \sum_{n=0}^{N-1} \chi_M(n).
\]

This quantity measures the lower asymptotic density with which the selector
exposes digits.  
Hybrid identities have positive balance, while null density identities have
balance zero.

\subsection{Basic properties}

The balance has two elementary properties.

\begin{itemize}
    \item It is invariant under tail modification beyond finite prefixes.
    \item It depends only on the selector stream $M$.
\end{itemize}

The balance is therefore an extended invariant that captures structural
information invisible to the collapse map.

\subsection{Lower semicontinuity}

The balance is not continuous with respect to the product topology.  
Small modifications to the selector can introduce arbitrarily large gaps or
arbitrarily many selected positions within a long prefix, which can shift the
density downward or upward.  
However, balance satisfies a one sided estimate.

\begin{proposition}[Lower Semicontinuity]
Let $(G_k)$ converge to $G$ in the product topology.  
Then
\[
\eta(G)
  \le \liminf_{k \to \infty} \eta(G_k).
\]
\end{proposition}

\begin{proof}
Fix $\varepsilon > 0$.  
Choose $N$ sufficiently large that
\[
\frac{1}{N} \sum_{n=0}^{N-1} \chi_M(n)
  < \eta(G) + \varepsilon.
\]
For all sufficiently large $k$, the identities $G_k$ agree with $G$ on the
first $N$ symbols.  
Hence
\[
\eta(G_k)
  \ge \frac{1}{N} \sum_{n=0}^{N-1} \chi_M(n)
  > \eta(G) - \varepsilon.
\]
Taking the limit inferior gives the claim.
\end{proof}

The failure of full continuity is a structural fact, but lower
semicontinuity is sufficient for all applications.

\section{Fluctuation Index}

The fluctuation index measures relative gap size between selected digit
positions.  
Let
\[
n_0 < n_1 < n_2 < \cdots
\]
be the indices at which $M(n) = D$ and define the successive gaps
\[
g_j = n_{j+1} - n_j.
\]

The fluctuation index of $G$ is the limit superior
\[
\phi(G)
  = \limsup_{j \to \infty} \frac{g_j}{n_j}.
\]

The quantity $g_j$ measures the absolute size of the gap following the
$j$th selected position, while the ratio $g_j / n_j$ measures the gap
relative to the position in the stream.  
High fluctuation indicates the presence of unusually large gaps between
selected digits, relative to scale.

\subsection{Basic properties}

The fluctuation index depends only on the selector stream and is unaffected by
any tail modifications that do not alter the sequence of selected positions.

The index captures the extent to which the selector allows sparse bursts of
digit exposure.  
Null density selectors typically have large fluctuation, while positive
density selectors have small fluctuation in many cases.

\subsection{Upper semicontinuity}

As with $\eta$, the fluctuation index is not continuous.  
A single large gap introduced in the tail can immediately increase the value
of $\phi$.  
However, the index satisfies upper semicontinuity.

\begin{proposition}[Upper Semicontinuity]
Let $(G_k)$ converge to $G$ in the product topology.  
Then
\[
\phi(G)
  \ge \limsup_{k \to \infty} \phi(G_k).
\]
\end{proposition}

\begin{proof}
Suppose $\phi(G) < c$ for some $c$.  
Then there exists $J$ such that
\[
\frac{g_j}{n_j} < c
\quad\text{for all}\quad j \ge J.
\]
The selected positions $n_j$ are determined by the selector stream.  
Agreement of $G_k$ with $G$ on sufficiently long prefixes ensures that the
positions of the first $J$ selected digits match, and that the ratios
$g_j / n_j$ for $j \le J$ match exactly.  
Therefore for all sufficiently large $k$,
\[
\phi(G_k) < c.
\]
The desired inequality follows by taking the limit superior.
\end{proof}

Upper semicontinuity reflects the fact that introducing a large gap is a
finite event that persists under limits, while eliminating a large gap is a
global tail operation that is not respected by the product topology.

\section{Selector Structure and Extended Invariants}

The invariants $\eta$ and $\phi$ reveal two complementary aspects of selector
behavior:
\begin{itemize}
    \item $\eta$ measures how frequently digits are exposed in a long term
          sense,
    \item $\phi$ measures how irregularly they are exposed relative to scale.
\end{itemize}

These quantities need not determine one another.  
For example, a selector may have positive density but also have occasional
large gaps, or a selector may have density zero but exhibit extremely regular
spacing.  
Both invariants occur densely in the generative space.

\section{Extended Invariants Inside Collapse Fibers}

Extended invariants vary widely inside a collapse fiber.  
Fix a computable real $x$.  
The fiber $\mathcal{F}_{\mathrm{eff}}(x)$ contains identities with:
\begin{itemize}
    \item positive balance,
    \item zero balance,
    \item small fluctuation indices,
    \item arbitrarily large fluctuation indices.
\end{itemize}

This diversity follows from the fact that extended invariants depend only on
the selector stream and are insensitive to positions where $M(n) = K$.  
Given any selector behavior consistent with infinite digit selection, one may
always construct an identity in the fiber by assigning the selected positions
the digits of $x$ in the correct order.

Thus the fiber contains identities with all possible behaviors permitted by
the definitions of $\eta$ and $\phi$.

\section{Summary}

The entropy balance and fluctuation index extend the collapse map by assigning
numerical values to the selector behavior of a generative identity.  
Both invariants are discontinuous in the product topology but satisfy natural
semicontinuity properties.  
These quantities illustrate the diversity of selector patterns that arise
inside collapse fibers and provide a bridge between the collapse quotient of
Part V and the generative geometry of the next chapter.

In Chapter 13 we study geometric embeddings of extended invariants and discuss
how these quantities characterize large scale features of the generative
space.
\clearpage{}

\chapter*{Part VI Summary}

Part VI introduces extended invariants that measure large scale features of
selector behavior and provide coarse geometric perspectives on the generative
space.  
Unlike collapse or finite observers, these invariants capture asymptotic
properties of the selector stream and therefore reveal structural features
that survive tail modification but remain invisible to continuous projections.

The first chapter presents the entropy balance $\eta$ and the fluctuation
index $\phi$.  
The balance measures the lower asymptotic density of digit exposures, while
the fluctuation index measures the growth of relative gaps between selected
positions.  
These invariants are discontinuous but satisfy natural semicontinuity
properties.  
Their values vary widely inside collapse fibers, which illustrates the
symbolic diversity hidden beneath classical magnitude.

The second chapter introduces geometric embeddings based on these invariants.
Plotting generative identities in the $(\eta, \phi)$ plane reveals large scale
structure in selector behavior.  
Hybrid and null density selectors occupy distinct regions, and identities with
high or low fluctuation index appear at very different geometric scales.
Higher dimensional embeddings are also possible using block statistics, gap
growth rates, or meta stream behavior.

The final chapter synthesizes the framework and outlines future directions.
Extended invariants and geometric embeddings provide new perspectives on the
generative representation of real numbers and suggest further investigation of
higher order invariants, connections to symbolic dynamics, and interactions
with computability and randomness.

Part VI therefore shows how generative identities can be analyzed using
structural, asymptotic, and geometric coordinates that lie beyond collapse and
beyond the reach of finite observers.
 \clearpage{}\chapter{Extended Invariants: Entropy Balance and Fluctuation}

\section{Introduction}

Extended invariants measure large scale features of the selector stream that
survive tail modification and reveal structure invisible to the collapse map.
Two such invariants are the entropy balance $\eta$, which measures the lower
asymptotic density of digit exposures, and the fluctuation index $\phi$, which
measures the relative growth of gaps between selected positions.  

In this chapter we introduce these invariants, establish basic properties,
prove semicontinuity, and place them in the slice geometry of the generative
space.  
Selector behavior may be analyzed through vertical slices (fixed prefixes),
horizontal slices (fixed invariant values), and fiber slices (fixed collapsed
value).  
Appendix~E contains many worked examples illustrating these slices and the
full range of possible behaviors.

\section{Entropy Balance}

Let $G = (M,D,K)$ be a generative identity and write
\[
\chi_M(n) =
\begin{cases}
1 & \text{if } M(n) = D, \\
0 & \text{otherwise}.
\end{cases}
\]

The entropy balance is the lower asymptotic density of digit exposures:
\[
\eta(G)
  = \liminf_{N\to\infty}
      \frac{1}{N} \sum_{n < N} \chi_M(n).
\]

A selector with $\eta(G) > 0$ exposes digits frequently, while $\eta(G) = 0$
indicates sparse exposure.  
Balance is invariant under tail modification beyond a finite prefix and
depends only on the selector.

\subsection{Lower semicontinuity}

The balance is not continuous in the product topology, but it satisfies a one
sided bound.

\begin{proposition}[Lower Semicontinuity]
If $G_k \to G$ in the product topology, then
\[
\eta(G) \le \liminf_{k\to\infty} \eta(G_k).
\]
\end{proposition}

\begin{proof}
Fix $\varepsilon > 0$ and choose $N$ such that
\[
\frac{1}{N}\sum_{n<N} \chi_M(n)
    < \eta(G) + \varepsilon.
\]
For $k$ large enough, $G_k$ agrees with $G$ on the first $N$ coordinates, so
\[
\eta(G_k)
  \ge \frac{1}{N}\sum_{n<N} \chi_M(n)
  > \eta(G) - \varepsilon.
\]
Taking the limit inferior yields the claim.
\end{proof}

\section{Fluctuation Index}

Let the selection indices be
\[
n_0 < n_1 < n_2 < \cdots,
\qquad
g_j = n_{j+1} - n_j.
\]

The fluctuation index measures the growth of relative gaps:
\[
\phi(G)
  = \limsup_{j\to\infty} \frac{g_j}{n_j}.
\]

Large $\phi(G)$ indicates the presence of long, infrequent bursts of digit
exposure relative to scale.

\subsection{Upper semicontinuity}

\begin{proposition}[Upper Semicontinuity]
If $G_k \to G$, then
\[
\phi(G) \ge \limsup_{k\to\infty} \phi(G_k).
\]
\end{proposition}

\begin{proof}
Suppose $\phi(G) < c$.  
Then for sufficiently large $j$,
\[
g_j < c n_j.
\]
Agreement of $G_k$ with $G$ on a large enough prefix ensures that the first
$J$ selected digits occur at the same positions.  
Thus for all sufficiently large $k$,
\[
\phi(G_k) < c.
\]
Taking the limit superior proves the claim.
\end{proof}

\section{Slice Geometry of Selector Behavior}

Extended invariants permit a geometric interpretation of the generative space
through slices that constrain different aspects of selector behavior.  
These slices provide intuition for how invariants, collapse fibers, and finite
prefixes interact.

\subsection{Vertical slices: cylinder sets}

A vertical slice fixes a finite prefix:
\[
\mathcal{C}(u)
  = \{ G \in \mathcal{X}^{*} : G[0..N-1] = u \}.
\]

These sets represent the regions observable by structural projections.  
Dependency bounds show that each observer samples only one vertical slice at a
time, so vertical slices encode the finite informational geometry underlying
incompleteness.

Vertical slices impose no restriction on $\eta$ or $\phi$, and their images
under the map $G \mapsto (\eta(G),\phi(G))$ can cover large regions of the
invariant plane.

\subsection{Horizontal slices: invariant level sets}

Fix $\alpha$ or $\beta$.  
Define
\[
\mathcal{H}_{\alpha}
  = \{ G : \eta(G) = \alpha \},
\qquad
\mathcal{H}^{\beta}
  = \{ G : \phi(G) = \beta \}.
\]

These sets identify identities with the same long term selector behavior even
if their finite prefixes differ.  
Horizontal slices cut across collapse fibers and vertical slices, illustrating
the independence of asymptotic structure from local structure.

In the invariant plane, these slices appear as vertical or horizontal lines.

\subsection{Fiber slices: fixing collapsed value}

Fix a real number $x$.  
The fiber slice is
\[
\mathcal{F}(x)
  = \{ G : \pi(G) = x \}.
\]

Since $\eta$ and $\phi$ depend only on the selector, not on collapse, the
image of $\mathcal{F}(x)$ in the invariant plane typically occupies a broad
region.  
This illustrates how collapse conceals most of the selector structure.

Appendix~E contains examples showing that for any pair $(\alpha,\beta)$, there
exists an identity in $\mathcal{F}(x)$ with $\eta = \alpha$ and $\phi =
\beta$.

\section{Selector Behavior Through Examples}

Appendix~E provides detailed examples demonstrating selectors with:

\begin{itemize}
    \item positive balance and low fluctuation,
    \item zero balance and bounded fluctuation,
    \item zero balance and unbounded fluctuation,
    \item oscillating densities,
    \item constructed pairs $(\eta,\phi)$ with prescribed values.
\end{itemize}

These examples show that extended invariants are flexible tools for describing
large scale selector structure.  
They also demonstrate that collapse fibers contain identities with all
admissible invariant values.

\section{Summary}

The entropy balance and fluctuation index provide numerical lenses through
which to view long term selector behavior.  
Their semicontinuity properties match their intuitive roles: balance is hard
to increase by small perturbations, while fluctuation is hard to decrease.  
Slice geometry offers a conceptual framework for understanding how vertical
prefix constraints, horizontal invariant constraints, and fiber constraints
interact.

Together with the examples of Appendix~E, these tools give a geometric
understanding of selector behavior that complements the structural and
computational perspectives developed in earlier parts of the monograph.
\clearpage{}
\clearpage{}\chapter{Synthesis and Outlook}

\section{Introduction}

The Generative Identity Framework offers a structural perspective on real
numbers that complements the usual analytic and combinatorial viewpoints.
Generative identities represent real numbers as collapsed outputs of symbolic
mechanisms composed of a selector stream, a digit stream, and a
meta-information stream.  
The collapse map extracts classical magnitude while discarding the majority of
the symbolic structure.  
This fundamental asymmetry between internal structure and classical value
drives the main results of the monograph.

In this final chapter we synthesize the central components of the framework
and outline directions for future research.  
The focus is not on summarizing all results but on clarifying the conceptual
roles played by the generative space, collapse fibers, projection theory, and
extended invariants.

\section{Collapse and Reconstruction}

A generative identity $G = (M, D, K)$ contains significantly more information
than its collapsed value $\pi(G)$.  
The selector identifies which digits of $D$ contribute to the canonical
output, while the meta stream carries additional symbolic content that is
completely invisible under collapse.

The collapse map is continuous, surjective, and highly non-injective.  
It identifies vast families of generative identities that share the same
canonical digit sequence.  
Reconstruction is therefore impossible: collapse fibers contain uncountably
many identities that differ in selector behavior, meta-information content,
and unobserved digits.  
The diagonalizer shows that much of this structure is irretrievably hidden
from finite observation.

\section{Effective Fibers and Observation}

The effective fiber $\mathcal{F}_{\mathrm{eff}}(x)$ associated with a computable
real number $x$ is a nonempty $\Pi^0_1$ class.  
It contains identities with a wide range of selector patterns and meta
streams.  
Continuous observers depend only on finite prefixes of the identity at any
fixed precision, and this finite information principle is the basis of
incompleteness.

The diagonalizer constructed in Part IV demonstrates that no finite family of
observers can distinguish all identities in the fiber.  
The Structural Incompleteness Theorem formalizes this into a general
statement: finite observation cannot recover the generative identity from its
collapsed value.

\section{Extended Invariants}

Extended invariants measure large scale features of the selector stream.  
Two such invariants, the entropy balance $\eta$ and the fluctuation index
$\phi$, capture long term density and relative gap size.  
These invariants are discontinuous but satisfy natural semicontinuity
properties.  
They provide a coarse geometric lens through which to view the generative
space.

Collapse fibers contain identities with all permitted values of $\eta$ and
$\phi$, which shows how little the collapse mechanism constrains selector
behavior.  
The embedding of identities into the $(\eta, \phi)$ plane illustrates the
diversity of generative structure that persists even after collapse.

\section{Generative Geometry}

The geometric viewpoint introduced in Chapter 13 suggests that extended
invariants may form coordinate axes in higher dimensional generative spaces.
Selectors may be analyzed through growth rates of gaps, block frequencies, or
meta-stream patterns.  
These invariants have the potential to organize the generative space along
new dimensions, providing refined classifications that go beyond collapse and
beyond the invariants introduced here.

Although the present framework focuses on selectors, similar geometric tools
could be applied to digit streams or meta streams.  
For example, meta-information could encode symbolic constraints, local
dependencies, or even probabilistic features.  
These possibilities point toward a broader program of generative analysis.

\section{Future Directions}

The results of this monograph raise several avenues for further study.

\subsection*{1. Higher order invariants}

Extended invariants may be generalized by considering block statistics,
empirical measures on the selector stream, or dimension-like quantities that
reflect scaling behavior.  
Understanding how these higher order invariants interact with collapse fibers
could lead to new forms of structural classification.

\subsection*{2. Connections to symbolic dynamics}

Selectors define subshifts of $\{D, K\}^{\mathbb{N}}$ with varying levels of
regularity.  
Interpreting generative identities as points in shift spaces may reveal
dynamical properties of collapse fibers and new connections to thermodynamic
formalism.

\subsection*{3. Computability and randomness}

The diagonalizer highlights the computational limits of observers.  
Investigating the interaction between selector behavior and algorithmic
randomness may clarify the relationship between generative structure and
Martin-Lof randomness in digitally represented reals.

\subsection*{4. Geometric and analytic embeddings}

Embedding generative identities into higher dimensional geometric spaces could
provide new ways of visualizing and classifying internal structure.  
Such embeddings may reveal patterns or invariants not captured by the
collapse map or the low dimensional coordinates introduced here.

\section{Conclusion}

The Generative Identity Framework provides a unified structure for analyzing
real numbers through symbolic generative mechanisms.  
Collapse reveals classical magnitude, while the internal behavior of
selectors, digits, and meta streams encodes a rich array of structural
information.  
Finite observation cannot recover this information.  
The collapse quotient hides far more than it reveals.

Extended invariants and geometric embeddings open the door to deeper study of
generative structure.  
They suggest that real numbers can be understood not only through magnitude,
dimension, or randomness, but also through the behavior of symbolic
mechanisms that generate them.

The framework developed here is only a beginning.  
It provides conceptual foundations and technical tools for a broader program
of generative analysis, one that aims to understand the continuum not simply
as a set of magnitudes but as the image of a vast symbolic space.
\clearpage{}
\chapter{The Fluctuation Index as a Tertiary Invariant}

\section{Introduction}

Entropy balance $\eta(G)$ measures \emph{how often} a selector chooses the digit
layer.  
But many selectors with the same digit density behave very differently:
some distribute digit selections uniformly, while others place them in bursts
separated by long gaps.

To capture this higher-order structure, we introduce the \emph{fluctuation
index}, a tertiary invariant that measures the irregularity or dispersion of
digit selections.  
This invariant refines entropy balance in the same way that variance refines
mean: it distinguishes selectors with identical limiting densities but
different internal patterns.

The fluctuation index satisfies all criteria for extended invariants introduced
in Chapter~12.  
It is continuous, prefix-determined, computable on the effective core, and
sensitive to structure invisible to entropy balance and collapse.  
It will serve as one of the key coordinates in the extended generative space
developed in Chapters 15 and 16.

\section{Gap Sequences and Dispersion}

Let $G = (M,D,K)$ be a generative identity.  
Write
\[
S_M = \{ n_0 < n_1 < n_2 < \cdots \}
\]
for the positions where $M$ selects the digit layer.  
Define the \emph{gap sequence}
\[
g_j = n_{j+1} - n_j.
\]

Digit density depends only on the asymptotic cardinality of $S_M$; the gap
sequence captures its \emph{shape}.  
Small gaps correspond to uniform usage; large gaps indicate bursts of meta-layer
dominance.

\section{Finite-Prefix Fluctuation}

We begin with a finite version that is well-defined on prefixes.

For any $n$ such that $S_M\cap[0,n)$ contains at least two digit selections,
define the partial gap sequence
\[
g_j^{(n)} = n_{j+1} - n_j
\quad\text{for } n_{j+1} < n.
\]

Let
\[
\Phi_n(G)
=
\begin{cases}
\max_j g_j^{(n)}, & \text{if at least two gaps appear in }[0,n),\\
n, & \text{otherwise}.
\end{cases}
\]

The value $\Phi_n(G)$ measures the largest digit-free region within the first
$n$ positions.  
Taking $\Phi_n(G)=n$ in the degenerate case ensures monotonicity and prefix
dependence.

\section{Definition of the Fluctuation Index}

\begin{definition}[Fluctuation Index]
The \emph{fluctuation index} of $G$ is
\[
\phi(G)
=
\limsup_{n\to\infty}
\frac{\Phi_n(G)}{n}.
\]
\end{definition}

Thus $\phi(G)$ measures the normalized size of the largest digit-free portion of
the initial segment.  
Values near zero indicate uniformity; values near one indicate extreme
irregularity or sparsity.

\section{Continuity and Prefix Dependence}

\begin{proposition}
The fluctuation index $\phi(G)$ is a prefix-determined, continuous structural
projection.
\end{proposition}

\begin{proof}
Fix $\varepsilon>0$.  
To determine whether $\phi(G)$ exceeds a threshold $\alpha$, it suffices to
inspect all gap lengths in the prefix of length $N = \lceil 1/\varepsilon\rceil
$.  
Agreement on this prefix ensures that the normalized $\Phi_n(G)/n$ values differ
by at most $\varepsilon$ for all $n \ge N$, proving continuity and prefix
determination.
\end{proof}

\section{Computability}

\begin{proposition}
The fluctuation index $\phi$ is computable on $\mathcal{G}_{\mathrm{eff}}$.
\end{proposition}

\begin{proof}
Given a computable selector $M$, we can compute all gap lengths $g_j^{(n)}$
within the prefix $M{\upharpoonright}n$.  
Thus $\Phi_n(G)$ is computable.  
Since $\phi(G)$ is obtained as the $\limsup$ of computable rational numbers
$\Phi_n(G)/n$, it is Type--2 computable.
\end{proof}

\section{Relationship to Entropy Balance}

Entropy balance and fluctuation index measure complementary aspects of the
selector.

\begin{itemize}
    \item $\eta(G)$ captures the \emph{overall frequency} of digit usage.
    \item $\phi(G)$ captures the \emph{distributional irregularity} of digit usage.
\end{itemize}

\begin{proposition}
$\eta(G)$ does not determine $\phi(G)$, and $\phi(G)$ does not determine
$\eta(G)$.
\end{proposition}

\begin{proof}
Selectors with identical densities may differ arbitrarily in the size of gaps;
similarly, sequences with identical gap structure may differ in density by
increasing or decreasing digit selections uniformly.
\end{proof}

Thus $\eta$ and $\phi$ are independent coordinates in the extended space.

\section{Behavior Inside Collapse Fibers}

As with entropy balance, fluctuation index varies fully within each collapse
fiber.

\begin{proposition}
For every $x\in[0,1]$ and $\alpha\in[0,1]$, there exists
$G\in\mathcal{F}(x)$ such that $\phi(G)=\alpha$.
\end{proposition}

\begin{proof}
Construct selectors with gap sequences that achieve maximal gap proportions
corresponding to $\alpha$, and align digit selections to the expansion of $x$.
\end{proof}

In the effective setting:

\begin{proposition}
If $x\in\mathbb{R}_c$ and $\alpha\in\mathbb{Q}\cap[0,1]$, then
$\mathcal{F}_{\mathrm{eff}}(x)$ contains an effective generator $G$ with
$\phi(G)=\alpha$.
\end{proposition}

\begin{proof}
Use periodic or computably sparse selectors whose gap structure realizes
$\alpha$, then assign digit and meta coordinates computably as in earlier
constructions.
\end{proof}

Thus $\phi$ is a refining invariant and a genuine tertiary coordinate.

\section{Projection-Lattice Structure}

\begin{proposition}
The fluctuation index is the supremum (in the refinement order) of the family
of projections
\[
G \longmapsto \frac{\Phi_n(G)}{n}.
\]
\end{proposition}

\begin{proof}
The $\limsup$ operation yields the least upper bound in the refinement order:
any projection that dominates each $\Phi_n(G)/n$ must dominate their
$\limsup$ as well.
\end{proof}

This positions $\phi$ naturally within the projection lattice:  
entropy balance is an infimum of finite-frequency projections, while fluctuation
index is a supremum of gap-size projections.

\section{Compatibility with Diagonalization}

Fluctuation index is sensitive to highly local changes in gap structure but is
still prefix-determined.  
Thus the diagonalizer of Chapter~9 can be adapted to preserve $\phi(G)$ while
evading any finite family of other projections.

\begin{proposition}
For any computable $\alpha\in[0,1]$ and any computable real $x$, there exists a
diagonalizing mechanism $G^\#\in\mathcal{F}_{\mathrm{eff}}(x)$ with
$\phi(G^\#)=\alpha$.
\end{proposition}

\begin{proof}
Choose a tail identity $A$ with $\phi(A)=\alpha$, and sew it into the
diagonalizer construction.  
Digit-index alignment preserves collapse, and prefix stability preserves gap
structure at all prescribed scales.
\end{proof}

\section{Outlook}

The fluctuation index enriches the generative coordinate system beyond entropy
balance.  
Selectors with identical frequency patterns can have vastly different
irregularity profiles, and $\phi(G)$ captures this tertiary layer of structure.

Chapter~15 combines $\eta$ and $\phi$ into a two-dimensional extended coordinate
system, drawing an analogy with the classical complex plane:  
$\pi(G)$ corresponds to the “real axis,” while $\eta(G)$ or $\phi(G)$ act as
imaginary directions that restore structure lost under collapse.

\chapter{Orthogonal Extensions and the Complex Analogy}

\section{Introduction}

Collapse extracts a single coordinate from a generative identity: its classical
magnitude.  
Entropy balance and fluctuation index extract structural information that
collapse discards.  
Together, these invariants begin to form a coordinate system on the generative
space, revealing a geometry richer than the one-dimensional continuum obtained
from the collapse quotient.

The purpose of this chapter is to formalize a conceptual analogy:  
\emph{adding a secondary invariant to collapse is analogous to extending the
real line to a plane}.  
This is not an isomorphism of structures, but a geometric metaphor: the
classical real number $\pi(G)$ is one coordinate, and an extended invariant
(such as $\eta(G)$ or $\phi(G)$) provides an orthogonal direction that restores
structure lost under collapse.

We develop this analogy rigorously by constructing two-dimensional embeddings of
the generative space.  
These embeddings highlight how extended invariants enrich the generative
representation without overcoming the fundamental limitations imposed by
structural incompleteness.

\section{Collapse as a One-Dimensional Projection}

The collapse map
\[
\pi : \mathcal{X}^* \to [0,1]
\]
is a structural projection that forgets nearly all internal structure.  
Viewed geometrically, collapse captures only the ``horizontal’’ coordinate of a
generative identity.  
Every collapse fiber is an entire vertical column of mechanisms projecting to
the same point.

\section{Adding a Secondary Coordinate}

Let $I : \mathcal{X}\to\mathbb{R}$ be an extended invariant such as entropy
balance $\eta$ or fluctuation index $\phi$.  
Both are continuous, prefix-determined, and non-collapsing.

We consider the map
\[
G \longmapsto (\pi(G), I(G)) \in \mathbb{R}^2.
\]

\begin{proposition}[Two-Dimensional Embedding]
If $I$ is non-collapsing, then the map
\[
\Theta_I(G) := (\pi(G), I(G))
\]
is an embedding of each collapse fiber into $\mathbb{R}^2$.
\end{proposition}

\begin{proof}
If $G,H\in\mathcal{F}(x)$ with $G\neq H$, then $\pi(G)=\pi(H)=x$ but
$I(G)\neq I(H)$ by non-collapse.  
Thus $\Theta_I$ is injective on the fiber.  
Continuity follows from continuity of $\pi$ and $I$.
\end{proof}

Thus adding a single extended invariant “lifts’’ each collapse fiber into an
interval of vertical values, restoring structure lost in the one-dimensional
collapse.

\section{Orthogonal Extension Analogy}

We now explain the complex-plane analogy carefully and rigorously.

\subsection*{The real line}

In classical mathematics:
\[
\mathbb{R} \quad\text{is one-dimensional.}
\]

\subsection*{The complex plane}

The complex plane arises by adding an orthogonal direction:
\[
\mathbb{C} = \mathbb{R} \oplus i\mathbb{R}.
\]

Geometrically this means:
- same horizontal coordinate (real part),
- second, independent vertical coordinate (imaginary part).

\subsection*{The generative analogy}

In the generative setting:

- $\pi(G)$ plays the role of the “horizontal” coordinate,
- an extended invariant $I(G)$ plays the role of a “vertical” coordinate.

The analogy is:

\[
\text{Collapse-only: } G \mapsto \pi(G)
\quad\leadsto\quad
\text{One-dimensional real axis.}
\]

\[
\text{Extended coordinates: } G \mapsto (\pi(G), I(G))
\quad\leadsto\quad
\text{Plane-like embedding restoring lost structure.}
\]

\begin{remark}
This analogy is conceptual:  
we are not claiming that $(\pi,I)$ forms a field, a vector space, or an
algebraic closure.  
The analogy concerns dimensional enrichment, not algebraic structure.
\end{remark}

\section{Choosing $I=\eta$ or $I=\phi$}

Both entropy balance and fluctuation index provide valid orthogonal extensions.

\subsection*{Entropy balance plane}

The map
\[
G \longmapsto (\pi(G),\eta(G))
\]
produces a plane in which:

- horizontal axis: classical magnitude,
- vertical axis: frequency of digit selections.

\begin{itemize}
    \item Hybrid identities ($\eta>0$) appear in the positive vertical region.
    \item Null-density identities ($\eta=0$) lie on the horizontal axis.
    \item Each classical real $x$ corresponds to the vertical line
    $\{x\}\times[0,1]$.
\end{itemize}

\subsection*{Fluctuation plane}

The embedding
\[
G \longmapsto (\pi(G),\phi(G))
\]
produces a different slice of structure:

- horizontal axis: magnitude,
- vertical axis: irregularity or dispersion.

These two planes emphasize complementary aspects of the generative space.

\section{Higher-Dimensional Embeddings}

We may combine multiple invariants:

\[
G \longmapsto (\pi(G), \eta(G), \phi(G)) \in \mathbb{R}^3.
\]

This embedding distinguishes:

\begin{itemize}
    \item magnitude (collapse),
    \item average digit density,
    \item long-run selector irregularity.
\end{itemize}

Many distinct invariants—meta-frequency statistics, pattern densities, or
computable subshift entropies—can be added as further axes.

\section{Limits of Dimensional Restoration}

The Structural Incompleteness Theorem remains in force.

\begin{proposition}
No finite-dimensional embedding
\[
\mathcal{X} \longrightarrow \mathbb{R}^d
\]
using computable structural projections is injective on any effective fiber.
\end{proposition}

\begin{proof}
Each coordinate of such an embedding is a computable structural projection.
Apply the Structural Incompleteness Theorem to the finite family of these
projections.
\end{proof}

Thus extended invariants enrich generative coordinates but cannot fully recover
the lost structure.  
This limitation mirrors the fact that the complex plane adds only one new axis
to the real line; it does not recover all structure lost in collapsing
$\mathbb{R}^2$ onto $\mathbb{R}$.

\section{Interpretation}

The analogy with the complex plane should be understood as follows:

\begin{quote}
Adding an independent invariant to $\pi$ produces a two-dimensional coordinate
system, just as adding an imaginary coordinate extends the real line to the
complex plane.  
Both enrich the representational landscape, revealing structure invisible to the
original projection.
\end{quote}

This conceptual picture clarifies the role of extended invariants: they provide
orthogonal directions in the generative geometry, expanding the classical
representation into a richer multidimensional framework.

\section{Outlook}

The final chapter, Chapter~16, investigates the geometry of extended
coordinates.  
Even as more invariants are added, the ability to recover structure diminishes
rapidly.  
Part~VI concludes by analyzing this phenomenon and explaining why the
generative framework supports an expanding hierarchy of invariants but no
finite system can fully classify generative identities.

\chapter{Diminishing Returns and Final Outlook}

\section{Introduction}

Part~VI introduced extended generative invariants—quantities such as entropy
balance and fluctuation index that recover aspects of structure lost under the
collapse map.  
These invariants enrich the generative coordinate system and allow us to embed
each collapse fiber into multidimensional spaces.  
Entropy balance captures long-term selector frequencies; fluctuation index
captures irregularity and dispersion; other invariants may measure
meta-patterns, combinatorial complexity, or effective entropy.

But Part~IV showed that no \emph{finite} system of computable invariants can
classify an effective collapse fiber.  
This tension creates a geometric phenomenon at the core of the generative
framework:  

\begin{quote}
\emph{Each new invariant recovers genuine structure—but the amount of structure
it can recover decreases rapidly as the number of invariants grows.}
\end{quote}

This chapter formalizes and interprets this phenomenon of \emph{diminishing
returns}.  
We conclude by synthesizing the entire generative viewpoint and outlining
possible directions for further research.

\section{The Geometry of Successive Refinements}

Let $\pi$ be collapse, and let
\[
I_1, I_2, \ldots, I_r
\]
be extended invariants (computable structural projections) added as higher
coordinates of the generative space.

Define the extended coordinate map
\[
\Theta_r(G) = (\pi(G), I_1(G), \ldots, I_r(G)).
\]

Each new invariant refines the fiber structure by identifying distinctions that
previous invariants do not capture.

However, the Structural Incompleteness Theorem implies:

\begin{quote}
For every finite $r$, there remain infinitely many effective generators that
$\Theta_r$ cannot distinguish.
\end{quote}

To understand the geometry, consider the fiber $\mathcal{F}_{\mathrm{eff}}(x)$
for a computable real $x$.

\section{Fiber Shrinkage Under Added Coordinates}

Adding one invariant collapses the fiber from a $\Pi^0_1$ class of infinite
size to a smaller (but still infinite) subset.  
Adding more invariants continues to shrink the fiber.

Let
\[
F_r(x) = \{ G \in \mathcal{F}_{\mathrm{eff}}(x) : \Theta_r(G) \text{ is fixed} \}.
\]

Then:

\begin{enumerate}
    \item $F_0(x) = \mathcal{F}_{\mathrm{eff}}(x)$ (only collapse is fixed),
    \item $F_1(x)$ (fixing $\eta$) is infinite,
    \item $F_2(x)$ (fixing both $\eta$ and $\phi$) is infinite,
    \item \ldots, and for any finite $r$, $F_r(x)$ is infinite.
\end{enumerate}

Thus each new invariant reduces—but never eliminates—the fiber’s internal
degrees of freedom.

\section{Projection-Lattice Interpretation}

In the projection lattice of Chapter~6:

- collapse $\pi$ is a coarse projection,
- each extended invariant $I_j$ refines the lattice by intersecting prefix
constraints,
- but the intersection of finitely many computable constraints is always too
coarse to produce a singleton.

This yields a lattice-theoretic restatement of diminishing returns:

\begin{proposition}
Let $\{\Phi_1,\ldots,\Phi_r\}$ be computable structural projections.  
Then their meet
\[
\Phi_1 \wedge \cdots \wedge \Phi_r
\]
is never injective on $\mathcal{F}_{\mathrm{eff}}(x)$ for any computable real
$x$.
\end{proposition}

Thus no finite meet of projections resolves all internal structure.

\section{Asymptotic Exhaustion of Structure}

We may view the sequence of refined fibers
\[
F_0(x) \supseteq F_1(x) \supseteq F_2(x) \supseteq \cdots
\]
as a descending chain of computably closed sets.  
Each step removes some ambiguity, but cannot eliminate it entirely.

\begin{proposition}
For any computable real $x$, the intersection
\[
\bigcap_{r=0}^{\infty} F_r(x)
\]
contains infinitely many effective generators.
\end{proposition}

\begin{proof}[Sketch]
If the intersection were finite—let alone a singleton—then a finite stage of
the coordinate system would already be injective, contradicting structural
incompleteness.  
The diagonalizer ensures infinitely many identities remain indistinguishable by
any finite set of invariants.
\end{proof}

Thus even an infinite hierarchy cannot resolve all structure if restricted to
computable invariants with finite prefix dependence.

\section{Interpretation: Dimensional Saturation}

Extended invariants provide “orthogonal directions’’ that lift collapse fibers
into higher-dimensional coordinate systems.  
But these axes suffer a phenomenon analogous to diminishing returns:

- The first axis ($\eta$) reveals a large amount of structure.  
- The second axis ($\phi$) reveals additional but less dramatic structure.  
- Further axes reveal still finer distinctions, but each contributes less than
the axes before it.

This resembles the spectral decay seen in principal-component analyses or the
entropy reduction curves in coding theory: the first coordinates dominate the
information content.

\section{Collapse as the Limiting Shadow}

The generative viewpoint can now be summarized:

\begin{enumerate}
    \item A generative identity contains enormous symbolic structure.
    \item Collapse forgets nearly all of it.
    \item Extended invariants retrieve systematic fragments of that lost
    structure.
    \item No finite set of invariants can reverse collapse.
    \item Even an unbounded sequence of computable invariants cannot fully
    classify fibers.
\end{enumerate}

Collapse is therefore a limiting shadow of a high-dimensional generative space:
extended invariants brighten the shadow but cannot fully reconstruct the
original object.

\section{Final Outlook}

The generative framework opens several directions for future research:

\begin{itemize}
    \item \textbf{Infinite Coordinate Systems.}  
    What happens if one considers transfinite or noncomputable invariants?

    \item \textbf{Measure-Theoretic Generative Models.}  
    How do extended invariants behave under probabilistic generative processes?

    \item \textbf{Operator Theory on Generative Space.}  
    Can one define linear or nonlinear operators acting on $(M,D,K)$-space
    that respect collapse and extended coordinates?

    \item \textbf{Descriptive-Set-Theoretic Complexity.}  
    What is the exact complexity of effective fibers, and how do extended
    invariants alter this classification?

    \item \textbf{Geometry of Extended Embeddings.}  
    Do extended invariants produce well-structured manifolds or fractal
    geometries inside $\mathbb{R}^d$?
\end{itemize}

The overarching insight of this monograph is that classical real numbers
represent the collapse shadow of a richer generative world.  
Extended invariants illuminate fragments of this world, but the symbolic
geometry underlying generative identities remains fundamentally higher
dimensional and resistant to finite classification.

Collapse is only the beginning; the generative structure continues far beyond.


\appendix
\clearpage{}\chapter{Type--2 Effectivity Essentials}

\section{Introduction}

This appendix summarizes the basic notions from Type--2 Effectivity (TTE) and
computable analysis that are used implicitly in the main text.  
The aim is not to give a complete treatment, but to explain the background
needed for structural projections, dependency bounds, effective fibers, and
the diagonalizer.  
Standard references include Weihrauch's monograph on computable analysis, the
work of Brattka, Hertling, and Weihrauch on represented spaces, and Pauly's
surveys on synthetic descriptive set theory.

We focus on three themes:

\begin{enumerate}
    \item names for real numbers and elements of product spaces,
    \item computable functionals on sequence spaces and their moduli of continuity,
    \item effective closed sets and $\Pi^{0}_{1}$ classes.
\end{enumerate}

Throughout, $\mathbb{N} = \{0,1,2,\ldots\}$ and sequences are indexed from
zero.

\section{Names and Represented Spaces}

\subsection{Baire space and Cantor space}

\emph{Baire space} is the set $\mathbb{N}^{\mathbb{N}}$ of all infinite
sequences of natural numbers.  
\emph{Cantor space} is the set $\{0,1\}^{\mathbb{N}}$ of infinite binary
sequences.  
Both spaces carry the product topology generated by basic open sets determined
by finite prefixes.

Baire and Cantor space are the standard domains for TTE.  
Elements of more complicated spaces, such as real numbers or continuous
functions, are represented by infinite sequences in these spaces.

\subsection{Represented spaces}

A \emph{represented space} is a pair $(X,\delta_X)$ where $X$ is a set and
\[
\delta_X : \subseteq \mathbb{N}^{\mathbb{N}} \to X
\]
is a partial surjective map.  
Elements $p \in \mathbb{N}^{\mathbb{N}}$ with $\delta_X(p) = x$ are called
\emph{names} of $x$.

Different representation maps encode different ways of describing objects in
$X$.  
In computable analysis, the usual Cauchy representation of real numbers is
obtained by interpreting $p \in \mathbb{N}^{\mathbb{N}}$ as a rapidly
converging sequence of rational approximations to a real number.

\subsection{Computable points}

A point $x \in X$ is \emph{computable} if it has a computable name, that is,
there is a computable sequence $p \in \mathbb{N}^{\mathbb{N}}$ such that
$\delta_X(p) = x$.

In the Generative Identity Framework, the effective core
$\mathcal{G}_{\mathrm{eff}}$ of the generative space plays the role of
computable elements.  
Each generative identity $G = (M,D,K)$ corresponds to a name that interleaves
the symbols of the three streams in a computable way, and vice versa.

\section{Type--2 Machines and Computable Maps}

\subsection{Type--2 Turing machines}

A Type--2 Turing machine is an abstract device that reads and writes infinite
sequences.  
It has:

\begin{itemize}
    \item a read only input tape containing $p \in \mathbb{N}^{\mathbb{N}}$,
    \item a write only output tape on which it produces $q \in \mathbb{N}^{\mathbb{N}}$,
    \item a work tape for finite internal computation.
\end{itemize}

The machine is required to produce each output symbol $q(n)$ after reading
only finitely many symbols of the input.  
This finite use condition ensures that the induced map on Baire space is
continuous in the product topology.

\subsection{Computable maps between represented spaces}

Let $(X,\delta_X)$ and $(Y,\delta_Y)$ be represented spaces.  
A (partial) function $f : X \to Y$ is \emph{computable} if there exists a
Type--2 Turing machine $M$ such that, for every $p$ with $\delta_X(p) = x$,
the output sequence $M(p)$ is defined and satisfies
\[
\delta_Y(M(p)) = f(x).
\]

Intuitively, given infinite access to a name of $x$, the machine produces a
name of $f(x)$ using only finite prefixes of the input at each step.

\subsection{Continuity and computability}

A basic theorem of TTE states that every computable function between
represented spaces is continuous with respect to the induced topologies.  
In many natural representations, the converse holds as well: any continuous
function with an effective modulus of continuity is computable.

In this monograph, structural projections are continuous real valued
functionals on a product space of symbolic sequences.  
When such functionals are computable, they admit effective moduli of
continuity that appear as dependency bounds.

\section{Moduli of Continuity and Dependency Bounds}

\subsection{Moduli of continuity on sequence spaces}

Let $f : \mathbb{N}^{\mathbb{N}} \to \mathbb{R}$ be continuous in the product
topology.  
For each $\varepsilon > 0$ there exists an $N$ such that any two sequences
that agree on their first $N$ terms have $f$ values within $\varepsilon$.

A \emph{modulus of continuity} is a function
\[
\mu : (0,1] \to \mathbb{N}
\]
such that agreement on the first $\mu(\varepsilon)$ indices implies
\[
|f(p) - f(q)| < \varepsilon.
\]
If $f$ is computable, then $\mu$ can be chosen to be computable as well.

\subsection{Dependency bounds in the generative space}

The generative space $\mathcal{X}^{*}$ is a product of discrete alphabets,
equipped with the product topology.  
A structural projection
\[
\Phi : \mathcal{X}^{*} \to \mathbb{R}
\]
is continuous if and only if there exists a function $B_{\Phi}$ such that
agreement of generative identities $G$ and $H$ on their first
$B_{\Phi}(\varepsilon)$ coordinates implies
\[
|\Phi(G) - \Phi(H)| < \varepsilon.
\]

In the main text, $B_{\Phi}$ is called a \emph{dependency bound}.  
This is exactly a modulus of continuity for $\Phi$ in the sense of TTE,
presented in a way that emphasizes its combinatorial meaning: the value of
$\Phi(G)$ to precision $\varepsilon$ depends only on a finite prefix of the
identity.

For a finite family of projections, a common bound is obtained by taking the
maximum of the individual bounds.  
This yields a uniform dependency bound that controls the entire family.

\subsection{Prefix stabilization and tail invariance}

If $B_{\Phi}$ is a dependency bound for $\Phi$, then for any $\varepsilon$,
agreement on the prefix of length $B_{\Phi}(\varepsilon)$ guarantees that
changes to the tail beyond this prefix cannot alter the value of $\Phi$ by
more than $\varepsilon$.

This prefix stabilization property is used repeatedly in Part~III and
Part~IV.  
It shows that continuous observers consume only finitely much information at
any fixed precision, which in turn permits tail modifications that preserve
all observations in a given finite family.

\subsection{Effective fibers as \texorpdfstring{$\Pi^{0}_{1}$}{Pi01} classes}

\subsection{Effective open and closed sets}

A subset $U \subseteq \mathbb{N}^{\mathbb{N}}$ is \emph{effectively open} (or
$\Sigma^{0}_{1}$) if it is a union of a computably enumerable family of basic
open sets.  
The complement of an effectively open set is \emph{effectively closed}
(or $\Pi^{0}_{1}$).

In Cantor or Baire space, basic open sets are specified by finite prefixes.
Thus an effectively closed set $F$ is one for which membership can be disproved
by finite evidence.  
To see that $p \notin F$, it suffices to find a finite prefix that forces
$p$ into the complement.

\subsection{Effective fibers as \texorpdfstring{$\Pi^{0}_{1}$}{Pi01} classes}

The effective core of the generative space,
\[
\mathcal{G}_{\mathrm{eff}} \subseteq \mathcal{X}^{*},
\]
inherits a natural representation from its presentation as a subspace of a
finite alphabet product.  
The effective fiber associated with a computable real $x$ is
\[
\mathcal{F}_{\mathrm{eff}}(x)
  = \{ G \in \mathcal{G}_{\mathrm{eff}} : \pi(G) = x \}.
\]

This set is a $\Pi^{0}_{1}$ class in the sense that $G \notin
\mathcal{F}_{\mathrm{eff}}(x)$ can be witnessed by a finite prefix where the
canonical output deviates from the prescribed digit sequence of $x$.

In the main text, this perspective is used to justify the existence of
computable identities inside the fiber and to support constructions that
modify tails while preserving membership in the fiber.

\section{Application to the Generative Identity Framework}

The TTE machinery summarized above arises in the monograph in the following
ways.

\begin{itemize}
    \item The generative space $\mathcal{X}^{*}$ is a represented space whose
          elements are generative identities.  
          The effective core $\mathcal{G}_{\mathrm{eff}}$ corresponds to those
          identities that admit computable names.

    \item Structural projections are continuous real valued functionals on
          $\mathcal{X}^{*}$.  
          When such projections are computable, they admit computable
          dependency bounds, which are moduli of continuity in the TTE sense.

    \item Effective fibers $\mathcal{F}_{\mathrm{eff}}(x)$ are $\Pi^{0}_{1}$
          classes of identities that collapse to a fixed real number.  
          The nonemptiness and internal structure of these classes are used in
          the construction of the meta diagonalizer.

    \item Prefix stabilization and tail invariance are direct consequences of
          continuity and the existence of dependency bounds.  
          These properties express the finite information content of
          observations and allow the diagonalizer to introduce divergence
          beyond the reach of any finite family of observers.
\end{itemize}

The framework of represented spaces and Type--2 computability therefore
provides a conceptual foundation for the generative identity framework.  
It explains why structural projections necessarily depend on finite prefixes,
why effective fibers admit rich internal structure, and why diagonalization
against continuous observers is possible.
\clearpage{}
\clearpage{}\chapter{Symbolic Dynamics Essentials}

\section{Introduction}

This appendix summarizes the symbolic dynamics concepts that appear implicitly
throughout the monograph.  
Although the generative identity framework uses selector streams rather than
traditional symbol blocks, many of its structural properties are naturally
expressed using tools from symbolic dynamics.  
The purpose of this appendix is to describe these tools and explain how they
interact with the generative space.

We begin with the shift map and the product topology on sequences.  
We then describe notions of frequency, recurrence, gap statistics, and
residual sets, all of which are used in the analysis of selector behavior.  
The appendix concludes with a discussion of block structures and how they
relate to extended invariants.

\section{Shift Spaces and the Product Topology}

\subsection{Full shift spaces}

Let $\mathcal{A}$ be a finite alphabet.  
The full shift over $\mathcal{A}$ is the space
\[
\mathcal{A}^{\mathbb{N}} = \{ x_0 x_1 x_2 \ldots : x_n \in \mathcal{A} \}.
\]
This space is equipped with the product topology generated by basic open sets
of the form
\[
[x_0 x_1 \ldots x_{k-1}]
  = \{ y \in \mathcal{A}^{\mathbb{N}} : y_i = x_i \text{ for } 0 \le i < k \},
\]
also called cylinder sets.

The product topology makes $\mathcal{A}^{\mathbb{N}}$ compact, totally
disconnected, and metrizable.  
These topological properties are inherited by the generative space
$\mathcal{X}$ and the digit selecting subspace $\mathcal{X}^{*}$.

\subsection{The shift map}

The shift map
\[
\sigma : \mathcal{A}^{\mathbb{N}} \to \mathcal{A}^{\mathbb{N}}
\]
is defined by $(\sigma(x))_n = x_{n+1}$.  
It is continuous, surjective, and preserves the cylinder structure.

In the generative identity framework, one may shift the selector, digit, or
meta stream independently.  
The shift is not used as a dynamical map in the main text, but the structural
intuition provided by shifting plays an important role.  
For example, asymptotic densities and gap statistics are shift invariant
properties.

\subsection{Subshifts}

A \emph{subshift} is a closed, shift invariant subset of $\mathcal{A}^{\mathbb{N}}$.  
Such sets are determined by specifying which finite blocks of symbols are
allowed or forbidden.

Selectors may be viewed informally as elements of the subshift
\[
\{D,K\}^{\mathbb{N}},
\]
and families of selectors with additional structural constraints form natural
subshifts within this space.

\section{Density, Frequencies, and Gap Structure}

\subsection{Lower and upper densities}

For $x \in \mathcal{A}^{\mathbb{N}}$ and $a \in \mathcal{A}$, the lower and
upper densities of the symbol $a$ are given by
\[
\underline{d}_{a}(x)
  = \liminf_{N\to\infty}
      \frac{1}{N} \bigl| \{ 0 \le n < N : x_n = a \} \bigr|,
\]
\[
\overline{d}_{a}(x)
  = \limsup_{N\to\infty}
      \frac{1}{N} \bigl| \{ 0 \le n < N : x_n = a \} \bigr|.
\]

In selector analysis, these become:
\[
\eta(G) = \underline{d}_{D}(M),
\]
the lower asymptotic density of digit exposures.

\subsection{Gap sequences}

Let $x \in \mathcal{A}^{\mathbb{N}}$ and fix a symbol $a \in \mathcal{A}$.  
List the positions at which $x_n = a$ as
\[
n_0 < n_1 < n_2 < \ldots.
\]
Define the gap sequence
\[
g_j = n_{j+1} - n_j.
\]

The \emph{relative gap growth} is the limit superior
\[
\phi(x)
  = \limsup_{j\to\infty} \frac{g_j}{n_j},
\]
which is the definition of the fluctuation index in the main text.

Gap statistics are classical objects in symbolic dynamics and informally
describe how sparsely a symbol may occur.

\subsection{Recurrence and regularity}

A symbol $a$ is \emph{recurrent} in $x$ if it appears infinitely often.  
Every selector in $\mathcal{X}^{*}$ has $D$ recurrent, since otherwise the
canonical output would be finite.

Selectors with positive density of $D$ are regular in the sense that their gap
sequence is bounded above by a linear function.  
Selectors of zero density admit much larger fluctuations.

\section{Block Structures and Empirical Measures}

\subsection{Blocks and patterns}

A block (or word) of length $k$ over $\mathcal{A}$ is an element of
$\mathcal{A}^{k}$.  
The set of blocks appearing in $x \in \mathcal{A}^{\mathbb{N}}$ is
\[
\mathcal{L}(x)
  = \bigcup_{k\ge0}
      \{ x_n x_{n+1} \ldots x_{n+k-1} : n \in \mathbb{N} \}.
\]

Selectors have block structures in $\{D,K\}^{k}$ that reflect which positions
expose digits and which do not.  
These blocks determine fine scale properties of selectors that are not
captured by density or fluctuation alone.

\subsection{Empirical frequency measures}

For a block $u \in \mathcal{A}^{k}$, the empirical frequency up to index $N$
is
\[
\mathrm{freq}_{N}(u,x)
  = \frac{1}{N}
    \bigl| \{ 0 \le n < N-k+1 : x_{n} \ldots x_{n+k-1} = u \} \bigr|.
\]

Empirical frequencies provide refined statistical information about symbolic
sequences.  
Although not used directly in the main text, these measures motivate the
extended invariants discussed in Part VI.

\section{Selector Behavior in Symbolic Terms}

Selectors are symbolic sequences in $\{D,K\}^{\mathbb{N}}$ with the additional
constraint that $D$ must occur infinitely often.  
Many properties of selectors are classical:

\begin{itemize}
    \item positive density selectors correspond to sequences in which $D$
          occurs with positive lower density,
    \item zero density selectors correspond to sparse symbol occurrences,
    \item large fluctuation selectors correspond to sequences with large gaps.
\end{itemize}

These behaviors are well studied in the context of return times, symbolic
recurrence, and sparse subshifts.  
The extended invariants $\eta$ and $\phi$ adapt classical notions to the
selector setting.

\section{Residual Structure and Typicality}

In classical symbolic dynamics, residual sets (dense $G_\delta$ sets) describe
typical behavior under the Baire category notion of genericity.  
Many selector-related properties are generic in the space $\{D,K\}^{\mathbb{N}}$:

\begin{itemize}
    \item irregular gap growth,
    \item oscillating frequencies,
    \item absence of limiting densities.
\end{itemize}

Although the generative identity framework does not rely directly on generic
properties, the irregularity of symbolic sequences supports the observation
that collapse fibers contain many identities with extreme or pathological
selector patterns.

\section{Interaction with the Generative Identity Framework}

The symbolic tools summarized above enter the monograph in the following ways.

\begin{itemize}
    \item Selector streams are symbolic sequences in a full shift
          $\{D,K\}^{\mathbb{N}}$, and their asymptotic properties determine the
          extended invariants $\eta$ and $\phi$.

    \item Lower densities, gap sequences, and limsup statistics are used to
          analyze large scale selector structure.

    \item Block structures and empirical frequency ideas motivate possible
          higher dimensional invariants that extend the geometric picture in
          Part VI.

    \item The product topology on symbolic sequences is the same topology used
          to define continuity of structural projections and to derive
          dependency bounds.

    \item Residual irregularity of symbolic sequences illustrates why collapse
          fibers contain identities with many distinct selector patterns,
          reinforcing the incompleteness phenomena of Part IV.
\end{itemize}

Symbolic dynamics therefore provides a conceptual and mathematical foundation
for understanding selector behavior, extended invariants, and the geometry of
the generative space.
\clearpage{}
\clearpage{}\chapter{Alignment and Sewing: Full Technical Proofs}

\section{Introduction}

This appendix provides full proofs of the technical lemmas used in
Chapter~8 and Chapter~9.  
These results justify the alignment of selected digits, the sewing of tails,
and the preservation of collapse fibers under controlled concatenation of
prefixes and tails.

The purpose of this appendix is to present these arguments in their natural
level of detail while keeping the main text focused on conceptual structure.

\section{Canonical Output and Selection Indices}

For a generative identity $G = (M,D,K)$, define its sequence of selected
positions by
\[
n_0 < n_1 < n_2 < \cdots,
\]
where $n_j$ is the $j$th index with $M(n_j) = D$.  
The canonical output of $G$ is the sequence
\[
d_0, d_1, d_2, \ldots,
\qquad\text{where } d_j = D(n_j).
\]
For identities in $\mathcal{X}^{*}$, this output yields a valid digit
expansion.

Two identities $H$ and $A$ lie in the same collapse fiber if and only if
\[
D_H(n_j^{H}) = D_A(n_j^{A}) = x_j
\]
for all $j$, where $n_j^{H}$ and $n_j^{A}$ are their respective selection
indices.

\section{Alignment of Selected Digits}

The first lemma states that identities in the same collapse fiber expose the
same canonical digits at potentially different positions.  
This basic fact allows us to use selection indices as alignment points.

\begin{lemma}[Alignment of Selection Indices]
\label{lem:alignment}
Let $H$ and $A$ be identities in the same collapse fiber
$\mathcal{F}(x)$.  
Let $n_j^{H}$ and $n_j^{A}$ be their respective $j$th selection indices.  
Then the digits exposed at these positions coincide:
\[
D_H(n_j^{H}) = D_A(n_j^{A}) = x_j.
\]
\end{lemma}

\begin{proof}
By definition of the collapse fiber,
\[
\pi(H) = \pi(A) = x.
\]
The canonical output of $\pi(H)$ is the sequence of digits
\[
x_0, x_1, x_2, \ldots,
\]
and the same holds for $\pi(A)$.  
Since $n_j^{H}$ and $n_j^{A}$ denote the $j$th positions where $H$ and $A$
expose their digits, the exposed digits must coincide with the $j$th digit of
$x$.  
Therefore
\[
D_H(n_j^{H}) = x_j = D_A(n_j^{A}),
\]
as required.
\end{proof}

This lemma provides the foundation for sewing: two identities in the same
collapse fiber may disagree on the positions where selected digits occur, but
their canonical output digits occur in the same order.

\section{Prefix Completion and Tail Extraction}

The following definition formalizes the process of replacing the tail of one
identity with the tail of another, starting at aligned selection indices.

\begin{definition}[Prefix Completion and Tail Extraction]
Let $H$ and $A$ be identities in $\mathcal{X}^{*}$ and let $j \in \mathbb{N}$.  
Define
\[
h_j = n_j^{H},
\qquad
a_j = n_j^{A}.
\]
We define the identity $G = H \,\widehat{\ }_{\,j}\, A$ by
\[
G(n) =
\begin{cases}
H(n) & n \le h_j, \\
A(n - h_j + a_j) & n > h_j.
\end{cases}
\]
\end{definition}

This construction preserves all symbols of $H$ up to the $j$th selected
position and then reproduces the symbolic behavior of $A$ starting at the
corresponding selected digit.

\section{Sewing Preserves the Collapse Fiber}

The next lemma shows that prefix completion and tail extraction preserve the
collapsed value when the identities lie in the same fiber.

\begin{lemma}[Tail Sewing Preserves Collapse]
\label{lem:sewing-preserves-collapse}
Let $H$ and $A$ lie in the collapse fiber $\mathcal{F}(x)$ and let $G =
H \,\widehat{\ }_{\,j}\, A$.  
Then $G \in \mathcal{F}(x)$.
\end{lemma}

\begin{proof}
Let $h_j$ and $a_j$ denote the $j$th selected positions of $H$ and $A$.  
The identity $G$ agrees with $H$ on every position $n \le h_j$.  
In particular, the first $j$ selected digits of $G$ occur at the same indices
as in $H$ and have the same values.

For $n > h_j$, the identity $G$ reproduces the behavior of $A$ starting at
index $a_j$.  
The $(j+1)$st selected digit in $G$ appears at the first position $m > h_j$
with $A(m - h_j + a_j) = D$, which corresponds to the $(j+1)$st selection
index of $A$.

Thus $G$ exposes the same canonical digit sequence as $A$, namely the digit
expansion of $x$.  
Hence $\pi(G) = x$ and $G \in \mathcal{F}(x)$.
\end{proof}

This result holds for every $j$ and for any choice of $A$ in the collapse
fiber.

\section{Dependency Bounds and Controlled Sewing}

The next lemma shows how dependency bounds combine with sewing to preserve the
values of structural projections.

\begin{lemma}[Controlled Sewing]
\label{lem:controlled-sewing}
Let $\mathcal{P}$ be a finite family of structural projections with uniform
dependency bound
\[
N = B_{\mathcal{P}}(\varepsilon).
\]
Let $H$ and $A$ lie in the collapse fiber $\mathcal{F}(x)$.  
Let $j$ satisfy $h_{j} \ge N$.  
Define $G = H \,\widehat{\ }_{\,j}\, A$.  
Then
\[
|\Phi(G) - \Phi(H)| < \varepsilon
\quad\text{for all } \Phi \in \mathcal{P}.
\]
\end{lemma}

\begin{proof}
Since $G$ and $H$ agree on all coordinates $n \le h_j$ and $h_j \ge N$, the
prefix agreement condition of the structural projections implies
\[
|\Phi(G) - \Phi(H)| < \varepsilon
\]
for each $\Phi \in \mathcal{P}$.  
The tail of $G$ beyond $h_j$ is irrelevant, since dependency bounds imply
that only the prefix of length $N$ influences the value of $\Phi$ to
precision $\varepsilon$.
\end{proof}

This lemma shows that sewing changes structure only beyond the observational
reach of the projections.

\section{Sewing with Dependency Bounds: A Technical Refinement}

In the diagonalizer construction, we need an explicit estimate relating $j$,
$N$, and the positions of selected digits.  
The following lemma provides this relationship.

\begin{lemma}[Selection Index Lower Bound]
\label{lem:selection-index-lower-bound}
Let $H$ be a generative identity with infinitely many selected digits.  
For any $N \in \mathbb{N}$, there exists a $j$ such that $h_j \ge N$.  
Moreover, if $H$ has positive selector density $\eta(H) > 0$, then
\[
h_j \le \frac{j}{\eta(H)}.
\]
\end{lemma}

\begin{proof}
Since $H$ exposes infinitely many digits, the sequence
\[
h_0 < h_1 < h_2 < \cdots
\]
is strictly increasing and unbounded.  
Thus for any $N$ there exists $j$ with $h_j \ge N$.

If $\eta(H) > 0$, then by definition of lower density,
\[
\frac{j}{h_j} \to \eta(H)
\quad\text{along a subsequence}.
\]
Equivalently,
\[
h_j \le \frac{j}{\eta(H)}
\]
for all sufficiently large $j$.
\end{proof}

This lemma ensures that we can always find an alignment index beyond the range
required by the dependency bounds.

\section{Full Sewing Lemma and Its Consequences}

We now combine the previous results into a single statement that is used in
the diagonalizer construction.

\begin{lemma}[Full Sewing Lemma]
\label{lem:full-sewing}
Let $\mathcal{P}$ be a finite family of structural projections with uniform
dependency bound $B_{\mathcal{P}}(\varepsilon) = N$.  
Let $H$ and $A$ lie in the collapse fiber $\mathcal{F}(x)$.  
Let $j$ satisfy $h_j \ge N$.  
Then the sewed identity $G = H \,\widehat{\ }_{\,j}\, A$ satisfies:
\begin{enumerate}
    \item $G \in \mathcal{F}(x)$,
    \item $|\Phi(G) - \Phi(H)| < \varepsilon$ for all $\Phi \in \mathcal{P}$.
\end{enumerate}
\end{lemma}

\begin{proof}
The first part follows from Lemma~\ref{lem:sewing-preserves-collapse}.  
The second part follows from Lemma~\ref{lem:controlled-sewing}.  
\end{proof}

The Full Sewing Lemma provides the key finite information control needed for
diagonalization: observers remain stable under changes to the identity beyond
a sufficiently long prefix.

\section{Computability of the Sewn Identity}

We finish with the computability properties of the sewing operation.

\begin{lemma}[Computability of Sewing]
\label{lem:computable-sewing}
If $H$ and $A$ are computable identities in $\mathcal{F}(x)$ and $j$ is
computable from $H$, then $H \,\widehat{\ }_{\,j}\, A$ is a computable identity.
\end{lemma}

\begin{proof}
Computable identities have computable selector, digit, and meta streams.
Given $j$ and the selection indices $h_j$ and $a_j$, which are computable from
$H$ and $A$, the definition of the sewed identity provides an explicit
algorithm to compute $G(n)$ for each $n$.  
Thus $G$ is computable.
\end{proof}

This lemma ensures that the diagonalizer constructed in the main text is
computable.

\section{Summary}

This appendix provided full proofs of the alignment and sewing lemmas that
support the diagonalizer construction.  
These results show that:

\begin{itemize}
    \item identities in the same collapse fiber expose the same canonical
          digits in the same order,
    \item tails may be replaced freely once alignment indices are chosen,
    \item dependency bounds ensure that observers are unaffected by tail
          modification,
    \item computable identities remain computable under sewing.
\end{itemize}

Together, these tools form the core technical machinery used to establish the
Structural Incompleteness Theorem.
\clearpage{}
\clearpage{}\chapter{Diagonalizer Construction Details}

\section{Introduction}

This appendix contains the full technical details of the meta diagonalizer
constructed in the main text.  
The purpose is to present the inductive machinery underlying the
construction in a complete and self contained form.  
We give precise definitions of stabilization indices, prefix bounds, and
alignment points, and we verify the computability of the final identity.

Throughout, $x$ denotes a computable real number with canonical digit
expansion $(x_j)_{j \ge 0}$, and $H$ is a fixed computable reference identity
in the effective collapse fiber $\mathcal{F}_{\mathrm{eff}}(x)$.

We assume an enumeration of all computable structural projections
\[
\Phi_0, \Phi_1, \Phi_2, \ldots,
\]
together with computable dependency bounds $B_k$ for each $\Phi_k$.

\section{Preliminaries}

\subsection{Effective fibers}

The set $\mathcal{F}_{\mathrm{eff}}(x)$ of computable identities that collapse
to $x$ is a $\Pi^{0}_{1}$ class.  
Elements of this class are represented by computable selector, digit, and meta
streams whose canonical outputs agree with the expansion of $x$.

\subsection{Divergent identities}

For each $k$ and each rational $\varepsilon > 0$, the Divergence Lemma from
Chapter~9 guarantees the existence of a computable identity $A \in
\mathcal{F}_{\mathrm{eff}}(x)$ such that
\[
|\Phi_k(A) - \Phi_k(H)| > 3\varepsilon.
\]
This provides the source of divergence at stage $k$.

\subsection{Selection indices}

For any identity $G$, let
\[
n_0^{G} < n_1^{G} < n_2^{G} < \cdots
\]
denote the selection indices corresponding to the positions where $M(n) = D$.

\section{Inductive Construction Overview}

We construct a sequence
\[
G_0, G_1, G_2, \ldots
\]
of computable identities satisfying:

\begin{enumerate}
    \item $G_0 = H$,
    \item $G_k \in \mathcal{F}_{\mathrm{eff}}(x)$ for all $k$,
    \item $G_{k+1}$ extends $G_k$ on a prefix of computable length $N_{k+1}$,
    \item $G_{k+1}$ introduces divergence on projection $\Phi_k$,
    \item the limit identity $G^{\sharp}$ defined coordinatewise is computable.
\end{enumerate}

\subsection{Tolerances}

Define the sequence of tolerances
\[
\varepsilon_k = 2^{-(k+2)},
\]
which is computable, strictly decreasing, and tends to zero.

\subsection{Prefix stabilization lengths}

Define
\[
N_0 = 0,
\]
and inductively
\[
N_{k+1} = \max \left( N_k,\ B_k(\varepsilon_k) \right).
\]
This ensures that agreement on the first $N_{k+1}$ symbols forces
agreement of $\Phi_k$ to within $\varepsilon_k$.

\section{Inductive Step}

Assume $G_k$ has been defined.  
We construct $G_{k+1}$ in several stages.

\subsection{Stage 1: Choosing the divergent identity}

By the Divergence Lemma, choose a computable identity
\[
A_k \in \mathcal{F}_{\mathrm{eff}}(x)
\]
such that
\[
|\Phi_k(A_k) - \Phi_k(H)| > 3\varepsilon_k.
\]

The identity $A_k$ will supply the tail divergence at stage $k$.

\subsection{Stage 2: Locating an alignment index}

Let
\[
n_0^{G_k} < n_1^{G_k} < n_2^{G_k} < \cdots
\]
and
\[
n_0^{A_k} < n_1^{A_k} < n_2^{A_k} < \cdots
\]
be the respective selection indices.

Since $G_k$ exposes infinitely many digits, there exists $j_k$ such that
\[
n_{j_k}^{G_k} \ge N_{k+1}.
\]

By alignment (Lemma C.1), the digit exposed at $n_{j_k}^{G_k}$ in $G_k$ is the
same as the digit exposed at $n_{j_k}^{A_k}$ in $A_k$, and both coincide with
the digit $x_{j_k}$.

\subsection{Stage 3: Sewing the tail}

Define $G_{k+1}$ by
\[
G_{k+1}(n) =
\begin{cases}
G_k(n) & n \le n_{j_k}^{G_k}, \\
A_k(n - n_{j_k}^{G_k} + n_{j_k}^{A_k}) & n > n_{j_k}^{G_k}.
\end{cases}
\]

\begin{itemize}
    \item By construction, $G_{k+1}$ matches $G_k$ on all indices
          $\le n_{j_k}^{G_k} \ge N_{k+1}$.
    \item By the sewing lemma, $G_{k+1} \in \mathcal{F}_{\mathrm{eff}}(x)$.
    \item By dependency bounds, $\Phi_k(G_{k+1})$ differs from $\Phi_k(G_k)$
          by less than $\varepsilon_k$.
    \item Since $A_k$ diverges by more than $3\varepsilon_k$ from $H$, the
          final divergence of $G_{k+1}$ from $H$ remains at least
          $\varepsilon_k$.
\end{itemize}

Thus $G_{k+1}$ satisfies all inductive requirements.

\section{Existence of the Limit Identity}

\subsection{Coordinate stabilization}

For each index $n$, there exists a stage $k$ such that
\[
N_k > n.
\]
Since $G_{k+1}$ and $G_k$ agree on all indices up to $N_{k+1} \ge N_k$, it
follows that for all $m \ge k$,
\[
G_m(n) = G_k(n).
\]
Thus every coordinate stabilizes.

\subsection{Definition of the limit}

Define $G^{\sharp} \in \mathcal{X}^{*}$ by
\[
G^{\sharp}(n) = \lim_{k\to\infty} G_k(n),
\]
where the limit is understood as coordinatewise stabilization.

The limit exists by the preceding argument.

\subsection{Membership in the effective fiber}

Every $G_k$ lies in $\mathcal{F}_{\mathrm{eff}}(x)$, and sewing preserves
membership.  
Since each $G_k$ exposes the canonical digits of $x$ at aligned positions, the
same holds for the limit.  
Thus
\[
G^{\sharp} \in \mathcal{F}_{\mathrm{eff}}(x).
\]

\section{Computability of the Diagonalizer}

\subsection{Computability of stabilization indices}

The sequence $(N_k)$ is computable because:

\begin{itemize}
    \item $B_k$ are computable dependency bounds,  
    \item $\varepsilon_k$ is computable,  
    \item $N_{k+1}$ depends only on $N_k$, $B_k(\varepsilon_k)$, and basic
          arithmetic.
\end{itemize}

Thus $N_k$ is uniformly computable.

\subsection{Computing selection indices}

For each computable identity, the selector stream is computable, so the
selection indices are computable by scanning until the required number of
$D$ symbols have been seen.

Thus the indices $n_j^{G_k}$ and $n_j^{A_k}$ are computable.

\subsection{Computability of the sewing operation}

Given $n$ and $k$, to compute $G_{k+1}(n)$:

\begin{itemize}
    \item check whether $n \le n_{j_k}^{G_k}$,  
    \item if so, return $G_k(n)$,  
    \item otherwise, return the aligned symbol from $A_k$.
\end{itemize}

All needed values are computable, so $G_{k+1}$ is computable.

\subsection{Computability of the limit identity}

For any fixed $n$, to compute $G^{\sharp}(n)$:

\begin{itemize}
    \item find $k$ such that $N_k > n$,  
    \item output $G_k(n)$.
\end{itemize}

Since $N_k$ is computable and increasing without bound, this procedure is
effective.

Thus $G^{\sharp}$ is computable.

\section{Summary}

This appendix presented the full details of the diagonalizer construction:

\begin{itemize}
    \item the inductive construction of $(G_k)$,
    \item the stabilization lengths $N_k$,
    \item alignment indices $j_k$,
    \item controlled sewing of tails,
    \item preservation of collapse fibers,
    \item computability of each $G_k$,
    \item computability of the limit identity $G^{\sharp}$.
\end{itemize}

These tools establish the existence of a computable identity in the collapse
fiber of $x$ that agrees with a reference identity on all observed prefixes
yet diverges along every computable structural projection.
\clearpage{}
\clearpage{}\chapter{Extended Invariants and Selector Geometry}

\section{Introduction}

This appendix provides explicit examples and geometric interpretations of the
extended invariants introduced in Part~VI.  
These invariants measure large scale properties of the selector stream and
reveal how generative identities distribute across the symbolic space.  
The appendix also develops a slice based geometric viewpoint of selector
behavior, which generalizes and formalizes earlier three dimensional intuition
using the modern terminology of the monograph.

Throughout, $G = (M,D,K)$ denotes a generative identity with selector stream
$M \in \{D,K\}^{\mathbb{N}}$.  
The entropy balance is
\[
\eta(G)
  = \liminf_{N\to\infty}
      \frac{1}{N} \sum_{n<N} \chi_M(n),
\]
and the fluctuation index is
\[
\phi(G)
  = \limsup_{j\to\infty} \frac{g_j}{n_j},
\]
where $n_j$ are the selection indices and $g_j = n_{j+1} - n_j$.

\section{Vertical, Horizontal, and Fiber Slices}

Extended invariants give rise to natural geometric slices through the
generative space.  
These slices provide useful conceptual pictures of selector behavior and
clarify how collapse fibers intersect large scale invariant structure.

\subsection{Vertical slices: fixing a prefix}

A vertical slice is a cylinder set of the form
\[
\mathcal{C}(u)
  = \{ G \in \mathcal{X}^{*} : G[0..N-1] = u \},
\]
where $u$ is a finite prefix of length $N$.  
Vertical slices represent the set of identities that agree on a finite
segment of the selector, digit, and meta streams.

Vertical slices are the regions that structural projections inspect.  
Dependency bounds imply that a projection samples only a vertical slice, and
prefix stabilization ensures that observers ignore all tails beyond the slice
depth.  
This interpretation connects directly to the incompleteness phenomena of
Part~IV.

\subsection{Horizontal slices: fixing invariant values}

Fix $\alpha \in [0,1]$ or $\beta \in [0,\infty]$.  
The horizontal slices
\[
\mathcal{H}_{\alpha}
  = \{ G : \eta(G) = \alpha \},
\qquad
\mathcal{H}^{\beta}
  = \{ G : \phi(G) = \beta \}
\]
represent level sets of extended invariants.  
These sets group identities by long term selector behavior and cut across many
collapse fibers.

When plotted in the $(\eta,\phi)$ plane, horizontal slices correspond to
vertical or horizontal lines.  
They reveal the large scale organization of selector patterns and illustrate
the diversity of possible behaviors.

\subsection{Fiber slices: fixing the collapsed value}

Fix a real number $x$.  
The fiber slice
\[
\mathcal{F}(x)
  = \{ G \in \mathcal{X}^{*} : \pi(G) = x \}
\]
represents all identities that collapse to $x$.  
This slice is a symbolic sheet that cuts through the invariant geometry of
the selector space.

Because extended invariants depend only on the selector, not on collapse, the
image of a fiber under the embedding
\[
G \mapsto (\eta(G), \phi(G))
\]
typically fills a large region of the $(\eta,\phi)$ plane.  
This geometric picture illustrates how collapse conceals nearly all selector
structure.

\section{Worked Examples of Extended Invariants}

\subsection{Periodic positive density example}

Let
\[
M(n) =
\begin{cases}
D & \text{if } n \text{ is even},\\
K & \text{otherwise}.
\end{cases}
\]
Then half of all positions expose digits:
\[
\eta(G) = \frac{1}{2}.
\]
The selection indices are $n_j = 2j$, so $g_j = 2$ and
\[
\phi(G) = 0.
\]
This identity represents regular, evenly spaced digit exposure.

\subsection{Positive density with mild irregularity}

Let $M$ be the periodic sequence obtained by repeating $DDK$.  
Then
\[
\eta(G) = \frac{2}{3},
\qquad
\phi(G) = 0,
\]
even though the exposure pattern is not evenly spaced.

\subsection{Zero density with bounded gaps}

Let $M(n) = D$ when $n$ is prime and $K$ otherwise.  
The density of primes is zero, so
\[
\eta(G) = 0.
\]
However, gaps grow only logarithmically, so
\[
\phi(G) = 0.
\]

\subsection{Zero density with large fluctuations}

Select digits at factorial indices
\[
n_j = j!.
\]
Then $\eta(G) = 0$, but
\[
\frac{g_j}{n_j} = j,
\quad
\phi(G) = \infty.
\]

\subsection{Oscillating density example}

Expose digits in blocks:
\[
D^{2^0}K^{2^0}D^{2^1}K^{2^1}D^{2^2}K^{2^2}\cdots
\]
Then the density oscillates between values near $0$ and near $1$, and
\[
\eta(G) = 0,
\qquad
\phi(G) = \infty.
\]

\section{Semicontinuity Demonstrations}

\subsection{Lower semicontinuity of \texorpdfstring{$\eta$}{eta}}


Let $G$ satisfy $\eta(G) = 0$.  
Define $G_k$ by copying the first $k$ bits of $M$ and then exposing digits
forever.  
Then $\eta(G_k) = 1$ for all $k$ and
\[
\eta(G)
  \le \liminf_{k\to\infty} \eta(G_k).
\]

\subsection{Upper semicontinuity of \texorpdfstring{$\phi$}{phi}}

Let $G$ have evenly spaced selected digits so $\phi(G) = 0$.  
Modify the tail of $M$ in $G_k$ by inserting a single gap of length
$\ell_k \to \infty$.  
Then
\[
\limsup_{k\to\infty} \phi(G_k) = \infty,
\qquad
\phi(G) \ge \limsup_{k\to\infty} \phi(G_k).
\]

\subsection{Failure of continuity}

Let $G_k$ select digit $D$ only at position $k$ and set $G$ to have no
selected digits.  
Then $G_k \to G$, but the invariants behave discontinuously.  
This illustrates that neither $\eta$ nor $\phi$ can be continuous in the
product topology.

\section{Extended Invariants Inside Collapse Fibers}

Since extended invariants depend only on the selector, each collapse fiber
contains identities with arbitrary invariant values.

\subsection{Arbitrary balance in a fiber}

Given any $\alpha \in [0,1]$, construct a selector $M$ with
\[
\eta(G) = \alpha,
\]
assign the canonical digits of $x$ at selected positions, and choose any meta
stream.  
The resulting identity lies in $\mathcal{F}(x)$.

\subsection{Arbitrary fluctuation in a fiber}

Given $\beta \in [0,\infty]$, construct a selector with
\[
\phi(G) = \beta
\]
by adjusting the growth of the gap sequence.  
Placing the digits of $x$ at selected indices yields an identity in
$\mathcal{F}(x)$.

\subsection{Simultaneous control}

Given any $(\alpha,\beta)$, selectors can be built to realize
\[
\eta(G) = \alpha,
\qquad
\phi(G) = \beta,
\]
and the corresponding identities lie in the collapse fiber of $x$.  
Thus fibers map to substantial regions of the invariant plane.

\section{Summary}

This appendix presented explicit examples illustrating the full range of
behavior for $\eta$ and $\phi$, as well as slice based geometric
interpretations that clarify how selectors populate the generative space.  
Vertical slices represent finite symbolic prefixes, horizontal slices represent
invariant level sets, and fiber slices show the richness of selector behavior
compatible with a fixed collapsed value.  
Together, these examples and geometric perspectives support the broader
conclusion that collapse conceals substantial symbolic structure.
\clearpage{}
\clearpage{}\clearpage{}

\backmatter
\bibliographystyle{plain}
\bibliography{references}

\end{document}
