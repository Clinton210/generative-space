\documentclass[11pt,openany]{book}


\usepackage{geometry}
\geometry{margin=1in}

\usepackage{amsmath, amssymb, amsthm, mathtools}

\usepackage{graphicx}
\usepackage{bm}
\usepackage{enumerate}

\usepackage[hidelinks]{hyperref}

\usepackage{silence}
\WarningFilter{latex}{Reference}
\WarningFilter{latex}{There were undefined references}


\usepackage{cite}


\theoremstyle{plain}
\newtheorem{theorem}{Theorem}[chapter]
\newtheorem{proposition}{Proposition}[chapter]
\newtheorem{corollary}{Corollary}[chapter]
\newtheorem{lemma}{Lemma}[chapter]

\theoremstyle{definition}
\newtheorem{definition}{Definition}[chapter]
\newtheorem{remark}{Remark}[chapter]
\newtheorem{example}{Example}[chapter]


\begin{document}

\frontmatter

\begin{titlepage}
    \centering
    \vspace*{2cm}
    {\Huge\bfseries The Generative Identity Framework\par}
    \vspace{1.5cm}
    {\Large Clinton Potter\par}
    \vfill
    {\large \today\par}
\end{titlepage}

\newenvironment{abstract}{
    \cleardoublepage
    \thispagestyle{plain}
    \begin{center}
        {\Large\bfseries Abstract}
    \end{center}
    \begingroup
}{\endgroup
    \cleardoublepage
}


\clearpage{}\begin{abstract}
This monograph develops the Generative Identity Framework, a structural
approach to real numbers based on symbolic generative mechanisms. A generative
identity is a triple $(M,D,K)$ of infinite sequences consisting of a selector,
a digit stream, and a meta-information stream. The classical real associated
with an identity is obtained by a continuous collapse map that reads only the
digits exposed by the selector. Collapse is surjective and highly
non-injective. Each real number $x$ corresponds to a symbolic fiber
$\mathcal{F}(x)$ containing many generative identities that share the same
canonical output.

The internal structure of these fibers is studied through continuous
observers. A structural projection is any continuous real valued functional on
the generative space. Dependency bounds from Type-2 Effectivity control the
finite prefix on which each observer depends. These bounds yield prefix
stabilization and tail invariance and show that every continuous observer
extracts only finitely many symbols at any fixed precision.

Using these tools, we construct a computable identity inside the effective
collapse fiber of a computable real $x$ that agrees with a reference identity
on arbitrarily long prefixes and is indistinguishable from it by every
computable structural projection. This yields the Indistinguishability
Theorem, which states that no finite family of continuous observers, even when
combined with the collapsed value, can determine the underlying generative
identity. Finite observation cannot recover the symbolic structure concealed by
collapse.

The monograph then introduces robust asymptotic invariants that measure
large scale selector behavior. The entropy balance $\eta$ describes the lower
asymptotic density of digit exposures, and the fluctuation index $\phi$
describes relative gap growth between selected positions. These invariants are
tail invariant and therefore robust under finite modification, but they are
everywhere discontinuous in the product topology. Their images provide coarse
geometric embeddings of selector behavior and illustrate the diversity that
persists inside each collapse fiber.

The framework offers a unified structural, computational, and geometric view
of real numbers. It presents the continuum as a quotient of a rich symbolic
space and establishes intrinsic limitations on what any finite observational
process can recover about generative structure.
\end{abstract}
\clearpage{}
\clearpage{}\chapter*{Acknowledgments}
\addcontentsline{toc}{chapter}{Acknowledgments}

The ideas developed in this monograph grew out of long periods of independent study and reflection that predate my formal training in mathematics.  
My academic background is in Industrial and Organizational Psychology, and I am completing an undergraduate degree in mathematics.  
The earliest versions of the concepts that eventually became the generative framework arose from efforts to understand how symbolic sequences can combine ordered and stochastic behavior.  
These intuitions matured into the program-based architecture presented here.

I made extensive use of contemporary AI systems during the preparation of this manuscript.  
These systems assisted with drafting, restructuring, and checking the exposition, and they helped convert informal ideas and partial sketches into precise mathematical statements.  
All conceptual advances, definitions, and theorems in this work originate with the author, and the responsibility for correctness lies entirely with me.

I am grateful to my family and friends for their patience, encouragement, and support during the development of this project.  
Their confidence made this work possible.
\clearpage{}

\tableofcontents

\chapter*{Prelude}

Real numbers are usually described by their magnitudes and by the symbolic
expansions that represent them. This monograph develops a different
perspective. Instead of viewing a real number as a static point on the
continuum, we regard it as the collapsed output of a symbolic generative
mechanism. Such a mechanism consists of a selector stream, a digit stream, and
a meta-information stream, all evolving in parallel. Only the digits exposed
by the selector survive the collapse to a classical real value. The remaining
symbolic structure forms a rich landscape that is invisible to classical
analysis.

The guiding idea of the Generative Identity Framework is that classical
magnitude hides substantial internal structure. A single real number may have
many generative identities that all produce the same digit sequence under
collapse but differ in how those digits are exposed, how gaps are distributed,
and what symbolic information is carried in unobserved layers. These
differences do not affect the collapsed value, yet they play a central role in
the behavior of observers that act on the generative representation.

Part I introduces the generative space, the collapse map, and the geometry of
collapse fibers. A collapse fiber collects all identities that produce the
same real number. These fibers are closed subsets of the ambient symbolic
space, and they contain identities with selector streams of positive density,
zero density, regular spacing, or extreme irregularity. The fiber therefore
records structure that collapse alone cannot access.

Parts II and III develop the finite observation theory. A structural
projection is a continuous observer that assigns a real value to a generative
identity based only on finite symbolic information. Dependency bounds
formalize this finite information principle by specifying which prefix an
observer must inspect to achieve a desired precision. Prefix stabilization
shows that once a long enough prefix is fixed, observers ignore the tail of
the identity. These tools provide a precise description of what finite
observation can and cannot detect.

Part IV establishes the central incompleteness phenomenon. Using alignment and
sewing methods that operate within a collapse fiber, a mimicry construction
produces a computable identity that agrees with a reference identity on every
prefix required by a given family of observers, yet differs from it in its
tail. This leads to the Structural Incompleteness Theorem, which states that
no finite family of continuous observers, even when combined with the
collapsed value, can recover the generative identity. Finite observation is
inherently limited by the topology of the generative space.

Part V describes the real continuum as a quotient of the generative space
under collapse. This quotient view clarifies why most symbolic structure is
invisible to classical magnitude and connects the framework to represented
spaces in computable analysis, where real numbers arise as equivalence classes
of symbolic descriptions.

Part VI introduces extended invariants that measure large scale features of
selector behavior. The entropy balance and fluctuation index capture
asymptotic density and relative gap growth. These invariants are nowhere
continuous in the product topology, reflecting the gap between asymptotic
structure and finite observation. Geometric embeddings based on these
invariants reveal the diversity of selector behavior inside collapse fibers
and illustrate how generative identities distribute across large scale
coordinates.

The Generative Identity Framework unifies symbolic, computational, and
geometric viewpoints on real numbers. It shows that a real number is not only
a magnitude but also the shadow of a richer symbolic identity. The framework
opens many directions for further study, including higher order invariants,
geometric embeddings, connections to symbolic dynamics, and interactions with
computability and randomness.

The chapters that follow develop these ideas systematically, beginning with
the foundations of the generative space and culminating in the structural
incompleteness of finite observation.
 
\mainmatter

\chapter*{Part I Summary}

Part I introduces the symbolic foundations of the Generative Identity
Framework. A generative identity is defined as a triple of infinite sequences
$(M,D,K)$ consisting of a selector stream, a digit stream, and a meta
information stream. These sequences form a full product space $\mathcal{X}$
equipped with the product topology, and the digit selecting subspace
$\mathcal{X}^{*}$ contains those identities whose selector exposes infinitely
many digits.

The collapse map extracts the classical real value associated with a
generative identity by reading the digits exposed by $M$ and interpreting them
as a base $b$ expansion. This map is continuous and surjective. Its fibers are
closed, perfect, and totally disconnected subsets of the generative space, and
each fiber contains many identities that differ sharply in their selector
behavior, spacing patterns, and meta streams while producing the same collapsed
value.

The geometry of these fibers provides the first indication that classical
magnitude conceals substantial symbolic structure. Fibers contain identities
with dense or sparse selectors, identities with regular or highly irregular
spacing, and identities with freely chosen meta information. These degrees of
freedom motivate the central question of the monograph: how much of this
structure can be detected by continuous observers that operate on finite
prefixes?

Part I therefore establishes the symbolic setting, the collapse mechanism, and
the foundational fiber geometry that support the analysis of structural
observers in Part III and the incompleteness phenomena developed in Part IV.
 \clearpage{}\chapter{The Generative Space}

\section{Introduction}

The Generative Identity Framework begins by treating real numbers not as
primitive points on the continuum, but as the collapsed shadows of richer
symbolic mechanisms.  
A \emph{generative identity} consists of three infinite sequences working in
parallel: a selector stream, a digit stream, and a meta-information stream.
Only fragments of these sequences determine the classical real number; the
remainder encode additional structure that becomes invisible after collapse.

The purpose of this chapter is to formally describe the ambient space in
which these identities live.  
We define the generative space as a Cantor-like product of symbolic layers,
introduce its effective (computable) core, and establish the topological
principles that underlie collapse, reconstruction, and structural
incompleteness.

Throughout, we fix a base $b \ge 2$ for numeral expansion, and we assume
$\Sigma$ is a finite meta-alphabet.

\section{Definition of the Generative Space}

A generative identity is a triple
\[
G = (M, D, K),
\]
where:
\begin{itemize}
    \item $M \in \{D,K\}^{\mathbb{N}}$ is the \emph{selector stream},
          indicating at each position whether the mechanism exposes a digit
          or a meta-symbol;

    \item $D \in \{0,1,\ldots,b-1\}^{\mathbb{N}}$ is the \emph{digit stream},
          an infinite reservoir from which classical digits are selected when
          $M(n) = D$;

    \item $K \in \Sigma^{\mathbb{N}}$ is the \emph{meta-information stream},
          carrying auxiliary symbolic structure not visible to the classical
          collapse.
\end{itemize}

Each coordinate is a sequence over a finite alphabet equipped with the
discrete topology.  
The generative space is the product
\[
\mathcal{X}
  = \{D,K\}^{\mathbb{N}}
    \times \{0,1,\ldots,b-1\}^{\mathbb{N}}
    \times \Sigma^{\mathbb{N}},
\]
endowed with the product (Cantor) topology.  
Basic open sets are determined by finite prefixes of the three streams.

This topology reflects the principle that every observation of a generative
identity accesses only finitely many symbols from each layer.

\section{The Canonical Output}

Although a generative identity contains three infinite sequences, only the
selector and digit layers contribute to the production of the classical digit
sequence.  
Define the \emph{canonical output} of $G$ as the infinite sequence
\[
X(G) = (d_G(j))_{j=0}^\infty,
\]
where $d_G(j)$ is the $j$th digit encountered among the positions $n$ with
$M(n) = D$, read in order.

Formally, let
\[
n_0 < n_1 < n_2 < \cdots
\]
be the increasing sequence of indices at which $M(n_k) = D$.  
Then
\[
d_G(j) = D(n_j).
\]

If $M$ selects digits only finitely often, the canonical output is finite.
Since classical real numbers require infinite expansions, we restrict our
attention to a natural subspace.

\section{The Digit-Selecting Subspace}

Define the \emph{digit-selecting subspace}
\[
\mathcal{X}^*
  = \{\, G \in \mathcal{X} : M \text{ selects } D \text{ infinitely often}\,\}.
\]

This subspace is closed under finite modifications and is topologically large
within $\mathcal{X}$.  
Every element of $\mathcal{X}^*$ yields an infinite canonical output sequence
and therefore a well-defined classical real number after collapse.

\section{The Effective Core}

The framework distinguishes between arbitrary symbolic identities and those
that are computably generated.  
A generative identity $G = (M,D,K)$ is \emph{computable} if each of the
streams $M$, $D$, and $K$ is a computable function
$\mathbb{N} \to \{D,K\}$, $\mathbb{N} \to \{0,\ldots,b-1\}$,
and $\mathbb{N} \to \Sigma$, respectively.

The \emph{effective core} of the generative space is the set
\[
\mathcal{G}_{\mathrm{eff}}
  = \{\, G \in \mathcal{X} : M,D,K \text{ are computable}\,\}.
\]

This subset plays a central role in the diagonalization and incompleteness
results developed later.  
It forms the computational analogue of the ambient space $\mathcal{X}$ and is
countable in contrast to the uncountable full product.

\section{Worked Examples}

Although the space $\mathcal{X}$ is infinite-dimensional, simple examples
illustrate the fundamental ideas.

\subsection*{Example 1: Alternating Selector}

Let $M$ alternate deterministically:
\[
M = D,K,D,K,D,K,\ldots,
\]
and let $D$ be the digit expansion of a real number $x$ in base $b$ repeated
infinitely, while $K$ carries arbitrary meta-symbols.

Then:
\begin{itemize}
    \item the canonical output $X(G)$ contains every other digit of $D$,  
    \item the collapse $\pi(G)$ produces a real number whose expansion consists
          of the even-indexed digits of $x$.
\end{itemize}

Different choices of the meta-layer $K$ yield distinct generative identities,
all collapsing to the same classical value.

\subsection*{Example 2: Null-Density Selector}

Fix a sequence of perfect squares $1,4,9,16,\ldots$ and define
\[
M(n) =
\begin{cases}
D & \text{if } n \text{ is a perfect square},\\
K & \text{otherwise}.
\end{cases}
\]

The selector exposes digit positions with asymptotic density~$0$.  
The canonical output still produces an infinite digit sequence, but only at a
slowly growing rate.  
This identity collapses to the same real number as the sequence of selected
digits, despite its extremely sparse structure.

\section{Summary}

The generative space $\mathcal{X}$ is a symbolic product space rich enough to
encode both the visible and invisible structure of real numbers.  
Its effective core $\mathcal{G}_{\mathrm{eff}}$ provides a computationally
tractable subspace with deep descriptive complexity.  
Every generative identity in $\mathcal{X}^*$ yields a canonical output and,
through it, a classical real number.

In the next chapter, we define the collapse map that translates these
identities into points of the continuum, initiating the central dichotomy
between internal structure and classical magnitude.
\clearpage{}
\clearpage{}\chapter{The Collapse Map}

\section{Introduction}

A generative identity contains far more symbolic structure than is visible in
its classical magnitude.  
The collapse map extracts a real number from a generative identity by reading
only the digits exposed by the selector stream.  
This operation forgets almost all of the internal generative behavior,
producing a single value in $[0,1]$ while leaving behind a large fiber of
distinct identities sharing the same classical output.

This chapter defines the collapse map, establishes its continuity, and shows
that every real number---computable or otherwise---arises as the collapse of
many different generative identities.

\section{Digit Selection}

Let $G = (M, D, K) \in \mathcal{X}^*$ be a digit-selecting generative identity.
Recall that the canonical output sequence is defined by enumerating the digits
appearing at positions where $M(n) = D$.

Let
\[
n_0 < n_1 < n_2 < \cdots
\]
be the increasing sequence of indices with $M(n_j) = D$, and define
\[
d_G(j) = D(n_j).
\]

The sequence $(d_G(j))_{j \ge 0}$ is an infinite sequence in
$\{0,1,\ldots,b-1\}^{\mathbb{N}}$ and will serve as the base-$b$ expansion of
the collapsed value.

\section{Definition of the Collapse Map}

For every $G \in \mathcal{X}^*$, define the \emph{collapse map}
\[
\pi(G)
  = \sum_{j=0}^{\infty} \frac{d_G(j)}{b^{j+1}}.
\]

When a real number has two base-$b$ expansions (a terminating expansion and a
repeating one), we adopt the standard convention of using the non-terminating
representation with trailing $(b-1)$s avoided.  
This ensures the collapse map is well defined.

The collapse map is the primary projection from the generative space to the
unit interval.  
It depends only on the canonical output and therefore only on the portions of
the digit stream selected by $M$.

\section{Continuity of Collapse}

The topology on $\mathcal{X}^*$ makes $\pi$ a continuous function onto
$[0,1]$.  
Given $\varepsilon > 0$, choosing $N$ large enough so that
$b^{-(N+1)} < \varepsilon$ shows that the first $N$ selected digits determine
$\pi(G)$ to within $\varepsilon$.

Since selected digits appear infinitely often, the first $N$ of them arise
within some initial prefix of $G$.  
Thus, for every $\varepsilon > 0$, there exists an integer $L$ such that any
two identities agreeing on their first $L$ symbols in each stream have
collapsed values within $\varepsilon$.

Therefore \( \pi : \mathcal{X}^* \to [0,1] \) is continuous.

\section{Surjectivity}

Every real number in $[0,1]$ arises as the collapse of many generative
identities.  
Fix any real number $x$ with base-$b$ expansion
\[
x = \sum_{j=0}^{\infty} \frac{x_j}{b^{j+1}}.
\]

Choose a selector $M$ that always selects digits:
\[
M(n) = D \quad \text{for all } n.
\]
Define the digit stream $D$ by $D(n) = x_n$ for all $n$, and let $K$ be any
meta-information sequence.

Then $G = (M, D, K) \in \mathcal{X}^*$ satisfies $\pi(G) = x$.  
Varying $K$ freely shows that the fiber $\pi^{-1}(\{x\})$ is uncountable.

\section{Effective Surjectivity}

The collapse map behaves correctly on the effective core.  
A real number $x \in [0,1]$ is computable if and only if it has a computable
base-$b$ expansion.  
Given such an expansion, the construction above produces a computable
generative identity $G \in \mathcal{G}_{\mathrm{eff}}$ satisfying
$\pi(G) = x$.

Conversely, if $G \in \mathcal{G}_{\mathrm{eff}} \cap \mathcal{X}^*$, then the
canonical output sequence $d_G(j)$ is computable, and so $\pi(G)$ is a
computable real.

Thus
\[
\pi(\mathcal{G}_{\mathrm{eff}} \cap \mathcal{X}^*)
  = \mathbb{R}_c,
\]
the set of computable reals.

\section{Fibers and Structural Redundancy}

The collapse map is many-to-one.  
For any $x \in [0,1]$, the fiber
\[
\mathcal{F}(x) = \pi^{-1}(\{x\})
\]
contains identities that may share no structural similarity beyond producing
the same output digits.

Two identities may:
\begin{itemize}
    \item select digits at completely different positions,
    \item carry unrelated meta-information streams,
    \item differ arbitrarily on unselected digits,
\end{itemize}
while still collapsing to the same real $x$.

This structural redundancy is essential for the development of the projection
theory and the incompleteness results of later parts.

\section{Summary}

The collapse map converts the symbolic structure of a generative identity into
a classical real number by selecting and aggregating digits according to the
selector stream.  
It is continuous, surjective, effectively surjective on computable identities,
and massively non-injective.  
The fibers of $\pi$ form the central objects of study in the Generative
Identity Framework.

The next chapter analyzes the internal geometry of these fibers and the
degrees of freedom that remain invisible after collapse.
\clearpage{}
\clearpage{}\chapter{Collapse Fibers and Ambient Compactness}
\label{chap:fibers}

\section{The Ambient Generative Space}

Let $\Sigma$ denote the finite alphabet used to encode the mixer, digit, and
meta streams of a generative identity. The full generative space is the
product
\[
\mathcal{X} = \Sigma^{\mathbb{N}}
\]
equipped with the product topology induced by the discrete topology on
$\Sigma$. By Tychonoff's theorem, $\mathcal{X}$ is compact and metrizable. A
convenient metric is
\[
d(G,G') = 2^{-N},
\]
where $N$ is the least index at which $G$ and $G'$ differ. This metric
generates the product topology.

We will often work with the subspace
\[
\mathcal{X}^{*}
=
\bigl\{
G \in \mathcal{X} :
M_G(n) = D \text{ for infinitely many } n
\bigr\},
\]
consisting of identities that select infinitely many digits. The set
$\mathcal{X}^{*}$ is a dense $G_{\delta}$ subset of $\mathcal{X}$ (see
\cite{Kechris} or \cite{Weihrauch}). It is not closed, and therefore not
compact.

The distinction between $\mathcal{X}$ and $\mathcal{X}^{*}$ is important.
Compactness arguments must take place in the ambient space $\mathcal{X}$.

\section{Collapse Map and Closedness of Fibers}

The collapse map
\[
\pi : \mathcal{X} \to [0,1]
\]
was defined in Chapter \texttt{\ref{chap:collapse}} by interpreting each identity
as an instruction to select digits of the output real number. Since the output
of $\pi$ depends only on the selected digits and these digits depend on finitely
many coordinates of the generative identity when viewed at any fixed precision,
the following fact is standard in symbolic dynamics.

\begin{proposition}
\label{prop:collapse-continuous}
The collapse map $\pi$ is continuous with respect to the product topology on
$\mathcal{X}$.
\end{proposition}

A proof may be found, for example, in Chapter 1 of \cite{LindMarcus} for shift
spaces and in Section 6 of \cite{Weihrauch} for continuous operators on Cantor
space.

Since singletons in $[0,1]$ are closed, the collapse fiber
\[
\mathcal{F}(x) = \pi^{-1}(\{x\})
\]
is a closed subset of the ambient generative space $\mathcal{X}$.

\begin{corollary}
\label{cor:fiber-compact}
For every $x \in [0,1]$, the collapse fiber $\mathcal{F}(x)$ is compact.
\end{corollary}

\begin{proof}
The ambient space $\mathcal{X}$ is compact and $\mathcal{F}(x)$ is closed in
$\mathcal{X}$, so $\mathcal{F}(x)$ is compact.
\end{proof}

This corrects the earlier intuition that compactness arises from
$\mathcal{X}^{*}$. The effective subspace $\mathcal{X}^{*}$ is not compact.
Compactness of the fiber is inherited from the ambient space $\mathcal{X}$.

\section{The Effective Fiber and Its Position in the Ambient Space}

Define the effective fiber by
\[
\mathcal{F}_{\mathrm{eff}}(x)
=
\mathcal{F}(x) \cap \mathcal{X}^{*}.
\]
This is the set of identities that collapse to $x$ and select infinitely many
digits. It is a dense subset of $\mathcal{F}(x)$ in the subspace topology.

The following subtlety is important.

\begin{proposition}
\label{prop:efffiber-not-closed}
The effective fiber $\mathcal{F}_{\mathrm{eff}}(x)$ is not closed in
$\mathcal{X}$ and is not closed in $\mathcal{X}^{*}$.
\end{proposition}

\begin{proof}
Consider a sequence of identities $G_k \in \mathcal{X}^{*}$ whose selectors
place the $j$th selected digit at position $n_j^{(k)}$ with
$n_j^{(k)} \to \infty$ as $k \to \infty$ for each fixed $j$. Pointwise limits
of such sequences may select only finitely many digits, so the limit lies in
$\mathcal{X} \setminus \mathcal{X}^{*}$. Since $\mathcal{F}(x)$ is closed, the
same phenomenon occurs inside fibers.
\end{proof}

Despite this lack of closedness, the effective fiber retains the same
topological richness as the full fiber.

\section{Perfectness and Cantor Geometry of the Fiber}

The collapse fiber $\mathcal{F}(x)$ is totally disconnected and has no
isolated points. This follows from standard arguments in Cantor space (see
\cite{LindMarcus} or \cite{Kechris}). We state the result here for reference.

\begin{proposition}
\label{prop:fiber-perfect}
For every $x \in [0,1]$, the collapse fiber $\mathcal{F}(x)$ is perfect,
totally disconnected, and uncountable.
\end{proposition}

\begin{proof}
Total disconnectedness follows from the product structure of $\mathcal{X}$.
Perfectness follows because arbitrary changes in the unselected coordinates or
in the meta-information stream after any finite index preserve the collapsed
value. Details are standard and may be found in the references above.
\end{proof}

Since the effective subspace $\mathcal{X}^{*}$ is dense in $\mathcal{X}$, the
effective fiber inherits these properties.

\begin{corollary}
\label{cor:efffiber-perfect}
The effective fiber $\mathcal{F}_{\mathrm{eff}}(x)$ is dense in the full fiber
and contains no isolated points.
\end{corollary}

\section{Tail Freedom Inside Collapse Fibers}

For any identity in the fiber and any finite prefix, there exist distinct
extensions in the fiber that share the prefix and differ afterward. This
property will be essential in later chapters.

\begin{proposition}
\label{prop:tail-freedom}
Let $x \in [0,1]$ and let $G \in \mathcal{F}(x)$. For every $N$ there exist
distinct identities $G'$ and $G''$ in $\mathcal{F}(x)$ such that
\[
G' \upharpoonright N
=
G'' \upharpoonright N
=
G \upharpoonright N.
\]
\end{proposition}

\begin{proof}
Modify the unselected digits or the meta stream beyond index $N$. These
changes do not alter the collapsed value. The resulting identities remain in
the fiber and are distinct.
\end{proof}

Tail freedom is one of the central sources of nondeterminacy in the generative
framework and forms the geometric basis for the indistinguishability
construction in Chapter \texttt{\ref{chap:indistinguishability}}.

\section{Summary}

This chapter clarified the topological setting of the collapse fibers. The key
points are:

\begin{itemize}
    \item The ambient generative space $\mathcal{X}$ is compact.
    \item The effective subspace $\mathcal{X}^{*}$ is dense and not compact.
    \item Collapse fibers $\mathcal{F}(x)$ are closed subsets of $\mathcal{X}$
    and are therefore compact.
    \item The effective fiber is dense inside the fiber and contains no
    isolated points.
    \item Tail freedom allows arbitrary variations after finite prefixes.
\end{itemize}

These properties form the foundation for the structural indistinguishability
results in Chapter \texttt{\ref{chap:indistinguishability}}.

\clearpage{}

\chapter*{Part II Summary}

Part II analyzes the behavior of selector streams, which determine when digits
of the digit stream are exposed and thus shape the symbolic structure of a
generative identity. The selector governs both the internal geometry of an
identity and the finite-information view available to continuous observers.

Two broad regimes of selector behavior are examined. Hybrid selectors expose
digits with positive asymptotic frequency, while null-density selectors expose
digits only sporadically, yet still infinitely often. Both regimes occur
densely in the generative space and in every collapse fiber. This shows that
collapse places essentially no constraint on how rapidly or irregularly digits
may be revealed.

The analysis in Part II emphasizes that selector patterns vary widely even
among identities sharing the same collapsed value. Within a single fiber one
finds identities with regular, evenly spaced exposures, as well as identities
with extreme sparsity or highly irregular gap growth. These differences are
structural: they persist regardless of how the digit or meta streams behave,
and they are invisible to classical magnitude.

This structural diversity motivates the central themes of Parts III and IV.
Continuous observers examine only finite prefixes, and selector behavior
demonstrates how much long-term structure can lie beyond finite
observational reach. Part II therefore lays the groundwork for projection
theory and for the incompleteness phenomena that arise when observers attempt
to measure identities using only finite information.
 \clearpage{}\chapter{Selector Patterns and Density Regimes}

\section{Introduction}

Generative identities differ not only in the symbols they carry but also in
the \emph{rate} at which their selectors expose digits from the underlying
digit stream.  
This rate---the asymptotic density of positions where $M(n) = D$---governs both
the structure of the canonical output and the degree of freedom present inside
the collapse fiber.

This chapter analyzes two fundamental regimes of selector behavior:
\begin{itemize}
    \item \emph{Hybrid selectors}, which expose digits with positive
          asymptotic density, and
    \item \emph{Null-density selectors}, which expose digits at vanishing
          density.
\end{itemize}

Although these two extremes lie on opposite ends of a broad spectrum, both
occur densely in the generative space.  
Understanding these regimes clarifies how generative identities with sharply
different internal behaviors can collapse to the same real number.

\section{Selector Density}

For a selector stream $M \in \{D,K\}^{\mathbb{N}}$, define the indicator
function
\[
\chi_M(n) =
\begin{cases}
1 & M(n) = D,\\[4pt]
0 & M(n) = K.
\end{cases}
\]

The \emph{selector density} of $M$ is the lower asymptotic density
\[
\eta(M)
  = \liminf_{N \to \infty}
      \frac{1}{N}
      \sum_{n=0}^{N-1} \chi_M(n).
\]

If $\eta(M) > 0$, the selector exposes digits at a positive rate; if
$\eta(M)=0$, exposure becomes increasingly sparse.

This density measures only the frequency of digit selections, not their
spacing; a selector may have dense clusters of selections followed by long
voids while still having positive or zero density.

\section{Hybrid Selectors}

\subsection{Definition}

A generative identity $G = (M,D,K)$ is \emph{hybrid} if $\eta(M) > 0$.
Equivalently, the indices $n$ with $M(n)=D$ have positive asymptotic density.

Hybrid identities expose digits regularly enough that, in the long run, a
non-negligible portion of the total stream contributes to the classical
output.

\subsection{Topological density}

Hybrid selectors occur densely in the generative space.

\begin{proposition}
For every nonempty basic open set in $\mathcal{X}$, there exists a hybrid
identity contained in it.
\end{proposition}

\begin{proof}
Let the open set be determined by finite prefixes of $(M, D, K)$.  
Extend these prefixes by placing $M(n) = D$ for all $n$ beyond the given
prefix.  
Then the extended identity is hybrid and remains inside the open set.  
Thus hybrid selectors form a dense subset of $\mathcal{X}$.
\end{proof}

\subsection{Interpretation}

Hybrid identities distribute their observed digits steadily throughout the
total stream.  
They represent the ``typical'' behavior of selectors when little is known
about their structure.

\section{Null-Density Selectors}

\subsection{Definition}

A generative identity is \emph{null-density} if its selector satisfies
$\eta(M) = 0$.

These selectors still expose infinitely many digits (since $G \in
\mathcal{X}^*$), but they do so with asymptotically negligible frequency.

\subsection{Examples}

A standard example uses the perfect squares:
\[
M(n) =
\begin{cases}
D & \text{if } n = k^2 \text{ for some } k,\\
K & \text{otherwise}.
\end{cases}
\]
Since the number of squares below $N$ is $\lfloor \sqrt{N} \rfloor$, the
density of $D$-positions is $N^{-1/2} \to 0$.

More intricate examples use rapidly growing computable sequences such as
$n_k = k!$, $n_k = 2^{2^k}$, or sparse polynomial-time patterns.

\subsection{Existence in every fiber}

Null-density selectors appear in every collapse fiber.

\begin{proposition}
For every $x \in [0,1]$, there exists a null-density generative identity
$G \in \mathcal{F}_{\mathrm{eff}}(x)$.
\end{proposition}

\begin{proof}
Fix the canonical expansion $(x_j)$ of $x$ and define a selector that exposes
digits only at perfect-square positions.  
At each such position $n_j$, set $D(n_j) = x_j$; elsewhere set $D$ arbitrarily.
Let $K$ be any computable meta-stream.  
This identity lies in $\mathcal{F}_{\mathrm{eff}}(x)$ and has density zero by
construction.
\end{proof}

\subsection{Interpretation}

Null-density selectors exhibit extreme sparsity.  
They expose infinitely many digits but at a rate too small to influence the
asymptotic distribution of symbols in the overall generative space.  
Such identities show that collapse fibers contain elements of dramatically
different structural complexity.

\section{Selector Diversity Inside a Fiber}

Hybrid and null-density identities coexist inside the same collapse fiber,
demonstrating that the classical output $x$ places almost no restrictions on
the internal rate of digit revelation.

Given any $x$, the effective fiber $\mathcal{F}_{\mathrm{eff}}(x)$ contains:
\begin{itemize}
    \item identities selecting digits frequently,
    \item identities selecting digits sparsely,
    \item identities with periodic or chaotic selection patterns,
    \item identities with arbitrary meta-information streams.
\end{itemize}

This freedom underscores the essential distinction between internal
generative structure and classical magnitude.

\section{Summary}

Selector density provides the first structural coordinate for generative
identities.  
Hybrid selectors expose digits with positive asymptotic density, whereas
null-density selectors do so sparsely.  
Both behaviors occur densely in the generative space and both appear in every
effective collapse fiber.  
The coexistence of such radically different regimes within a single fiber
illustrates the vast internal variability hidden beneath the collapse.

The next chapter introduces structural projections, continuous observers that
measure generative properties without disrupting the underlying identity.
\clearpage{}
\clearpage{}\chapter{Structural Projections and the Projection Lattice}

\section{Introduction}

The collapse map extracts the classical value of a generative identity while
discarding most of its internal structure.  
To understand which aspects of this structure can be detected by continuous
observers, we introduce the general notion of a \emph{structural projection}.
These projections form a lattice under pointwise comparison and represent
effective measurements that respect the topology of the generative space.

The framework developed in this chapter draws on ideas from Type-2
Effectivity, where continuous functionals on sequence spaces are understood
through their finite information content.  
This finite information principle, central in the work of Weihrauch and
Pauly on represented spaces, appears here in an explicit combinatorial form.
It allows projections to be analyzed through their dependency on finite
prefixes and serves as the foundation for the incompleteness results proved
later.

\section{Structural Projections}

A \emph{structural projection} is any continuous function
\[
\Phi : \mathcal{X}^* \to \mathbb{R},
\]
where $\mathcal{X}^*$ carries the product topology defined in Part I.  
Continuity ensures that the value $\Phi(G)$ is determined to any fixed
precision by a finite prefix of $G$.

More precisely, for every $\varepsilon > 0$, continuity provides an integer
$B_\Phi(\varepsilon)$ such that
\[
G[0..B_\Phi(\varepsilon)]
  = H[0..B_\Phi(\varepsilon)]
\quad\Longrightarrow\quad
|\Phi(G) - \Phi(H)| < \varepsilon.
\]

The function $B_\Phi$ plays the role of a computable modulus of continuity in
the sense of Type-2 computability, which is the standard framework for
analyzing real-valued functionals on symbolic spaces.

\section{Basic Examples}

Several projections arise naturally from the structure of a generative
identity.

\subsection*{Collapse}

The collapse $\pi$ is the foundational projection.  
Its continuity was established in Chapter 2 and follows from the classical
theory of real number representations.

\subsection*{Digit statistics}

Fix a digit $a \in \{0,\ldots,b-1\}$.  
Define
\[
\Phi_a(G)
  = \liminf_{N \to \infty}
      \frac{1}{N}
      \sum_{j=0}^{N-1} \mathbf{1}[\, d_G(j) = a \,].
\]
This projection measures the lower asymptotic frequency of the digit $a$ in
the canonical output.  
Other variants include limsup frequency or empirical block frequencies.

Such projections resemble classical invariants in symbolic dynamics, where
frequency statistics determine measure-theoretic properties of subshifts.
The exposition of Lind and Marcus provides many examples of these quantities
in the context of shift spaces.

\subsection*{Selector statistics}

Define
\[
\Psi(G)
  = \liminf_{N \to \infty}
      \frac{1}{N}
      \sum_{n=0}^{N-1} \mathbf{1}[\, M(n) = D \,].
\]
This projection measures the asymptotic density with which the selector
exposes digits.  
It coincides with the selector density studied in Chapter 4 but now viewed as
an observer on $\mathcal{X}^*$.

\section{Dependency Bounds}

Dependency bounds measure the amount of information an observer requires to
determine its output to a given precision.

\begin{definition}[Dependency Bound]
Let $\Phi : \mathcal{X}^* \to \mathbb{R}$ be continuous.  
A function $B_\Phi : (0,1] \to \mathbb{N}$ is a \emph{dependency bound} for
$\Phi$ if
\[
G[0..B_\Phi(\varepsilon)]
  = H[0..B_\Phi(\varepsilon)]
\quad\Longrightarrow\quad
|\Phi(G) - \Phi(H)| < \varepsilon
\]
for all $\varepsilon > 0$.
\end{definition}

If $\Phi$ is computable, classical results from Type-2 Effectivity imply that
$B_\Phi$ can be chosen to be computable as well.  
This follows from the fact that computable functionals on Baire space admit
computable moduli of continuity.

Dependency bounds quantify the finite information content of observers and
provide the mechanism by which projections can be frozen at finite stages in
the diagonalizer construction of Part IV.

\section{The Projection Lattice}

Given two projections $\Phi$ and $\Psi$, define
\[
\Phi \le \Psi
\quad\Longleftrightarrow\quad
\Phi(G) = \Phi(H) \text{ whenever } \Psi(G) = \Psi(H).
\]
This relation expresses that $\Psi$ distinguishes at least as much structure
as $\Phi$.

\begin{proposition}
The set of structural projections on $\mathcal{X}^*$ ordered by $\le$ forms a
complete lattice.
\end{proposition}

\begin{proof}
For any family of projections $(\Phi_i)$, the pointwise supremum
\[
\Phi(G) = \sup_i \Phi_i(G)
\]
is still continuous and therefore a structural projection.  
This projection is the least upper bound with respect to $\le$.  
Similarly, pointwise infima provide greatest lower bounds.
\end{proof}

This algebraic structure parallels the lattice of continuous real-valued
functionals on represented spaces and has been extensively studied in the
context of Weihrauch degrees.  
Here it provides the organizational framework for understanding how different
projections capture different aspects of generative structure.

\section{Summary}

Structural projections are continuous observers on the generative space.  
Their finite dependency on prefixes gives rise to computable dependency
bounds, and their collective structure forms a complete lattice.  
These properties reflect classical results from Type-2 computability and
symbolic dynamics but are here adapted to the generative identity setting.

In the next chapter we formalize prefix stabilization and show how the finite
dependency of observers enables the controlled constructions that drive the
incompleteness phenomena in Part IV.
\clearpage{}

\chapter*{Part III Summary}

Part III develops the theory of structural projections, which formalize how
continuous observers extract information from generative identities.  
A structural projection is any continuous real valued functional on the
generative space.  
Such observers depend only on finite prefixes of an identity at any fixed
precision, and this finite information principle is captured by computable
dependency bounds.

Dependency bounds provide explicit control over the amount of symbolic data
required to determine the value of an observer within a given error.  
This leads to prefix stabilization, which states that once two identities
agree on a sufficiently long prefix, all observers in a finite family must
agree on their values to within any chosen tolerance.  
Tail modification beyond this prefix has no effect on the output of the
observers.

Different observers impose different finite constraints on generative
identities.  
These constraints may conflict in a single prefix, producing projective
incompatibility.  
For example, one observer may require frequent digit exposures, while another
requires long gaps.  
Such conflicts show that no finite prefix can simultaneously satisfy all
structural demands and provide the combinatorial mechanism that allows
controlled divergence inside collapse fibers.

Part III therefore establishes the observational limits imposed by
continuity, provides the finite information tools that govern the behavior of
observers, and sets the stage for the diagonalizer construction in Part IV.
 \clearpage{}\chapter{Dependency Bounds and Prefix Stabilization}

\section{Introduction}

Structural projections evaluate generative identities using only finitely many
symbols at any fixed precision.  
This finite information principle is central to Type-2 computability, where
continuous functionals on Baire space are understood through their moduli of
continuity.  
In the generative setting, these moduli appear naturally as \emph{dependency
bounds}.  

This chapter develops the machinery that allows observers to be controlled at
finite stages.  
We formalize prefix stabilization, show how dependency bounds govern
finite-stage agreement, and explain how these properties prepare the ground
for the construction of the meta-diagonalizer in Part IV.

\section{Finite Information and Dependency Bounds}

Let $\Phi : \mathcal{X}^* \to \mathbb{R}$ be a structural projection.  
Continuity implies that for every $\varepsilon > 0$ there exists an integer
$B_\Phi(\varepsilon)$ such that agreement on the first $B_\Phi(\varepsilon)$
symbols of the identity forces agreement of the projections within
$\varepsilon$:
\[
G[0..B_\Phi(\varepsilon)]
  = H[0..B_\Phi(\varepsilon)]
\quad\Longrightarrow\quad
|\Phi(G) - \Phi(H)| < \varepsilon.
\]

When $\Phi$ is computable, the classical results of Pour-El, Richards,
Weihrauch, and Pauly guarantee that the map $\varepsilon \mapsto
B_\Phi(\varepsilon)$ may be chosen computably.  
This computability requirement is essential for the effective diagonalization
argument, where observers must be controlled by explicit finite parameters.

\section{Uniform Bounds for Finite Families}

Many arguments involve finite families of projections that must be handled
simultaneously.

\begin{definition}[Uniform Dependency Bound]
Given a finite family
\[
\mathcal{P} = \{ \Phi_1, \ldots, \Phi_k \}
\]
of projections, a function $B_{\mathcal{P}} : (0,1] \to \mathbb{N}$ is a
\emph{uniform dependency bound} if
\[
G[0..B_{\mathcal{P}}(\varepsilon)]
  = H[0..B_{\mathcal{P}}(\varepsilon)]
\quad\Longrightarrow\quad
|\Phi_i(G) - \Phi_i(H)| < \varepsilon
\]
for all $i$.
\end{definition}

Since the family is finite, we may take
\[
B_{\mathcal{P}}(\varepsilon)
  = \max_i B_{\Phi_i}(\varepsilon),
\]
which is computable if each $\Phi_i$ is.

Uniform bounds allow us to freeze a finite family of observers at a single
precision parameter.  
This operation is repeated at increasing precision in the diagonalizer
construction.

\section{Prefix Stabilization}

The key structural property of projections is that agreement beyond the
dependency bound is irrelevant to their evaluation.

\begin{proposition}[Prefix Stabilization]
Let $\Phi$ be a structural projection.  
Fix $\varepsilon > 0$ and set $N = B_\Phi(\varepsilon)$.  
If $G$ and $H$ agree on their first $N$ symbols, then their projections differ
by less than $\varepsilon$:
\[
G[0..N] = H[0..N]
\quad\Longrightarrow\quad
|\Phi(G) - \Phi(H)| < \varepsilon.
\]
\end{proposition}

\begin{proof}
This is exactly the definition of continuity in the product topology.  
The basic open neighborhoods of $G$ are determined by finite prefixes.
Choosing $N$ as the length of such a prefix gives the desired result.
\end{proof}

Prefix stabilization encodes the idea that projections observe only a finite
window of the identity at any fixed resolution.  
The unobserved tail may contain arbitrary structure without being detected by
the observer.

\section{Stability Under Tail Modification}

Tail modification is the process of replacing the portion of a generative
identity beyond some index $N$ with an arbitrary tail.

\begin{proposition}
Let $\Phi$ be a structural projection, let $\varepsilon > 0$, and let
$N = B_\Phi(\varepsilon)$.  
If $G$ and $H$ agree on $[0..N]$, then replacing the tail of $G$ by the tail
of $H$ beyond $N$ produces a new identity $\tilde{G}$ that satisfies
\[
|\Phi(\tilde{G}) - \Phi(G)| < \varepsilon.
\]
\end{proposition}

\begin{proof}
Since $\tilde{G}$ and $G$ agree on their first $N$ symbols, the conclusion
follows from prefix stabilization.
\end{proof}

This invariance under tail modification is one of the central structural
properties of projections.  
It ensures that observers can be satisfied at finite stages, while the tail
remains available for divergence, which is essential for diagonalization.

\section{Interaction with Selector Density}

Many structural projections depend only on selected digits.  
For such projections, the relevant prefixes are determined by the positions
where $M(n) = D$, not by the raw index $n$.  
This leads to selector-dependent versions of dependency bounds, which appear
later when controlling density and fluctuation observers.

The general principle remains unchanged: agreement on the relevant finite
prefix of the canonical output determines agreement of the projection at the
corresponding precision.

\section{Summary}

Dependency bounds capture the finite information content of observers on the
generative space.  
Prefix stabilization and tail invariance show that structural projections
depend only on finite prefixes at any fixed precision.  
These properties enable finite-stage control of observers and are the key
technical tools for the alignment and sewing constructions that begin in the
next chapter and culminate in the meta-diagonalizer of Part IV.
\clearpage{}
\clearpage{}\chapter{Projective Incompatibility}
\label{chap:projective-incompatibility}

\section{Introduction}

Structural projections extract different aspects of a generative identity.
Some measure digit frequencies, others examine spacing patterns, and others
recover classical information through collapse. Although each projection
depends only on a finite prefix of the generative identity at any prescribed
precision, their finite-prefix requirements may conflict.

This chapter develops a formal notion of such conflicts. Distinct observers
often demand incompatible local structures from the selector or digit streams.
These incompatibilities arise from the way dependency bounds determine the
finite windows that observers examine, and they play a central role in the
structural indistinguishability results of Chapter
\texttt{\ref{chap:indistinguishability}}. They express the basic limitation
that no finite window can satisfy all observers simultaneously.

The phenomenon is analogous to familiar situations in symbolic dynamics, where
different combinatorial or ergodic invariants require incompatible blocks to
appear in a shift space. Here the same principle applies to observational
functionals rather than to subshifts.

\section{Observer Requirements}

Let $\Phi$ and $\Psi$ be two structural projections with dependency bounds
$B_{\Phi}$ and $B_{\Psi}$. Fix a precision $\varepsilon > 0$. To approximate
$\Phi(G)$ and $\Psi(G)$ within $\varepsilon$, we must satisfy
\[
G[0..B_{\Phi}(\varepsilon)] \text{ determines } \Phi(G),
\qquad
G[0..B_{\Psi}(\varepsilon)] \text{ determines } \Psi(G).
\]
If the projections extract unrelated forms of structural information, these
finite windows may demand incompatible patterns.

\subsection*{Example: density versus spacing}

Let $\Phi$ be a digit density projection and let $\Psi$ be a spacing or
fluctuation projection. To approximate $\Phi(G)$ with small error, the initial
prefix must contain many positions where $M(n) = D$. To approximate $\Psi(G)$
with small error, the prefix must contain a long interval in which
$M(n)=K$, so that the gap ratio can be measured accurately. A single prefix
cannot realize both requirements simultaneously.

This type of conflict is common in combinatorics on words. Local constraints
on symbol frequencies and local constraints on block lengths do not always
admit a common finite witness.

\section{Formal Definition of Incompatibility}

\begin{definition}[Projective incompatibility]
Two projections $\Phi$ and $\Psi$ are incompatible at precision $\varepsilon$
if no prefix of length
\[
L = \max\bigl\{ B_{\Phi}(\varepsilon), B_{\Psi}(\varepsilon) \bigr\}
\]
can simultaneously satisfy the finite-prefix requirements needed to approximate
both projections to within $\varepsilon$ at their target values.
\end{definition}

Incompatibility therefore expresses a finite informational impossibility:
within the window $[0..L]$ the observers demand conflicting symbolic patterns.

\section{Concrete Instances}

The density versus spacing example above is representative. Let
$N_{\Phi} = B_{\Phi}(\varepsilon)$ and $N_{\Psi} = B_{\Psi}(\varepsilon)$.
If we examine the prefix $[0..L]$ with
$L = \max(N_{\Phi}, N_{\Psi})$, then a density requirement may force many
selected positions in this interval, whereas a spacing requirement may force a
long unselected block. These demands cannot be met by the same prefix.

This incompatibility is entirely local. It depends only on the finite window
used by each observer, not on the behavior of the generative identity outside
this window.

\section{Finite Families of Observers}

Finite families of projections may also contain internal conflicts.

\begin{proposition}
Let $\mathcal{P}$ be a finite family of projections. If $\mathcal{P}$
contains two projections that are incompatible at precision $\varepsilon$,
then no prefix of length
\[
B_{\mathcal{P}}(\varepsilon)
=
\max_{\Phi \in \mathcal{P}} B_{\Phi}(\varepsilon)
\]
can satisfy all projections in the family at precision $\varepsilon$.
\end{proposition}

\begin{proof}
Any prefix satisfying the family must satisfy each member individually. If two
members of the family impose incompatible requirements on the prefix, there is
no prefix that satisfies all of them.
\end{proof}

Thus incompatibility propagates across finite families of observers.

\section{Lack of Interval Structure in Projective Images}

In earlier versions of this chapter it was natural to expect that the image of
a collapse fiber under a projection forms an interval. This is not the case.
Continuous maps from zero-dimensional compact spaces can have highly
disconnected images (see \cite{Kechris}). Therefore no general structural
statement can be made about $\Phi(\mathcal{F}(x))$ beyond continuity.

This observation reinforces the finite-prefix viewpoint: projection behavior is
governed by dependency bounds, not by global geometric structure of the fiber.

\section{Implications for Indistinguishability}

Prefix incompatibility has a direct consequence for observational limits.
Since different observers require different finite windows, the prefixes needed
to satisfy growing families of observers also grow. This monotonic expansion
of prefix requirements implies that observers reveal only finite structural
information about generative identities.

In Chapter \texttt{\ref{chap:indistinguishability}} we use this observation to
construct identities that agree with a reference identity on all relevant
prefixes for any finite family of observers, while differing in their
tails. This shows that generative structure beyond these prefixes remains
undetectable.

\section{Summary}

Different structural projections impose distinct finite-prefix constraints. When
these constraints cannot be realized by a single prefix, the projections are
incompatible. This incompatibility is a local symbolic phenomenon and reflects
the fact that observers examine finite windows of the generative identity at
finite precision. These finite-prefix effects provide the conceptual foundation
for the indistinguishability results of Chapter
\texttt{\ref{chap:indistinguishability}}.
\clearpage{}
\clearpage{}\chapter{Alignment and Tail Sewing Inside Fibers}

\section{Introduction}

The collapse fiber $\mathcal{F}(x)$ contains a vast collection of generative
identities that all yield the same classical real number.  
The diagonalizer developed in the next chapter constructs a new identity
inside the effective fiber that matches a reference identity on all observed
prefixes while diverging arbitrarily in its unobserved tail.  
To carry out this construction, we need two technical tools.

The first tool is an alignment procedure.  
Since the collapse depends only on the sequence of selected digits in the
order they appear, we must ensure that when we splice the tail of one
identity onto the prefix of another, the resulting identity produces the same
canonical output.  
The second tool is a sewing procedure, which replaces the tail of one
identity with the tail of another while retaining membership in the same
collapse fiber.

These constructions rely on the fact that identities in a fiber agree on
their selected digits when listed in order, even though the positions of
these digits in the raw sequence may differ.  
This kind of alignment appears in various areas of symbolic dynamics, in
particular in the study of synchronized shift spaces, but here it plays a
more basic role.  
The alignment and sewing tools allow us to replace long tails without
changing the collapsed value.

\section{Alignment of Selected Digits}

Let $H$ and $A$ be two identities in the fiber $\mathcal{F}(x)$, and let
\[
d_H(0), d_H(1), d_H(2), \ldots
\quad\text{and}\quad
d_A(0), d_A(1), d_A(2), \ldots
\]
be their canonical output sequences.  
Since $H$ and $A$ lie in the same fiber, these sequences are identical and
represent the expansion of $x$.

Let
\[
h_0 < h_1 < h_2 < \cdots
\quad\text{and}\quad
a_0 < a_1 < a_2 < \cdots
\]
be the indices at which $H$ and $A$ select digits.  
For any $k$, both identities expose the $k$th digit of $x$ at their
respective indices $h_k$ and $a_k$.

\begin{proposition}[Index Alignment]
For any $k$, there exist positions in $H$ and $A$ at which the $k$th canonical
digit is selected, namely $h_k$ and $a_k$.  
Thus an identity obtained by taking the prefix of $H$ up to $h_k$ and the tail
of $A$ beginning at $a_k$ produces the same canonical output as $H$.
\end{proposition}

\begin{proof}
Since both identities lie in $\mathcal{F}(x)$, the value of the $k$th selected
digit in each must be $x_k$.  
Therefore the alignment indices $h_k$ and $a_k$ exist by definition of the
canonical output.
\end{proof}

This proposition ensures that splicing the two identities at matching digit
indices preserves the canonical output sequence.

\section{Sewing of Tails}

Given two identities $H$ and $A$ in the same fiber, consider the identity
$\tilde{G}$ that agrees with $H$ up to $h_k$ and with $A$ beyond $a_k$.  
Alignment ensures that the canonical output of $\tilde{G}$ equals that of
$H$, so $\tilde{G}$ lies in $\mathcal{F}(x)$.

\begin{proposition}[Tail Sewing]
Fix $k \in \mathbb{N}$.  
Let $G$ be the identity defined by
\[
G(n) =
\begin{cases}
H(n) & n \le h_k,\\
A(n - h_k + a_k) & n > h_k.
\end{cases}
\]
Then $G \in \mathcal{F}(x)$.
\end{proposition}

\begin{proof}
The identity $G$ agrees with $H$ on the prefix containing the first $k$ selected
digits.  
Beyond that prefix it reproduces the $(k+1)$st, $(k+2)$nd, and all later
selected digits of $A$ in order.  
Since $A$ and $H$ have the same canonical output, $G$ reproduces this same
sequence.  
Therefore $\pi(G) = x$.
\end{proof}

This construction replaces the tail of one identity with that of another
without altering the canonical output.  
The ability to modify the tail freely inside the fiber is one of the key
structural freedoms used in the diagonalizer.

\section{Controlled Tail Replacement}

In diagonalization, we do not splice tails arbitrarily.  
Instead, we choose $A$ to satisfy a specific structural property that we want
the final identity to inherit, and we sew its tail onto a reference identity
$H$ after a sufficiently long prefix.

Let $\mathcal{P}$ be a finite family of projections that we wish to match up
to precision $\varepsilon$.  
Let $N = B_{\mathcal{P}}(\varepsilon)$ be the uniform dependency bound.  
If $H$ and $A$ agree on their first $N$ symbols, then sewing the tail of $A$
onto the prefix of $H$ at any alignment point beyond $N$ preserves the
projections to within $\varepsilon$.

\begin{proposition}[Controlled Tail Sewing]
Let $\mathcal{P}$ be a finite family of projections with uniform dependency
bound $B_{\mathcal{P}}$.  
Fix $\varepsilon > 0$ and set $N = B_{\mathcal{P}}(\varepsilon)$.  
Let $h_k$ be the $k$th selection index for $H$, and choose $k$ such that
$h_k \ge N$.  
Similarly, let $a_k$ be the $k$th selection index for $A$.  
Define $G$ by sewing the prefix of $H$ up to $h_k$ to the tail of $A$ from
$a_k$ onward.  
Then for every $\Phi \in \mathcal{P}$,
\[
|\Phi(G) - \Phi(H)| < \varepsilon.
\]
\end{proposition}

\begin{proof}
Since $G$ and $H$ agree on their first $h_k$ symbols and $h_k \ge N$, we have
agreement on the first $N$ symbols.  
By definition of $B_{\mathcal{P}}$, agreement on the first $N$ symbols ensures
agreement of all projections in the family to within $\varepsilon$.
\end{proof}

This shows that once observers are satisfied on the prefix of length $N$, the
tail may be replaced freely without altering their outputs at the chosen
precision.  
This powerful freedom is the main technical ingredient of the diagonalizer.

\section{Summary}

Alignment of selected digits ensures that identities in the same fiber expose
their canonical digits in a coherent order.  
Tail sewing uses this alignment to replace the entire tail of one identity
with the tail of another while remaining inside the collapse fiber.

When combined with dependency bounds and prefix stabilization, these tools
allow us to construct identities that satisfy any finite family of observers
on arbitrarily long prefixes while diverging freely in the unobserved tail.
The next chapter uses these tools to build the meta-diagonalizer, which
demonstrates the impossibility of recovering generative structure from any
finite collection of continuous observers.
\clearpage{}

\chapter*{Part IV Summary}

Part IV establishes the central incompleteness phenomenon of the Generative
Identity Framework. The collapse map determines the classical real value
associated with a generative identity, but it reveals only a small portion of
the symbolic structure encoded by the selector, digit, and meta streams. This
part shows that no finite collection of continuous observers can recover the
hidden generative identity from its collapsed value.

The first chapter develops the alignment and sewing tools that operate inside
collapse fibers. Identities in the same fiber expose the canonical digits of
their collapsed value in the same order, even when the positions of those
exposures differ. This shared output allows selected digits to be aligned,
after which the tail of one identity may be replaced with the tail of another
without affecting the collapsed value. These operations ensure that finite
prefix agreement can always be preserved while symbolic differences are
introduced beyond the reach of observers.

The second chapter presents the mimicry construction. Given an enumeration of
computable structural projections, the construction builds a computable
identity that agrees with a reference identity on all prefixes required by the
observers, yet differs from the reference in its tail. Dependency bounds
guarantee that this agreement on finite prefixes forces observers to assign
identical values, even though the two identities are structurally distinct.
This yields a computable identity that is observationally indistinguishable
from the reference while not being equal to it.

The final chapter proves the Structural Incompleteness Theorem. For any
computable real number $x$ and any finite family of computable continuous
observers, there exist distinct identities in the effective collapse fiber
$\mathcal{F}_{\mathrm{eff}}(x)$ that produce the same observations for every
observer in the family. Observers cannot distinguish these identities because
their measurements depend only on finite prefixes, while the symbolic
differences lie entirely beyond those prefixes.

Part IV therefore shows that generative structure is fundamentally invisible
to finite continuous observation. This incompleteness arises from the topology
of the generative space and the finite information inherent in continuous
functionals, not from randomness or approximation.
 \clearpage{}\chapter{Structural Indistinguishability}
\label{chap:indistinguishability}

\section{Introduction}

Collapse fibers $\mathcal{F}(x)$ contain uncountably many generative identities
that encode the same classical real number. Earlier chapters established that
computable structural projections---the observers of the framework---depend
only on finitely many generative coordinates when queried at any fixed
precision. Each observer sees only a finite window into an identity.

This chapter proves the central incompleteness phenomenon of the generative
framework: \emph{finite observation cannot recover generative structure}.
We construct a computable identity inside the effective fiber of a computable
real $x$ that is generically different from a fixed reference identity, yet
simulates it so precisely that no computable structural projection can
distinguish them.

Unlike classical diagonalization---which builds an object designed to
\emph{evade} a list of properties---our construction builds an object that
\emph{mimics} a given reference. By satisfying the finite-prefix requirements
of every observer in an effective enumeration, we show that internal
generative information (selector geometry, meta-information, tail structure)
is fundamentally invisible to finitary analysis.

\section{Setup}

Fix a computable real number $x$ and a computable identity
\[
H \in \mathcal{F}_{\mathrm{eff}}(x)
\]
that will serve as our reference.  
Let
\[
\{\Phi_k\}_{k \in \mathbb{N}}
\]
be an effective enumeration of all computable secondary projections on
$\mathcal{G}_{\mathrm{eff}}$, each equipped with a computable dependency bound
$B_k(\varepsilon)$ as in Chapter~\ref{chap:dependency-bounds}.

Our goal is to construct a computable identity
\[
G^\sharp \in \mathcal{F}_{\mathrm{eff}}(x),
\qquad
G^\sharp \neq H,
\]
such that for every $k$,
\[
\lim_{k \to \infty}
\bigl|\Phi_k(G^\sharp) - \Phi_k(H)\bigr| = 0.
\]

In fact, we will achieve the stronger property
\[
\bigl|\Phi_k(G^\sharp) - \Phi_k(H)\bigr|
< \varepsilon_k,
\qquad
\varepsilon_k = 2^{-(k+1)},
\]
for each $k$.

Thus $G^\sharp$ lies arbitrarily close to $H$ from the perspective of every
computable observer, yet differs from $H$ at infinitely many positions.

\section{Distinctness in the Effective Fiber}

A key structural fact is that collapse fibers have no isolated points.

\begin{lemma}[Effective Non-Isolation]
\label{lem:distinct}
For every computable identity $H \in \mathcal{F}_{\mathrm{eff}}(x)$ and every
prefix length $N$, there exists a computable identity $A \in \mathcal{F}_{\mathrm{eff}}(x)$ such that
\[
A \upharpoonright N = H \upharpoonright N
\quad\text{and}\quad
A \neq H.
\]
\end{lemma}

\begin{proof}
Collapse fibers are closed subsets of the ambient Cantor product space, and by
the alignment and sewing lemmata from Chapter~\ref{chap:alignment}, one may
modify any generative tail beyond index $N$ while preserving membership in
$\mathcal{F}(x)$. Since the modifications can be carried out effectively,
$A$ may be chosen computable.
\end{proof}

This lemma guarantees that we can always introduce genuine generative
differences beyond any fixed prefix while keeping the collapse value $x$
unchanged.

\section{Mimicry Construction}

We construct a sequence
\[
G_0, G_1, G_2, \dots
\]
of identities in $\mathcal{F}_{\mathrm{eff}}(x)$ converging to the desired
limit identity $G^\sharp$.

\subsection{Initialization}

Let $G_0 = H$ and $N_0 = 0$.  
Set $\varepsilon_k = 2^{-(k+1)}$.

\subsection{Inductive Step}

Assume $G_k$ and $N_k$ are defined.

\paragraph{Step 1: Update the dependency horizon.}
To ensure that $G_{k+1}$ and $H$ agree with respect to observer $\Phi_k$ at
precision $\varepsilon_k$, compute
\[
L_k = B_k(\varepsilon_k),
\qquad
N_{k+1} = \max(N_k, L_k) + 1.
\]

\paragraph{Step 2: Freeze the prefix.}
Require
\[
G_{k+1} \upharpoonright N_{k+1}
=
H \upharpoonright N_{k+1}.
\]
Any extension of this prefix automatically satisfies
\[
\bigl|\Phi_k(G_{k+1}) - \Phi_k(H)\bigr| < \varepsilon_k.
\]

\paragraph{Step 3: Force distinctness.}
Apply Lemma~\ref{lem:distinct} to obtain a computable identity
$A_k \in \mathcal{F}_{\mathrm{eff}}(x)$ with the same prefix but
$A_k \neq H$.  
Set $G_{k+1} = A_k$.

This ensures both conditions:
\[
G_{k+1} \upharpoonright N_{k+1} =
H \upharpoonright N_{k+1},
\qquad
G_{k+1} \neq H.
\]

\subsection{Existence of the Limit}

Since $N_{k+1} > N_k$ and $G_{k+1}$ agrees with $G_k$ on coordinates
$0,\dots,N_k$, the sequence $(G_k)$ stabilizes coordinatewise and therefore
converges in the product topology to a limit identity $G^\sharp$.

Each $G_k$ belongs to $\mathcal{F}_{\mathrm{eff}}(x)$, and the fiber is closed
in the ambient product, so
\[
G^\sharp \in \mathcal{F}_{\mathrm{eff}}(x).
\]

By construction, $G^\sharp$ differs from $H$ at infinitely many coordinates.

\section{The Structural Indistinguishability Theorem}

\begin{theorem}[Structural Indistinguishability]
\label{thm:indistinguishability}
Let $x$ be a computable real number and $H \in \mathcal{F}_{\mathrm{eff}}(x)$
a fixed computable identity.  
Then there exists a computable identity
\[
G^\sharp \in \mathcal{F}_{\mathrm{eff}}(x),
\qquad
G^\sharp \neq H,
\]
such that for every computable secondary projection $\Phi$,
\[
\Phi(G^\sharp) = \Phi(H).
\]
\end{theorem}

\begin{proof}
Fix any computable projection $\Phi_m$.  
For every $k \ge m$, the construction guarantees that
\[
G_k \upharpoonright N_k = H \upharpoonright N_k
\quad\text{with}\quad
N_k \ge B_m(\varepsilon_k).
\]
Hence
\[
\bigl|\Phi_m(G_k) - \Phi_m(H)\bigr| < \varepsilon_k.
\]
Taking limits yields
\[
\Phi_m(G^\sharp) = \Phi_m(H).
\]
Since $m$ was arbitrary, this equality holds for every computable observer.
Distinctness follows from the fact that the disagreement positions were forced
to diverge to infinity.
\end{proof}

\section{Interpretation}

The theorem shows that collapse fibers contain identities that are
\emph{observationally indistinguishable} from one another under any finite or
computably infinite battery of observers.

Observers operate through finite windows; generative differences appear only in
tails. Thus no finite observation protocol---no matter how extensive the menu
of computable structural projections---can recover the full generative
structure of an identity.

In classical analysis, the slogan is:
\[
\text{continuous observers separate points.}
\]
In the generative framework, the slogan is reversed:
\[
\text{finite observers see only finite prefixes.}
\]
The collapse map destroys structure; structure can be \emph{simulated}
arbitrarily well inside the fiber.

This establishes the fundamental incompleteness principle of the framework:
generative information exists in a strict sense but lies beyond the reach of
classical analytic observation.
\clearpage{}
\clearpage{}\chapter{The Continuum as a Collapse Quotient}
\label{chap:quotient}

\section{Introduction}

The collapse map sends a generative identity to a real number by selecting and
interpreting the digits exposed by its selector stream. This chapter examines
the relationship between the ambient generative space $\mathcal{X}$ and the
classical continuum $[0,1]$, viewed through the lens of the collapse map. The
continuum arises as a quotient of $\mathcal{X}$ in which all identities that
produce the same canonical output are identified. This viewpoint highlights the
fact that classical magnitude is a coarse shadow of a richer symbolic space.

The quotient interpretation matches standard constructions in computable
analysis and represented space theory. There a real number is given by an
equivalence class of names. Here the equivalence relation is induced by the
canonical output mechanism defined in Chapter \texttt{\ref{chap:collapse}}.

\section{The Collapse Equivalence Relation}

The collapse map $\pi : \mathcal{X} \to [0,1]$ induces an equivalence relation
\[
G \sim H
\quad\Longleftrightarrow\quad
\pi(G) = \pi(H).
\]
The equivalence class of $G$ is the collapse fiber
\[
[\![G]\!]
=
\mathcal{F}(\pi(G)).
\]

Two identities lie in the same class exactly when they generate the same
canonical digit sequence, and this sequence determines the collapsed real
number in the usual base $b$ interpretation.

\section{The Quotient Map}

We equip $\mathcal{X}$ with the product topology and $[0,1]$ with the Euclidean
topology. By Proposition \texttt{\ref{prop:collapse-continuous}} the collapse
map $\pi$ is continuous and surjective. The induced quotient map
\[
q : \mathcal{X} \to \mathcal{X}/\!\sim
\]
is continuous in the quotient topology. The quotient $\mathcal{X}/\!\sim$
inherits compactness from $\mathcal{X}$, since the domain is compact and the
equivalence relation is closed. These are standard facts from general topology
(see \cite{Kechris}).

A classical result then yields the following.

\begin{proposition}
\label{prop:quotient-homeo}
The quotient space $\mathcal{X}/\!\sim$ is homeomorphic to the closed interval
$[0,1]$.
\end{proposition}

\begin{proof}
The map $\pi$ is a continuous surjection and identifies exactly the elements
of each fiber. Its universal property shows that $\pi$ factors through the
quotient map $q$, and the induced map $\tilde{\pi} : \mathcal{X}/\!\sim \to [0,1]$
is continuous and bijective. Since the domain is compact and the codomain is
Hausdorff, $\tilde{\pi}$ is a homeomorphism.
\end{proof}

Although the quotient has the simple topology of an interval, the equivalence
classes are highly structured. The quotient construction collapses complex
symbolic data into a single classical value.

\section{Structure of Collapse Fibers}

Each collapse fiber is a compact, perfect, and totally disconnected subset of
$\mathcal{X}$. These properties were established in Chapter
\texttt{\ref{chap:fibers}}. The selector and meta-information streams may vary
freely beyond any finite index without altering the collapsed value, so the
fiber typically resembles a product of Cantor-like sets with additional
constraints arising from the selection mechanism.

This internal richness plays a central role in the structural
indistinguishability theorem of Chapter \texttt{\ref{chap:indistinguishability}}.
Finite observers examine only finitely many coordinates and therefore cannot
recover tail structure inside a fiber. The quotient viewpoint makes this
limitation explicit.

\section{Computability Perspective}

From the viewpoint of computable analysis, the equivalence classes induced by
the collapse map correspond to sets of names for real numbers. If $x$ is a
computable real number, then the effective fiber $\mathcal{F}_{\mathrm{eff}}(x)$
contains a computable identity. Such an identity serves as a computable name
for $x$ in the sense of Type-2 Effectivity (see \cite{Weihrauch}).

Conversely, if $x$ is not computable, then $\mathcal{F}_{\mathrm{eff}}(x)$
contains no computable elements. The fiber may still have complicated
structure, but none of its identities provide an effective name.

This connection aligns the generative framework with classical represented
space theory while emphasizing that the generative space contains far more
structure than standard naming systems. The quotient collapses this structure,
retaining only classical magnitude.

\section{Summary}

Real numbers arise as equivalence classes of generative identities under the
collapse map. The quotient of the ambient generative space by this relation is
homeomorphic to $[0,1]$, even though its equivalence classes are rich symbolic
subsets of a higher-dimensional space. This quotient interpretation highlights
why classical magnitude cannot recover generative structure and motivates the
analysis of asymptotically sensitive invariants in the next part of the
monograph.
\clearpage{}

\chapter*{Part V Summary}

Part V develops the quotient perspective that connects the infinite
dimensional generative space to the classical continuum. The collapse map
reads the digits exposed by the selector stream of a generative identity and
interprets them as a real number in base $b$. Although the generative space
contains extensive symbolic structure, the collapse map identifies many
distinct identities and assigns them the same classical value.

The equivalence classes of the collapse map are the collapse fibers. Each
fiber is a closed subset of the ambient compact product space $\mathcal{X}$,
hence compact, perfect, and totally disconnected. These fibers contain
identities with a wide range of selector behaviors, including positive
density, zero density, regular spacing, and large irregular gaps. These
structural differences are invisible to collapse but play essential roles in
the behavior of observers and in the incompleteness phenomena established in
Part IV.

The quotient of $\mathcal{X}^{*}$ by collapse equivalence is naturally
homeomorphic to the interval $[0,1]$. This parallels the viewpoint of
represented spaces in computable analysis, where classical objects are treated
as equivalence classes of symbolic descriptions. In this setting, each real
number corresponds to the entire fiber of its generative representations, not
to a single canonical identity.

Part V therefore shows that the classical continuum is a coarse image of a
rich symbolic space. The structure of collapse fibers, together with the
freedom of selector behavior within them, prepares the ground for Part VI,
where extended invariants are used to organize and analyze generative
identities through large scale numerical and geometric coordinates.
 \clearpage{}\chapter{Extended Invariants: Asymptotic Density and Fluctuation}
\label{chap:invariants-eta-phi}

\section{Introduction}

The collapse map records the classical real value determined by a generative
identity, but collapse does not reveal the internal structure of the selector
stream. Part IV established that no finite collection of continuous observers
can recover this structure. In this chapter we introduce two extended
invariants that capture large scale features of selector behavior: the
asymptotic density of exposed digits and the fluctuation index of successive
gaps.

Both invariants depend only on the selector stream and measure its
asymptotic behavior. They are tail dependent and therefore invisible to any
fixed finite observer. Their extreme sensitivity to tail modification is a
consequence of the product topology on $\mathcal{X}$: agreement on long
prefixes imposes no restrictions on long term behavior. As a result, both
invariants are discontinuous everywhere and take all admissible values inside
any nonempty open set.

\section{Asymptotic Density}

Let $G = (M,D,K)$ be a generative identity. Define the indicator
\[
\chi_{M}(n)
=
\begin{cases}
1 & \text{if } M(n)=D,\\[3pt]
0 & \text{if } M(n)=K.
\end{cases}
\]
The asymptotic density (or balance) of $G$ is the lower limit
\[
\eta(G)
=
\liminf_{N\to\infty}
\frac{1}{N}\sum_{n=0}^{N-1}\chi_{M}(n).
\]
This invariant measures the long term frequency of exposed digits. Positive
values correspond to selectors with sustained exposure, while $\eta(G)=0$
indicates arbitrarily long intervals of nonselection.

\subsection{Basic properties}

The quantity $\eta(G)$ has two simple features:
\begin{itemize}
    \item it depends only on the selector stream $M$,
    \item it is invariant under modifications of $M$ beyond any finite prefix.
\end{itemize}
Thus $\eta$ is an extended invariant that captures structure outside the reach
of the collapse map.

\section{Fluctuation Index}

Let
\[
0 \le n_{0} < n_{1} < n_{2} < \dots
\]
be the positions where $M(n)=D$. Define the successive gaps
\[
g_{j} = n_{j+1} - n_{j}.
\]
The fluctuation index of $G$ is the upper limit
\[
\phi(G)
=
\limsup_{j\to\infty}\frac{g_{j}}{n_{j}}.
\]
The ratio $g_{j}/n_{j}$ measures the scale adjusted size of the gap following
the $j$th selected digit. Large values indicate that the selector permits long
intervals of nonselection relative to position.

\subsection{Basic properties}

Like $\eta$, the fluctuation index depends only on the selector stream and is
unchanged by tail modifications that preserve the selected positions. Positive
density selectors often yield small fluctuation. Sparse selectors can produce
arbitrarily large fluctuation.

\section{Asymptotic Sensitivity and Nowhere Continuity}

Asymptotic invariants on symbolic streams are almost never continuous in the
product topology. Agreement on finite prefixes places no constraints on tail
behavior. The invariants $\eta$ and $\phi$ illustrate this phenomenon sharply:
both are discontinuous at every point.

\begin{theorem}[Nowhere continuity of $\eta$]
\label{thm:eta-nowhere}
Let $U$ be a nonempty basic open set in $\mathcal{X}$. For every $\alpha$ in
$[0,1]$ there exists an identity $G$ in $U$ with $\eta(G)=\alpha$.
\end{theorem}

\begin{proof}
Let $U$ be determined by a prefix $w$ of length $N$. Extend $w$ by constructing
a selector tail whose asymptotic density equals $\alpha$. For rational
$\alpha$ one may use a periodic sequence; for irrational $\alpha$ one may use
a Sturmian sequence. The prefix contributes a bounded amount to the average,
which vanishes as $N\to\infty$, so the full selector has asymptotic density
$\alpha$.
\end{proof}

\begin{theorem}[Nowhere continuity of $\phi$]
\label{thm:phi-nowhere}
Let $U$ be a nonempty basic open set in $\mathcal{X}$. For every $\beta$ in
$[0,\infty]$ there exists an identity $G$ in $U$ with $\phi(G)=\beta$.
\end{theorem}

\begin{proof}
Let $U$ be determined by a prefix of length $N$. To obtain $\phi(G)=0$ select
every position beyond $N$. To obtain a finite $\beta>0$ place selected digits
at positions approximately satisfying $n_{j+1}\approx(1+\beta)n_{j}$. To
obtain $\beta=\infty$ let $n_{j}=j!$ for large $j$. In all cases the prefix is
respected and the tail determines the fluctuation index.
\end{proof}

These results show that $\eta$ and $\phi$ vary freely inside any open set.
The invariants are asymptotically sensitive: arbitrarily small changes in the
prefix leave room for arbitrary changes in the tail, and therefore arbitrary
values of the invariants.

\section{Extended Invariants Inside Collapse Fibers}

Collapse fibers contain identities with widely varying selector behavior. Fix
a real number $x$. The fiber $\mathcal{F}(x)$ contains identities with:
\begin{itemize}
    \item every possible asymptotic density in $[0,1]$,
    \item every possible fluctuation index in $[0,\infty]$.
\end{itemize}

This follows by combining tail freedom inside collapse fibers with the
constructions of Theorems \texttt{\ref{thm:eta-nowhere}} and
\texttt{\ref{thm:phi-nowhere}}. Any selector stream with infinitely many
selected positions can be combined with the canonical digits of $x$ to create
a valid identity in the fiber. Thus extended invariant values occur densely
throughout the fiber.

\section{Interpretation}

The extended invariants $\eta$ and $\phi$ quantify aspects of generative
structure invisible to collapse. They are determined entirely by the tail and
are unaffected by finite-prefix perturbations. This aligns with the
indistinguishability principle of Chapter
\texttt{\ref{chap:indistinguishability}}: finite observers detect only finite
prefixes and cannot see the asymptotic features described by $\eta$ and
$\phi$.

The discontinuity of these invariants therefore reflects a deep mismatch
between the product topology and the asymptotic geometry of the selector
stream. The product topology governs finite observation. The extended
invariants measure infinite scale structure.

\section{Summary}

The asymptotic density $\eta$ and the fluctuation index $\phi$ are extended
invariants that describe large scale features of selector behavior. Both are
determined by the tail of the selector stream and are discontinuous at every
point of $\mathcal{X}$. Within any open set, all admissible invariant values
occur.

These invariants illustrate the asymptotic richness that survives inside
collapse fibers and motivate the geometric study of invariant pairs in the
next chapter.
\clearpage{}

\chapter*{Part VI Summary}

Part VI introduces extended invariants that measure large scale features of
selector behavior and provide coarse geometric perspectives on the generative
space.  
Unlike collapse or finite observers, these invariants capture asymptotic
properties of the selector stream and therefore reveal structural features
that survive tail modification but remain invisible to continuous projections.

The first chapter presents the entropy balance $\eta$ and the fluctuation
index $\phi$.  
The balance measures the lower asymptotic density of digit exposures, while
the fluctuation index measures the growth of relative gaps between selected
positions.  
These invariants are discontinuous but satisfy natural semicontinuity
properties.  
Their values vary widely inside collapse fibers, which illustrates the
symbolic diversity hidden beneath classical magnitude.

The second chapter introduces geometric embeddings based on these invariants.
Plotting generative identities in the $(\eta, \phi)$ plane reveals large scale
structure in selector behavior.  
Hybrid and null density selectors occupy distinct regions, and identities with
high or low fluctuation index appear at very different geometric scales.
Higher dimensional embeddings are also possible using block statistics, gap
growth rates, or meta stream behavior.

The final chapter synthesizes the framework and outlines future directions.
Extended invariants and geometric embeddings provide new perspectives on the
generative representation of real numbers and suggest further investigation of
higher order invariants, connections to symbolic dynamics, and interactions
with computability and randomness.

Part VI therefore shows how generative identities can be analyzed using
structural, asymptotic, and geometric coordinates that lie beyond collapse and
beyond the reach of finite observers.
 \clearpage{}\chapter{Slice Geometry of Asymptotic Invariants}
\label{chap:slice-geometry}

\section{Introduction}

Extended invariants provide numerical summaries of the long term behavior of
the selector stream. Chapter \texttt{\ref{chap:invariants-eta-phi}} introduced
the asymptotic density $\eta$ and the fluctuation index $\phi$, both of which
depend only on the tail of the selector and therefore measure structural
information invisible to the collapse map. These invariants are asymptotically
sensitive and discontinuous at every point of the generative space.

In this chapter we place the invariants in a geometric context by examining
three natural families of slices through the generative space: vertical
slices that fix finite prefixes, horizontal slices that fix invariant values,
and fiber slices that fix collapsed magnitude. Together these slices describe
how local symbolic structure, global asymptotic behavior, and classical value
interact.

\section{Vertical Slices: Finite Prefix Constraints}

A vertical slice fixes a finite prefix:
\[
\mathcal{C}(u)
=
\{
G \in \mathcal{X} : G[0..N-1]=u
\}.
\]

These sets represent the regions accessible to structural projections, since
dependency bounds constrain each observer to examine only one vertical slice
at a time. Vertical slices are clopen in the product topology and encode the
finite dimensional geometry that governs observational limits and
indistinguishability.

Vertical slices place no constraints on the invariants $\eta$ or $\phi$. By
the nowhere continuity results of Chapter \texttt{\ref{chap:invariants-eta-phi}},
every value of $\eta$ and every value of $\phi$ occurs densely in each vertical
slice.

\section{Horizontal Slices: Invariant Level Sets}

Fix $\alpha$ in $[0,1]$ or $\beta$ in $[0,\infty]$. Define the horizontal
slices
\[
\mathcal{H}_{\alpha}
=
\{ G : \eta(G)=\alpha \},
\qquad
\mathcal{H}^{\beta}
=
\{ G : \phi(G)=\beta \}.
\]

These sets identify identities with identical long term selector behavior even
when their finite prefixes differ. Because $\eta$ and $\phi$ are tail
dependent, horizontal slices cross all vertical slices densely. They cut across
finite prefix classes and collapse fibers alike.

Geometrically, the map
\[
G \mapsto (\eta(G),\phi(G))
\]
projects the generative space into the invariant plane. Horizontal slices
correspond to lines in this plane, and their dense intersection with each
vertical slice reflects the asymptotic sensitivity of the invariants.

\section{Fiber Slices: Fixing Collapsed Value}

Fix a real number $x$. The fiber slice
\[
\mathcal{F}(x)
=
\{ G : \pi(G)=x \}
\]
collects all identities that collapse to $x$. Since $\eta$ and $\phi$ depend
only on the selector and not on the canonical output, the image of
$\mathcal{F}(x)$ under the invariant map is typically a large region of the
invariant plane.

By combining tail freedom inside collapse fibers with the constructions of
Chapter \texttt{\ref{chap:invariants-eta-phi}}, one finds that for any pair
$(\alpha,\beta)$ with $\alpha$ in $[0,1]$ and $\beta$ in $[0,\infty]$, there
exists an identity in $\mathcal{F}(x)$ with $\eta=\alpha$ and $\phi=\beta$.
Thus fiber slices contain identities exhibiting the full range of invariant
values.

\section{Geometric Interpretation}

The three families of slices illustrate the independence of finite prefix
information, asymptotic selector behavior, and collapsed magnitude.

\begin{itemize}
    \item Vertical slices constrain finite symbolic structure but do not
    restrict $\eta$ or $\phi$.
    \item Horizontal slices constrain asymptotic behavior but intersect every
    finite prefix class.
    \item Fiber slices constrain collapsed magnitude but allow all admissible
    invariant values.
\end{itemize}

Together these slices show that the invariants $\eta$ and $\phi$ capture
features orthogonal to finite observation and independent of collapse. They
provide a geometric language for understanding which aspects of selector
structure persist under prefix agreement and which aspects remain completely
unobservable to continuous projections.

\section{Summary}

The slice geometry of vertical, horizontal, and fiber slices provides a
geometric interpretation of the asymptotic invariants introduced in Chapter
\texttt{\ref{chap:invariants-eta-phi}}. Vertical slices represent finite
prefix constraints. Horizontal slices represent invariant constraints. Fiber
slices represent collapsed value constraints. The invariants vary freely in
all of these slices, reflecting their asymptotic nature and their insensitivity
to prefix information.

These geometric insights prepare the way for the study of joint invariant
behavior in the next chapter. Appendix~E contains explicit examples that
illustrate the full range of selector behaviors and invariant combinations.
\clearpage{}
\clearpage{}\chapter{Synthesis and Outlook}

\section{Introduction}

The Generative Identity Framework offers a structural perspective on real
numbers that complements the usual analytic and combinatorial viewpoints.
Generative identities represent real numbers as collapsed outputs of symbolic
mechanisms composed of a selector stream, a digit stream, and a
meta-information stream.  
The collapse map extracts classical magnitude while discarding the majority of
the symbolic structure.  
This fundamental asymmetry between internal structure and classical value
drives the main results of the monograph.

In this final chapter we synthesize the central components of the framework
and outline directions for future research.  
The focus is not on summarizing all results but on clarifying the conceptual
roles played by the generative space, collapse fibers, projection theory, and
extended invariants.

\section{Collapse and Reconstruction}

A generative identity $G = (M, D, K)$ contains significantly more information
than its collapsed value $\pi(G)$.  
The selector identifies which digits of $D$ contribute to the canonical
output, while the meta stream carries additional symbolic content that is
completely invisible under collapse.

The collapse map is continuous, surjective, and highly non-injective.  
It identifies vast families of generative identities that share the same
canonical digit sequence.  
Reconstruction is therefore impossible: collapse fibers contain uncountably
many identities that differ in selector behavior, meta-information content,
and unobserved digits.  
The diagonalizer shows that much of this structure is irretrievably hidden
from finite observation.

\section{Effective Fibers and Observation}

The effective fiber $\mathcal{F}_{\mathrm{eff}}(x)$ associated with a computable
real number $x$ is a nonempty $\Pi^0_1$ class.  
It contains identities with a wide range of selector patterns and meta
streams.  
Continuous observers depend only on finite prefixes of the identity at any
fixed precision, and this finite information principle is the basis of
incompleteness.

The diagonalizer constructed in Part IV demonstrates that no finite family of
observers can distinguish all identities in the fiber.  
The Structural Incompleteness Theorem formalizes this into a general
statement: finite observation cannot recover the generative identity from its
collapsed value.

\section{Extended Invariants}

Extended invariants measure large scale features of the selector stream.  
Two such invariants, the entropy balance $\eta$ and the fluctuation index
$\phi$, capture long term density and relative gap size.  
These invariants are discontinuous but satisfy natural semicontinuity
properties.  
They provide a coarse geometric lens through which to view the generative
space.

Collapse fibers contain identities with all permitted values of $\eta$ and
$\phi$, which shows how little the collapse mechanism constrains selector
behavior.  
The embedding of identities into the $(\eta, \phi)$ plane illustrates the
diversity of generative structure that persists even after collapse.

\section{Generative Geometry}

The geometric viewpoint introduced in Chapter 13 suggests that extended
invariants may form coordinate axes in higher dimensional generative spaces.
Selectors may be analyzed through growth rates of gaps, block frequencies, or
meta-stream patterns.  
These invariants have the potential to organize the generative space along
new dimensions, providing refined classifications that go beyond collapse and
beyond the invariants introduced here.

Although the present framework focuses on selectors, similar geometric tools
could be applied to digit streams or meta streams.  
For example, meta-information could encode symbolic constraints, local
dependencies, or even probabilistic features.  
These possibilities point toward a broader program of generative analysis.

\section{Future Directions}

The results of this monograph raise several avenues for further study.

\subsection*{1. Higher order invariants}

Extended invariants may be generalized by considering block statistics,
empirical measures on the selector stream, or dimension-like quantities that
reflect scaling behavior.  
Understanding how these higher order invariants interact with collapse fibers
could lead to new forms of structural classification.

\subsection*{2. Connections to symbolic dynamics}

Selectors define subshifts of $\{D, K\}^{\mathbb{N}}$ with varying levels of
regularity.  
Interpreting generative identities as points in shift spaces may reveal
dynamical properties of collapse fibers and new connections to thermodynamic
formalism.

\subsection*{3. Computability and randomness}

The diagonalizer highlights the computational limits of observers.  
Investigating the interaction between selector behavior and algorithmic
randomness may clarify the relationship between generative structure and
Martin-Lof randomness in digitally represented reals.

\subsection*{4. Geometric and analytic embeddings}

Embedding generative identities into higher dimensional geometric spaces could
provide new ways of visualizing and classifying internal structure.  
Such embeddings may reveal patterns or invariants not captured by the
collapse map or the low dimensional coordinates introduced here.

\section{Conclusion}

The Generative Identity Framework provides a unified structure for analyzing
real numbers through symbolic generative mechanisms.  
Collapse reveals classical magnitude, while the internal behavior of
selectors, digits, and meta streams encodes a rich array of structural
information.  
Finite observation cannot recover this information.  
The collapse quotient hides far more than it reveals.

Extended invariants and geometric embeddings open the door to deeper study of
generative structure.  
They suggest that real numbers can be understood not only through magnitude,
dimension, or randomness, but also through the behavior of symbolic
mechanisms that generate them.

The framework developed here is only a beginning.  
It provides conceptual foundations and technical tools for a broader program
of generative analysis, one that aims to understand the continuum not simply
as a set of magnitudes but as the image of a vast symbolic space.
\clearpage{}

\appendix
\clearpage{}\chapter{Type 2 Effectivity and Computable Structure}
\label{appendix:tte}

\section{Introduction}

This appendix summarizes the basic notions from Type 2 Effectivity (TTE) and
computable analysis that appear throughout the monograph. The purpose is not
to provide a complete treatment but to outline the background needed for
structural projections, dependency bounds, effective fibers, and prefix
indistinguishability. Standard references include the works of Weihrauch,
Brattka, Hertling, and Pauly on represented spaces and computable analysis.

Three themes recur in the main text:

\begin{enumerate}
    \item names for real numbers and elements of product spaces,
    \item computable functionals and effective moduli of continuity,
    \item effective closed sets and $\Pi^{0}_{1}$ classes.
\end{enumerate}

Throughout, $\mathbb{N}=\{0,1,2,\ldots\}$ and sequences are indexed from zero.

\section{Represented Spaces and Names}

\subsection{Baire space and Cantor space}

Baire space is the sequence space $\mathbb{N}^{\mathbb{N}}$. Cantor space is
the binary sequence space $\{0,1\}^{\mathbb{N}}$. Both carry the product
topology generated by basic open sets determined by finite prefixes. These
spaces serve as standard domains for TTE, and more complicated mathematical
objects are represented by infinite sequences in them.

\subsection{Represented spaces}

A represented space is a pair $(X,\delta_{X})$ where $X$ is a set and
\[
\delta_{X} : \subseteq \mathbb{N}^{\mathbb{N}} \to X
\]
is a partial surjection. A sequence $p$ with $\delta_{X}(p)=x$ is called a
\emph{name} of $x$. Different representation maps correspond to different
codings of objects in $X$.

For real numbers, the standard Cauchy representation interprets $p$ as a
rapidly converging sequence of rational approximations to a real value.

\subsection{Computable points}

A point $x\in X$ is \emph{computable} if it has a computable name. In the
Generative Identity Framework, the effective core
$\mathcal{G}_{\mathrm{eff}}$ consists of identities whose mixer, digit, and
meta streams can be encoded by computable sequences. Names can be obtained by
interleaving these streams into a single sequence in Baire space.

\section{Type 2 Machines and Computable Maps}

\subsection{Type 2 Turing machines}

A Type 2 Turing machine reads an infinite input sequence, writes an infinite
output sequence, and performs finite internal computation. The output symbol
$q(n)$ must be produced after inspecting only finitely many input symbols.
This finite use condition ensures the induced map on Baire space is continuous
in the product topology.

\subsection{Computable maps between represented spaces}

Let $(X,\delta_{X})$ and $(Y,\delta_{Y})$ be represented spaces. A function
$f:X\to Y$ is computable if there exists a Type 2 Turing machine that converts
any name of $x$ into a name of $f(x)$, respecting the representation maps.
Intuitively, the machine computes the value of $f(x)$ to any fixed precision
using only finite information from the name of $x$.

\subsection{Continuity and computability}

A foundational theorem of TTE states that every computable function between
represented spaces is continuous with respect to the induced topologies.
Conversely, whenever a continuous map admits an effective modulus of
continuity, it is computable.

In the monograph, structural projections are continuous real valued maps on a
product space of symbolic sequences. When these projections are computable,
they have computable moduli of continuity, which appear as dependency bounds.

\section{Moduli of Continuity and Dependency Bounds}

\subsection{Moduli of continuity}

Let $f:\mathbb{N}^{\mathbb{N}}\to\mathbb{R}$ be continuous. For each
$\varepsilon>0$ there exists $N$ such that agreement on the first $N$
coordinates guarantees the values differ by less than $\varepsilon$. A
function
\[
\mu : (0,1]\to \mathbb{N}
\]
with this property is called a modulus of continuity. If $f$ is computable,
$\mu$ may be chosen computable.

\subsection{Structural projections and dependency bounds}

The ambient generative space $\mathcal{X}$ is a product of discrete alphabets.
A structural projection
\[
\Phi:\mathcal{X}\to\mathbb{R}
\]
is continuous precisely when there exists a dependency bound $B_{\Phi}$ such
that agreement of two identities on their first $B_{\Phi}(\varepsilon)$
coordinates implies
\[
|\Phi(G)-\Phi(H)| < \varepsilon.
\]

Dependency bounds express the finite informational nature of continuous
observation. They guarantee prefix stabilization: changes to the tail beyond a
certain index do not affect the value of the observer beyond a specified
tolerance.

For a finite family of projections, a uniform bound is obtained by taking the
maximum of the individual bounds.

\section{Effective Closed Sets and \texorpdfstring{$\Pi^{0}_{1}$}{Pi01} Classes}

\subsection{Effective open and closed sets}

A set $U\subseteq\mathbb{N}^{\mathbb{N}}$ is \emph{effectively open} (or
$\Sigma^{0}_{1}$) if it is a computably enumerable union of basic open sets.
Its complement is \emph{effectively closed} (or $\Pi^{0}_{1}$). Membership in
a $\Pi^{0}_{1}$ set can be disproved by finite evidence: observing a finite
prefix that forces the sequence into the complement.

\subsection{Effective fibers as \texorpdfstring{$\Pi^{0}_{1}$}{Pi01} classes}

The effective fiber associated with a computable real $x$ is
\[
\mathcal{F}_{\mathrm{eff}}(x)
=
\{ G\in\mathcal{G}_{\mathrm{eff}} : \pi(G)=x \}.
\]
Since deviations from the canonical digit sequence of $x$ can be detected by
finite prefixes, the set of valid identities is effectively closed. Thus
$\mathcal{F}_{\mathrm{eff}}(x)$ is a $\Pi^{0}_{1}$ class.

This perspective explains why computable identities exist inside each fiber
and why tails can be modified without leaving the fiber, provided the selected
digits remain consistent with the canonical expansion of $x$.

\section{Application to the Generative Identity Framework}

The concepts summarized above support the technical development of the
framework in several ways.

\begin{itemize}
    \item The ambient generative space $\mathcal{X}$ is a represented space.
          The effective core $\mathcal{G}_{\mathrm{eff}}$ corresponds to those
          elements with computable names.

    \item Structural projections are continuous real valued maps with computable
          dependency bounds. These bounds capture the prefix dependence of
          observers and underlie the principle of prefix indistinguishability.

    \item Effective fibers $\mathcal{F}_{\mathrm{eff}}(x)$ are $\Pi^{0}_{1}$
          classes. Their nonemptiness and internal flexibility permit tail
          modification constructions inside collapse fibers.

    \item Prefix stabilization and the finiteness of dependency bounds explain
          why observers cannot distinguish identities that agree on observable
          prefixes. These properties enable the indistinguishability
          construction of Chapter \texttt{\ref{chap:indistinguishability}}.
\end{itemize}

The machinery of represented spaces and Type 2 computability therefore
provides the formal foundation for the generative identity framework. It
clarifies why continuous observation accesses only finite information, why
collapse fibers admit rich internal variation, and why structural
indistinguishability is an inherent feature of generative structure.
\clearpage{}
\clearpage{}\chapter{Symbolic Dynamics Essentials}
\label{appendix:symbolic-dynamics}

\section{Introduction}

This appendix summarizes the symbolic dynamics concepts that appear implicitly
throughout the monograph. Although the generative identity framework uses
selector streams rather than general symbolic blocks, many structural features
of selector behavior are naturally expressed in symbolic terms. The purpose of
this appendix is to outline these tools and indicate how they interact with
the generative space and with the asymptotic invariants developed in
Part~VI.

We begin with full shift spaces and the product topology. We then describe
densities, gap statistics, and block structures, and conclude with a brief
discussion of residual sets and typicality in symbolic dynamics.

\section{Shift Spaces and the Product Topology}

\subsection{Full shifts}

Let $\mathcal{A}$ be a finite alphabet. The full shift is the space
\[
\mathcal{A}^{\mathbb{N}}
=
\{\, x_0 x_1 x_2 \cdots : x_n \in \mathcal{A} \,\}.
\]
Basic open sets are cylinders
\[
[x_0 x_1 \cdots x_{k-1}]
=
\{\, y : y_i = x_i \text{ for } 0 \le i < k \,\}.
\]

The product topology makes $\mathcal{A}^{\mathbb{N}}$ compact, totally
disconnected, and metrizable. These properties hold for the ambient generative
space $\mathcal{X}$, which is also a full shift on a finite alphabet. The
subspace $\mathcal{X}^{*}$, which requires infinitely many digit exposures, is
dense but not compact. This distinction is important in the analysis of
collapse fibers and convergence.

\subsection{The shift map}

The shift map
\[
\sigma(x)_n = x_{n+1}
\]
is continuous, surjective, and preserves cylinder sets. Although the shift is
not used as a dynamical map in the monograph, it provides structural
intuition. Properties such as densities, recurrence, and gap growth are
shift-invariant features.

Selectors are sequences in $\{D,K\}^{\mathbb{N}}$, and shifting corresponds
to advancing the decision of which positions expose digits. This perspective
connects the generative setting to classical symbolic tools.

\subsection{Subshifts}

A subshift is a closed, shift invariant subset of $\mathcal{A}^{\mathbb{N}}$.
Such sets arise by forbidding finite blocks. Families of selectors with
additional constraints (such as prescribed asymptotic density or regular gap
control) form natural subshifts. These subshifts offer a symbolic framework
for describing structured selector behavior.

\section{Densities and Gap Structure}

\subsection{Lower and upper densities}

For $x \in \mathcal{A}^{\mathbb{N}}$ and $a \in \mathcal{A}$, define
\[
\underline{d}_{a}(x)
=
\liminf_{N \to \infty}
\frac{1}{N}
\left|\{ 0 \le n < N : x_n = a \}\right|,
\]
\[
\overline{d}_{a}(x)
=
\limsup_{N \to \infty}
\frac{1}{N}
\left|\{ 0 \le n < N : x_n = a \}\right|.
\]

For selectors $M \in \{D,K\}^{\mathbb{N}}$, the asymptotic density
\[
\eta(G) = \underline{d}_{D}(M)
\]
plays the role of the lower frequency of digit exposures. Chapter
\texttt{\ref{chap:invariants-eta-phi}} analyzes $\eta$ as an asymptotic
invariant and shows that it is discontinuous at every point of the generative
space.

\subsection{Gap sequences}

List the indices at which $x_n=a$ as
\[
n_0 < n_1 < n_2 < \cdots.
\]
The gap sequence is $g_j = n_{j+1} - n_j$. The relative gap growth is
\[
\phi(x)
=
\limsup_{j \to \infty}
\frac{g_j}{n_j},
\]
which is the fluctuation index in the main text. Large values of $\phi$
correspond to sporadic occurrences of $a$ relative to scale.

Gap sequences and related statistics are classical tools for studying sparse
occurrences in symbolic sequences. In the generative framework, they describe
the long range behavior of selectors.

\subsection{Recurrence}

A symbol $a$ is recurrent in $x$ if it appears infinitely often. For selectors
in $\mathcal{X}^{*}$, the requirement that $D$ be recurrent corresponds to the
obligation that digits be exposed infinitely many times in order to produce a
complete canonical expansion. Selectors of zero density admit arbitrarily long
gaps, while positive density selectors exhibit more regular spacing.

\section{Block Structures and Empirical Measures}

\subsection{Blocks}

A block of length $k$ is an element of $\mathcal{A}^{k}$. The set of blocks
appearing in a symbolic sequence $x$ is
\[
\mathcal{L}(x)
=
\bigcup_{k \ge 0}
\{\, x_n x_{n+1} \cdots x_{n+k-1} : n \in \mathbb{N} \,\}.
\]

Selector block structures in $\{D,K\}^{k}$ describe finite patterns of
exposures and suppressions. These local patterns determine fine scale features
that are not captured by invariants such as $\eta$ or $\phi$.

\subsection{Empirical measures}

Given $u \in \mathcal{A}^{k}$, the empirical frequency is
\[
\mathrm{freq}_{N}(u,x)
=
\frac{1}{N}
\left|
\{\, 0 \le n < N-k+1 : x_n x_{n+1} \cdots x_{n+k-1}=u \,\}
\right|.
\]

Empirical frequency ideas motivate possible higher order invariants, such as
block frequencies or empirical measures on selector streams. These quantities
extend the geometric framework of extended invariants discussed in Part~VI.

\section{Residual Structure and Irregularity}

Residual sets, or dense $G_{\delta}$ subsets, describe typical behavior in the
sense of Baire category. In classical symbolic dynamics, many forms of
irregularity are residual:

\begin{itemize}
    \item unbounded fluctuations in gap growth,
    \item oscillating symbol frequencies,
    \item absence of limiting densities.
\end{itemize}

Although the generative identity framework does not rely directly on residual
genericity, the prevalence of irregular symbolic behavior demonstrates that
collapse fibers contain identities with extreme or pathological selector
patterns. This supports the structural indistinguishability results.

\section{Interaction with the Generative Identity Framework}

Symbolic dynamics interacts with the generative framework in several ways.

\begin{itemize}
    \item Selector streams are symbolic sequences in a full shift space, and
          their long term behavior determines the asymptotic invariants
          $\eta$ and $\phi$.

    \item Density and gap statistics describe tail behavior of selectors,
          which is central to invariant analysis and prefix indistinguishability.

    \item Block structures and empirical frequency ideas motivate refined
          invariants beyond those studied in the monograph and suggest future
          directions for generative geometry.

    \item The product topology on symbolic sequences is the same topology used
          to define continuity of structural projections and to obtain
          dependency bounds.

    \item The irregularity typical of symbolic sequences helps explain why
          collapse fibers contain identities with many distinct selector
          patterns.
\end{itemize}

Symbolic dynamics therefore provides a natural mathematical language for
describing selector behavior, asymptotic invariants, and the geometry of the
generative space.
\clearpage{}
\clearpage{}\chapter{Alignment and Sewing: Full Technical Proofs}

\section{Introduction}

This appendix provides full proofs of the technical lemmas used in
Chapter~8 and Chapter~9.  
These results justify the alignment of selected digits, the sewing of tails,
and the preservation of collapse fibers under controlled concatenation of
prefixes and tails.

The purpose of this appendix is to present these arguments in their natural
level of detail while keeping the main text focused on conceptual structure.

\section{Canonical Output and Selection Indices}

For a generative identity $G = (M,D,K)$, define its sequence of selected
positions by
\[
n_0 < n_1 < n_2 < \cdots,
\]
where $n_j$ is the $j$th index with $M(n_j) = D$.  
The canonical output of $G$ is the sequence
\[
d_0, d_1, d_2, \ldots,
\qquad\text{where } d_j = D(n_j).
\]
For identities in $\mathcal{X}^{*}$, this output yields a valid digit
expansion.

Two identities $H$ and $A$ lie in the same collapse fiber if and only if
\[
D_H(n_j^{H}) = D_A(n_j^{A}) = x_j
\]
for all $j$, where $n_j^{H}$ and $n_j^{A}$ are their respective selection
indices.

\section{Alignment of Selected Digits}

The first lemma states that identities in the same collapse fiber expose the
same canonical digits at potentially different positions.  
This basic fact allows us to use selection indices as alignment points.

\begin{lemma}[Alignment of Selection Indices]
\label{lem:alignment}
Let $H$ and $A$ be identities in the same collapse fiber
$\mathcal{F}(x)$.  
Let $n_j^{H}$ and $n_j^{A}$ be their respective $j$th selection indices.  
Then the digits exposed at these positions coincide:
\[
D_H(n_j^{H}) = D_A(n_j^{A}) = x_j.
\]
\end{lemma}

\begin{proof}
By definition of the collapse fiber,
\[
\pi(H) = \pi(A) = x.
\]
The canonical output of $\pi(H)$ is the sequence of digits
\[
x_0, x_1, x_2, \ldots,
\]
and the same holds for $\pi(A)$.  
Since $n_j^{H}$ and $n_j^{A}$ denote the $j$th positions where $H$ and $A$
expose their digits, the exposed digits must coincide with the $j$th digit of
$x$.  
Therefore
\[
D_H(n_j^{H}) = x_j = D_A(n_j^{A}),
\]
as required.
\end{proof}

This lemma provides the foundation for sewing: two identities in the same
collapse fiber may disagree on the positions where selected digits occur, but
their canonical output digits occur in the same order.

\section{Prefix Completion and Tail Extraction}

The following definition formalizes the process of replacing the tail of one
identity with the tail of another, starting at aligned selection indices.

\begin{definition}[Prefix Completion and Tail Extraction]
Let $H$ and $A$ be identities in $\mathcal{X}^{*}$ and let $j \in \mathbb{N}$.  
Define
\[
h_j = n_j^{H},
\qquad
a_j = n_j^{A}.
\]
We define the identity $G = H \,\widehat{\ }_{\,j}\, A$ by
\[
G(n) =
\begin{cases}
H(n) & n \le h_j, \\
A(n - h_j + a_j) & n > h_j.
\end{cases}
\]
\end{definition}

This construction preserves all symbols of $H$ up to the $j$th selected
position and then reproduces the symbolic behavior of $A$ starting at the
corresponding selected digit.

\section{Sewing Preserves the Collapse Fiber}

The next lemma shows that prefix completion and tail extraction preserve the
collapsed value when the identities lie in the same fiber.

\begin{lemma}[Tail Sewing Preserves Collapse]
\label{lem:sewing-preserves-collapse}
Let $H$ and $A$ lie in the collapse fiber $\mathcal{F}(x)$ and let $G =
H \,\widehat{\ }_{\,j}\, A$.  
Then $G \in \mathcal{F}(x)$.
\end{lemma}

\begin{proof}
Let $h_j$ and $a_j$ denote the $j$th selected positions of $H$ and $A$.  
The identity $G$ agrees with $H$ on every position $n \le h_j$.  
In particular, the first $j$ selected digits of $G$ occur at the same indices
as in $H$ and have the same values.

For $n > h_j$, the identity $G$ reproduces the behavior of $A$ starting at
index $a_j$.  
The $(j+1)$st selected digit in $G$ appears at the first position $m > h_j$
with $A(m - h_j + a_j) = D$, which corresponds to the $(j+1)$st selection
index of $A$.

Thus $G$ exposes the same canonical digit sequence as $A$, namely the digit
expansion of $x$.  
Hence $\pi(G) = x$ and $G \in \mathcal{F}(x)$.
\end{proof}

This result holds for every $j$ and for any choice of $A$ in the collapse
fiber.

\section{Dependency Bounds and Controlled Sewing}

The next lemma shows how dependency bounds combine with sewing to preserve the
values of structural projections.

\begin{lemma}[Controlled Sewing]
\label{lem:controlled-sewing}
Let $\mathcal{P}$ be a finite family of structural projections with uniform
dependency bound
\[
N = B_{\mathcal{P}}(\varepsilon).
\]
Let $H$ and $A$ lie in the collapse fiber $\mathcal{F}(x)$.  
Let $j$ satisfy $h_{j} \ge N$.  
Define $G = H \,\widehat{\ }_{\,j}\, A$.  
Then
\[
|\Phi(G) - \Phi(H)| < \varepsilon
\quad\text{for all } \Phi \in \mathcal{P}.
\]
\end{lemma}

\begin{proof}
Since $G$ and $H$ agree on all coordinates $n \le h_j$ and $h_j \ge N$, the
prefix agreement condition of the structural projections implies
\[
|\Phi(G) - \Phi(H)| < \varepsilon
\]
for each $\Phi \in \mathcal{P}$.  
The tail of $G$ beyond $h_j$ is irrelevant, since dependency bounds imply
that only the prefix of length $N$ influences the value of $\Phi$ to
precision $\varepsilon$.
\end{proof}

This lemma shows that sewing changes structure only beyond the observational
reach of the projections.

\section{Sewing with Dependency Bounds: A Technical Refinement}

In the diagonalizer construction, we need an explicit estimate relating $j$,
$N$, and the positions of selected digits.  
The following lemma provides this relationship.

\begin{lemma}[Selection Index Lower Bound]
\label{lem:selection-index-lower-bound}
Let $H$ be a generative identity with infinitely many selected digits.  
For any $N \in \mathbb{N}$, there exists a $j$ such that $h_j \ge N$.  
Moreover, if $H$ has positive selector density $\eta(H) > 0$, then
\[
h_j \le \frac{j}{\eta(H)}.
\]
\end{lemma}

\begin{proof}
Since $H$ exposes infinitely many digits, the sequence
\[
h_0 < h_1 < h_2 < \cdots
\]
is strictly increasing and unbounded.  
Thus for any $N$ there exists $j$ with $h_j \ge N$.

If $\eta(H) > 0$, then by definition of lower density,
\[
\frac{j}{h_j} \to \eta(H)
\quad\text{along a subsequence}.
\]
Equivalently,
\[
h_j \le \frac{j}{\eta(H)}
\]
for all sufficiently large $j$.
\end{proof}

This lemma ensures that we can always find an alignment index beyond the range
required by the dependency bounds.

\section{Full Sewing Lemma and Its Consequences}

We now combine the previous results into a single statement that is used in
the diagonalizer construction.

\begin{lemma}[Full Sewing Lemma]
\label{lem:full-sewing}
Let $\mathcal{P}$ be a finite family of structural projections with uniform
dependency bound $B_{\mathcal{P}}(\varepsilon) = N$.  
Let $H$ and $A$ lie in the collapse fiber $\mathcal{F}(x)$.  
Let $j$ satisfy $h_j \ge N$.  
Then the sewed identity $G = H \,\widehat{\ }_{\,j}\, A$ satisfies:
\begin{enumerate}
    \item $G \in \mathcal{F}(x)$,
    \item $|\Phi(G) - \Phi(H)| < \varepsilon$ for all $\Phi \in \mathcal{P}$.
\end{enumerate}
\end{lemma}

\begin{proof}
The first part follows from Lemma~\ref{lem:sewing-preserves-collapse}.  
The second part follows from Lemma~\ref{lem:controlled-sewing}.  
\end{proof}

The Full Sewing Lemma provides the key finite information control needed for
diagonalization: observers remain stable under changes to the identity beyond
a sufficiently long prefix.

\section{Computability of the Sewn Identity}

We finish with the computability properties of the sewing operation.

\begin{lemma}[Computability of Sewing]
\label{lem:computable-sewing}
If $H$ and $A$ are computable identities in $\mathcal{F}(x)$ and $j$ is
computable from $H$, then $H \,\widehat{\ }_{\,j}\, A$ is a computable identity.
\end{lemma}

\begin{proof}
Computable identities have computable selector, digit, and meta streams.
Given $j$ and the selection indices $h_j$ and $a_j$, which are computable from
$H$ and $A$, the definition of the sewed identity provides an explicit
algorithm to compute $G(n)$ for each $n$.  
Thus $G$ is computable.
\end{proof}

This lemma ensures that the diagonalizer constructed in the main text is
computable.

\section{Summary}

This appendix provided full proofs of the alignment and sewing lemmas that
support the diagonalizer construction.  
These results show that:

\begin{itemize}
    \item identities in the same collapse fiber expose the same canonical
          digits in the same order,
    \item tails may be replaced freely once alignment indices are chosen,
    \item dependency bounds ensure that observers are unaffected by tail
          modification,
    \item computable identities remain computable under sewing.
\end{itemize}

Together, these tools form the core technical machinery used to establish the
Structural Incompleteness Theorem.
\clearpage{}
\clearpage{}\chapter{Mimicry Construction Details}
\label{appendix:mimicry}

\section{Introduction}

This appendix provides full technical details for the mimicry construction
used in Chapter~9 to prove the Structural Indistinguishability Theorem. The
goal is to construct a computable identity inside a collapse fiber that is
distinct from a given reference identity but indistinguishable from it by any
computable structural projection.

The construction uses three ingredients developed earlier in the text:

\begin{itemize}
    \item dependency bounds for structural projections,
    \item alignment and sewing tools from Appendix~\ref{appendix:alignment-sewing},
    \item the perfectness of effective collapse fibers.
\end{itemize}

Throughout, $x$ is a computable real with canonical digit expansion
$(x_j)_{j \ge 0}$, and $H$ is a fixed computable identity in the effective
fiber $\mathcal{F}_{\mathrm{eff}}(x)$. We assume a computable enumeration of
all computable structural projections
\[
\Phi_0, \Phi_1, \Phi_2, \ldots,
\]
each equipped with a computable dependency bound $B_k(\varepsilon)$.

\section{Preliminaries}

\subsection{Effective fibers}

The effective fiber $\mathcal{F}_{\mathrm{eff}}(x)$ is a nonempty
$\Pi^{0}_{1}$ class containing all computable identities that collapse to $x$.
It is perfect, so for any identity $H$ in the fiber and any finite prefix
length $N$ there exists another computable identity $A$ in the fiber that
agrees with $H$ on $[0..N]$ but differs at some later coordinate.

This property supplies the tail variation needed for the mimicry construction.

\subsection{Selection indices}

For any identity $G$, let
\[
n_0^{G} < n_1^{G} < n_2^{G} < \cdots
\]
list the indices where $M(n)=D$. Sewing operations from
Appendix~\ref{appendix:alignment-sewing} use selection indices to align the
tails of two identities inside the same collapse fiber.

\section{Overview of the Mimicry Construction}

We construct a sequence
\[
G_0, G_1, G_2, \ldots
\]
with the following properties:

\begin{enumerate}
    \item $G_0 = H$,
    \item $G_k \in \mathcal{F}_{\mathrm{eff}}(x)$ for all $k$,
    \item $G_{k+1}$ extends $G_k$ on a prefix of length $N_{k+1}$,
    \item for each $k$,
    \[
    |\Phi_k(G_{k+1}) - \Phi_k(H)| < \varepsilon_k,
    \]
    \item $G_{k+1}$ is chosen to be distinct from $H$ by varying the tail.
\end{enumerate}

The limit identity $G^{\sharp}$ will agree with $H$ on arbitrarily long
prefixes and therefore be indistinguishable from $H$ by every computable
projection, while still being symbolically distinct.

\subsection{Tolerances}

Define the error tolerances
\[
\varepsilon_k = 2^{-(k+2)}.
\]
These form a computable, strictly decreasing sequence tending to zero.

\subsection{Prefix stabilization lengths}

Define
\[
N_0 = 0,
\qquad
N_{k+1} = \max \left( N_k,\, B_k(\varepsilon_k) \right) + 1.
\]
Thus $N_{k+1}$ strictly increases and is computable. Agreement on
$[0..N_{k+1}]$ guarantees agreement of $\Phi_k$ to within $\varepsilon_k$.

\section{Inductive Step}

Assume $G_k$ has been defined.

\subsection{Step 1: Preserving observer accuracy}

The next identity $G_{k+1}$ must agree with $G_k$ (and therefore with $H$) on
$[0..N_{k+1}]$. This ensures
\[
|\Phi_k(G_{k+1}) - \Phi_k(H)| < \varepsilon_k.
\]

\subsection{Step 2: Selecting a distinct tail inside the fiber}

By perfectness of $\mathcal{F}_{\mathrm{eff}}(x)$, there exists a computable
identity
\[
A_k \in \mathcal{F}_{\mathrm{eff}}(x)
\]
such that
\[
A_k[0..N_{k+1}] = H[0..N_{k+1}]
\quad\text{and}\quad
A_k \neq H.
\]

This identity provides controlled tail variation while preserving collapse.

\subsection{Step 3: Locating an alignment index}

Let
\[
n_0^{G_k} < n_1^{G_k} < \cdots,
\qquad
n_0^{A_k} < n_1^{A_k} < \cdots
\]
denote the selection indices. Since both identities have infinitely many
exposed digits, there exists $j_k$ such that
\[
n_{j_k}^{G_k} \ge N_{k+1}.
\]

By the alignment lemma in Appendix~\ref{appendix:alignment-sewing}, the digits
exposed at these aligned positions coincide.

\subsection{Step 4: Sewing the tail}

Define
\[
G_{k+1}(n)=
\begin{cases}
G_k(n) & n \le n_{j_k}^{G_k}, \\
A_k(n - n_{j_k}^{G_k} + n_{j_k}^{A_k}) & n > n_{j_k}^{G_k}.
\end{cases}
\]

By the Full Sewing Lemma:

\begin{itemize}
    \item $G_{k+1} \in \mathcal{F}_{\mathrm{eff}}(x)$,
    \item $G_{k+1}[0..N_{k+1}] = G_k[0..N_{k+1}]$,
    \item $\Phi_k(G_{k+1})$ lies within $\varepsilon_k$ of $\Phi_k(H)$,
    \item $G_{k+1}$ differs from $H$ at some tail coordinate.
\end{itemize}

\section{Existence of the Limit Identity}

For each index $n$, choose $k$ such that $N_k > n$. For all $m \ge k$,
\[
G_m(n) = G_k(n),
\]
so the coordinate stabilizes. Define
\[
G^{\sharp}(n) = \lim_{k\to\infty} G_k(n).
\]

\subsection{Membership in the effective fiber}

Each $G_k$ collapses to $x$, and sewing preserves the canonical output. Thus
\[
G^{\sharp} \in \mathcal{F}_{\mathrm{eff}}(x).
\]

\subsection{Indistinguishability}

For any computable projection $\Phi_m$, agreement of $G^\sharp$ and $H$ on
prefixes of length at least $B_m(\varepsilon_k)$ ensures
\[
|\Phi_m(G^{\sharp}) - \Phi_m(H)| < \varepsilon_k
\quad\text{for all } k \ge m.
\]
Therefore
\[
\Phi_m(G^{\sharp}) = \Phi_m(H)
\quad\text{for all } m.
\]

\subsection{Distinctness}

Since at each stage the tail is chosen to differ from $H$, one can ensure that
$G^{\sharp} \neq H$ by avoiding a fixed forbidden coordinate or pattern.

\section{Computability}

\subsection{Computability of prefix bounds}

The values $N_k$ are computable because the dependency bounds $B_k$ and the
tolerances $\varepsilon_k$ are computable.

\subsection{Computability of alignment}

Selection indices of computable identities are computable by scanning the
selector until the $j$th exposure is found.

\subsection{Computability of sewing}

The sewing operation is computable coordinatewise using the computable indices
$n_{j_k}^{G_k}$ and $n_{j_k}^{A_k}$.

\subsection{Computability of the limit}

To compute $G^{\sharp}(n)$, find $k$ with $N_k > n$ and output $G_k(n)$.
Since $N_k$ grows without bound and is computable, this procedure yields a
computable name for $G^{\sharp}$.

\section{Summary}

This appendix provided the full technical development of the mimicry
construction:

\begin{itemize}
    \item computation of stabilization lengths,
    \item selection of distinct tails inside the collapse fiber,
    \item alignment at selection indices,
    \item sewing of tails while preserving collapse,
    \item preservation of observer values via dependency bounds,
    \item coordinatewise convergence,
    \item computability of the limit identity.
\end{itemize}

These tools establish the existence of a computable identity in
$\mathcal{F}_{\mathrm{eff}}(x)$ that is observationally indistinguishable from
a given identity but symbolically distinct, proving the Structural
Indistinguishability Theorem.
\clearpage{}
\clearpage{}\chapter{Extended Invariants and Selector Geometry}

\section{Introduction}

This appendix provides explicit examples and geometric interpretations of the
robust asymptotic invariants introduced in Part~VI. These invariants measure
large scale properties of the selector stream and illustrate how generative
identities distribute across symbolic space. The appendix also develops a
slice based geometric viewpoint that clarifies how finite prefix structure,
asymptotic selector statistics, and collapse fibers interact.

Throughout, $G = (M,D,K)$ is a generative identity with selector
$M \in \{D,K\}^{\mathbb{N}}$. The robust asymptotic invariants are

\[
\eta(G)
    = \liminf_{N\to\infty}
        \frac{1}{N}\sum_{n<N} \chi_M(n),
\qquad
\phi(G)
    = \limsup_{j\to\infty} \frac{g_j}{n_j},
\]

where $n_j$ lists the selected positions and $g_j = n_{j+1} - n_j$.

\section{Vertical, Horizontal, and Fiber Slices}

Extended invariants give rise to three natural types of slices through the
generative space. These slices provide conceptual maps of selector behavior
and explain why finite observation cannot constrain asymptotic structure.

\subsection{Vertical slices: fixing a prefix}

A vertical slice is a cylinder set

\[
\mathcal{C}(u)
  = \{ G \in \mathcal{X}^{*} : G[0..N-1] = u \},
\]

where $u$ is a finite prefix. Vertical slices represent the observable region
available to any continuous projection at fixed precision. Dependency bounds
ensure that each observer samples only a single vertical slice at a time, and
so vertical slices encode the finite information geometry underlying
indistinguishability.

Vertical slices impose no restriction on $\eta$ or $\phi$. Invariant values
can vary freely inside any cylinder.

\subsection{Horizontal slices: fixing invariant values}

Fix $\alpha \in [0,1]$ or $\beta \in [0,\infty]$. The level sets

\[
\mathcal{H}_{\alpha}
  = \{ G : \eta(G) = \alpha \},
\qquad
\mathcal{H}^{\beta}
  = \{ G : \phi(G) = \beta \}
\]

group identities by large scale selector behavior. These slices cut across
collapse fibers and across all vertical slices. When drawn in the
$(\eta,\phi)$ plane, horizontal slices appear as straight lines and illustrate
how asymptotic structure is decoupled from the finite structure observed by
continuous projections.

\subsection{Fiber slices: fixing the collapsed value}

Fix a real number $x$. The fiber slice

\[
\mathcal{F}(x)
    = \{ G \in \mathcal{X}^{*} : \pi(G) = x \}
\]

contains all identities whose selected digits give $x$. Extended invariants
depend only on the selector, not on collapse. Consequently, the image of
$\mathcal{F}(x)$ under the map

\[
G \longmapsto (\eta(G), \phi(G))
\]

typically occupies a large subset of the invariant plane. This illustrates
why the classical value of a real number constrains only a small portion of
its generative structure.

\section{Worked Examples of Extended Invariants}

\subsection{Periodic positive density example}

Let

\[
M(n) =
\begin{cases}
D & \text{if } n \text{ is even},\\
K & \text{otherwise}.
\end{cases}
\]

Then $\eta(G) = 1/2$. The indices are $n_j = 2j$, so $g_j = 2$ and

\[
\phi(G) = 0.
\]

This selector has positive density and perfectly regular spacing.

\subsection{Positive density with mild irregularity}

Let $M$ repeat the pattern $DDK$. Then

\[
\eta(G) = 2/3,
\qquad
\phi(G) = 0.
\]

The pattern is not uniform but its gap growth is bounded.

\subsection{Zero density with bounded gaps}

Let $M(n)=D$ when $n$ is prime. The density of primes is zero, so
$\eta(G)=0$, but since gaps grow sublinearly,

\[
\phi(G) = 0.
\]

\subsection{Zero density with large fluctuations}

Select digits at factorial indices:

\[
n_j = j!.
\]

Then $\eta(G)=0$ and

\[
\frac{g_j}{n_j} = j,
\qquad
\phi(G) = \infty.
\]

\subsection{Oscillating density example}

Expose digits in blocks

\[
D^{2^0}K^{2^0}D^{2^1}K^{2^1}D^{2^2}K^{2^2}\cdots
\]

Then the density oscillates between near zero and near one. One obtains

\[
\eta(G) = 0,
\qquad
\phi(G) = \infty.
\]

These examples show that $\eta$ and $\phi$ capture independent aspects of
asymptotic selector behavior.

\section{Robustness and Discontinuity}

The invariants $\eta$ and $\phi$ are robust under finite tail modifications
beyond any fixed prefix, but they are maximally discontinuous in the product
topology. The following examples demonstrate this behavior.

\subsection{Lower asymptotic density}

Let $G$ satisfy $\eta(G)=0$. Define $G_k$ by copying the first $k$ symbols of
$M$ and then exposing digits at all later positions. Then

\[
\eta(G_k) = 1
\quad\text{for all } k,
\]

yet $G_k \to G$ in the product topology. This shows that $\eta$ is not upper
semicontinuous and exhibits strong discontinuity.

\subsection{Relative gap growth}

Let $G$ have evenly spaced selected digits, so $\phi(G)=0$. Modify the tail of
$M$ in $G_k$ by inserting a single gap of length $\ell_k \to \infty$. Then

\[
\phi(G_k) = \infty
\quad\text{for all } k,
\]

and again $G_k \to G$ in the product topology. This demonstrates
discontinuity of $\phi$.

\subsection{No possibility of continuity}

Consider $G_k$ selecting $D$ at a single position $k$ and $G$ selecting $D$ at
no positions. Then $G_k \to G$, yet $\eta(G_k) = 1/k$ while
$\eta(G)=0$, and similarly for fluctuation. Both invariants fail continuity in
the strongest possible sense.

\section{Extended Invariants Inside Collapse Fibers}

Since invariants depend only on the selector, each fiber contains identities
with all allowable invariant values.

\subsection{Arbitrary density in a fiber}

Fix $\alpha \in [0,1]$. Construct a selector $M$ with $\eta(M)=\alpha$ and
place the digits of $x$ at selected positions. This yields a generative
identity in $\mathcal{F}(x)$ with invariant $\eta=\alpha$.

\subsection{Arbitrary fluctuation in a fiber}

Fix $\beta \in [0,\infty]$. Construct a selector with $\phi=\beta$ and again
place the digits of $x$ at selected positions. This produces an identity with
the desired fluctuation inside $\mathcal{F}(x)$.

\subsection{Simultaneous control}

Given any $(\alpha,\beta)$ in the invariant plane, build a selector realizing
both values and place the digits of $x$ accordingly. Thus each collapse fiber
maps to a substantial region in $(\eta,\phi)$ space.

\section{Summary}

This appendix presented explicit examples illustrating the full range of
possible values for the robust asymptotic invariants $\eta$ and $\phi$, as
well as slice based geometric interpretations that clarify how selectors
populate the generative space. Vertical slices represent finite symbolic
prefixes, horizontal slices represent long term invariant values, and fiber
slices reveal the symbolic variety compatible with a fixed collapsed value.
These perspectives support the broader conclusion that collapse conceals the
vast majority of generative structure.
\clearpage{}


\backmatter
\bibliographystyle{plain}
\bibliography{references}

\end{document}
