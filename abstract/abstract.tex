\begin{abstract}
This monograph develops the Generative Identity Framework, a structural approach
to real numbers based on symbolic generative mechanisms.  
A generative identity is a triple $(M, D, K)$ of infinite sequences: a selector
stream, a digit stream, and a meta-information stream.  
The classical real number associated with an identity is obtained by a
continuous collapse map that reads only the digits exposed by the selector.
Collapse is surjective and highly non-injective, and each collapsed value $x$
corresponds to a large symbolic fiber $\mathcal{F}(x)$ containing many
generative identities.

We study the internal structure of these fibers through continuous
observers.  
A structural projection is any continuous real-valued functional on the
generative space, and its dependence on finite prefixes is controlled by
computable dependency bounds in the sense of Type-2 Effectivity.  
These bounds imply prefix stabilization and tail invariance, which together
give a finite-information description of all continuous observers.

Using these tools, we construct a computable identity inside the effective
fiber of a computable real $x$ that agrees with a reference identity on
arbitrarily long prefixes, yet diverges along every computable structural
projection.  
This diagonalizer yields the Structural Incompleteness Theorem: no finite
family of continuous observers, even when combined with the collapsed value,
can recover the generative identity.  
Finite observation cannot capture the symbolic structure hidden beneath
collapse.

Finally, we introduce extended invariants that measure large-scale selector
behavior.  
The entropy balance $\eta$ (lower asymptotic density of digit exposures) is
lower semicontinuous, and the fluctuation index $\phi$ (relative gap growth)
is upper semicontinuous.  
Although discontinuous, these invariants provide coarse geometric embeddings
of generative identities and illustrate the diversity that persists inside
each collapse fiber.

The framework offers a unified structural, computational, and geometric view
of real numbers, revealing the continuum as a quotient of a rich symbolic
space and exposing intrinsic limits on what any finite process can observe.
\end{abstract}
