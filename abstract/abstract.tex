\begin{abstract}
This monograph develops the Generative Identity Framework, a structural
approach to real numbers based on symbolic generative mechanisms. A generative
identity is a triple $(M,D,K)$ of infinite sequences consisting of a selector,
a digit stream, and a meta-information stream. The classical real associated
with an identity is obtained by a continuous collapse map that reads only the
digits exposed by the selector. Collapse is surjective and highly
non-injective. Each real number $x$ corresponds to a symbolic fiber
$\mathcal{F}(x)$ containing many generative identities that share the same
canonical output.

The internal structure of these fibers is studied through continuous
observers. A structural projection is any continuous real valued functional on
the generative space. Dependency bounds from Type-2 Effectivity control the
finite prefix on which each observer depends. These bounds yield prefix
stabilization and tail invariance and show that every continuous observer
extracts only finitely many symbols at any fixed precision.

Using these tools, we construct a computable identity inside the effective
collapse fiber of a computable real $x$ that agrees with a reference identity
on arbitrarily long prefixes and is indistinguishable from it by every
computable structural projection. This yields the Indistinguishability
Theorem, which states that no finite family of continuous observers, even when
combined with the collapsed value, can determine the underlying generative
identity. Finite observation cannot recover the symbolic structure concealed by
collapse.

The monograph then introduces robust asymptotic invariants that measure
large scale selector behavior. The entropy balance $\eta$ describes the lower
asymptotic density of digit exposures, and the fluctuation index $\phi$
describes relative gap growth between selected positions. These invariants are
tail invariant and therefore robust under finite modification, but they are
everywhere discontinuous in the product topology. Their images provide coarse
geometric embeddings of selector behavior and illustrate the diversity that
persists inside each collapse fiber.

The framework offers a unified structural, computational, and geometric view
of real numbers. It presents the continuum as a quotient of a rich symbolic
space and establishes intrinsic limitations on what any finite observational
process can recover about generative structure.
\end{abstract}
