\begin{abstract}
This monograph develops the Generative Identity Framework, a structural theory
that interprets real numbers as collapsed values of symbolic generative
identities. A generative identity is a triple of infinite coordinate streams.
The collapse map reads only the exposed values in one chosen coordinate and
produces a classical real number. Collapse is continuous and surjective, and
each real number corresponds to a collapse fiber that contains many identities
with the same canonical output sequence.

Structural projections are continuous real valued observers on the ambient
product space. Results from Type 2 Effectivity show that every such observer
has a computable dependency bound that determines the finite prefix on which
its value depends. Dependency bounds imply prefix stabilization and tail
invariance. Observers therefore extract only finitely many coordinates at any
fixed precision. They recover only a small portion of the structure encoded in
a generative identity.

This finite information principle leads to structural incompleteness. Using
alignment and sewing methods inside effective collapse fibers, the monograph
constructs a computable identity that matches a reference identity on every
prefix inspected by a tower of computable observers yet differs on infinitely
many later coordinates. Every computable projection assigns the same value to
both identities. The result is the Indistinguishability Theorem, which states
that no finite or computable collection of continuous observers, even when
combined with the collapsed value, can reconstruct the symbolic identity from
which the value was produced.

Beyond collapse and finite observers, the monograph introduces extended
invariants that describe large scale behavior. The density invariant measures
the lower asymptotic frequency of exposures, and the fluctuation index measures
relative gap growth. These invariants depend only on the tail of the exposure
mechanism. They are invariant under finite modifications and are discontinuous
everywhere in the product topology. They reveal asymptotic behavior that is
invisible to finite observational processes and that varies freely within each
collapse fiber.

The Generative Identity Framework provides a unified topological and
computability theoretic perspective on real number representations. It views
the continuum as a quotient of a larger symbolic space and establishes intrinsic
limits on what finite information methods can extract from generative data.
Classical magnitude displays only a small portion of the structure present in
symbolic representations of real numbers.
\end{abstract}
