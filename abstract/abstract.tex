\begin{abstract}
This monograph develops the Generative Identity Framework, a structural theory
that interprets real numbers as collapsed images of symbolic generative
mechanisms. A generative identity is a triple $(M,D,K)$ consisting of a
selector stream, a digit stream, and a meta-information stream. The classical
real associated with an identity is produced by a continuous collapse map that
reads only the digits exposed by the selector. Collapse is surjective and
highly non-injective, and each real number $x$ corresponds to a collapse fiber
$\mathcal{F}(x)$ containing many identities that share the same canonical
digit sequence.

The geometry of collapse is complemented by the geometry of observation. A
structural projection is a continuous real valued functional on the ambient
generative space. Results from Type 2 Effectivity yield computable dependency
bounds that control the finite prefix on which each projection depends.
Dependency bounds imply prefix stabilization and tail invariance. They show
that all continuous observers extract only finitely many symbols at any fixed
precision and therefore access only a small portion of the internal structure
of a generative identity.

This finite information principle leads to a general form of structural
incompleteness. Using alignment and sewing techniques inside effective
collapse fibers, the monograph constructs a computable identity that agrees
with a reference identity on arbitrarily long prefixes yet remains
symbolically distinct. Every computable structural projection assigns the same
value to both identities, which proves the Indistinguishability Theorem. No
finite family of continuous observers, even when augmented with the collapsed
value, can recover the full symbolic structure that produced it.

Beyond collapse and observation, the monograph introduces extended invariants
that capture large scale selector behavior. The entropy balance $\eta$ is the
lower asymptotic density of exposed digits, and the fluctuation index $\phi$
measures relative gap growth between exposures. These invariants are invariant
under finite modification of the selector and describe features that survive
inside collapse fibers. They are everywhere discontinuous in the product
topology and illustrate the asymptotic richness that collapse conceals.

The Generative Identity Framework provides a unified structural, computational,
and geometric account of real numbers. It presents the continuum as a quotient
of a symbolic space and establishes intrinsic limits on what any finite
observational procedure can determine about generative structure. The results
demonstrate that classical magnitude reveals only a small portion of the
structure present in symbolic representations of real numbers.
\end{abstract}
