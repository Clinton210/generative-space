\begin{abstract}
This monograph develops the generative framework for representing real numbers through layered mechanisms.  
The generative space $\mathcal{X}$ consists of mixer, digit, and meta sequences equipped with the product topology, and its effective core $\mathcal{G}_{\mathrm{eff}}$ consists of computable mechanisms.  
Classical magnitude arises as the collapse of a generative identity and defines the primary invariant of the framework.  
Collapse maps $\mathcal{X}$ onto the continuum and maps $\mathcal{G}_{\mathrm{eff}}$ onto the computable reals, producing fibers that contain rich internal structure.  
Hybrid identities, which select digits with positive density, and ghost identities, which select digits with density zero, illustrate the range of internal behaviors compatible with a fixed magnitude.  
Secondary projections provide coordinate systems that summarize aspects of this structure but depend on only finite prefixes of effective identities.  
Using these finite dependence properties, a meta diagonalizer is constructed that evades any finite family of computable projections.  
This yields the Structural Incompleteness Theorem, which shows that no finite coordinate system can classify effective generative identities.  
Classical analysis appears as a quotient of the generative space under collapse, revealing magnitude as a coarse invariant of a much richer internal mechanism.  
The final chapter outlines measure-theoretic, dynamical, and computability-theoretic directions for future research.
\end{abstract}
